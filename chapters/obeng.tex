\documentclass[output=paper,colorlinks,citecolor=brown]{langscibook} 
\ChapterDOI{10.5281/zenodo.6393801}
\author{Samuel Gyasi Obeng\affiliation{Indiana University}}
\title[Discursive strategies for managing bad news]
      {Discursive strategies for managing bad news: Exemplification from Akan (Ghana)} 
\abstract{Bad news is a problem for both news bearers and news recipients, especially in situations where apprehensions run high given that it may run counter to people's in situ social and psychological needs \citep{Maynard2003}. The object of this paper was to examine the discursive strategies used by diseased individuals and their caregivers to deliver and manage their bad news. In pursing the above objective, transcripts of narratives collected from diseased individuals and their caregivers were subjected to empirical inspection with the view to determining the communicative strategies they employed to deal with their special situation. The study was done within the framework of language and liberty \citep{Obeng2018, Obeng2020} and the results showed that disease and ``powerful'' actors intrude on diseased individuals and care-givers’ negative liberty (by encroaching on their fundamental freedoms) and positive liberty (by preventing them from participating in their family and communal lives). Common linguistic strategies used in talking about disease and in seeking and protecting participants liberty include: silence, hesitations, reduplication, adjectives of quality, adverbs and intensifiers, verbs denoting physical sensation, and factive formulae (for evidentiality and credence). Discourse-pragmatic strategies for delivering bad news and for seeking liberty include the speech acts of complaining, blaming and assuring. Other strategies include avoidance, inferencing and polyvocality. It is concluded that to protect diseased individuals’ liberty from and liberty to, there is the need to put in place rights that protect these freedoms and empower diseased individuals to participate in their family and communal lives. Also, society must understand the communicational mores surrounding bad news delivery and management and be “educated” about the intertwining nature of language and care-giving.}

\IfFileExists{../localcommands.tex}{
 \addbibresource{localbibliography.bib}
 \usepackage{langsci-optional,langsci-branding}
\usepackage{langsci-gb4e}
% \usepackage{langsci-textipa}
% \usepackage{langsci-glyphs}
\usepackage[linguistics]{forest}
\usepackage{tabto}
\usepackage{multirow}
\usepackage{bbding}

\usepackage[normalem]{ulem}

\usepackage{tikz-qtree}

\usepackage{enumitem}

\usepackage{multicol}
\usepackage{stmaryrd} %double brackets

\makeatletter
\let\pgfmathModX=\pgfmathMod@
\usepackage{pgfplots,pgfplotstable}%
\let\pgfmathMod@=\pgfmathModX
\makeatother
\usepgfplotslibrary{colorbrewer}
\usetikzlibrary{fit}

\usepackage{jambox}
\usepackage{tikz-qtree-compat}
\usetikzlibrary{arrows, arrows.meta}
\usepackage{longtable}
\usepackage{subcaption}

 \makeatletter
\let\thetitle\@title
\let\theauthor\@author
\makeatother

\newcommand{\togglepaper}[1][0]{
%   \bibliography{../localbibliography}
  \papernote{\scriptsize\normalfont
    \theauthor.
    \thetitle.
    To appear in:
    Change Volume Editor \& in localcommands.tex
    Change volume title in localcommands.tex
    Berlin: Language Science Press. [preliminary page numbering]
  }
  \pagenumbering{roman}
  \setcounter{chapter}{#1}
  \addtocounter{chapter}{-1}
}

\newcommand{\bari}{\ipabar{\i}{.5ex}{1.1}{}{}}
\newcommand{\notipa}[1]{\textnormal{#1}}

\newcommand{\agre}{\textsc{agr}-\ol{eene}}

\renewcommand{\emph}[1]{\textit{#1}} % resetting a setting from ling-macros-modified (I think?)

% forest settings to make compact but (mostly) straight-spined trees:
\forestset{
fairly nice empty nodes/.style={
            delay={where content={}{shape=coordinate,for parent={
                  for children={anchor=north}}}{}}
, angled/.style={content/.expanded={$<$\forestov{content}$>$}}
}}

\forestset{sn edges/.style={for tree={parent anchor=south, child anchor=north}}}

\newcommand{\bex}{\begin{exe}}
\newcommand{\fex}{\end{exe}}

\newcommand{\bxl}{\begin{exe}}
\newcommand{\fxl}{\end{exe}}

\newcommand{\ix}[1]{\textsubscript{#1}}
\newcommand{\alert}[1]{\textbf{#1}}
\newcommand{\ol}[1]{\textit{#1}}


			\usetikzlibrary{shapes,arrows,positioning,decorations,decorations.pathmorphing,intersections}
\forestset{
nice empty nodes/.style={
    for tree={calign=fixed edge angles},
    delay={where content={}{shape=coordinate,for siblings={anchor=north}}{}}
},
}

\definecolor{dark-gray}{gray}{0.3}

%\usepackage{dingbat,pifont}


%%%%%%%%%%%%For arrows%%%%%%%%%%%%%

\newcommand\Tikzmark[2]{%
  \tikz[remember picture]\node[inner sep=0pt,outer sep=0pt] (#1) {#2};%
}
\NewDocumentCommand\DrawArrow{O{}mmmmO{3}}{
\tikz[remember picture,overlay]
  \draw[->,line width=0.8pt,shorten >= 2pt,shorten <= 2pt,#1]
    (#2) -- ++(0,-#6\ht\strutbox) coordinate (aux) -- node[#4] {#5} (#3|-aux) -- (#3);
}
\NewDocumentCommand\DrawDotted{O{}mmmmO{3}}{
\tikz[remember picture,overlay]
  \draw[->,line width=0.9pt,dotted,shorten >= 2pt,shorten <= 2pt,#1]
    (#2) -- ++(0,-#6\ht\strutbox) coordinate (aux) -- node[#4] {#5} (#3|-aux) -- (#3);
}
\NewDocumentCommand\DrawLine{O{}mmmmO{3}}{
\tikz[remember picture,overlay]
  \draw[line width=0.8pt,shorten >= 2pt,shorten <= 2pt,#1]
    (#2) -- ++(0,-#6\ht\strutbox) coordinate (aux) -- node[#4] {#5} (#3|-aux) -- (#3);
}
%%%%%%%%%%%%%%%%%%%%%%%%%%%%%%%%%%%%%


\newcommand{\baru}{ʉ}
\newcommand{\baruH}{\'\baru}
\newcommand{\baruL}{\`\baru}

\newcommand{\ep}{ε}
\newcommand{\epH}{\'\ep}
\newcommand{\epL}{\`\ep}

\newcommand{\schwa}{ə}
\newcommand{\schwaH}{\'ə}
\newcommand{\schwaL}{\`ə}

\newcommand{\oo}{ɔ}
\newcommand{\ooH}{\'\oo}
\newcommand{\ooL}{\`\oo}

\newcommand{\ds}{\textsuperscript{
	\hspace*{-2pt}\begin{tikzpicture}
		\draw[-{>[scale=0.5]}] (0,0.4) --(0,0.25);
	\end{tikzpicture}}}

\newcommand{\ch}{t͡ʃ}
\newcommand{\dz}{d͡ʒ}

\newcommand{\tgl}{ʔ}

%shortcuts for the complementizers
\newcommand{\mbuL}{mb\baruL}
\newcommand{\mbuHL}{mb\baruH\baruL}
\newcommand{\mbuLH}{mb\baruL\baruH}
\newcommand{\la}{lá}
\newcommand{\nda}{ndà}

\newcommand{\tsc}[1]{\textsc{#1}}
\renewcommand{\textscb}{ʙ}
\newcommand{\ipa}[1]{#1} %disable IPA

\newcommand{\SM}[1]{#1}

\DeclareNewSectionCommand
  [
    counterwithin = chapter,
    afterskip = 2.3ex plus .2ex,
    beforeskip = -3.5ex plus -1ex minus -.2ex,
    indent = 0pt,
    font = \usekomafont{section},
    level = 1,
    tocindent = 1.5em,
    toclevel = 1,
    tocnumwidth = 2.3em,
    tocstyle = section,
    style = section
  ]
  {appendixsection}

\renewcommand*\theappendixsection{\Alph{appendixsection}}
\renewcommand*{\appendixsectionformat}
              {\appendixname~\theappendixsection\autodot\enskip}
\renewcommand*{\appendixsectionmarkformat}
              {\appendixname~\theappendixsection\autodot\enskip}

\renewcommand{\lsChapterFooterSize}{\footnotesize}
 
 %% hyphenation points for line breaks
%% Normally, automatic hyphenation in LaTeX is very good
%% If a word is mis-hyphenated, add it to this file
%%
%% add information to TeX file before \begin{document} with:
%% %% hyphenation points for line breaks
%% Normally, automatic hyphenation in LaTeX is very good
%% If a word is mis-hyphenated, add it to this file
%%
%% add information to TeX file before \begin{document} with:
%% %% hyphenation points for line breaks
%% Normally, automatic hyphenation in LaTeX is very good
%% If a word is mis-hyphenated, add it to this file
%%
%% add information to TeX file before \begin{document} with:
%% \include{localhyphenation}
\hyphenation{
affri-ca-te
affri-ca-tes 
Līk-pāk-páln
pro-sod-ic
phe-nom-e-non
Chi-che-wa
Lu-bu-ku-su
Ngbu-gu
Boyel-dieu
Mat-chi
pho-neme
Mil-em-be
Nyan-chera
Mc-Pher-son
Tsoo-tso
Sku-pin
dis-tin-guishes
con-ser-va-tion
Me-dum-ba
}

\hyphenation{
affri-ca-te
affri-ca-tes 
Līk-pāk-páln
pro-sod-ic
phe-nom-e-non
Chi-che-wa
Lu-bu-ku-su
Ngbu-gu
Boyel-dieu
Mat-chi
pho-neme
Mil-em-be
Nyan-chera
Mc-Pher-son
Tsoo-tso
Sku-pin
dis-tin-guishes
con-ser-va-tion
Me-dum-ba
}

\hyphenation{
affri-ca-te
affri-ca-tes 
Līk-pāk-páln
pro-sod-ic
phe-nom-e-non
Chi-che-wa
Lu-bu-ku-su
Ngbu-gu
Boyel-dieu
Mat-chi
pho-neme
Mil-em-be
Nyan-chera
Mc-Pher-son
Tsoo-tso
Sku-pin
dis-tin-guishes
con-ser-va-tion
Me-dum-ba
}
 
 \togglepaper[1]%%chapternumber
}{}

\begin{document}
\SetupAffiliations{mark style=none}
\maketitle

\section{Introduction}\label{sec:obeng:1}\largerpage

Communication between care-givers, patients and families is an essential component of any high-quality care, especially in cases where the illness is serious, and anxieties are high. \citet{MaynardFreese2012}, in a study about how participants in interaction manage affective experiences and reactions to good news and bad news delivery noted how the delivery or receipt of news (both good and bad) momentarily disrupts recipients’ involvement in a social world whose contours and features they typically take for granted, as well as the power of such news to evoke and display strong emotion in the news recipient. 

Bad news includes news that conveys the results of a deadly and/or socially stigmatized disease, news that discusses and/or provides information about disease and death, or any form of tragedy or misfortune related to disease management, death and dying. In an earlier study, \citet{Maynard2003} noted that breaking bad news is a real problem for both news bearers and news recipients, especially in situations where anxieties run high; a claim supported by \citet{FallowfieldJenkins2004} and \citet{BrownEtAl2009}. Indeed, as \citet{Maynard2003} elucidates, people are reluctant to transmit bad news because of its communicative difficulty, the face threat it imposes on both the news bearer and the news recipient, and the impact (social, financial, etc.) it has on news recipients and their immediate families. According to \citet{TesserRosen1975} and \citet{WeenigEtAl2014, WeenigEtAl2001}, bad news with indefinite consequences is transmitted more often than bad news with definite consequences.

Research has shown that even though bad news delivery is so important, its acquisition and management within the care-giving professions are often not included in the training curriculum of care-giving institutions. Current studies into the training of medical personnel all emphasize the need to educate such personnel about appropriate and effective ways of breaking and managing bad news. For example, in a study that involved a communication skills workshop for nephrology fellows focusing on delivering bad news and helping patients define care goals, including end-of-life preferences, \citet{SchellEtAl2013} discovered that less than one-third of the studied nephrologists reported prior palliative care training. All respondents felt that communication skills were important to being a “great nephrologist,” and that an essential part of communication must involve ability to deliver bad news, express empathy, and discuss dialysis initiation and withdrawal. Another study by \citet{BaysEtAl2014} also emphasized the importance of providing training for health care personnel in breaking and managing bad news to, and with their patients, with the researchers concluding that communication skills intervention was associated with improvement in trainees' skills in giving bad news and expressing empathy. 

Another study that dealt with the need to have effective communication skills in breaking and managing bad news was that by \citet{WalczakEtAl2016} wherein the authors identified and synthesized evidence for interventions including communication skills training, education, advance care planning, and structured practice changes that targeted end-of-life communication. Participants targeted included patients, caregivers, healthcare professionals and multiple stakeholders. The researchers discovered that interventions targeting patients and caregivers are essential to breaking and managing bad news; however, they also found out that barriers to end-of-life communication may more effectively be removed via multi-focal interventions.

In another study about patients’ perception regarding the disclosure of news about their cancer as regards physician counselling and how they perceived the flow of information between hospital‐based and family physicians, \citet{SpiegelEtAl2009} discovered that 37.7\% of their respondents viewed the news disclosure about their cancer as being presented to them “very empathically” or “empathically.” Two-thirds of the patients (62.3\%) stated that the news disclosure about their cancer was presented to them “not so empathically” or “not at all empathically.” Most importantly, the researchers observed the important role of family physicians in breaking bad news to the patients. Specifically, they discovered that patients were more likely to state that the bad news about their cancer had been done “very empathically” or “empathically” when counselled by family physicians in contrast to when they had been counselled by hospital‐oncologists or self‐employed specialists. 

Furthermore, a great majority of patients (81.8\%) felt that they had been given adequate opportunity to ask the questions that they considered to be important when they were counselled by a family physician. Only 43.5\% felt that they had been given adequate opportunity to ask the questions that they considered to be important when they were counselled by a hospital‐oncologist or a self‐em\-ployed specialist (44.3\%). Also discovered by the researchers was the fact that 56.8\% of the patients preferred to discuss the suggested cancer therapies with an oncologist. The above discoveries led the researchers to conclude that it is essential for oncologists to involve family physicians in breaking bad news to patients. 

Within discourse theory, managing bad news falls under therapeutic discourse that involves talk-in-interaction that represents the social practice between clinicians and clients \citep{Leahy2004}, talk-in-interaction between diseased individuals, their caregivers and other discourse types with the potential to bring emotional, social, and physical relief, as well as discourse types aimed at assisting victims to cope with and\slash or adapt to their difficult situations \citep{Obeng2008}. 

On the discursive strategies for breaking and managing bad news, research has shown that within the western medical discourse ecology, hesitations, qualifiers, and maneuver or circumlocution \citep{Maynard2003, BrownEtAl2009} and prosody, especially volume and pitch \citep{MaynardFreese2012}, are some of the common strategies used in delivering and managing bad news. In Akan (Ghana), \citet{Obeng2008} identified “speaking to the wind” (addressing God, a deity or a person who is not in a ratified state of talk with you), songs, conventional indirectness, non-conventional indirectness (especially, “twisted speech” e.g., idioms and proverbs), use of different \textit{wh}-question types as being some of the strategies for breaking and managing bad news. \citeauthor{Obeng2008} also noted that in Ghanaian society, there is the tendency for diseased individuals to blame their disease on other people (sometimes even on their caregivers), and that being able to blame others offers some relief since it takes responsibility away from the diseased individuals as being the source of their illness even if such communicative behavior creates tensions between them and those they blame for their illness. Indeed, on the conveyance of diseased individuals’ emotional state, \citet{KalmykovaMergenthaler1998} argue that narratives serve as one of the commonest and best communicative means to convey speakers’ emotional states given that they are able to stimulate responses of listeners. 

Another strategy for breaking bad news, according to \citet{Movahedi1996}, is through a second language given that such a language provides space where certain personal and cultural facets of a diseased individual or a caregiver may be more easily verbalized than through their first language. Specifically, there is the tendency for a diseased bilingual to resort to code-switching when breaking and talking about bad news since such bad news tends to be more ‘tellable’ in a less familiar language or register.

The World Health Organization’s Constitution \citep{World-Health-Organization1946} observes the right to health as one of a set of internationally agreed human rights standards which is inseparable or “indivisible” from other rights such as that to be free from intrusion from the state or institutions affiliated with a state or the right to participate in one’s communal affairs. Other rights include the right to vote, the right to free speech, and the right to provide for the underlying determinants of health, such as safe and potable water, sanitation, food, housing, health-related information and education, and gender equality. However, thus far, there has been no study of the nature and\slash or type of language used by diseased individuals to seek liberty from their care-givers (including health personnel) or people around them while breaking and dealing with their bad news or when dealing with their unique burdensome health communication; this study seeks to fill the gap. 

\section{Theoretical framework}\label{sec:obeng:2}

The study is done within the framework of language and liberty \citep{Obeng2020, Obeng2018}. I view health as both a \textit{human right} (by justifiably belonging to every person), and a \textit{civil right} (by requiring society to ensure the right to equality of health for all). Obeng’s theory on language and liberty is inspired by Sir Isaiah  \citegen{Berlin1960} theory on \textit{liberty} which includes \textit{liberty from} (\textit{negative liberty} which involves the protection of individuals from the intrusions of the society and others into their fundamental freedoms), and \textit{liberty to} (also called \textit{positive liberty}, which guarantees the right of individuals to participate in governance and to share in the political power of their communities). Positive liberty includes the right to freedom and/or independence at the various levels of an individual’s community. 

In working within the theory of \textit{language and liberty}, it is important to establish the fact that besides liberty being a philosophical, political and juridical concept, it is also a health concept, and that language is used to express liberty and the above associated concepts just as liberty depends on language to become a reality. Therefore, in working within the theory of language and liberty, I posit: (a) that disease, caregivers (community), and sometimes the state, may encroach on diseased individuals’ negative liberty by forcing them into certain states of health (physical, emotional etc.), and thereby preventing them from doing what they can otherwise do; and (b) that disease, caregivers (community), and sometimes the state, may deny diseased individuals their positive liberty by preventing them from exercising their right to participate and share in their family and communal lives. As a linguistic concept, language and liberty are intertwined. In particular, liberty informs and is informed by language and relies on language to become reality. 

Working within the theory of language and liberty requires me to examine such important issues as the nature of material communicative and culturally congruent conditions put in place (or made available) to ensure that diseased individuals can seek and maintain liberty (that is, be guaranteed liberty). In particular, I examine the available culturally congruent communicational freedom mores and/or framework which protect individuals’ liberties and thus guarantee liberty in both the negative and positive senses. Specifically, I examine the linguistic and discourse-pragmatic features that are available and allowable for use by diseased individuals to vent their socio-emotional frustrations, challenge their care-givers if need be, and seek material support to enable them to deal with their situation, among others. The absence of such culturally congruent material and communicational freedom mores in a community creates a situation in which people in power (care-givers, lineage elders, etc.) may needlessly infringe on diseased individuals’ negative and positive liberties with impunity, especially when such individuals hold divergent views about the nature and/or type of their care. Two important questions I attempt to answer are: 

\begin{enumerate}\sloppy
\item What linguistic and discourse-pragmatic strategies are used by participants (diseased persons and/or their caregivers) to seek their negative and positive liberty?
\item Through what discursive strategies is liberty denial made reality?
\end{enumerate}

\section{Aim and method}\label{sec:obeng:3}

This paper aims at recapitulating and extending work I did on Akan therapeutic discourse \citep{Obeng2008} by examining the communicative strategies used by diseased individuals, caregivers, and persons they interact with, to seek liberty while breaking and managing bad news. In pursuing the above aims, I examine:

\begin{enumerate}\sloppy
     \item \textit{The linguistic strategies} (sound types or patterns, silence and pausal phenomena, reduplication, prosodies, vocabulary, phrasal and sentential formulae/categories);
     \item \textit{Discourse-pragmatic strategies} (inferences and presuppositions derived from utterances), various forms of speech acts such as \textit{complaining, blaming, assuring, avoidance}, etc.; and
     \item \textit{Metacommunicative strategies} (such as indirectness and (non)verbal communicative stimuli that help support or interpret that which is said)
\end{enumerate}

that are used by victims of socially stigmatized diseases and their immediate care-givers, relatives or friends to seek liberty while managing their bad news. 

Data for the study are made up of transcripts of 6 recorded interactions collected in Akan-Twi, in Asuom and Accra (Ghana). Data collection spanned August 2015 to October 2015 and February 2016 to March 2016. Participants included 6 diseased individuals and 13 caregivers. The data content dealt with sickness and death. Specifically, the data consisted of discourses involving diseased individuals’ lived experiences and those of their caregivers, and interactions about death, especially, impending death. A form of the Akan orthography is used in the transcription and participants’ initials are used to create anonymity. 

The data were chosen because they involved the delivery and management of bad news and were produced by care-givers and diseased individuals who lived with and thus experienced bad news. Participants’ lived experiences either as diseased persons or as close relatives caring for the diseased persons meant they experienced the denial of liberty either as agents or patients thereby making their assertions and claims about liberty authentic. Also, the data are replete with speech acts that are common in healthcare management discourse. Among such speech acts are: complaints relating to uncertainty about diagnoses, insufficient care, or lack of care; criticism of the nature of care of care-givers or of the diseased individuals; request for help (financial or moral support); assurance; among others. Through the data we are led into the studied participants’ worldview about lived experiences relating to the pursuit of liberty and the various discursive strategies via which liberty was pursued.

\section{Texts and discussion}\label{sec:obeng:4}

The first hypothesis I put forward relates to the interconnectedness between language and liberty and is stated as follows: in asymmetrical medical discourse, actors without power (diseased individuals and care-givers who bear bad news) have linguistic and pragmatic strategies for seeking liberty; that is, speaking the unspeakable (tabooed expressions) irrespective of whether or not their pronouncements will be viewed as ungrateful about their care, dissatisfied with their care, and in the case of care-givers, viewed as uncaring. It is argued that being able to speak the unspeakable offers the bad news bearers relief (communicational as well as emotional). We posit further that through their language, we are brought into the bad news bearers’ worldview about how language and liberty inform each other. Even though two long extracts are cited below in support of the above observations, it is important to note that the cited extracts represent the many cases that help illustrate the above observations and others.

\subsection{Excerpt 1}\label{sec:obeng:4.1}
Context: Breaking the bad news about a tabooed disease (cancer). Recorded in Asuom (Ghana) September 12, 2015.

\ea\label{ex:obeng:1}%1
    T1 KN:\\
    \gll    Me sewaa, mate sɛ AMD yare.\\
            my aunt I’ve.heard that AMD sick \\
            (4.0)\\
\ex\label{ex:obeng:2}
    T2 AB:\\
    \gll    Hm! Deɛ aba deɛ ɛso sen me.\\
            hm what has.happened \textsc{foc} it.big(ger) than me\\
\ex\label{ex:obeng:3}
    T3 KN:\\
    \gll    Yadeɛ no hyɛ ne he?\\
            disease the be.\textsc{loc} her where\\
            (3.0) \\
\ex\label{ex:obeng:4}
    T4 KN:\\
    \gll    Ɛdeɛn na ɛha no?\\
            what \textsc{foc} it.trouble her\\
            (3.0)\\
\ex\label{ex:obeng:5}
    T5 AB:
    \ea
    \gll    Obiara nnim. Ebi se ɛyɛ duabɔ, ebi nso se ɛyɛ nkrɔfoɔ no biribi no bi.\\ 
            everyone \textsc{neg}.know some say it.be curse some also say it.is those.people their thing that some\\
    \ex
    \gll    Sɛ ɛyɛ deɛn, sɛ ɛyɛ deɛn, obi nhu.\\
            whatever it.be what whatever it.be what someone \textsc{neg}.know\\
    \z
\ex\label{ex:obeng:6}
    T6 KN:\\
    \gll    Mode no akɔ dɔkota?\\
            you.take her has.gone hospital \\
\ex\label{ex:obeng:7}
    T7 AB:\\
    \gll    Yɛde no akɔ baabi ara Obiara nhu adekodeɛ.\\
            we.take her has.gone every where everyone not.see thing.exactly\\
\ex\label{ex:obeng:8}
    T8 AG:\\
    \ea
    \gll    Onipa wote hɔ na wo honam ahyehye ayɛ tumtumtumtum.\\
            Human you-be there \textsc{conj} your skin has.burnt.burnt has.become black.black.black.black \\
    \ex
    \gll    ayɛ ammɔdin \\
            become unmentionable \\
    \z
\ex\label{ex:obeng:9}
    T9 AB:
    \ea
    \gll    Sɛ ɔakɔtia biribi so o, sɛ deɛn o, obiara nnim. \\
            whether she.has.gone.step something on \textsc{conj} whether what \textsc{conj} everyone not.know\\
    \ex
    \gll    Yɛadeɛ yi deɛ ɛmfiri ha! \\
            disease this as.for it.not.from here \\
    \z
\ex\label{ex:obeng:10}
    T10 AG:\\
    \gll    Seesei deɛ Onyame n’adom oo, n’adom! \\
            Now as.for God his.grace \textsc{interj} his.grace\\
\ex\label{ex:obeng:11}
    T11 KN:\\
    \gll    Na ɔwɔ he?\\
            And she-be where\\
\ex\label{ex:obeng:12}
    T12 AB:\\
    \gll    Ɔhyɛ dan no mu hɔ \\
            she.be.\textsc{loc} room the in there\\
    \z
    
   \noindent\relax         [KN enters AMD’s room]
\ea\label{ex:obeng:13}
    T13 KN:\\
    \gll    AMD ɛte sɛn?\\
            AMD it.be how?\\
\ex\label{ex:obeng:14}
    T14 AMD:\\
    \gll    =Ei, Braa! aa mekonkɔn hɔ.\\
            \textsc{interj} Brother well I.hang.in.hang.in there \\
            (2.0)\\
\ex\label{ex:obeng:15}
    T15 KN:\\
    \gll    Apɔ mu te sɛn? \\
            joints in be.stative how\\
            (4.0)\\
\ex\label{ex:obeng:16}
    T16 AMD:\\
    \gll    Braa, me na meni oo! Hm! (1.0)\\
            brother I \textsc{emp} I.be (like.this) \textsc{interj} ɦm (well)\\
            (2.0)\\
\ex\label{ex:obeng:17}
    T17 KN:\\
    \gll    Wo ho ho yɛ wo den.\\
            your self will.be you strong\\
            (3.0)\\
\ex\label{ex:obeng:18}
    T18 AMD:\\
    \ea
    \gll    Sɛ wo ara wonim; obi a mewɔ m’dwuma a menhia obiara mmoa.\\
            as you \textsc{emp} you.know someone who I.have my.work who I.not.need anyone’s help\\
    \ex
    \gll    ɛnnɛ hwɛ! Ɛno ara ne sɛ mete faako \\
            now/today look that \textsc{emp} be that I.stay one/same.place\\
    \z
\ex\label{ex:obeng:19}
    T19 KN:\\
    \gll    Na wɔn de wo akɔ dɔkota?\\
            and they take you have.gone hospital\\
\largerpage[2]
\ex\label{ex:obeng:20}
    T20 AMD:\\
    \gll    Ɛno deɛ mete wɔn so a mesɔre. Obi ara nim sɛ wɔn ayɛ nea wɔn bɛtumi!\\
            that as.for I.sit them on if I.get.up everyone knows that they have.done what they can\\
\ex\label{ex:obeng:21}
    T21 KN:\\
    \gll    Na mokɔɛ no wɔn se dɛn na ɛɛkɔɔso yadeɛ yi?\\
            and you-went when they say what \textsc{emp} it.\textsc{prog}.cause disease this\\
            (2.0) \\
\ex\label{ex:obeng:22}
    T22 AMD:
    \ea
    \gll    Hm! (2.0) Obi ara ka ne deɛ. Ebi se kansa, ebi se sei, ebi se sei. \\
            ɦm (2.0) every one says his/her thing some say cancer some say this some say that \\
    \ex
    \gll    Edin bebree. Wobɔ din a ɛnyɛ yie.\\
            names many you.mention name if it.\textsc{neg}.be possible \\
    \z
\ex\label{ex:obeng:23}
    T23 KN:\\
    \gll    Na dɔkotafoɔ no yɛɛ ho biribi maa wo?\\
            and doctors the do.\textsc{pst} self something for you\\
\ex\label{ex:obeng:24}
    T24 AMD:\\
    \ea
    \gll    Deɛ wɔn kaeɛ ara ne sɛ yɛmfa me mmra fie. \\
            what they said only be that we.bring me come home\\
    \ex     
    \gll    Yadeɛ yi ama me asa ama me ayɛ abɔfra.\\
            disease this has.made me encumbered has-made me become child\\
            (2.0)
    \z
\ex\label{ex:obeng:25}
    T25 KN:\\
    \gll    Me sewaanom boa wo anaa? \\
            my aunt.them help you \textsc{q}\\
            (4.0)\\
\ex\label{ex:obeng:26}
    T26 AMD:\\
    \gll    Aaa ɛyɛ. Wo ara wonim; ayɛ sɛ wo nsa atɔ ɔsaman aduane mu.\\
            well it.okay you \textsc{emp} you.know (it).has.become like your hand has.fallen ghost’s food in\\
\ex\label{ex:obeng:27}
    T27 KN:\\
    \ea
    \gll    Kyerɛ sɛ ɛnyɛ papa bi ara. (4.0) \\
            meaning that it.not.be good exactly very. \\ 
    \ex     
    \gll    ɦm (2.0) it.will.be well/okay\\
            Hm (2.0) Ɛbɛyɛ yie\\
    \z
\ex\label{ex:obeng:28}
    T28 AMD:\\
    \gll    Aaa anomaa bi ne ne su; ɔse: “Aaa yɛrehwɛ.”\\
            Well bird a and its cry; it.says Well we.are.looking \\
\z

\subsection{Excerpt 1 (Translation)}

\begin{exe}
    \exr{ex:obeng:1} T1 KN: My Aunt, I’ve heard that AMD is sick. (4.0)
    \exr{ex:obeng:2} T2 AB: Well, What’s happened is bigger than me (is beyond me).
    \exr{ex:obeng:3} T3 KN: Where exactly (in her body) is the disease? (3.0) 
    \exr{ex:obeng:4} T4 KN: What is wrong with her? (3.0)
    \exr{ex:obeng:5} T5 AB: No one knows. Some say it is a curse, others say it is that sickness that affects the other people (i.e., a disease of foreign nature). Whatever it is, whatever it is, no one knows.
    \exr{ex:obeng:6} T6 KN: Have you taken her to the hospital?
    \exr{ex:obeng:7} T7 AB: We’ve taken her everywhere. No one knows what it is.
    \exr{ex:obeng:8} T8 AG: A human being whose body has burnt all over and turned into coal-black. It has become an unmentionable.
    \exr{ex:obeng:9} T9 AB: If she has stepped on something (juju/voodoo) or whatever it is, no one knows! As for this disease, it is not from here (it is a foreign disease).
    \exr{ex:obeng:10} T10 AG: As it is now, we’re counting on God’s grace (that is, only God can save her).
    \exr{ex:obeng:11} T11 KN: And where is she?
    \exr{ex:obeng:12} T12 AB: She's inside the room over there.
\end{exe}

\noindent[KN Enters AMD’s Room]

\begin{exe}
    \exr{ex:obeng:13} T13 KN: AMD How are you?=
    \exr{ex:obeng:14} T14 AMD: =Wow! Brother! (2.0) Well, I’m hanging in.
    \exr{ex:obeng:15} T15 KN: How are you? (4.0)
    \exr{ex:obeng:16} T16 AMD: Brother Look at me (Can you imagine what’s happened to me)! (1.0) Well! (2.0)
    \exr{ex:obeng:17} T17 KN: You will be well. (3.0)
    \exr{ex:obeng:18} T18 AMD: You very well know; Someone (a person) with my own business who did not need help from anyone! Now look! All I do is stay at the same place (I’m encumbered by the disease).
    \exr{ex:obeng:19} T19 KN: Have they taken you to the hospital?
    \exr{ex:obeng:20} T20 AMD: They’ve done their best. Everyone knows they’ve done their best!
    \exr{ex:obeng:21} T21 KN: And when you went (to the hospital) what did they say is causing this disease? (2.0)
    \exr{ex:obeng:22} T22 AMD: Well! (2.0) Everyone had their own theory about the cause. Some say it is cancer, some say this, others say that. So many names. Some are just unmentionable.
    \exr{ex:obeng:23} T23 KN: And did the doctors do something about it?
    \exr{ex:obeng:24} T24 AMD: All they said was that they should bring me home. This disease has encumbered me; it has made become like a child (dependent on others). (2.0)
    \exr{ex:obeng:25} T25 KN: Does my Aunt and others help you? (4.0)
    \exr{ex:obeng:26} T26 AMD: Well it’s okay. You very well know. It’s like putting your hands in a ghost’s food (It’s like a forced commitment not on their own free will).
    \exr{ex:obeng:27} T27 KN: Meaning the help is not very good. (4.0) Well (2.0) it will be well\slash okay.
    \exr{ex:obeng:28} T28 AMD: A bird and its song/cry: it sings “Well, we’ll see'': (I’ll believe it if I see it).
\end{exe}

An observation of the above excerpt reveals the use of various pausal phenomena in an attempt to hold back the bad news given its communicative difficulty. Prominent among the different types of pausal phenomena are silence, hesitation or voiced pauses, and a combination of both. What is unique about the silent pauses is the duration. Specifically, the silent pauses are much longer than what is generally seen as normal in Akan interaction which is between 0.1 and 0.5 seconds \citep{Obeng1987, Obeng1989, Obeng1999} . In American English conversations, \citet{jefferson1983} identified 0.5 seconds as being the normal silent pause between a current turn and the next turn. The 4.0-second pause between KN’s question \textit{ɛbaa no sɛn?} ‘what happened?’ in T1 and AB’s turn (T2) signals the difficulty in breaking the bad news. In fact, AB’s first utterance is a voiced pause \textit{hm} [ɦm] produced with a piano volume and a low pitch height which, in Akan, is used to preface or signal an upcoming constrained utterance. Thus, the use of the voiced pause, \textit{hm} [ɦm], acts as a hedge and points to the fact that it is culturally neither congruent nor appropriate for AB to deliver the bad news at that point in the interaction given that her liberty to do so is impeded by the cultural conventions of delivering bad news. Besides the phonetic cues of silent and voiced pauses, we also have the syntactic cue of focus marking to project the bad news. The expression:\largerpage

\ea\label{ex:obeng:29}
    \gll    Deɛ aba deɛ, ɛso sen me.\\
            what hascome \textsc{foc} it-big(ger) than me\\
    \glt    `This news/tragedy is bigger/beyond my capability to break/deal with.'
\z

Thus, we see from T1 and T2 the restriction placed on the bad news bearer in breaking the bad news. Note that an observation of T3 to T5 also confirms the communicative difficulty associated with the bad news delivery. Both T3 and T4 are produced by the same speaker, KN. Within conversational analysis (CA) theory and practice, the 3.0-second pauses between T3 and T4, and that between T4 and T5 are referred to as initiative time latency pause and point to a current speaker’s termination of her turn and a next speaker’s refusal to assume turn ownership thereby forcing the current speaker to continue talking. The fact that AB comes in only after KN rephrases the question in T3 and poses it again in T4 points to the level of difficulty in delivering the news about AMD’s illness. AB’s answer (T5), \textit{Obiara nnim. Ebi se ɛyɛ duabɔ, ebi se ɛyɛ nkrɔfoɔ no biribi no bi. Sɛ ɛyɛ deɛn, sɛ ɛyɛ deɛn, obi nhu} ‘No one knows. Some say it is a curse, others say it is that (disease) belongs to the other people’, is of considerable ethnopragmatic import: First, she resorts to avoidance via a factive construction \textit{Obiara nnim }‘No one knows’ in which the subject of the sentence, \textit{Obiara}, literally meaning everyone but idiomatically means ‘no one’ due to the negative particle \textit{n}- that precedes the verb \textit{nim} ‘know.’ Note that in Akan medical discourse use of the quantifier \textit{Obiara} is a form of number game used to support an assertion. In the current discourse context, the expression \textit{Obiara nnim} ‘No one knows’ is used to signify the difficulty in breaking the bad news. If no one knows, then who is she, a non-medical person, or one without the power of divination about illness, to claim knowledge of it?

AB’s next sentence in T5, \textit{Ebi se ɛyɛ duabɔ, ebi se ɛyɛ nkrɔfoɔ no biribi no bi,} ‘Some say it is a curse, others say it is that (sickness) which belongs to those people’ also involves avoidance via the use of the non-specific determiner, \textit{ebi} ‘some (people).’ Note also that the use of avoidance also points to the intrusions on AB’s liberty; she is not permitted to name the source or cause of AMD’s illness given that delivering such news could result in communal disintegration since in the studied community such diseases are attributable to relatives who are either jealous of the diseased individual and therefore use a curse to make her sick or even cause her death. Not naming such people helps maintain harmony; even if on the outside. Note also that AB is not responsible for the inferences and conclusions others may draw or have regarding the disease’s causation. She expresses her non-responsibility for such conclusions by engaging in a disclaimer by way of attributing the conclusions drawn about the disease to others. The expression ɛyɛ \textit{nkrɔfoɔ no biribi no bi} ‘it is that (sickness) which belongs to those people’ attributes the cause or source of a disease to a foreign source or other(s) rather than self or one’s own group and enables the caregiver to blame those people from whom the disease originated for the disease’s burden.

When asked whether they had taken the diseased person to the hospital (T6), AB responds in T7:
\textit{Yɛde no akɔ baabi ara. Obiara nhu adekodeɛ.} ‘We’ve taken her to every place. No one knows what it is.’ Use of the first person plural pronoun prefix, \textit{yɛ}- ‘we’, suggests that the caregiving had been communal and that there had been no apathy. The second sentence which recapitulates her earlier statement about no one knowing what the disease was, is a marker of frustration and a request. The words \textit{baabi ara }‘everywhere’ and \textit{Obiara} ‘no one’ which preface \textit{nhu adekodeɛ} ‘(not) know what it is,’ point to the exhaustion of all possible options about cure by all persons with no success; a waste of scarce communal resources for a hopeless cause. The above expressions also signal to KN that if he knows of any place (other than those already tried by the caregivers) that they (the caregivers) may not be aware of, then he must, as required by custom, take the diseased individual there for treatment. Thus, this is both an indirect request for KN to help and also a warning for KN not to blame the caregivers for not trying their best to help.

In T8, AG, another caregiver, describes the disease without mentioning the diseased person’s name; she refers to her with the noun \textit{onipa} ‘person.’ Use of the non-specific reference, \textit{onipa} ‘person,’ instead of the third person pronoun, \textit{ɔnʊ} ‘she,’ or the diseased person’s name, affords the speaker the liberty to talk about the disease without getting into the specifics of naming the diseased person. Use of the reduplicative \textit{ahyehye} ‘multiple “burns”’ describes the intensity of the disease as well as the multiple places where the melanoma ‘burns’ have affected the diseased person. Note also that the reduplication of the word \textit{tumm} ‘black’ four times also describes the extent of the coloration of the skin by the disease; something which she goes on to describe as an unmentionable (that is, a taboo). Like the caregivers before her, she ends by also attributing the disease to a foreign source saying, \textit{Yɛadeɛ yi deɛ ɛmfiri ha!} ‘As for this disease, it is not from here.’ She could simply have said ‘it’s not from here’ or ‘this disease is not from here.’ Use of the focus marking expression \textit{Yɛadeɛ yi deɛ}  ‘as for this disease,’ adds to the uncertainty about the nature of the disease and its source. By attributing it to a foreign source, she is given the liberty to talk about the disease since any repercussion or shame associated with it is deflected to an unknown source.

It is important to note that up till T13, the disease, cancer, had not been named because it is considered a tabooed disease in Akan society. The fact that the name of the disease was avoided by the care-givers leads us to posit our second observation which is that: care-givers whose liberty to deliver bad news may be constrained, may leave the delivery of the bad news to the diseased individuals themselves in order to obviate crises. It is only after KN speaks with AMD that she names the disease. Specifically, it is in T22 that AMD mentions cancer, but even then, she resorts to uncertainty. When asked about what disease she had and what was causing it (T21), she hesitates first, pauses, and then mentions the possibility of it being cancer. She then goes on to explain the uncertainty about the disease type with the expression, \textit{ebi se sei, ebi se sei. Edin bebree. Wobɔ din a ɛnyɛ yie.} ‘some say this, others say that, so many names, it is an unmentionable.’ The repetition of the construction \textit{ebi se sei}, is to create and amplify the extent of doubt about the exact nature of the disease. Via the repetition, she appears to be saying that no one really knows what the disease is. Also, the expression, \textit{Edin bebree} ‘so many names’, adds to the uncertainty about the exact nature/type of disease. It also points to the extent of the taboo nature of the disease or its incurable nature. If the disease has that many names, and if caregivers have not settled on a name, then it is, to say the least, a bad disease. Finally, the expression, \textit{Wobɔ din a ɛnyɛ yie} ‘it is impossible to mention its name’ that is, ‘it is a tabooed disease,’ lends a further measure of support to the dangerous and/or terrible nature of the disease and how restraining its effect had been on AMD’s life. In fact, one could argue that the nature of the disease intrudes on AMD’s liberty to even name it without resorting to avoidance and circumlocution.

On the denial of her liberty, the following three expressions that were produced by AMD,

\begin{enumerate}
    \item[a.] \textit{Braa, me na meni oo!} (1.0) Hm! ‘Brother, can you believe it’s me you’re looking at!’ in T14; 
    \item[b.] \textit{Sɛ wo ara wonim; obi a mewɔ m’dwuma a menhia obiara mmoa; ɛnnɛ hwɛ! Ɛno ara ne sɛ mete faako} ‘You know very well I had my own business and was never in need, look at me today! I’m just stuck at one place (because of the diseases)’ in T16; and
    \item[c.] \textit{Yadeɛ yi ama me asa ama me ayɛ abɔfra} ‘This disease has made me look like a child (weak and in need of help)’ in T22; 
\end{enumerate}
are discursively most significant given that they each express the diseased individual’s (AMD’s) recognition of her disease intruding on her liberty by making her needy, ruining her business, immobilizing her, and making her dependent on others for care and sustenance. From the discourse-pragmatic perspective, the phrase \textit{me na me ni oo} ‘Can you believe it’s me you’re looking at?’ expresses incredulity about her physical appearance brought about by the disease. 

From the above excerpts, we observe how AMD’s utterances index the extent to which the disease had encroached on her liberty (negative and positive). From the point of view of valence, AMD’s utterance, \textit{me na me ni oo}, expresses the emotional burden brought on her by the disease and from the socio-economic point of view, the downward spiral of her social status from self-sufficiency to that of dependency. The interjection \textit{oo} expresses self-pity.

KN’s utterance in T17, \textit{Wo ho bɛyɛ wo den,} ‘You will be well,' is a speech act of assurance done via a declarative sentence to suggest confidence in what is uttered and intended to help AMD to not give in to the disease but to emotionally manage it. AMD’s self-sufficiency and independence are expressed by the sentence, \textit{Sɛ wo ara wonim; obi a mewɔ m’dwuma a menhia obiara mmoa} ‘You know very well; some(one) who had her own business and was never in need.’ This construction is made up of a factive formula, \textit{Sɛ wo ara wonim} ‘You know very well,’ which establishes the truth or credence of the following utterance, \textit{obi a mewɔ m’dwuma a menhia obiara mmoa} ‘(some(one)) who had my own business and was never in need;’ her self-sufficiency. Her last utterance, \textit{ɛnnɛ hwɛ! Ɛno ara ne sɛ mete faako} ‘today/now look, all I do is stay at the same place,’ suggests that her right to physical mobility has been taken away from her by the disease. Even though she does not mention the disease as having restricted her to her home, it is implicitly stated. The expression, \textit{ɛnnɛ hwɛ!} ‘today/now look,’ which was accompanied by a hand gesture of showing both palms upwards and drawing them to herself, communicates her frustration at being paralyzed and bedridden. 

In T25, KN asks about care-giving and one sees from AMD’s answer the communicative burden inherent in the question. Responding directly that the care provided her by the relatives was insufficient and also not done of their own free will but out of ‘social compulsion’ would have marred her face and those of her caregivers. She seeks communicative liberty by: (a) resorting to hesitation, as in the sentence, \textit{Aaa ɛyɛ} ‘Well, it’s okay;’ and (b) using a factive expression, \textit{Wo ara wonim} ‘you very well know’ which presupposes the truth of the following statement and consequently makes it impossible to refute the propositional content of the following complaint, \textit{ayɛ sɛ wo nsa ato ɔsaman aduane mu} ‘it’s like putting your hands in a ghost’s food'; meaning committing to something and hence not being able to discontinue the care-giving. Via inferencing, AMD appears to be saying that the care-givers are providing her with care because they are socially obliged and not because they care so much about her welfare. Indeed, KN, in T27, resorts to inferencing to explain AMD’s previous utterance (T26) by explicitly saying ‘that means the care isn’t very good’ and that forces AMD to be more direct by saying, \textit{Kyerɛ sɛ ɛnyɛ papa bi ara} ‘Meaning, it’s not very good.’ 

Phonetically, the long silent pause of 4.0 seconds within KN’s turn points to the fact that he was done with his turn and wanted AMD to take over the turn ownership. The fact that AMD did not take over the floor suggests the communicative burden placed on her by KN’s line of questioning. Not being grateful to her care-givers is tabooed no matter how insufficient the care might be. When KN continues, he engages in the use of a voiced pause \textit{Hm} [ɦm] which is also a hesitation and signals that the discourse thus far may have involved a face threat. Pausing within one’s turn for 2.0 seconds after the issuance of the voiced pause also confirms the face-threat inherent in his utterance and hence a change in the discourse content from asking about the nature of the care-giving to providing assurance that all will be well via the expression, \textit{Ɛbɛyɛ yie.} ‘It’ll be well.’

In AMD’s final utterance she resorts to polyvocality via the aphorism, \textit{Aaa anomaa bi ne ne su; ɔse: “Aaa yɛrehwɛ”} ‘Well, it is a certain bird that cries/sings: We’ll see,’ where she attributes her doubt about a change in the nature of care to the ‘cry’ (song) of a bird that says, ‘We’ll see.’ This aphorism has public knowledge of wide accessibility given that it is known by all competent adult native speakers of Akan. Basically, AMD appears to be saying something like, ‘I will believe it when I see it.’

Next, we examine another excerpt with a view to also identifying the linguistic and discourse-pragmatic strategies used in delivering and managing bad news about a diseased individual with mental/emotion problems, and the communicative strategies used to deny and seek liberty.

\subsection{Excerpt 2}\label{sec:obeng:4.2}
Context: Breaking the bad news involving a socio-emotional and/or mental challenge. Recorded September 19, 2015.

\ea T1 KO:\\\label{ex:obeng:30}
    \gll    AC, ɛte sɛn?\\
            AC (it)-be how \\
\ex T2 AC:\\\label{ex:obeng:31}
    \gll    Hm (.) Deɛ aka ara ni oo. \\
            ɦm (.) What left \textsc{emp} this \textsc{voc}.\\
\ex T3 KO:\\\label{ex:obeng:32}
    \gll    Aberanteɛ no ho te sɛn? \\
            gentleman the self be how\\
            (4.0)\\
\ex T4 AC:\\\label{ex:obeng:33}
    \gll    Seesei deɛ ɛnyɛ koraakoraakoraakoraa. \\
            Now as.for (it).\textsc{neg}.good whatsoever-whatsoever-whatsoever-whatsoever\\
\ex
    T5 AS:\\\label{ex:obeng:34}
    \ea
    \gll    Wɔfa nipa bi yɛ nipa bɔne papapapapapa. \\
            Uncle people some be people bad very.very.very \\
    \ex
    \gll    Aberanteɛ fɛfɛfɛfɛfɛfɛ hwɛ deɛ wɔn ayɛ no! \\
            Gentleman nice.nice.nice.nice look what they have.made him \\
            (2.0)\\
    \z
\ex T6 KO:\\\label{ex:obeng:35}
    \gll    Seesei ɔwɔ he? \\
            now he.be where \\
            (3.0)\\
\ex T7 AC:\\\label{ex:obeng:36}
    \gll    ɛba mu saa a, ɔnkasa obiara ho; yɛama no akɔhyɛ dam mu \\
            (It).come in.(it.happens) that if he.neg.speak anyone self we.have.made him go.be room in \\
\ex T8 KO:\\\label{ex:obeng:37}
    \gll    Moma me nkyea no. \\
            You.\textsc{pl}.allow me greet him \\
\ex T9 AS:\\\label{ex:obeng:38}
    \gll    Wɔfa ɛnha wo ho na ɔremmua wo. \\
            Uncle not.worry your self for he’ll.\textsc{neg}.respond you \\
\sn[]{{\ob}KO enters room{\cb}}

\ex T10 KO:\\\label{ex:obeng:39}
    \gll    YG; ɛte sɛn? \\
            YG (it).be how \\

\sn[]{{\ob}KO returns from room{\cb}}

\ex T11 AS:\\\label{ex:obeng:40}
    \gll    Obuaa wo?\\
            He.responded you?\\
            (1.0)\\
\ex T12 KO:\\\label{ex:obeng:41}
    \gll    Daabi. \\
            no \\
            (3.0)\\
\ex
    T13 KO:\\\label{ex:obeng:42}
    \gll    Mode no akɔ hɔspitl anaa? \\
            You.\textsc{pl}.take him have.gone hospital Q \\
\ex
    T14 AC:\\\label{ex:obeng:43}
    \gll    Baabi ara nni hɔ a yɛmfa no nkɔeɛ. \\
            Some.where every \textsc{neg}.be there that we.\textsc{neg}.take him neg-go \\
\ex
    T15 KO:\\\label{ex:obeng:44}
    \gll    Wɔn se adeɛ bɛn pɔtee na ɛreha no? \\
            they say thing what specifically \textsc{emp} (it).\textsc{prog}.trouble him \\
\ex
    T16 AC:\\\label{ex:obeng:45}
    \gll    Wɔn se adeɛ no akɔ hyɛ amee no mu pɛɛ. Kyerɛ sɛ asɛm aba ho. \\
            They say thing the has.gone be.in brain the in exactly meaning that problem has-come self \\
\ex
    T17 KO:\\\label{ex:obeng:46}
    \gll    Enti seesei yɛreyɛ no dɛn? \\
            So now we.\textsc{prog}.do it what \\
\ex T18 AS:\label{ex:obeng:47}
    \ea 
    \gll    Wɔfa wode no bɛfa baabi a. \\
            Uncle you.take him will.go somewhere if \\
    \ex
    \gll    ɔbarima sei, sɛ ɔsɔ dadeɛ mu  a, nso yadeɛ yi nti ɔte faako\\
            man this if he.holds machete in if but illness this because.of he.stays one.place \\
    \z
\ex T19 AC:\label{ex:obeng:48}
    \ea
    \gll    Baabi bɛn bio? Adeɛ no deɛ wo ara wonim sɛ. \\
            Some-place where else thing that as.for you \textsc{emp} you.know that \\
    \ex
    \gll    Asɛe awie; na ɛhe bio?\\
            (it).has.spoilt finish and.so where else \\
    \z
\ex T20 AS:\label{ex:obeng:49}
    \ea
    \gll    Wɔfa ayɛ den ama yɛn yie! Sɛ ne yadeɛ no ba a, \\
            Uncle it.become difficult for us very If his illness the come if \\
    \ex
    \gll    na yɛn nyinaa yɛn ho  hyehye yɛn; obiara repɛ baabi akɔtɛ.\\
            then us all our selves burn.burn us everyone \textsc{prog}.want somewhere go.hide\\
    \z
\ex\label{ex:obeng:50}
    T21 KO:
    \ea
    \gll    ɛha deɛ baabi ara nni hɔ a mode no bɛkɔ \\
            Here as.for somewhere any (n.where) \textsc{neg}.be there that you.take him will.go \\
    \ex
    \gll    akɔ gya na moahome kakra\\
            go leave so.that you.rest little \\
    \z
\ex\label{ex:obeng:51}
    T22 AC:\\
    \gll    Daabi oo. Mmerɛ bi wɔ hɔ mpo a, bosome koraa na mennaeɛ.\\
            No \textsc{voc/interj} Time some be there even if month even \textsc{emp} I.\textsc{neg}.sleep \\
\ex\label{ex:obeng:52}
    T23 KO:\\
    \gll    Hm. Oh diɛ! dis iz nɔt guud!\\
            ɦm oh dear this is not good\\
\ex\label{ex:obeng:53}
    T24 AS:
    \ea
    \gll    Den deɛ, ayɛ den, nso ɔboafoɔ biara nni ha oo.\\
            Hard as.for (it).has.been hard but helper any \textsc{neg}.be here \textsc{voc}.\\
    \ex
    \gll    Woba sei a, na anidasoɔ aba. \\
            You-come like.this if then hope has.come\\
    \z
\ex\label{ex:obeng:54}
    T25 KO:\\
    \gll    Mɛhwɛ deɛ mɛtuni ayɛ.\\
 	        I’ll.look what I.can do\\
\z

\subsection{Excerpt 2 (Translation)}\largerpage

\begin{exe}
    \exr{ex:obeng:30} T1 KO: AC, How are you?
    \exr{ex:obeng:31} T2 AC: Well (.) This is what is left (Just fine).
    \exr{ex:obeng:32} T3 KO: The Gentleman, How is he? (4.0)
    \exr{ex:obeng:33} T4 AC: As for now, it’s EXTREMELY bad!
    \exr{ex:obeng:34} T5 AS: Uncle, some people are EXTREMELY wicked/bad! Such a HANDSOME gentleman; look at what they’ve done to him (Look at how they’ve bewitched him!)
    \exr{ex:obeng:35} T6 KO: Where is he now? (3.0)
    \exr{ex:obeng:36} T7 AC: When he has an attack, he talks to no one; we make him go into a room (we confine him to a room)
    \exr{ex:obeng:37} T8 KO: Can I greet him?
    \exr{ex:obeng:38} T9 AS: Uncle, don’t waste your time; he’ll ignore you (he’ll not respond to your greeting)
\end{exe}
\noindent [KO enters room]
\begin{exe}
    \exr{ex:obeng:39} T10 KO: YG; How are you?
\end{exe}
\noindent [KO returns from room]
\begin{exe}
    \exr{ex:obeng:40} T11 AS: Did he respond? (1.0)
    \exr{ex:obeng:41} T12 KO: No (3.0) 
    \exr{ex:obeng:42} T13 KO: Have you taken him to the hospital?
    \exr{ex:obeng:43} T14 AC: There isn’t a place where we’ve not taken him. (We’ve taken him everywhere!)
    \exr{ex:obeng:44} T15 KO: What exactly do they say is wrong with him?
    \exr{ex:obeng:45} T16 AC: They said that thing (the disease) has gone into his brain; meaning there is trouble.
    \exr{ex:obeng:46} T17 KO: So, what are we doing about it?
    \exr{ex:obeng:47} T18 AS: Uncle, if you can send him to a place, you should. Such a strong man, if he holds a cutlass (he is well-built for farming); yet, because of the disease, he is stuck at this place.
    \exr{ex:obeng:48} T19 AC: Where else? You know very well that the die is cast (i.e., the disease has already finished him); where else are you talking about? (Where else should we take him?)
    \exr{ex:obeng:49} T20 AS: Uncle, it’s become so difficult for us! When he has an attack, we all feel uncomfortable and look for a place to hide.
    \exr{ex:obeng:50} T21 KO: As for this place (town) there is no place where you can leave him so that you can have a little rest?
    \exr{ex:obeng:51} T22 AC: Not really; never! There are times that for a whole month I’d have no sleep at all!
    \exr{ex:obeng:52} T23 KO: Well, Oh Dear! This is not good!
    \exr{ex:obeng:53} T24 AS: As for difficult, it is! (It is really difficult); but there are no helpers here, you know. Once you’ve come (here), there is hope.
    \exr{ex:obeng:54} T25 KO: I’ll see what I can do.
\end{exe}

A systematic observation of above excerpt shows similarities with those of Excerpt 1 in terms of the features that are used to deliver bad news and to seek and protect diseased individuals’ liberty. Thus, as in Excerpt 1, as in this excerpt (Excerpt 2), the linguistic features used for managing the bad news include pausal phenomena, reduplication, factive expressions and Akan-English code-switching. We begin by with pausal phenomena. 

\subsection{Pausal phenomena}\label{sec:obeng:4.3}

In T2, AC signals the bad news first by using the voiced pause, \textit{ɦm}, a hesitating pause that is used as a hedge and a signal to the upcoming bad news. This voiced pause is then followed by a short silent pause notated as (.). The long pauses of between 2.0 and 4.0 seconds that occur between the turns are used to signal or project upcoming bad news. The longer the pause the more difficult it is for the bad news bearer to deliver it. For example, when asked how YG (the diseased individual) was doing (T3), AC in T4 paused for 4.0 seconds signaling that YG was not doing well or that his condition of health had not improved. When KO repeated his question about the whereabouts of YG in T6, AC paused for 3.0 seconds again to signal the bad news about YG’s health. What is important in this extract is how the long pause was followed by an evasive answer. Note that there are several places such as T11 and T12 where a long pause of 3.0 second duration, an initiative time latency pause, is used because AC did not assume the turn occupancy and KO had to continue as a result of the extent of the bad news regarding YG’s health and his behavior of not being communicative when suffering an (emotional/panic) attack from the disease.

\subsection{Reduplication}\label{sec:obeng:4.4}

Throughout the discourse reduplication is used by the studied participants to show the intensity or extent of badness of the news or the extent to which a diseased individual’s physical or positive attribute has been destroyed by the negative effects of the disease. In T4, AC says: \textit{Seesei deɛ ɛnyɛ koraakoraakoraakoraa} ‘As for now, it is extremely bad.’ Repeating the word \textit{koraa} four times shows an extreme level of badness of YG’s health status. Also, in T5, AS employs reduplication to show the extent of badness or evil nature of some people (witches or people with the spiritual capability to cause others to be sick). In the utterance, she notes: \textit{Wɔfa nipa bi yɛ nipa bɔne papapapapapa.} ‘Uncle, some people are ˈ\textsc{extremely} bad.’ I have used uppercase letters and bold together with a stress marker on the word extremely to show emphasis and extent of badness as expressed by AS. The intensifier, \textit{pa} ‘very,’ is repeated six times so we have the following structure: [\textsc{det}\,+\,Noun\,+\,Copula\,+\,\textsc{int}\,+\,\textsc{int}\,+\,\textsc{int}\,+\,\textsc{int}\,+\,\textsc{int}\,+\,\textsc{int} + \textsc{adj}] i.e., some\,+\,people\,+\,are\,+\,very\,+\,very\,+\,very\,+\,very\,+\,very\,+\,very\,+\,bad. Repeating the intensifier six times suggests excessive badness. Thus, by engaging in reduplication we are led into AS’ world view about disease causation (in this case, witches) and the extent of badness or evil nature of such persons. In her following utterance, AS engages in another form of reduplication by saying: \textit{Aberanteɛ fɛfɛfɛfɛfɛfɛ hwɛ deɛ wɔn ayɛ no!} ‘Such a ˈ\textsc{handsome} gentleman! Look at what they’ve done to him!’ Here, AS uses reduplication to describe the extent of handsomeness of YG, and by implication to suggest the debilitating effect of the disease on him (YG). Like extremely, handsome in the above sentence has been bolded, capitalized and marked with a stress diacritic to show the extent of handsomeness. Note that \textit{fɛ} means handsome or beautiful so repeating it six times shows extreme beauty or handsomeness. Thus, AS appears to be saying that if a gentleman as handsome as YG could be turned into such a monster such that his care-givers have to either confine him to a room or run away from him, then one sees the extent to which disease can intrude upon the liberty of both the diseased individuals and their care-givers.

Finally, in T20 AS notes, \textit{yɛn nyinaa yɛn ho hyehye yɛn} ‘literally, we all, our bodies burn-burn us’ meaning ‘we all feel uneasy’ to project the bad news about the difficulty in caring for YG. The verb \textit{hyehye} is a physical verb that denotes physical sensation and unveils both the physical and emotional burden of disease on all stakeholders in the care-giving.

\subsection{Factive expressions}\label{sec:obeng:4.5}

Factive expressions such as \textit{wo ara wonim} ‘you very well know’ are often used to show credence or provide evidence about upcoming bad news, news about care, or a point about one’s physical or emotional condition. In T19, AC notes: \textit{Baabi bɛn bio? Adeɛ no deɛ wo ara wonim sɛ asɛe awie; na ɛhe bio} ‘Where else? As for that thing (the disease), you very well know that it has already destroyed him).’ AC uses the factive formula, ‘you very well know’ for evidentiality. She appears to be saying something like: if it is common knowledge that the disease has already destroyed YG, then there is no need to seek a cure or medical attention anywhere else; after all the die is cast! 

\subsection{Akan--English code-switching}\label{sec:obeng:4.6}

As noted earlier, code-switching may be employed to deliver bad news given that it is easier to deliver bad news in a foreign language than in one’s own language. In Excerpt 2 above, we observe the bad news, the FTA, and/or the communicative difficulty being delivered in English. Thus, KO switches from Akan in T21 to English in T23 by responding, \textit{Hm. Oh diɛ! dis iz nɔt guud!} ‘Hm, oh  dear; this is not good’ to acknowledge receipt of the bad news. By acknowledging the news was bad via English, the burden of speaking the unspeakable is lessened \citep{Movahedi1996}. 

From the discourse-pragmatic point of view, bad news is managed through avoidance whereby the diseased individual’s name is not mentioned, and he is referred to by a non-specific name such as \textit{Aberanteɛ no} (T3) ‘the Gentleman.’ The caregivers may also resort to evasion as observed in T6 and T7 where instead of answering KO’s question as to the whereabouts of YG (the diseased individual), AC rather talked about what happened if YG (the diseased individual) was having a bad day.

On denial of liberty, we observed that the interactants used utterances related to forced imprisonment whereby the care-givers constrained the diseased individuals into rooms and denied them the opportunity to come out as shown in T7 where AC responded to KO’s question about the whereabouts of YG (the diseased individual) with the statement, \textit{ɛba mu saa a, ɔnkasa obiara ho na yɛama no akɔhyɛ dam mu} ‘When he has an attack, he speaks to no one and we force him into a room.’ By the action of the caregivers, we are led into the experience of the diseased individual and thus shown how a disease intrudes upon the positive liberty of diseased individuals (in this case, YG) by preventing them from participating in their family and communal lives.

Also, denial of liberty was also seen in situations where caregivers were forced to flee from a diseased individual and hide for their own safety as expressed in the utterances in T20 where AS, a caregiver notes; \textit{Sɛ ne yadeɛ no ba a, na yɛn nyinaa yɛn ho hyehye yɛn; obiara repɛ baabi akɔtɛ} ‘If he has an attack, then we all feel trapped and we seek a hiding place (to be away from him.)’

\section{Summary and conclusion}\label{sec:obeng:5}

From the cited texts and discussion, we observed that the linguistic strategies used to break and manage bad news include such phonetic features as silent and voiced pauses of various lengths, and reduplication or repetition which is employed to show frustration, anger, and emotional valence. Lexico-syntactic features used in bad news delivery and management included adjectives of quality used both attributively and predicatively, intensifiers (often repeated), verbs denoting physical sensation, and factive formulae (for evidentiality and credence). With respect to discourse-pragmatic features used to break and manage bad news, we identified vague reference forms, avoidance (which was done via giving up on words, or pronoun mismatch where a pronoun such as \textit{you} was used to index another pronoun such as she in order to avoid direct reference. Others included the speech acts of complaining, blaming (which was done through the use of distal deictics and/or innuendo), requesting, blame-shifting (that is, blaming others including witches for causing the disease or pain), use of quantifiers, and code-switching from Akan to English given the fact that face-threatening acts (FTAs) were perceived as being more tellable in a less familiar language or register \citep{Movahedi1996, Obeng2008}. Other discourse-pragmatic strategies used to break and manage bad news included polyvocality and inferencing through the use of such a non-specific pronoun as \textit{biribi} ‘something.’

Article 25 of the 1948 Universal Declaration of Human Rights mentions health as part of the right to an adequate standard of living. The United Nations General Assembly’s (1966) International Covenant on Economic, Social and Cultural Rights also recognizes the right to health as a human right \citep{United-Nations-General-Assembly1948}. Given the above declarations and in view of the data examined for this study, it is true to argue that disease encroaches on both the negative and positive liberty of diseased individuals and their caregivers. Thus, disease exposes diseased individuals’ and sometimes their caregivers’ freedoms to encroachment and takes away the right of diseased individuals (and sometimes their caregivers) to participate and share in their personal and communal activities. 

Furthermore, we learn from this study that managing bad news requires extreme care in determining what questions to ask and how to ask them, how to assign blame, how to assert and assure, among others, given that someone’s life could be on the line. Given the socio-emotional, financial and cultural burdens that disease puts on diseased individuals and their care-givers, it is recommended that an opportunity be made available for such individuals to interact with others in view of the fact that such an interactional opportunity could be therapeutic and serve as a communicative means to convey and manage speakers’ socio-emotional and physical states \citep{KalmykovaMergenthaler1998}.

It is argued further that bad news management has relevance for language and liberty. Specifically, through bad news management, liberty informs language and through language, liberty becomes a reality. Indeed, viewed from the point of view of liberty \citep{Berlin1960} we have demonstrated that via their complaints, requests, and other speech acts, the studied diseased individuals and their care-givers sought both negative and positive liberty. On the one hand, the diseased individuals requested the right to be free from intrusions from their diseases and from their care-givers’ actions such as forced confinement (negative liberty) and also sought the right to participate in their private, family and professional (business) lives. On the other hand, the care-givers also sought the right to be free from attacks by diseased individuals and a reprieve or lessening of their care-giving burden as seen in AS’ utterance in T24 in Excerpt 2 where she said, \textit{Woba sei a, na anidasoɔ aba} ‘When you visit like this, then there is hope (of help or reprieve).’ What is unique about AS’ utterance is that it is an implicit request for help and KO’s response, \textit{Mɛhwɛ deɛ mɛtuni ayɛ} ‘I’ll see what I can do,’ in that communicative context, was a promise to help. 

What is also unique about this study is that even though Akan language ideology assumes that diseased individuals and people who are generally in need of help are not as communicatively powerful as their care-givers, in these recorded discourses, both the diseased individuals and their care-givers sometimes ignored the power asymmetry and sought both their positive and negative liberties. It is possible that both sides saw each other as being in it together and therefore ignored the power relations. Most importantly, we have learned from this study that the interdependence nature of the Akan society and the socio-cultural requirements placed on members of the community to assist each other in times of need places members at the same camera angle and that trumps the societal power asymmetry. 

From the point of view of the larger Ghanaian society, it may be argued that to protect diseased individuals’ negative liberty and positive liberty, there is the need to put in place rights that prevent the effects of ‘disease’ and people from encroaching on the freedoms of diseased individuals as well as rights that empower them to participate in their family and communal lives. Also, medical personnel (especially, doctors and nurses), social workers, end of life care-givers, and family members caring for their sick relatives as well as scholars working in the health area must understand the discursive mores surrounding bad news delivery and bad news management in order to be educated about the intertwining nature of language and care-giving and to guarantee the liberty of all stakeholders in the care-giving ecology.

\section*{Abbreviations}
\begin{multicols}{3}
\begin{tabbing}
\textsc{interj}\hspace{1ex} \= Interjection \kill
    \textsc{conj} \> Conjunction \\
    \textsc{emp} \> Empathy \\
    \textsc{interj} \> Interjection \\
    \textsc{neg} \> Negation \\
    \textsc{pl} \> Plural \\
    \textsc{prog} \> Progressive \\
    \textsc{voc} \> Vocative
\end{tabbing}
\end{multicols}

{\sloppy\printbibliography[heading=subbibliography,notkeyword=this]}

\end{document}
