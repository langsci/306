\documentclass[output=paper,colorlinks,citecolor=brown]{langscibook}

\title{Counting mass nouns in Guébie}

\author{Hannah Sande\affiliation{Georgetown University} and Virginia Dawson\affiliation{University of California, Berkeley}}

\abstract{This paper contributes to the growing body of work on countability properties of nouns across languages by investigating the three-way countability distinction in Guébie, an Eastern Kru language spoken in Southwest Côte d'Ivoire. Guébie distinguishes three core categories of noun, which we call \textit{true mass, count}, and \textit{countable mass nouns}, and possesses a singulative suffix which converts countable mass nouns into count nouns. We use a mereological model to capture this three-way distinction, and the effects of the singulative suffix.}

\IfFileExists{../localcommands.tex}{
  \addbibresource{localbibliography.bib}
  \usepackage{langsci-optional,langsci-branding}
\usepackage{langsci-gb4e}
% \usepackage{langsci-textipa}
% \usepackage{langsci-glyphs}
\usepackage[linguistics]{forest}
\usepackage{tabto}
\usepackage{multirow}
\usepackage{bbding}

\usepackage[normalem]{ulem}

\usepackage{tikz-qtree}

\usepackage{enumitem}

\usepackage{multicol}
\usepackage{stmaryrd} %double brackets

\makeatletter
\let\pgfmathModX=\pgfmathMod@
\usepackage{pgfplots,pgfplotstable}%
\let\pgfmathMod@=\pgfmathModX
\makeatother
\usepgfplotslibrary{colorbrewer}
\usetikzlibrary{fit}

\usepackage{jambox}
\usepackage{tikz-qtree-compat}
\usetikzlibrary{arrows, arrows.meta}
\usepackage{longtable}
\usepackage{subcaption}

  \makeatletter
\let\thetitle\@title
\let\theauthor\@author
\makeatother

\newcommand{\togglepaper}[1][0]{
%   \bibliography{../localbibliography}
  \papernote{\scriptsize\normalfont
    \theauthor.
    \thetitle.
    To appear in:
    Change Volume Editor \& in localcommands.tex
    Change volume title in localcommands.tex
    Berlin: Language Science Press. [preliminary page numbering]
  }
  \pagenumbering{roman}
  \setcounter{chapter}{#1}
  \addtocounter{chapter}{-1}
}

\newcommand{\bari}{\ipabar{\i}{.5ex}{1.1}{}{}}
\newcommand{\notipa}[1]{\textnormal{#1}}

\newcommand{\agre}{\textsc{agr}-\ol{eene}}

\renewcommand{\emph}[1]{\textit{#1}} % resetting a setting from ling-macros-modified (I think?)

% forest settings to make compact but (mostly) straight-spined trees:
\forestset{
fairly nice empty nodes/.style={
            delay={where content={}{shape=coordinate,for parent={
                  for children={anchor=north}}}{}}
, angled/.style={content/.expanded={$<$\forestov{content}$>$}}
}}

\forestset{sn edges/.style={for tree={parent anchor=south, child anchor=north}}}

\newcommand{\bex}{\begin{exe}}
\newcommand{\fex}{\end{exe}}

\newcommand{\bxl}{\begin{exe}}
\newcommand{\fxl}{\end{exe}}

\newcommand{\ix}[1]{\textsubscript{#1}}
\newcommand{\alert}[1]{\textbf{#1}}
\newcommand{\ol}[1]{\textit{#1}}


			\usetikzlibrary{shapes,arrows,positioning,decorations,decorations.pathmorphing,intersections}
\forestset{
nice empty nodes/.style={
    for tree={calign=fixed edge angles},
    delay={where content={}{shape=coordinate,for siblings={anchor=north}}{}}
},
}

\definecolor{dark-gray}{gray}{0.3}

%\usepackage{dingbat,pifont}


%%%%%%%%%%%%For arrows%%%%%%%%%%%%%

\newcommand\Tikzmark[2]{%
  \tikz[remember picture]\node[inner sep=0pt,outer sep=0pt] (#1) {#2};%
}
\NewDocumentCommand\DrawArrow{O{}mmmmO{3}}{
\tikz[remember picture,overlay]
  \draw[->,line width=0.8pt,shorten >= 2pt,shorten <= 2pt,#1]
    (#2) -- ++(0,-#6\ht\strutbox) coordinate (aux) -- node[#4] {#5} (#3|-aux) -- (#3);
}
\NewDocumentCommand\DrawDotted{O{}mmmmO{3}}{
\tikz[remember picture,overlay]
  \draw[->,line width=0.9pt,dotted,shorten >= 2pt,shorten <= 2pt,#1]
    (#2) -- ++(0,-#6\ht\strutbox) coordinate (aux) -- node[#4] {#5} (#3|-aux) -- (#3);
}
\NewDocumentCommand\DrawLine{O{}mmmmO{3}}{
\tikz[remember picture,overlay]
  \draw[line width=0.8pt,shorten >= 2pt,shorten <= 2pt,#1]
    (#2) -- ++(0,-#6\ht\strutbox) coordinate (aux) -- node[#4] {#5} (#3|-aux) -- (#3);
}
%%%%%%%%%%%%%%%%%%%%%%%%%%%%%%%%%%%%%


\newcommand{\baru}{ʉ}
\newcommand{\baruH}{\'\baru}
\newcommand{\baruL}{\`\baru}

\newcommand{\ep}{ε}
\newcommand{\epH}{\'\ep}
\newcommand{\epL}{\`\ep}

\newcommand{\schwa}{ə}
\newcommand{\schwaH}{\'ə}
\newcommand{\schwaL}{\`ə}

\newcommand{\oo}{ɔ}
\newcommand{\ooH}{\'\oo}
\newcommand{\ooL}{\`\oo}

\newcommand{\ds}{\textsuperscript{
	\hspace*{-2pt}\begin{tikzpicture}
		\draw[-{>[scale=0.5]}] (0,0.4) --(0,0.25);
	\end{tikzpicture}}}

\newcommand{\ch}{t͡ʃ}
\newcommand{\dz}{d͡ʒ}

\newcommand{\tgl}{ʔ}

%shortcuts for the complementizers
\newcommand{\mbuL}{mb\baruL}
\newcommand{\mbuHL}{mb\baruH\baruL}
\newcommand{\mbuLH}{mb\baruL\baruH}
\newcommand{\la}{lá}
\newcommand{\nda}{ndà}

\newcommand{\tsc}[1]{\textsc{#1}}
\renewcommand{\textscb}{ʙ}
\newcommand{\ipa}[1]{#1} %disable IPA

\newcommand{\SM}[1]{#1}

\DeclareNewSectionCommand
  [
    counterwithin = chapter,
    afterskip = 2.3ex plus .2ex,
    beforeskip = -3.5ex plus -1ex minus -.2ex,
    indent = 0pt,
    font = \usekomafont{section},
    level = 1,
    tocindent = 1.5em,
    toclevel = 1,
    tocnumwidth = 2.3em,
    tocstyle = section,
    style = section
  ]
  {appendixsection}

\renewcommand*\theappendixsection{\Alph{appendixsection}}
\renewcommand*{\appendixsectionformat}
              {\appendixname~\theappendixsection\autodot\enskip}
\renewcommand*{\appendixsectionmarkformat}
              {\appendixname~\theappendixsection\autodot\enskip}

\renewcommand{\lsChapterFooterSize}{\footnotesize}
 
  %% hyphenation points for line breaks
%% Normally, automatic hyphenation in LaTeX is very good
%% If a word is mis-hyphenated, add it to this file
%%
%% add information to TeX file before \begin{document} with:
%% %% hyphenation points for line breaks
%% Normally, automatic hyphenation in LaTeX is very good
%% If a word is mis-hyphenated, add it to this file
%%
%% add information to TeX file before \begin{document} with:
%% %% hyphenation points for line breaks
%% Normally, automatic hyphenation in LaTeX is very good
%% If a word is mis-hyphenated, add it to this file
%%
%% add information to TeX file before \begin{document} with:
%% \include{localhyphenation}
\hyphenation{
affri-ca-te
affri-ca-tes 
Līk-pāk-páln
pro-sod-ic
phe-nom-e-non
Chi-che-wa
Lu-bu-ku-su
Ngbu-gu
Boyel-dieu
Mat-chi
pho-neme
Mil-em-be
Nyan-chera
Mc-Pher-son
Tsoo-tso
Sku-pin
dis-tin-guishes
con-ser-va-tion
Me-dum-ba
}

\hyphenation{
affri-ca-te
affri-ca-tes 
Līk-pāk-páln
pro-sod-ic
phe-nom-e-non
Chi-che-wa
Lu-bu-ku-su
Ngbu-gu
Boyel-dieu
Mat-chi
pho-neme
Mil-em-be
Nyan-chera
Mc-Pher-son
Tsoo-tso
Sku-pin
dis-tin-guishes
con-ser-va-tion
Me-dum-ba
}

\hyphenation{
affri-ca-te
affri-ca-tes 
Līk-pāk-páln
pro-sod-ic
phe-nom-e-non
Chi-che-wa
Lu-bu-ku-su
Ngbu-gu
Boyel-dieu
Mat-chi
pho-neme
Mil-em-be
Nyan-chera
Mc-Pher-son
Tsoo-tso
Sku-pin
dis-tin-guishes
con-ser-va-tion
Me-dum-ba
}
 
  \togglepaper[1]%%chapternumber
}{}

\begin{document}
\maketitle 

\section{Introduction}\label{sec:sande:1}

This paper investigates the countability properties of nouns in Guébie, an Eastern Kru language spoken in Southwest Côte d'Ivoire. Guébie distinguishes three core categories of noun, based on number marking. We adopt a mereological model based on properties of cumulativity and divisibility to account for the behavior of these nouns. Additionally, we situate Guébie's system in the emerging typology of countability distinctions cross-linguistically.

Guébie is an endangered Kru language spoken by no more than 7,000 speakers in Côte d'Ivoire. There is one known monolingual speaker, while other speakers are bilingual in Guébie and French, and often other neighboring Kru languages. The data presented here was collected over the past five years in Sande's work with the Guébie community \citep{Sande2017}. The specific forms in this paper have each been confirmed by at least two male speakers, ages $\sim$30 and $\sim$40.

In \sectref{sec:sande:2} we present the morphological number marking and syntactic distribution facts for the three categories of nouns in Guébie. \sectref{sec:sande:3} lays out a semantic analysis of the three degrees of countability in Guébie, based in a mereological approach. \sectref{sec:sande:4} briefly situates Guébie within the growing typology of number marking, and \sectref{sec:sande:5} concludes.

\section{Guébie number marking}\label{sec:sande:2}

In this section we show that Guébie distinguishes three noun categories based on number marking:

\begin{enumerate}
	\item Count nouns
	\item True mass nouns
	\item ``Countable'' mass nouns
\end{enumerate}

The diagnostics for these three categories are based primarily on their compatibility with Guébie's number morphology: the plural marker (/-a/ or /-i/), and the singulative marker (/-je/ or /-\ipa{{\ds}bə}/). The two plural markers and two singulative markers are allomorphs and do not differ in meaning \citep{Sande2017}.\footnote{The two singulative markers do not seem to differ in meaning, and there are phonological traits which explain their distribution. However, one speaker expresses an intuition that nouns that take /-je/ are often small, while nouns that take /-\ipa{{\ds}bə}/ are often large and/or round. However, this intuition does not hold up across the collected data. More work will be done in the future to explore this area. If a difference in size is found to be conveyed, a classifier-like analysis of the singular markers might be more appropriate than the one presented here; though see \sectref{sec:sande:4} on how classifiers are semantically similar to the singular marker in Guébie.}

\subsection{Count nouns}\label{sec:sande:2.1}

Count nouns in Guébie have a singular individual interpretation in their bare form. These include words for humans, large animals, and items that typically do not come in groups, i.e. [\ipa{ŋ}ʷ\ipa{ɔnɔ}$^{4.4}$] \textit{woman}, [\ipa{{\ds}bə}$^{31}$] \textit{plate}, [\ipa{mεɔ}$^{3.1}$] \textit{tongue}.\footnote{Guébie has four distinct tone heights, marked with numbers 1--4, where 4 is high.} Bare count nouns cannot have a plural or substance interpretation. This is shown in (1), where the bare form of [\ipa{{\ds}bə}$^{31}$] \textit{plate} cannot be predicated on a plural subject.

\ea[*]{%1
    \label{ex:sande:1}
	\gll    \ipa{liene}$^{3.3.1}$ \ipa{εja}$^{2.3}$ \ipa{lieko}$^{3.3.1}$ \ipa{{\ds}bə}$^{31}$ \ipa{mɔ}$^{1}$\\
	        \textsc{dem.pro.prox} with \textsc{dem.pro.dist} plate be.\textsc{emph}\\
	\glt    Intended: `This thing and that thing are plate(s).'}
\z

These nouns combine directly with the plural suffix (/-a/ or /-i/) to yield a plural reading. Example \REF{ex:sande:2}, in contrast to \REF{ex:sande:1}, shows that morphologically plural-marked count nouns can be predicated of plural subjects.

\ea%2
    \label{ex:sande:2}
    \gll    \ipa{liene}$^{3.3.1}$ \ipa{εja}$^{2.3}$ \ipa{lieko}$^{3.3.1}$ \ipa{{\ds}bə-i}$^{3.12}$ \ipa{mɔ}$^{1}$\\
	        \textsc{dem.pro.prox} with \textsc{dem.pro.dist} plate\textsc{-pl} be.\textsc{emph}\\
	\glt    `This thing and that thing are plates.'
\z

\tabref{tab:sande:1} shows a selection of count nouns in their bare form, and with the PL suffix.\footnote{Both plural suffixes in Guébie are associated with a level tone 2. When attached to a root, if the root is associated with more underlying tone heights than syllables (e.g. two tone levels on a monosyllabic word as in example a in \tabref{tab:sande:1}), then we see one-to-one association of syllables to tone heights beginning at the left, and any leftover tone heights form a contour together with the plural level 2 at the right edge.}

\begin{table}
	\begin{tabular}{lllll}
	\lsptoprule
		& \textbf{Singular} & \textbf{Plural} & \textbf{Translation}\\
		& Root & Root-PL & \\
		\midrule
		a. & \ipa{{\ds}bə}$^{31}$ & \ipa{{\ds}bə-i}$^{3.12}$ & `plate'\\
		b. & \ipa{cu}$^{3}$ & \ipa{cu-i}$^{3.2}$ & 	`month'\\
		c. & \ipa{sabala}$^{3.3.3}$ & \ipa{sabala-i}$^{3.3.3.2}$ & `shoe'\\
		d. & \ipa{ɟak}ʷ\ipa{εlε}$^{2.3.1}$ & \ipa{ɟak}ʷ\ipa{εlε-ɪ}$^{2.3.1.2}$ & `tarantula'\\
		e. & \ipa{mεɔ}$^{3.1}$ & \ipa{mεɔ-ɪ}$^{3.1.2}$ & `tongue'\\
		f. & \ipa{goji}$^{3.1}$ & \ipa{goji-a}$^{3.1.2}$ & 	`dog'\\
		g. & \ipa{du}$^{2}$ & \ipa{du-a}$^{2.2}$ & `city'\\
		%	g. & \ipa{ɲεɲɪ^{3.2}} & \ipa{ɲεɲɪ-a^{3.2.2}} & `sin'\\
	\lspbottomrule
	\end{tabular}
    \caption{Count nouns in Guébie}
    \label{tab:sande:1}
\end{table}

Count nouns cannot combine with the singulative suffix, as shown in \REF{ex:sande:3}.

\ea%3
    \label{ex:sande:3}
    \textbf{*noun-SG}
    \ea[*]{%3a
    \label{ex:sande:3a}
    \gll    \ipa{mεɔ}$^{3.1}$-\ipa{{\ds}bə/je}$^{1}$\\
	        tongue-\textsc{sg}\\
	\glt    Intended: `A tongue'}
    \ex[*]{%3b
    \label{ex:sande:3b}
    \gll    \ipa{{\ds}bə}$^{31}$-\ipa{{\ds}bə/je}$^{1}$\\
	        plate-\textsc{sg}\\
	\glt    Intended: `A plate'}
    \z
\z

Only the plural form of a count noun can combine with a numeral greater than one, as shown in \REF{ex:sande:4}.

\ea%4
    \label{ex:sande:4}
    \textbf{Numerals only combine with plural-marked count nouns}
    \ea[]{%4a
    \label{ex:sande:4a}
    \gll    \ipa{mεɔ-ɪ}$^{3.1.2}$ ta$^{3}$\\
	        tongue-\textsc{pl} three\\
	\glt    `Three tongues'}
    \ex[*]{%4b
    \label{ex:sande:4b}
    \gll    \ipa{mεɔ}$^{3.1}$ ta$^{3}$\\
		    tongue three\\
	\glt    Intended: `Three tongues'}
    \ex[]{%4c
    \label{ex:sande:4c}
    \gll    \ipa{{\ds}bə-i}$^{3.12}$ ta$^{3}$\\
	        plate-\textsc{pl} three\\
	\glt    `Three plates'}
    \ex[*]{%4d
    \gll    \ipa{{\ds}bə}$^{31}$ ta$^{3}$\\
            plate three\\
	\glt    Intended: `Three plates'}
    \z
\z

Similarly, only the plural form of a count noun can combine with an `all' or  `many' quantifier, as shown in \REF{ex:sande:4}. The translations marked with `\#' are impossible interpretations of these utterances.

\ea%5
    \label{ex:sande:5}
    \textbf{Quantifiers only combine with plural-marked count nouns}
    \ea[]{%5a
    \label{ex:sande:5a}
    \gll    \ipa{{\ds}bə-i}$^{3.12}$ \ipa{a{\ds}ba}$^{4.2}$\\
			plate-\textsc{pl} all\\
	\glt    `all the plates', \#`all the plate'}
    \ex[]{%5b
    \label{ex:sande:5b}
    \gll    \ipa{{\ds}bə-i}$^{3.12}$ \ipa{{\ds}butugba}$^{3.1.1}$\\
			plate-\textsc{pl} much\\
	\glt    `many plates', \#`much plate'}
    \ex[*]{%5c
    \gll    \ipa{{\ds}bə}$^{31}$ \ipa{a{\ds}ba}$^{4.2}$/\ipa{{\ds}butugba}$^{3.1.1}$\\
            plate all/much\\
    \glt    Intended: `all/much plate' or `all/many plates'}
    \z
\z
%same for cow, cat, plate, man, tongue

\todo[inline]{Inconsistent capitalization of translations throughout the chapter}

In sum, count nouns in Guébie act much like count nouns in English. They have a singular interpretation in their bare form and a plural interpretation when combined with plural morphology. In the latter case, they can appear with a numeral greater than one, or with quantifiers `all, many'.

\subsection{True mass nouns}

The second class of nouns in countability terms in Guébie are the true mass nouns. These nouns refer to substances, including liquids like \textit{blood, oil}, and those consisting of very tiny particles like \textit{sand} and \textit{salt}.

True mass nouns can only surface in their bare form. Unlike count nouns, mass nouns cannot combine directly with the plural suffix. Additionally, mass nouns cannot combine with the singulative suffix, as shown in \tabref{tab:sande:2}.

\begin{table}
    \begin{tabular}[h]{llllll}
    \lsptoprule
    & \textbf{Mass} & \textbf{Plural} & \textbf{Translation}\\
    & Root & *Root-PL & *Root-SG \\
    \midrule
    %	a. & \ipa{ɲu^{4}} & *\ipa{ɲu-a}, *\ipa{ɲu-i} & `water'\\
    	a. & \ipa{dolo}$^{1.1}$ & \ipa{*dolo-a, *dolo-i} & \ipa{*dodo-je, *dodo-{\ds}bə} & `blood'\\
    	b. & \ipa{dodo}$^{3.2}$ & \ipa{*dodo-a, *dodo-i} & \ipa{*dolo-je, *dolo-{\ds}bə} & `sand'\\
    	c. & \ipa{kpə}$^{4}$  & \ipa{*kpə-a, *kpə-i} & \ipa{*kpə-je, *kpə-{\ds}bə} & `oil'\\
    	d. & \ipa{ɟuru}$^{2.2}$ & \ipa{*ɟuru-a, *ɟuru-i} & \ipa{*ɟuru-je, *ɟuru-{\ds}bə} & `salt'\\
    %	d. & \ipa{novi^{2.3}} & *\ipa{novi-a, *novi-i} & 	`bees'\\
    %	e. & \ipa{kuk^{w}e^{4.1}} & *\ipa{kuk^{w}e-a, *kuk^{w}e-i} & `ants'\\
    %	f. & \ipa{wʊlε^{3.1}} & *\ipa{wʊlε-a, *wʊlε-ɪ}  & `fingers'\\
    %	g. & \ipa{ɟe^{3}} & *\ipa{ɟe-a, *ɟe-i} & 	`stars'\\
    %	h. & \ipa{ɟa^{31}} & *\ipa{ɟa-a, ??ɟa-i} & `coconuts'\\
    %	i. & \ipa{tro{\ds}bi^{3.2}} & *\ipa{tro{\ds}bi-a, *tro{\ds}bi-i} & `eggplants'\\
    \lspbottomrule
    \end{tabular}
    \caption{True mass nouns in Guébie}
    \label{tab:sande:2}
\end{table}

%Neither can nouns in this class combine with a singulative suffix.

%\begin{table}
%\caption{True mass nouns}\label{uncountmass}
%	\begin{tabular}{llllll}
%		& \textbf{Mass} & \textbf{Singular} & \textbf{Plural} & \textbf{Translation}\\
%		\hline
%		& Root & Root-SG & Root-SG-PL & \\
%		\hline
%		a. & \ipa{dodo}$^{1.1}$ & *\ipa{dodo-je, *dodo-{\ds}bə} & *\ipa{dodo(-je)-i/a, *dodo(-{\ds}bə)-i/a} & `blood'\\
%		b. & \ipa{dolo}$^{3.2}$ & *\ipa{dolo-je, *dolo-{\ds}bə} & *\ipa{dolo(-je)-i/a, *dolo(-{\ds}bə)-i/a} & `sand'\\
%		c. & \ipa{kpə}$^{4}$ & *\ipa{kpə-je, *kpə-{\ds}bə} & *\ipa{kpə(-je)-i/a, *kpə(-{\ds}bə)-i/a} & `oil'\\
%		d. & \ipa{ɟuru}$^{2.2}$ & *\ipa{ɟuru-je, *ɟuru-{\ds}bə} & *\ipa{ɟuru(-je)-i/a, *ɟuru(-{\ds}bə)-i/a} & `salt'
%	\end{tabular}
%\end{table}

True mass nouns can never combine with numerals in Guébie, as shown in \REF{ex:sande:6}.

\ea%6
    \label{ex:sande:6}
    \textbf{Numerals cannot modify bare mass nouns}
    \ea[]{%6a
    \label{ex:sande:6a}
    \gll    \ipa{dodo}$^{3.2}$ \ipa{la}$^{2}$ \ipa{ci-ə}$^{2.2}$ \ipa{ta}$^{3}$\\
	        sand of type-\textsc{pl} three\\
	\glt    `three types of sand'}
    \ex[*]{%6b
    \label{ex:sande:6b}
    \gll    \ipa{dodo}$^{3.2}$ \ipa{ta}$^{3}$\\
	        sand three\\
	\glt    Intended:`three sands'}
    \z
\z

Unlike count nouns, which cannot combine with quantifiers `all, many' in their bare form \REF{ex:sande:5}, bare mass nouns combine with quantifiers \REF{ex:sande:7}.

\ea%7
    \label{ex:sande:7}
    \textbf{Quantifiers can modify bare mass nouns}
    \ea[]{%7a
    \label{ex:sande:7a}
    \gll    \ipa{dolo}$^{1.1}$ \ipa{a{\ds}ba}$^{4.2}$\\
			blood all\\
	\glt    `all the blood'}
    \ex[]{%7b
    \label{ex:sande:7b}
    \gll    \ipa{dolo}$^{1.1}$ \ipa{{\ds}butugba}$^{3.1.1}$\\
			blood much\\
	\glt    `a lot of blood'}
    \ex[]{%7c
    \label{ex:sande:7c}
    \gll    \ipa{dodo}$^{3.2}$ \ipa{a{\ds}ba}$^{4.2}$\\
			sand all\\
	\glt    `all the sand'}
    \z
\z

In sum, true mass nouns never appear with number-marking morphology, and they cannot be modified by numerals. Unlike count nouns, they can be modified by quantifiers in their bare form.

%GD: for descriptive purposes we should say how mass nouns are counted

\subsection{``Countable'' mass nouns}\label{sec:sande:2.3}

The third class of nouns, which we call \textit{countable} mass nouns, shows split behavior: bare countable mass nouns pattern with mass nouns, while SG-marked countable mass nouns pattern with count nouns.

The countable mass class makes up a large part of the Guébie lexicon, consisting of individuals that typically come in groups. These include insects, small animals, body parts, fruits and vegetables, grains and nuts, stars, ashes, etc.\footnote{Interestingly, \textit{water} also falls into this class: when it combines with the SG suffix, it refers to a body of water such as a lake. For the present, we set \textit{water} aside, as we are unsure to what extent coercion plays a role.}

Like mass nouns, bare countable mass nouns cannot combine directly with the plural suffix, as shown in \tabref{tab:sande:3}.

\begin{table}
	\begin{tabular}{llllll}
	\lsptoprule
		& \textbf{Mass} & \textbf{Plural} & \textbf{Translation}\\
		& Root & *Root-PL &  \\
		\midrule
%			a. & \ipa{ɲu^{4}} & *\ipa{ɲu-a}, *\ipa{ɲu-i} & `water'\\
		%b. & \ipa{dolo^{1.1}} & *\ipa{dolo-a, *dolo-i} & `blood'\\
		%c. & \ipa{dodo^{3.2}} & *\ipa{dodo-a, *dodo-i} & `sand'\\
			a. & \ipa{novi}$^{2.3}$ & *\ipa{novi-a, *novi-i} & 	`bees'\\
			b. & \ipa{kuk}ʷe$^{4.1}$ & *\ipa{kuk}ʷe-a, *kukʷe-i & `ants'\\
			c. & \ipa{wʊlε}$^{3.1}$ & *\ipa{wʊlε-a, *wʊlε-ɪ}  & `fingers'\\
			d. & \ipa{ɟe}$^{3}$ & *\ipa{ɟe-a, *ɟe-i} & 	`stars'\\
			e. & \ipa{ɟa}$^{31}$ & *\ipa{ɟa-a, ??ɟa-i} & `coconuts'\\
			f. & \ipa{tro{\ds}biə}$^{3.2.2}$ & *\ipa{tro{\ds}biə-a, *tro{\ds}biə-i} & `eggplants'\\
	\lspbottomrule
	\end{tabular}
    \caption{Countable mass nouns in Guébie}
    \label{tab:sande:3}
\end{table}

Again like mass nouns, and unlike count nouns, bare countable mass nouns cannot combine with numerals, but can combine with quantifiers. This is shown in \REF{ex:sande:8}.

\ea%8
    \label{ex:sande:8}
    \ea[*]{%8a
    \label{ex:sande:8a}
    \gll    \ipa{ɟa}$^{31}$ \ipa{ta}$^{3}$ \\
	        coconuts three\\
	\glt    Intended: `Three coconuts'}
    \ex[]{%8b 
    \label{ex:sande:8b}
    \gll    \ipa{ɟa}$^{31}$ \ipa{a{\ds}ba}$^{4.2}$ \\
	        coconuts all\\
	\glt `All coconut'\footnote{More data is needed to know whether this has a definite interpretation similar to using a universal quantifier with a mass noun in English, and whether \REF{ex:sande:8b} is interpreted differently than \REF{ex:sande:12a}.}}
    \z
\z

Unlike both other classes of nouns, countable mass nouns can combine with the SG suffix to yield a singular individual reading. Just like bare count nouns, these SG-marked nouns cannot be predicated of plural subjects, as shown in \REF{ex:sande:9}.

\ea[*]{%9
    \label{ex:sande:9}
    \gll    \ipa{liəne}$^{3.3.1}$ \ipa{εja}$^{2.3}$ \ipa{liəko}$^{3.3.1}$ \ipa{ɟa-{\ds}bə}$^{3.1}$ \ipa{mɔ}$^{1}$\\
	        \textsc{dem.pro.prox} with \textsc{dem.pro.dist} coconuts\textsc{-sg} be.\textsc{emph}\\
	\glt    Intended: `This thing and that thing are coconut.'}
\z

However this SG form can then be pluralized with the /-a, -i/ plural marker, in which case it can surface as the predicate of a plural subject\footnote{See \cite[88--89]{Marchese1979} for a 2-way split in other Kru languages between countable nouns that take a plural suffix directly, and countable mass nouns which take -SG-PL suffixes.}, as in \REF{ex:sande:10}.

\ea%10
    \label{ex:sande:10}
    \gll    \ipa{liəne}$^{3.3.1}$ \ipa{εja}$^{2.3}$ \ipa{liəko}$^{3.3.1}$ \ipa{ɟa-{\ds}bə-i}$^{3.1.2}$ \ipa{mɔ}$^{1}$\\
	        \textsc{dem.pro.prox} with \textsc{dem.pro.dist} coconuts\textsc{-sg-pl} be.\textsc{emph}\\
	\glt    `This thing and that thing are coconuts.'
\z

\tabref{tab:sande:4} shows these number marking patterns for a selection of countable mass nouns.

\begin{table}
	\begin{tabularx}{\textwidth}{lXXXll}	
	\lsptoprule
		& \textbf{Mass} & \textbf{Singular} & \textbf{Plural} & \textbf{Translation}\\
		& Root & Root-SG & Root-SG-PL & \\
		\midrule
		a. & \ipa{ɟa}$^{31}$ & \ipa{ɟa-{\ds}bə}$^{3.1}$ & \ipa{ɟa-{\ds}bə-i}$^{3.1.2}$ & `coconut'\\
		b. & \ipa{tro{\ds}biə}$^{3.2.2}$ & \ipa{tro{\ds}biə-je}$^{3.2.2.1}$ & \ipa{tro{\ds}biə-je-i}$^{3.2.2.1.2}$ & `eggplant'\\
		c. & \ipa{novi}$^{2.3}$ & \ipa{novi-je}$^{2.3.1}$ & \ipa{novi-je-i}$^{2.3.1.2}$ & 	`bee'\\
		d. & \ipa{kuk}ʷe$^{4.1}$ & \ipa{kuk}ʷe-je$^{4.1.1}$ & \ipa{kuk}ʷe-je-i$^{4.1.1.2}$ & `ant'\\
		e. & \ipa{wʊlε}$^{3.1}$ & \ipa{wʊlε-je}$^{3.1.1}$ & \ipa{wʊlε-je-ɪ}$^{3.1.1.2}$ & `finger'\\
		f. & \ipa{ɟe}$^{3}$ & \ipa{ɟalɪ-je}$^{3.1}$ & \ipa{ɟalɪ-je-i}$^{3.1.2}$ & 	`star'\\
	\lspbottomrule
	\end{tabularx}
    \caption{Singular and Plural on countable mass nouns}
    \label{tab:sande:4}
\end{table}

%\begin{itemize}
%%	\item A subset of the \textit{mass} class can take a singular suffix (/-je/ or /-\ipa{{\ds}bə}/) for a singular reading.
%	\item The set of nouns that can take a singular suffix can take a plural suffix outside the singular one when talking about a specific quantity: bare-\textsc{sg-pl}.
%	\item We call this set of nouns \textit{countable mass} nouns.
%	\item Count nouns, as in (\ref{count}) can never take a singular suffix.
%\end{itemize}


%	\item The bare form of a countable mass noun yields a substance reading, rather than a plural reading
%	\begin{itemize}
%		\item For example, \ipa{ɟa^{31}} ``coconut'' refers to coconut substance, not to specific coconuts
%		\item This specific plural reading is only available for the bare-\textsc{sg-pl} form \ipa{ɟa-{\ds}bə-i^{3.1.2}} ``coconuts''
%	\end{itemize}

Like plural count nouns, PL-marked countable mass nouns (noun-\textsc{sg-pl}) can combine with numerals greater than one and quantifiers, but a noun-\textsc{sg} form cannot. This is shown in \REF{ex:sande:11} and \REF{ex:sande:12}.

\ea%11
    \label{ex:sande:11}
    \textbf{-\textsc{SG-PL} mass nouns with numerals}
    \ea[]{%11a
    \label{ex:sande:11a}
    \gll    \ipa{ɟa-{\ds}bə-i}$^{3.1.2}$ \ipa{ta}$^{3}$ \\
		    coconuts-\textsc{sg-pl} three\\
	\glt    `Three coconuts'}
		%	\item[b.] *\ipa{ɟa-i^{3.12}} \ipa{ta^{3}}
		%	\item[c.] *\ipa{ɟa^{31}} \ipa{ta^{3}}
    \ex[*]{%11b
    \label{ex:sande:11b}
    \gll    \ipa{ɟa-{\ds}bə}$^{3.1}$ \ipa{ta}$^{3}$\\
		    coconut-\textsc{sg} three\\
	\glt    Intended: `three coconut(s)'}
    \z
\z

\ea%12
    \label{ex:sande:12}
    \textbf{-\textsc{SG-PL} mass nouns with quantifiers}
    \ea[]{%12a
    \label{ex:sande:12a}
    \gll    \ipa{ɟa-{\ds}bə-i}$^{3.1.2}$ \ipa{a{\ds}ba}$^{4.2}$\\
			coconuts-\textsc{sg-pl} all\\
	\glt    `all coconuts'}
    \ex[*]{%12b
    \gll    \ipa{ɟa-{\ds}bə}$^{3.1}$ \ipa{a{\ds}ba}$^{4.2}$\\
			coconuts-\textsc{sg} all\\
	\glt    Intended: `all coconuts'}
    \z
\z

To summarize, bare countable mass nouns pattern with true mass nouns in that they cannot take plural marking or be modified by a numeral. By contrast, the SG marked form of a countable mass noun patterns with count nouns. The SG marked form yields a singular individual interpretation, it can take plural marking, and it can be modified by a numeral (by the numeral one in the noun-\textsc{sg} form, and by any numeral greater than one in the noun\textsc{-sg-pl} form). These properties are summarized in \tabref{tab:sande:5}.

\begin{table}
	\begin{tabularx}{\textwidth}{lcccc}
	\lsptoprule
    	& \textbf{Indiv. interp.} & \textbf{-PL} & \textbf{N-PL Numeral} & \textbf{N Quantifier} \\
    	\midrule
    	\textbf{Count} & X & X & X & \\
    	\textbf{True mass} & & & & X\\
    	\textbf{Countable mass (bare)} & & & & X\\
    	\textbf{Countable mass (-SG)} & X & X & X & \\
    \lspbottomrule
	\end{tabularx}
    \caption{Properties of noun types in Guébie}
    \label{tab:sande:5}
\end{table}

\subsection{Summary}\label{sec:sande:2.4}

Based on the distribution of singular and plural suffixes as well as numerals, we have seen that there is at least a three-way distinction in countability across nouns in Guébie: count nouns (e.g. `plate, woman'), countable mass nouns (e.g. `coconut, finger'), and true mass nouns (e.g. `blood, sand').

\section{Semantics}\label{sec:sande:3}

An analysis of the above data must account for (i) the different distribution and behavior of count nouns, true mass nouns, and countable mass nouns, and (ii) the distribution of SG and its semantic effect (i.e. that it takes a countable mass noun and turns it into a count noun). We assume here that the PL marker in Guébie is analagous to PL marking in languages like English.
%		\item Account for the distribution of PL and its semantic effect

\subsection{Count nouns vs. true mass nouns}\label{sec:sande:3.1}

A concrete way to model countability distinctions relies on notions of cumulativity and divisibility.\footnote{See \citealt{Quine1960, Cheng1973, Link1983, Krifka1989, Doetjes1997, Grimm2012Diss}; and \citealt{Deal2017}, among others.} These properties are defined in  \REF{ex:sande:13} and \REF{ex:sande:14} respectively.

\ea%13
    \label{ex:sande:13}
    A noun is cumulative iff it denotes a cumulative predicate.\\
    A predicate \textit{p} is cumulative iff any sum of parts that are \textit{p} is also \textit{p}. \hfill{\citep[128]{Deal2017}}
	%A nominal property is cumulative if, for any A that the property holds of, and any B that the property holds of, the property also holds of A+B together
\z

\ea%14
    \label{ex:sande:14}
    A noun is divisive iff it denotes a divisive predicate.\\
    A predicate \textit{p} is divisive iff any part of something that is \textit{p} is also \textit{p}. \hfill{\citep[129]{Deal2017}}
	%A nominal property is divisive if, for any A that the property holds of, the property also holds of any subpart of A
\z

Noun denotations that are neither cumulative nor divisive have been termed \textit{quantized} (\citealt{Krifka1989, Deal2017}), while those that are both cumulative and divisive have been termed \textit{homogeneous} (\citealt{Bunt1985, Deal2017}). These properties distinguish English singular count nouns and mass nouns respectively.

For example, consider the count noun ``plate''. If some thing A can be truly described as a plate, and B can also be truly described as a plate, it does not follow that A+B are a plate. Instead, A+B are truly described as plates. This shows that the English noun ``plate'' is not cumulative. Likewise, if A can be truly described as a plate, it does not follow that some subpart of A is also a plate. Instead, it would be described as part of a plate. This shows that English ``plate'' is not divisive.

In contrast, consider the mass noun ``sand''. If there is some thing A that can be truly described as sand, and B can also be truly described as sand, it follows that A+B are sand. Unlike ``plate'', the English noun ``sand'' is cumulative. Likewise, if A can be described as sand, it follows that some subpart of A is also sand. The English noun ``sand'' is also divisive.

This is summarized in \REF{ex:sande:15} and \REF{ex:sande:16}.

\ea%15
    \label{ex:sande:15}
    English singular count nouns are not cumulative and not divisive (i.e. they are quantized)
    \ea%15a
    \label{ex:sande:15a}
    A is a plate, and B is a plate, but A+B are not a plate
    \ex%15b
    \label{ex:sande:15b}
    A is a plate, but any subpart of A is not a plate
    \z
\z

\ea%16
    \label{ex:sande:16}
    English mass nouns are both cumulative and divisive (i.e. they are homogeneous)
    \ea%16a
    \label{ex:sande:16a}
    A is sand, and B is sand, and A+B is sand
    \ex%16b
    \label{ex:sande:16b}
    A is sand, and any subpart of A is sand
    \z
\z

We can schematize these properties of count and mass nouns as in \REF{ex:sande:17}. The denotation of a quantized noun like ``plate'' contains only non-overlapping individuals: while individual plates a, b, and c are in the denotation of ``plate'', their sums and subparts are not. In contrast, the denotation of a cumulative noun like ``sand'' only contains members that overlap with other members: each member of the denotation of ``sand'' is a subpart of another member, and shares each of its subparts with another member.

\ea%17
    \label{ex:sande:17}
    \ea%17a
    \label{ex:sande:17a}
    $\llbracket$plate$\rrbracket$ = \{a, b, c\}
    \ex%17b
    \label{ex:sande:17b}
    $\llbracket$sand$\rrbracket$ = \{ab, bc, ac, abc\}
    \z
\z

This analysis of the English count/mass distinction extends nicely to Guébie's count nouns and true mass nouns. Just like in English, Guébie's count nouns are quantized (i.e. neither divisive nor cumulative), and its true mass nouns are homogeneous (i.e. both divisive and cumulative). This is schematized in \REF{ex:sande:18}.

\ea%18
    \label{ex:sande:18}
    \ea%18a
    \label{ex:sande:18a}
    $\llbracket$\ipa{{\ds}bə}$^{31}$ ``plate''$\rrbracket$ = \{a, b, c\}
    \ex%18b
    \label{ex:sande:18b}
    $\llbracket$\ipa{dolo}$^{3.2}$ ``sand''$\rrbracket$ = \{ab, bc, ac, abc\}
    \z
\z

This analysis allows us to account for the distributional differences of PL between count nouns and true mass nouns: just like in English, PL can only combine with quantized denotations.\footnote{The role of PL is to add sums to the denotation, and thus makes the resulting denotation cumulative. There is debate in the literature about the exact nature of PL (e.g. whether the resulting denotation includes atoms as well as sums; see \citealt{SauerlandEtAl2005, FarkasSwart2010}), that we do not wish to address here. The Guébie PL data are compatible with analyses that account for English PL.} It also allows us to capture the restriction on numeral modification: numerals can only modify quantized denotations.\footnote{This assumes that only sets with non-overlapping members (i.e. quantized denotations) can be counted (\citealt{Chierchia1998, Landman2011}). For languages that have PL inflection on nouns that are modified by numerals $>$1, that PL marking is taken to be either purely morphosyntactic \citep{Krifka1989} or semantically undone by the numeral modification \citep{Chierchia1998}.}

%GD: Quantifiers?

\subsection{Countable mass nouns and SG}\label{sec:sande:3.2}

Bare countable mass nouns behave like mass nouns, but when they are marked with the SG suffix, they behave like count nouns. Modeling noun meanings in terms of cumulativity and divisiveness allows us capture this. Just like true mass nouns, countable mass noun denotations are cumulative. For example, arbitrarily large groups of coconuts and ants can be referred to with a bare countable mass noun. However, like count nouns and unlike true mass nouns, countable mass noun denotations are not divisive: they contain non-overlapping minimal parts. These properties can be captured by assuming that the denotations of countable mass nouns in Guébie contain both non-overlapping individual members and sums of those individual members. A countable mass noun denotation is schematized in \REF{ex:sande:19}, where individual letters \textit{a, b} and \textit{c} represent atomic individuals, such as individual coconuts or ants, and combinations of those letters represent sums of those individuals, such as a sum of two or three individual coconuts or ants.%\footnote{This type of denotation has been proposed for ``fake'' mass nouns in English (e.g. \textit{furniture}) and for individual-denoting nouns in ``classifier'' languages like Chinese and Japanese. See Section 5 for discussion.}

\ea%19
    \label{ex:sande:19}
    $\llbracket$\ipa{ɟa}$^{31}$ ``coconut''$\rrbracket$ = \{a, b, c, ab, bc, ac, abc\}
\z

Since these denotations are cumulative, they cannot combine with PL or be directly modified by numerals, just like true mass nouns. They are crucially different from mass nouns, however, in that their denotations do contain non-overlapping minimal parts. This kind of cumulative but non-divisive noun denotation is also found in English (for ``fake mass'' nouns like \textit{furniture} and \textit{jewelry}) and in ``classifier'' languages like Chinese and Japanese (see \citealt{Doetjes1997, Landman2011, Deal2017}). `Furniture' plus more `furniture' is still called `furniture' in English (cumulativity), but a sub-part of `furniture' such as the leg of a chair is not `furniture' (non-divisive). Just like in Guébie, \textit{furniture} cannot be marked PL (*furnitures) or be directly modified by numerals (*three furniture(s)). We return to the cross-linguistic picture in the following section.

Finally, we propose that this difference is what allows countable mass nouns (but not true mass nouns) to combine with the SG suffix. Specifically, the role of the SG suffix is to take in a countable mass noun denotation like in \REF{ex:sande:19}, and remove all non-atomic members. The result is the quantized denotation in \REF{ex:sande:20}, which, like the denotation of a count noun, only contains non-overlapping individuals (i.e. individual coconuts or ants).%\footnote{A reviewer raised the possibility that

\ea%20
    \label{ex:sande:20}
    $\llbracket$\ipa{ɟa-{\ds}bə}$^{3.1}$ ``coconut''$\rrbracket$ = \{a, b, c\}
\z

Since a SG-marked countable mass noun is now quantized, it can combine with PL marking, just like the quantized bare count nouns. Importantly, SG cannot attach to true mass nouns because their denotations do not contain these non-overlapping minimal parts.

The analysis presented here also allows us to capture the distribution of the quantifiers [\ipa{a{\ds}ba}$^{4.2}$] `all' and [\ipa{{\ds}butugba}$^{3.1.1}$] `many'. We propose that these quantifiers can only combine with cumulative noun denotations. This allows these quantifiers to combine with the homogeneous denotations of true mass nouns, and with the cumulative but non-divisive denotations of bare countable mass nouns, PL-marked count nouns, and SG-PL-marked countable mass nouns. In contrast, these quantifiers cannot combine with the quantized denotations of bare count nouns and SG-marked countable mass nouns.


\section{The cross-linguistic picture}\label{sec:sande:4}

We have seen that Guébie has a core three-way countability distinction in its nominal semantics, and that this three-way distinction can be captured in terms of cumulativity and divisiveness. Similar three way distinctions are also found in other languages. For example, in addition to the binary mass/count distinction, English also distinguishes a smaller class of ``fake mass'' nouns like \textit{jewelry, furniture,} and \textit{footwear}.
Welsh \citep{Grimm2012Chapter} has a larger class of nouns that are interpreted plural in their bare form, and require a SG suffix for singular reference. This contrasts with nouns that are interpreted singular in their bare form (count nouns), and those that cannot take the SG suffix (mass nouns).

Other languages appear to only make a two way distinction. For example, ``classifier'' languages, like Chinese and Japanese, make a countability distinction in terms of divisiveness, but not cumulativity.\footnote{For evidence of countability distinctions in Chinese and Japanese, see \citealt{ChengSybesma1998, InagakiBarner2009}; and \citealt{CheungEtAl2010}. For an explicit proposal in terms of cumulativity and divisiveness, see \citealt{Deal2017}.} These languages lack quantized noun denotations; typical count nouns like ``plate'' are cumulative in these languages, as indicated in (\ref{chinese})\todo{Undefined reference}. Note that this kind of analysis lends itself to an explanation of the typical absence of PL marking in such languages, and that all nouns in such languages require classifiers in numeral modification.

\ea%21
    \label{ex:sande:21}
    Noun denotations in classifier languages
    \ea%21a
    \label{ex:sande:21a}
    Individual-denoting nouns (e.g. \textit{plate}): \{a, b, c, ab, bc, ac, abc\}
    \ex%21b
    \label{ex:sande:21b}
    Substance-denoting nouns (e.g. \textit{sand}): \{ab, bc, ac, abc\}
    \z
\z

While cumulative but non-divisive noun denotations are commonly attested cross-linguistically, languages differ in how they treat such denotations.
In the first place, languages differ in what objects are assigned cumulative, non-divisive denotations. This class is small in English (\textit{furniture, jewelry, footwear} and \textit{mail}, among some others), with most nouns either truly mass or count. Languages like Guébie and Welsh, in contrast, have very large classes of such nouns, consisting of a wide variety of objects that typically come in groups. Classifier languages like Chinese and Japanese assign all non-substance nouns such denotations.

Second, languages differ in how they allow such nouns to be modified by a numeral. English uses measure words (e.g. \textit{three pieces of furniture}), while Chinese and Japanese have dedicated classifiers. In contrast, Guébie and Welsh have SG suffixes that convert a cumulative, non-divisive noun into a quantized noun.

Finally, while both Guébie and Welsh employ similar strategies for allowing such nouns to be modified by numerals (via a SG suffix), they also show an interesting difference: SG-marked nouns in Guébie can be further pluralized, but are not in Welsh.


%\begin{table}
%\caption{Cross-linguistic countability classes}
%\begin{tabular}{|l|cccc|}
%\hline
%& \textbf{English} & \textbf{Guébie} & \textbf{Welsh} & \textbf{Classifier langs}\\
%\hline
%\textbf{Cumulative \& Divisive} & $\checkmark$ (mass) & $\checkmark$ (mass) & $\checkmark$ (mass) & $\checkmark$ (substance)\\
%\textbf{Neither Cum. nor Div.} & $\checkmark$ (count) & $\checkmark$ (count)& $\checkmark$ (count)& \\
%\textbf{Cum. but not Div.} & $\checkmark$ (fake mass) & $\checkmark$ (countable mass) & $\checkmark$ (collective) & $\checkmark$ (individual) \\
%\hline
%\end{tabular}
%\end{table}

\section{Conclusion}\label{sec:sande:5}

Guébie shows a core, three-way countability distinction in its nominal semantics, based on number morphology and numeral modification. A singulative suffix takes ``countable'' mass nouns and turns them into count nouns. We model these distinctions in terms of cumulativity and divisiveness, which are useful concepts for modeling countability across languages.

\section*{Acknowledgements}

We are grateful to the Guébie community for sharing their time and their language. Thanks also to the audience at ACAL 49 in Michigan for their feedback.

\section*{Abbreviations}
\begin{tabularx}{.55\textwidth}{ll}
\textsc{dem} & demonstrative\\
\textsc{dist} & distal\\
\textsc{emph} & emphatic\\
\textsc{pro} & pronoun\\
\end{tabularx}
\begin{tabularx}{.45\textwidth}{ll}
\textsc{prox} & proximate\\
\textsc{sg} & singular\\
\textsc{pl} & plural\\
\\
\end{tabularx}

\newpage
\section*{Appendix A -- List of countable mass nouns in Guébie}
\begin{table}[!ht]
\fittable{
\begin{tabularx}{\textwidth}{p{20pt}lll}
\lsptoprule
& \textbf{Bare} & \textbf{Bare-SG-PL} & \textbf{Gloss}\\
\midrule
\multicolumn{2}{l}{\textbf{Body parts}} \\
 a. & \ipa{wʊlε}$^{3.1}$ & \ipa{wʊlε-je-ɪ}$^{3.1.1.2}$ & `finger'\\
 b. & \ipa{gala}$^{3.3}$ & \ipa{gala-je-i}$^{3.3.1.2}$ & `tooth'\\
 c. & \ipa{jiri}$^{2.3}$ & \ipa{jiri-je-i}$^{2.3.1.2}$ & `eye'\\
 d. & \ipa{juk}ʷe$^{3.3}$ & \ipa{juk}ʷe-je-i$^{3.3.1.2}$ & `ear'\\
 e. & \ipa{{\ds}bɔgɔ}$^{3.1}$ & \ipa{{\ds}bɔg}ʷ-e-i$^{3.1.1.2}$ & `leg'\\
 f. & \ipa{ɲi}$^{4}$ & \ipa{ɲi-je-i}$^{4.1.2}$ & `hair'\\
\midrule %\addlinespace[2ex]
\multicolumn{2}{l}{\textbf{Fruit and vegetables}} \\
g. & \ipa{ɟa}$^{31}$ & \ipa{ɟa-{\ds}bə-i}$^{3.1.2}$ & `coconut'\\
h. & \ipa{tro{\ds}biə}$^{3.2.2}$  & \ipa{tro{\ds}biə-je-i}$^{3.2.2.1.2}$ & `eggplant'\\
 i. & \ipa{dibo}$^{2.3}$ & \ipa{ɟiote-je-i}$^{2.2.3.1.2}$ & `plantain'\\
 j. & \ipa{gbajɔ}$^{3.1}$ & \ipa{gbajɔ-je-i}$^{3.1.1}$ & `okra'\\
 k. & \ipa{ŋatε}$^{3.1}$ & \ipa{ŋatε-je-i}$^{3.1.1.2}$ & `yam'\\
 l. & \ipa{gbajɪsɔ}$^{2.2.3}$ & \ipa{gbajɪsɔ-{\ds}bə-i}$^{2.2.3.1.2}$ & `papaya'\\
 m. & \ipa{dio}$^{3.3}$ & \ipa{dio-{\ds}bə-i}$^{3.3.1.2}$ & `pineapple'\\
% \end{tabular}

% %Also pepper, corn? millet ɲO^{23}? peanut?
% \begin{tabular}{lllll}

\midrule%\addlinespace[2ex]
\multicolumn{2}{l}{\textbf{Grains/Nuts}}\\
 n. & \ipa{saka}$^{3.3}$ & \ipa{saka-je-i}$^{3.3.1.2}$ & `rice'\\
 o. & \ipa{g}ʷi$^{3}$ & \ipa{g}ʷi-je-i$^{3.1.2}$ & `palm grain'\\
 p. & \ipa{gʊ}$^{3}$ & \ipa{gʊ-je-i}$^{3.1.2}$ & `kola nut'\\
 q. & \ipa{dodo}$^{2.3}$ & \ipa{dodo-je-i}$^{2.3.1.2}$ & `corn'\\

\midrule%\addlinespace[2ex]
\multicolumn{2}{l}{\textbf{Animals}}\\
r. & \ipa{novi}$^{2.3}$ &  \ipa{novi-je-i}$^{2.3.1.2}$ & 	`bee'\\
s. & \ipa{kuk}ʷe$^{4.1}$ &  \ipa{kuk}ʷe-je-i$^{4.1.1.2}$ & `ant'\\
 t. & \ipa{sio}$^{3.1}$ & \ipa{sio-je-i}$^{3.1.2}$ & `snail'\\
 u. & \ipa{popi}$^{3.1}$ & \ipa{popi-je-i}$^{3.1.1.2}$ & `bat'\\
 v. & \ipa{kaŋɪ}$^{3.1}$ & \ipa{kaŋɪ-je-i}$^{3.1.1.2}$ & `mosquito'\\

\midrule%\addlinespace[2ex]
\multicolumn{2}{l}{\textbf{Other}}\\
w. & \ipa{ɟe}$^{3}$ &  \ipa{ɟalɪ-je-i}$^{3.1.2}$ & 	`star'\\
 x. & \ipa{sɪka}$^{2.3}$ & \ipa{sɪka-je-i}$^{2.3.1.2}$ & `gold'\\
 y. & \ipa{gbaɟuk}ʷ\ipa{ə}$^{3.2.2}$ & \ipa{gbaɟuk}ʷ\ipa{ə-je-i}$^{3.2.2.1.2}$ & `grass'\\
 z. & \ipa{kakɔ}$^{3.1}$ & \ipa{kakɔ-je-i}$^{3.1.1.2}$ & `ember'\\
\lspbottomrule
\end{tabularx}%also beads
}
\end{table}

\printbibliography[heading=subbibliography,notkeyword=this]

\end{document}
