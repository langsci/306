\documentclass[output=paper,colorlinks,citecolor=brown]{langscibook}
\title{Focus marking strategies in Igbo}

\author{%
 Mary Amaechi\affiliation{University of Potsdam}\lastand
 Doreen Georgi\affiliation{University of Potsdam}
}

\abstract{In this paper we describe the encoding of term focus in the Benue-Kwa language Igbo. Next to a discussion of focus marking devices that are available in the language and their different pragmatic usage conditions, we highlight the fact that the observed subject / non-subject split in focus encoding provides novel insights into the scope and generality of the focus marking generalization put forward in \citet{FiedlerEtAl2010}. We argue that the distribution of focus markers is not solely regulated by pragmatic principles (viz. to avoid a default topic interpretation especially for subjects), but also by the syntactic position of the focus marker, its morphological realization conditions as well as a ban on too local subject movement.}

\IfFileExists{../localcommands.tex}{
  \addbibresource{localbibliography.bib}
  \usepackage{langsci-optional,langsci-branding}
\usepackage{langsci-gb4e}
% \usepackage{langsci-textipa}
% \usepackage{langsci-glyphs}
\usepackage[linguistics]{forest}
\usepackage{tabto}
\usepackage{multirow}
\usepackage{bbding}

\usepackage[normalem]{ulem}

\usepackage{tikz-qtree}

\usepackage{enumitem}

\usepackage{multicol}
\usepackage{stmaryrd} %double brackets

\makeatletter
\let\pgfmathModX=\pgfmathMod@
\usepackage{pgfplots,pgfplotstable}%
\let\pgfmathMod@=\pgfmathModX
\makeatother
\usepgfplotslibrary{colorbrewer}
\usetikzlibrary{fit}

\usepackage{jambox}
\usepackage{tikz-qtree-compat}
\usetikzlibrary{arrows, arrows.meta}
\usepackage{longtable}
\usepackage{subcaption}

  \makeatletter
\let\thetitle\@title
\let\theauthor\@author
\makeatother

\newcommand{\togglepaper}[1][0]{
%   \bibliography{../localbibliography}
  \papernote{\scriptsize\normalfont
    \theauthor.
    \thetitle.
    To appear in:
    Change Volume Editor \& in localcommands.tex
    Change volume title in localcommands.tex
    Berlin: Language Science Press. [preliminary page numbering]
  }
  \pagenumbering{roman}
  \setcounter{chapter}{#1}
  \addtocounter{chapter}{-1}
}

\newcommand{\bari}{\ipabar{\i}{.5ex}{1.1}{}{}}
\newcommand{\notipa}[1]{\textnormal{#1}}

\newcommand{\agre}{\textsc{agr}-\ol{eene}}

\renewcommand{\emph}[1]{\textit{#1}} % resetting a setting from ling-macros-modified (I think?)

% forest settings to make compact but (mostly) straight-spined trees:
\forestset{
fairly nice empty nodes/.style={
            delay={where content={}{shape=coordinate,for parent={
                  for children={anchor=north}}}{}}
, angled/.style={content/.expanded={$<$\forestov{content}$>$}}
}}

\forestset{sn edges/.style={for tree={parent anchor=south, child anchor=north}}}

\newcommand{\bex}{\begin{exe}}
\newcommand{\fex}{\end{exe}}

\newcommand{\bxl}{\begin{exe}}
\newcommand{\fxl}{\end{exe}}

\newcommand{\ix}[1]{\textsubscript{#1}}
\newcommand{\alert}[1]{\textbf{#1}}
\newcommand{\ol}[1]{\textit{#1}}


			\usetikzlibrary{shapes,arrows,positioning,decorations,decorations.pathmorphing,intersections}
\forestset{
nice empty nodes/.style={
    for tree={calign=fixed edge angles},
    delay={where content={}{shape=coordinate,for siblings={anchor=north}}{}}
},
}

\definecolor{dark-gray}{gray}{0.3}

%\usepackage{dingbat,pifont}


%%%%%%%%%%%%For arrows%%%%%%%%%%%%%

\newcommand\Tikzmark[2]{%
  \tikz[remember picture]\node[inner sep=0pt,outer sep=0pt] (#1) {#2};%
}
\NewDocumentCommand\DrawArrow{O{}mmmmO{3}}{
\tikz[remember picture,overlay]
  \draw[->,line width=0.8pt,shorten >= 2pt,shorten <= 2pt,#1]
    (#2) -- ++(0,-#6\ht\strutbox) coordinate (aux) -- node[#4] {#5} (#3|-aux) -- (#3);
}
\NewDocumentCommand\DrawDotted{O{}mmmmO{3}}{
\tikz[remember picture,overlay]
  \draw[->,line width=0.9pt,dotted,shorten >= 2pt,shorten <= 2pt,#1]
    (#2) -- ++(0,-#6\ht\strutbox) coordinate (aux) -- node[#4] {#5} (#3|-aux) -- (#3);
}
\NewDocumentCommand\DrawLine{O{}mmmmO{3}}{
\tikz[remember picture,overlay]
  \draw[line width=0.8pt,shorten >= 2pt,shorten <= 2pt,#1]
    (#2) -- ++(0,-#6\ht\strutbox) coordinate (aux) -- node[#4] {#5} (#3|-aux) -- (#3);
}
%%%%%%%%%%%%%%%%%%%%%%%%%%%%%%%%%%%%%


\newcommand{\baru}{ʉ}
\newcommand{\baruH}{\'\baru}
\newcommand{\baruL}{\`\baru}

\newcommand{\ep}{ε}
\newcommand{\epH}{\'\ep}
\newcommand{\epL}{\`\ep}

\newcommand{\schwa}{ə}
\newcommand{\schwaH}{\'ə}
\newcommand{\schwaL}{\`ə}

\newcommand{\oo}{ɔ}
\newcommand{\ooH}{\'\oo}
\newcommand{\ooL}{\`\oo}

\newcommand{\ds}{\textsuperscript{
	\hspace*{-2pt}\begin{tikzpicture}
		\draw[-{>[scale=0.5]}] (0,0.4) --(0,0.25);
	\end{tikzpicture}}}

\newcommand{\ch}{t͡ʃ}
\newcommand{\dz}{d͡ʒ}

\newcommand{\tgl}{ʔ}

%shortcuts for the complementizers
\newcommand{\mbuL}{mb\baruL}
\newcommand{\mbuHL}{mb\baruH\baruL}
\newcommand{\mbuLH}{mb\baruL\baruH}
\newcommand{\la}{lá}
\newcommand{\nda}{ndà}

\newcommand{\tsc}[1]{\textsc{#1}}
\renewcommand{\textscb}{ʙ}
\newcommand{\ipa}[1]{#1} %disable IPA

\newcommand{\SM}[1]{#1}

\DeclareNewSectionCommand
  [
    counterwithin = chapter,
    afterskip = 2.3ex plus .2ex,
    beforeskip = -3.5ex plus -1ex minus -.2ex,
    indent = 0pt,
    font = \usekomafont{section},
    level = 1,
    tocindent = 1.5em,
    toclevel = 1,
    tocnumwidth = 2.3em,
    tocstyle = section,
    style = section
  ]
  {appendixsection}

\renewcommand*\theappendixsection{\Alph{appendixsection}}
\renewcommand*{\appendixsectionformat}
              {\appendixname~\theappendixsection\autodot\enskip}
\renewcommand*{\appendixsectionmarkformat}
              {\appendixname~\theappendixsection\autodot\enskip}

\renewcommand{\lsChapterFooterSize}{\footnotesize}
 
  %% hyphenation points for line breaks
%% Normally, automatic hyphenation in LaTeX is very good
%% If a word is mis-hyphenated, add it to this file
%%
%% add information to TeX file before \begin{document} with:
%% %% hyphenation points for line breaks
%% Normally, automatic hyphenation in LaTeX is very good
%% If a word is mis-hyphenated, add it to this file
%%
%% add information to TeX file before \begin{document} with:
%% %% hyphenation points for line breaks
%% Normally, automatic hyphenation in LaTeX is very good
%% If a word is mis-hyphenated, add it to this file
%%
%% add information to TeX file before \begin{document} with:
%% \include{localhyphenation}
\hyphenation{
affri-ca-te
affri-ca-tes 
Līk-pāk-páln
pro-sod-ic
phe-nom-e-non
Chi-che-wa
Lu-bu-ku-su
Ngbu-gu
Boyel-dieu
Mat-chi
pho-neme
Mil-em-be
Nyan-chera
Mc-Pher-son
Tsoo-tso
Sku-pin
dis-tin-guishes
con-ser-va-tion
Me-dum-ba
}

\hyphenation{
affri-ca-te
affri-ca-tes 
Līk-pāk-páln
pro-sod-ic
phe-nom-e-non
Chi-che-wa
Lu-bu-ku-su
Ngbu-gu
Boyel-dieu
Mat-chi
pho-neme
Mil-em-be
Nyan-chera
Mc-Pher-son
Tsoo-tso
Sku-pin
dis-tin-guishes
con-ser-va-tion
Me-dum-ba
}

\hyphenation{
affri-ca-te
affri-ca-tes 
Līk-pāk-páln
pro-sod-ic
phe-nom-e-non
Chi-che-wa
Lu-bu-ku-su
Ngbu-gu
Boyel-dieu
Mat-chi
pho-neme
Mil-em-be
Nyan-chera
Mc-Pher-son
Tsoo-tso
Sku-pin
dis-tin-guishes
con-ser-va-tion
Me-dum-ba
}
 
  \togglepaper[1]%%chapternumber
}{}

\begin{document}
\maketitle 


\section{Introduction}\label{sec:amaechi:1}

This paper investigates focus marking in the Benue-Kwa language Igbo spoken in Southern Nigeria. Focus is an information-structural category; the constituent in focus is the most salient part of an utterance in a given discourse and signals the presence of alternatives that are relevant in the discourse for the interpretation of an utterance (see among others \citealt{Jackendoff1972, Dik1997, Rooth1985, Krifka2008, ZimmermannOnea2011}). We will be concerned with focus marking in Igbo, i.e. the linguistic encoding of focus by grammatical devices \citep{FiedlerEtAl2010}. Furthermore, we will concentrate on the term focus (viz. the encoding of focus on arguments and adjuncts) and leave verb and VP-focus for future research. Igbo is of interest for the study of focus marking because it is relatively rich in  morphosyntactic devices that are available to mark focus, and to a certain extent the different strategies encode different pragmatic types of focus. But apart from describing the focus marking system of Igbo, the main aim of this paper is to highlight a subject / non-subject asymmetry in focus marking. While such a split is cross-linguistically common, especially in (West) African languages, as documented in \citet{FiedlerEtAl2010} and \citet{Kalinowski2015}, the Igbo split provides (partial) counter-evidence for the generalization on focus marking splits put forward in \citet{FiedlerEtAl2010}. They observe that in contrast to non-subject focus, subject focus must always be marked in some way in West African languages. In Igbo, however, subjects cannot and sometimes must not be marked for focus by the usual devices applied in the language. This shows that the focus marking asymmetry between subjects and non-subjects can also have other sources than the pragmatic one identified in \citet{FiedlerEtAl2010} (viz. the avoidance of a topic interpretation for subjects). We argue that the asymmetry in Igbo results from an interaction of the syntactic position of the focus marker in the left periphery, a morphological realization condition on the focus head and a general constraint on too local movement (anti-locality).

Before we can investigate focus marking strategies in Igbo, we must first introduce the basics of its morphosyntax (see e.g. \citealt{GreenIgwe1963, Carrel1970, Manfredi1991, Mbah2006, Emenanjo2015}). The basic word order in a thetic sentence in Igbo is SBJ-V-DO-ADJ(uncts), see \REF{ex:amaechi:1}:

\ea%1
    \label{ex:amaechi:1}
    \gll    Òbí hụ̀rụ̀ Àdá n'-áhíá.\\
            Obi saw Ada P-market\\
    \glt    `Òbí saw Àdá at the market.'
\z

The language does not have verb-argument-agreement but rich derivational morphology \citep{Uwalaka1988}. The case system is highly reduced with Nom-Acc distinctions in some personal pronouns. Igbo is a tone language that distinguishes high (\'a), low (\`a) and downstep (\=a) tone; these encode both grammatical and lexical distinctions. We assume the clause-structure in \REF{ex:amaechi:2} for an information-structurally neutral sentence with a transitive verb \citep{AmaechiGeorgi2019}:

\ea%2
    \label{ex:amaechi:2}
    $[_{CP}$ C [$_{TP}$ DP$_{ext}$ [$_{T'}$ V+v+T [$_{vP}$ <DP$_{ext}$> [$_v'$ <v> [$_{VP}$ <V> DP$_{int}$ ]]]]]]
\z

The verb moves successive-cyclically through v to T (lower copies indicated in angled brackets); the structurally highest argument (the external argument) undergoes obligatory EPP-movement to SpecT (see \citealt{AmaechiGeorgi2019} for empirical arguments for these assumptions).

The paper is structured as follows: In \sectref{sec:amaechi:2} we introduce the various focus marking strategies of Igbo and document which pragmatic types of focus they can express. \sectref{sec:amaechi:3} shows that the term questions follow the same encoding strategies as focus, with an interesting difference with respect to (local) subjects. \sectref{sec:amaechi:4} discusses the subject/non-subject marking asymmetry and its relevance for cross-linguistic generalizations on focus marking splits. In \sectref{sec:amaechi:5} we briefly summarize our analysis of a (subset of the) focus marking strategies that derives the observed asymmetry.

\section{The expression of focus in Igbo}\label{sec:amaechi:2}

In this section we will describe how arguments and adjuncts can be focused in Igbo. We summarize both the morphosyntactic means as well as the rough discourse-pragmatic use of the various strategies.\footnote{A subset of the basic morphosyntactic facts has already been described in the mostly descriptive literature on Igbo, although with a focus on question formation, see e.g. \citet{Goldsmith1981, Ikekeonwu1987, Uwalaka1991, Mmaduagwu2012, Nwankwegu2015}. However, these sources do not provide a systematic overview, do not take into account all focus marking devices or pragmatic usage factors; and most importantly, they do not offer a detailed study and analysis for the observed  subject / non-subject split in focus marking. The data in this paper come from one of the authors, Mary Amaechi, who is a native speaker of Igbo. The data have been verified with several other native speakers, see the acknowledgements.}

\subsection{Morphosyntactic properties}\label{sec:amaechi:2.1}

Focus on arguments and adjuncts in Igbo can be expressed in a number of ways that we will refer to as the in-situ strategy, the ex-situ strategy, and the cleft strategy, respectively. We address each of them in turn. In the in-situ strategy, the element that is focused occurs in its canonical position, i.e. in the position it also occupies in an all new / out-of-the-blue-sentence (where the corresponding constituent alone is not in focus), see \REF{ex:amaechi:3}; the focused elements are represented in small caps in the English translations. Hence, there is no syntactic marking of focus in the sentence; there is also neither a morphological indication of focus (e.g. by a focus marker) nor phonological highlighting (e.g. by stress) of the focused constituent. This strategy is used frequently in answers to questions and it is only available for non-subjects, i.e. direct objects, adjuncts (see \ref{ex:amaechi:3}) as well as indirect objects, but not for (local) subjects without further changes (see below):

\newpage

\ea%3
    \label{ex:amaechi:3}
    In-situ focus (Igbo)\\
    \ea\label{ex:amaechi:3a}
    Context: Òbí hụ̀rụ̀ ònyé n'-áhíá? -- `Who did Obi see at the market?'\\
    \gll    Òbí hụ̀rụ̀ Àdá n'-áhíá.\\
            Obi saw Ada P-market\\
    \glt    `Òbí saw \textsc{Àdá} at the market.' \hfill{\small \textsc{do} focus}
    \ex\label{ex:amaechi:3b}
    Context: Òbí hụ̀rụ̀ Àdá n'èbé\=e? -- `Where did Obi see Ada?'\\
    \gll    Òbí hụ̀rụ̀ Àdá n'-áhíá.\\
            Obi saw Ada P-market\\
    \glt    `Òbí saw Àdá \textsc{at the market}.' \hfill{\small \textsc{adj} focus}
    \ex[*]{\label{ex:amaechi:3c}
    Context: Ònyé hụ̀rụ̀ Àdá  n'-áhíá? -- `Who saw Ada at the market?'\\
    \gll    Òbí hụ̀rụ̀ Àdá n'-áhíá.\\
            Obi saw Ada P-market\\
    \glt    `\textsc{Òbí} saw Àdá at the market.' \hfill{\small \textsc{sbj} focus}}
    \z
\z

It is possible to focus (local) subjects in-situ after all if they are accompanied by a focus-sensitive particle like \textit{sọ̀ọ́sọ̀}, `only':

\ea%4
    \label{ex:amaechi:4}
    \gll    Sọ̀ọ́sọ̀ Òbí hụ̀rụ̀ Àdá n'-áhíá.\\
            only Obi saw Ada P-market\\
    \glt    `Only \textsc{Òbí} saw Àdá at the market.'
\z

In the ex-situ strategy, see \REF{ex:amaechi:5}, the focused phrase occurs in the clause-initial position and must be followed by the morpheme \textit{k\`a} (which we will identify as a focus marker below). Note that this strategy is also not available for (local) subjects: they occupy the clause-initial position anyway due to Igbo's  SVO word order, but they cannot co-occur with the morpheme \textit{k\`a}. For all other XPs (direct objects, indirect objects, adjuncts) the construction is available.

\enlargethispage{0.5\baselineskip}
\ea%5
    \label{ex:amaechi:5}
    Ex-situ focus (Igbo)\\
    \ea[]{\label{ex:amaechi:5a}
    \gll    Àdá \textbf{*(kà)} Òbí hụ̀rụ̀ {\longrule}   n'-áhíá.\\
            Ada \textsc{foc} Obi saw {} P-market\\
    \glt    `Òbí saw \textsc{Àdá} at the market.' \hfill{\small \textsc{do} focus}}
    \ex[]{\label{ex:amaechi:5b}
    \gll    N'-áhíá \textbf{*(kà)} Òbí hụ̀rụ̀ Àdá  {\longrule}. \\
            P-market \textsc{foc} Obi saw Ada {}\\
    \glt    `Òbí saw Àdá \textsc{at the market}.' \hfill{\small \textsc{adj} focus}}
    \ex[*]{\label{ex:amaechi:5c}
    \gll    Òbí (\textbf{kà})  hụ̀rụ̀ Àdá  n'-áhíá.\\
            Obi \textsc{foc} saw Ada P-market\\
    \glt    `\textsc{Òbí} saw Àdá at the market.' \hfill{\small \textsc{sbj} focus}}
    \z
\z

In the ex-situ strategy focus is thus indicated both syntactically (by a change in the position of the focused element) as well as by morphological means (i.e. by the marker \textit{k\`a} that follows the focused constituent).

Finally, all grammatical functions in Igbo can be focused by means of a cleft structure. Clefts in Igbo are biclausal: The main clause is introduced by the invariant 3sg nominative personal pronoun \textit{ọ́} followed by the copula  \textit{bụ̀} (that usually occurs in identificational copula clauses). It embeds a CP in which focus is expressed by the ex-situ strategy, i.e. the focused XP occurs in clause-initial position and is followed by the morpheme \textit{k\`a}, see \REF{ex:amaechi:6}. Note that subjects can also be focused in a cleft, even though they can still not co-occur with \textit{k\`a} in the embedded clause, so we seem to be dealing rather with the in-situ strategy for focused subjects in the embedded clause of a cleft. Note that in contrast to what the English translation might suggest, these clefts in Igbo do not include a relative clause (see \citealt{Amaechi2018} for arguments).

\ea%6
    \label{ex:amaechi:6}
    Cleft strategy (Igbo)\\
    \ea\label{ex:amaechi:6a}
    \gll    Ọ́ bụ̀ Àdá kà Òbí hụ̀rụ̀ nà {\`m}gbède.\\
            it \textsc{cop} Ada \textsc{foc} Obi saw P evening\\
    \glt    `It is Àdá that Òbí saw in the evening.' \hfill{\textsc{do} focus}
    \ex\label{ex:amaechi:6b}
    \gll    Ọ́ bụ̀ nà {\`m}gbède kà Òbí hụ̀rụ̀ Àdá.\\
            it \textsc{cop} P evening \textsc{foc} Obi saw Ada\\
    \glt    `It is in the evening that Òbí saw Àdá.' \hfill{\textsc{adj}  focus}
    \ex\label{ex:amaechi:6c}
    \gll    Ọ́ bụ̀ Òbí hụ̀rụ̀ Àdá nà {\`m}gbède.\\
            it \textsc{cop} Obi saw Ada P evening\\
    \glt    `It is Òbí who saw Àdá in the evening.' \hfill{\textsc{sbj}-cleft}
    \z
\z

The cleft strategy is the only way in Igbo to focus (local) subjects without the need of an additional focus-sensitive particle.

The ex-situ and the cleft strategy can both be applied long-distance (i.e. the focused element can occur in a structurally higher clause than the one to which it is thematically related) and to all grammatical functions, see \REF{ex:amaechi:7} and \REF{ex:amaechi:8} for object and subject focus, respectively (the same holds for adjuncts and indirect objects):

\ea%7
    \label{ex:amaechi:7}
    Long ex-situ focus (Igbo)\\
    \ea\label{ex:amaechi:7a}
    \gll    Àdá \textbf{*(kà)} Úchè chèrè nà Òbí hụ̀rụ̀ {\longrule} n'-áhíá.\\
            Ada \textsc{foc} Uche think that Obi saw {} P-market\\
    \glt    `Úchè thinks that Òbí saw \textsc{Àdá} at the market.' \hfill{\small long \textsc{do} focus}
    \ex\label{ex:amaechi:7b}
    \gll    Òbí kà Úchè chèrè (*nà) {\longrule} hụ̄rụ̄ Àdá n'-áhíá.\\
            Obi \textsc{foc} Uche think (*that) {} saw Ada  P-market\\
    \glt    `Úchè thinks that \textsc{Òbí} saw Àdá at the market.' \hfill{\small long \textsc{sbj} focus}
    % \ex\label{ex:amaechi:7c}
    % \gll    N'-áhíá \textbf{*(kà)} Úchè chèrè nà Òbí hụ̀rụ̀ Àdá {\longrule}.\\
    %         P-market \textsc{foc} Uche think that Obi saw Ada {}\\
    % \glt    `Úchè thinks that Òbí saw Àdá \textsc{At the market} .' \hfill{\small long \textsc{adj} focus}
    \z
\z

\ea%8
    \label{ex:amaechi:8}
    Long clefts (Igbo)\\
    \ea\label{ex:amaechi:8a}
    \gll    Ọ́ bụ̀ Òbí kà Úchè chèrè nà Àdá hụ̀rụ̀ {\longrule}.\\
	        it \textsc{cop} Obi foc Uche thinks that Ada saw {}\\
    \glt    `It is Òbí that Úchè thinks that Àdá saw.' \hfill{\small long \textsc{do} focus}
    \ex\label{ex:amaechi:8b}
    \gll    Ọ́ bụ̀ Àdá kà Úchè chèrè (*nà) {\longrule} hụ̄rụ̄ Òbí.\\
	        it \textsc{cop} Ada \textsc{foc} Uche thinks (*that) {} saw Obi\\
    \glt    `It is Àdá that Úchè thinks saw Òbí.' \hfill{\small long \textsc{sbj} focus}
    \z
\z

Note that while local subjects cannot be focused with the ex-situ strategy, long-distance ex-situ focus is possible for subjects, see \REF{ex:amaechi:7}. Crucially, however, a long ex-situ focused subject must be accompanied by the marker \textit{k\`a}, just like (locally and non-locally) focused non-subjects in the ex-situ construction. This observation also provides evidence against the traditional view put forward in the descriptive literature on Igbo that the presence or absence of the morpheme \textit{k\`a} is driven by the grammatical function of the focused constituent: non-subjects combine with \textit{k\`a}, while subjects cannot do so. Since long-distance displaced focused subjects have to take \textit{k\`a} as well, the decisive factor cannot be grammatical function; see below for an alternative proposal.

\subsection{Semantic focus types}\label{sec:amaechi:2.2}

Focus expresses the presence of contextually salient alternatives that are relevant for the interpretation of a sentence \citep{Rooth1985, Rooth1992}. According to \citet{ZimmermannOnea2011} additional semantic and pragmatic factors can come into play and lead to different types of foci. \citet{Bazalgette2015} distinguishes between simple focus (that has no function besides triggering alternatives) and pragmatic focus that is associated with implicatures (e.g. contrast, exclusivity) or with presuppositions (e.g. existence, exhaustivity). \citet{VanDerWal2016} summarizes (and criticizes) tests used in the literature to identify these semantic and pragmatic focus types. We applied some of these tests to Igbo and checked which of the three syntactic focus strategies can be used in which function. The result is summarized in \tabref{tab:amaechi:1} (\langscicheckmark means that the strategy can be used to express this focus type, $\ast$ means that the strategy cannot be used in this context). In fact, the various syntactic strategies differ in the focus type they (preferably) express.

%Table 1
\begin{table}
    \begin{tabularx}{\textwidth}{Xllll}
    \lsptoprule
        test/function&in-situ&ex-situ&cleft \\
    \midrule
        { answer to question (new information)}&\langscicheckmark&\langscicheckmark&\langscicheckmark\\
        %&&&\\\midrule
        { alternative question  (selective focus)}&{\langscicheckmark}&\langscicheckmark&{$\ast$}\\
        %&&&\\\midrule
        % { mention some     (non-exhaustive)}&&&\\\midrule
        %&&&\\\midrule
        {  negative response  (non-presuppositional)}&\langscicheckmark&\langscicheckmark&$\ast$\\
        %&&&\\\midrule
        {  compatible with `only'   (exhaustive)}&\langscicheckmark&\langscicheckmark&\langscicheckmark \\
        % &&&\\\midrule
        {  compatible with additive `also' (non-exhaustive)}&\langscicheckmark&$\ast$&$\ast$ \\
        % (non-exhaustive)&&&\\\midrule
        { compatible with scalar `even' (non-exclusive)} &\langscicheckmark&$\ast$&$\ast$ \\
        %&&&\\\midrule
        { numeral interpreted as `exactly' in focus}&$\ast$&\langscicheckmark&\langscicheckmark \\
        %{ (non-exhaustive)}&&&\\\midrule
        % { weak quantifiers in focus (exclusive, exhaustive)} &&&\\\midrule
        % (exclusive, exhaustive)&&&\\\midrule
        {  exclusive/exhaustive co-text}&$\ast$&\langscicheckmark&\langscicheckmark\\
        { non-exhaustive co-text}&\langscicheckmark&$\ast$&$\ast$\\
        %{ unexpectedness}&&&\\\midrule
        { correction (of a previous statement)}&\langscicheckmark&\langscicheckmark&\langscicheckmark\\
    \lspbottomrule
\end{tabularx}
    \caption{Usage of the focus strategies}
    \label{tab:amaechi:1}
\end{table}

Space limitations prevent us from illustrating all contexts; we provide two below:

    %\ea
        %Negative response:\\
        %Context: Your son's friend is visiting and you invite her for dinner. You have prepared yam and cocoyam. You tell your son to ask his friend what she would like to eat. Later you ask him:
    %\begin{multicols}{2}
    %\ea
        %Question:\\
        %\gll   O choro ji ka o bu ede\\
                %3\textsc{sg} want yam or 3\textsc{sg} \textsc{cop} cocoyam.\\
        %\glt   `Does she want yam or cocoyam?'
    %\columnbreak
    %\ex
        %Answer:\\
        %O choro ji. \textit{in-situ}\\
         %Ji ka o choro. \textit{ex-situ}\\
         %\#O bu ji ka o choro. \textit{cleft}
    %\z
    %\end{multicols}
    %\z

\todo[inline]{Multicols environment removed due to the significant asymmetry between the two columns below.}

\ea%9
    \label{ex:amaechi:9}
    Correction:\\\vspace*{-0.2cm}
% \columnsep -3.3cm
% \begin{multicols}{2}
    \ea\label{ex:amaechi:9a}
    Statement A:\\
    \gll    Òbí hụ̀rụ̀ Àdá n'-áhíá.\\
            Obi saw Ada P-market\\
    \glt    `Òbí saw {Àdá} at the market.'\\
% ~\\
% ~
% \columnbreak
    \ex\label{ex:amaechi:9b}
    Corrective statement B:\\
        \ea\label{ex:amaechi:9bi}
        \gll    {\`M}bà, Òbí hụ̀rụ̀ Úchè n'-áhíá.\\
                no Obi saw Uche P-market\\
        \glt    `No, Ò. saw \textsc{Úchè} at the market.' \textit{in-situ}
        \ex\label{ex:amaechi:9bii}
                {\`M}bà, Úchè kà Òbí hụ̀rụ̀   n'-áhíá. ~~\textit{ex-situ}
        \ex\label{ex:amaechi:9biii}
                O bu Úchè kà Òbí hụ̀rụ̀   n'-áhíá. ~~~~\textit{cleft}
                %no Ada \textsc{foc} Obi saw {} P-market\\
        %\glt `No, Òbí saw \textsc{Úchè} at the market.' \hfill\textit{focus ex-situ}
        \z
    \z
% \end{multicols}
\z

\ea%10
    \label{ex:amaechi:10}
    Numeral interpretation:\\
    \ea\label{ex:amaechi:10a}
    \gll    \d{\'O} n\`a-\`enw\'et\'a \'ot\`u \`nd\`e n'\d{\'o}nw\'a.\\
            3\textsc{sg} \textsc{ipfv}-\textsc{nmzl}.have.\textsc{dir} one million P-moon\\
    \glt    `S/he earns (at least) one million a month.' \hfill{in-situ, upward entailing}
    \ex\label{ex:amaechi:10b}
    \gll    \'Ot\`u \`nd\`e k\`a \d{\'o} n\`a-\`enw\'et\'a n'\d{\'o}nw\'a.\\
            one million \textsc{foc} 3\textsc{sg} \textsc{ipfv}-\textsc{nmzl}.have.\textsc{dir} P-moon\\
    \glt    `S/he earns (exactly) one million a month.' \hfill{in-situ,~not~upward~entail.}
    \ex\label{ex:amaechi:10c}
    \gll    \d{\'O} b\d{\`u} \'ot\`u \`nd\`e k\`a \d{\'o} n\`a-\`enw\'et\'a n'\d{\'o}nw\'a.\\
            3\textsc{sg} \textsc{cop} one million \textsc{foc} 3\textsc{sg} \textsc{ipfv}-\textsc{ipfv}.have.\textsc{dir} P-moon\\
    \glt    `S/he earns (exactly) one million a month.' \hfill{cleft, not upward entailing}
    \z
\z

%The first subtype is simple focus, which only triggers a set of alternatives and does nothing else. - does not necessarily order these alternatives or exclude any of them as not true
%The second type of focus Bazalgette describes triggers alternatives and in
%addition has an associated implicature of contrast, exclusivity or unexpectedness. - exhaustivity
%The third type of focus triggers alternatives and additionally is associated with a presupposition (e.g. of existence or exhaustivity
%The fourth type concerns the content of the common ground (Krifka 2007b): in addition to triggering a set of alternatives there are operations on that set of alternatives, which can result in a scalar, exhaustive or exclusive reading, and which have a truth-conditional effect.
%The literature on focus distinguishes between semantic and
%simple focus, pragmatical flavours, semantic/truth conditional (exhaustive/contrastive focus)

%\newpage
%One can distinguish between different pragmatic types of focus (\citealt{Rooth:85, Schwarzschild:99, Dik:97, Rochemont:86, ZimmermannOnea:11:focus}). Most commonly, the following three subtypes are postulated: (i) new information focus as in answers to questions (see \ref{ex:F} for a question), (ii) corrective focus where a different element  from the set of contextually relevant alternatives replaces a  previously mentioned alternative (see \ref{ex:amaechi:9}), and (iii) contrastive focus where alternatives are contrasted in their properties (see \ref{ex:H}).

%\ea \label{ex:F}
%\ea
%\gll Ònyé *(\textbf{kà}) Òbí hụ̀rụ̀ n'-áhíá.\\
%who \textsc{foc} Obi saw P-market\\
%\glt `Who did Òbí see at the market?'\hfill{\small \textsc{do} question}
%\ex
%\gll Àdá \textbf{*(kà)} Òbí hụ̀rụ̀ {\longrule}   n'-áhíá.\\
%Ada \textsc{foc} Obi saw {} P-market\\
%\glt `Obi saw \textsc{Ada} at the market.' \hfill{\small \textsc{do} focus, new information}
%\z
%\z

%\ea \label{ex:amaechi:9}
%\ea
%{Speaker A:}\\
%\gll Òbí hụ̀rụ̀ Àdá n'-áhíá\\
%Obi saw Ada P-market\\
%\glt `Òbí saw \textsc{Àdá} at the market.'
%\ex
%{Speaker B:}\\
%\gll {\`M}bà, Úchè \textbf{*(kà)} Òbí hụ̀rụ̀ {\longrule}   n'-áhíá\\
%no Ada \textsc{foc} Obi saw {} P-market\\
%\glt `No, Òbí saw \textsc{Úchè} at the market.' \hfill{\small \textsc{do} focus, corrective}
%\z
%\z

%\ea \label{ex:H}
%\gll Òbí hụ̀rụ̀ Àdá n'-áhíá mà {\`U}g{\'o} *\textbf{(kà)} Úchè hụ̀rụ̀ n'-áhíá.\\
%Obi saw Ada P-market but Ugo \textsc{foc} Uche saw P-market\\
%\glt `Òbí saw Àdá at the market but Úchè saw \textsc{{\`U}g{\'o}} at the market.'\\\vspace*{-0.2cm}\phantom{~}\hfill{\small \textsc{do} focus, contrastive}
%\z

%The three pragmatic focus types are illustrated here with the ex-situ strategy because this is perceived as the most natural option in the three cases. However, all three focus types can in principle also be expressed by using the in-situ or the cleft strategy. Nevertheless, in-situ focus is used preferably to express new information focus rather than contrast or correction. The difference in the usage between the ex-situ and the cleft strategy is the following: As in many other languages, clefts induce exhaustiveness, viz. they indicates that the predicate holds solely for the mentioned alternative but for none of the other alternatives from the contextually salient set. This can be shown by applying various tests for exhaustivity (see e.g. \citealt{Kiss:98}). For example, one cannot add other alternatives in an afterthought to a cleft, but this is possible after a sentence with ex-situ focus, cf. the answers in \ref{ex:J} to the question in \ref{ex:I}:

%\ea\label{ex:I}
%\gll Ònyé kà Àdá hụ̀rụ̀.\\
%	who \textsc{foc} Ada saw\\
%\glt `Who did Àdá see?'
%\z

%\ea\label{ex:J}
%\ea
%\gll Òbí kà Àdá hụ̀rụ̀. Ọ́ hụ̀-kwà-rà Úchè.\\
%Obi \textsc{foc} Ada saw it see-also-\textsc{pst} Uche\\
%\glt `Àdá saw \textsc{Òbí}. And (she saw) \textsc{Úchè}, too.' \hfill{\textit{ex-situ \textsc{do} focus}}
%\ex
%\gll Ọ́ bụ̀ Òbí kà Àdá hụ̀rụ̀. \# Ọ́ hụ̀-kwà-rà  Úchè.\\
%it \textsc{cop} Obi \textsc{foc} Ada saw {} it see-also-\textsc{pst} Uche\\
%\glt `It is \textsc{Òbí} who Àdá saw. And (she saw) \textsc{Úchè}, too.'\hfill{\textit{\textsc{do} focus, cleft}}
%\z
%\z

\enlargethispage{1.0\baselineskip}

Thus, we can see that the morphosyntactic focus marking strategies differ in the semantic / pragmatic focus types they express.
%,.even though the association between focus marking type and pragmatic focus type in Igbo is rather loose, the marking type (in-situ focus mainly for new information focus) as well as the pragmatic function (exhaustivity)  have an influence on the choice of the focus marking strategy.

\section{Focus marking in questions and the morpheme \textit{kà}}\label{sec:amaechi:3}

As noted in \citet{FiedlerEtAl2010}, focus marking is very often not only found in focus constructions but also has other functions. Indeed, the same marking strategies described above for focus can also be found in constituent questions in Igbo. This is not surprising in light of the fact that wh-elements are usually considered to be inherently focused (see e.g. \citealt{Rochemont1986, Horvath1986, Tuller1986, Beck2006, Haida2007}). When asking a constituent question in Igbo, the corresponding wh-pronoun can either remain in-situ, be moved to the clause-initial position and must then be followed by the morpheme \textit{kà}, or can be expressed by means of a cleft, see \ref{ex:amaechi:11} for subject and direct object questions:\footnote{In addition to the strategies listed in \ref{ex:amaechi:11}, Igbo also has other means to form questions, especially the so-called \textit{kèd{\'u}}-construction, which shows different properties than the constructions discussed here and is also syntactically very different, viz. potentially a biclausal structure with an embedded relative clause, see among others \citet{Ikekeonwu1987, Ndimele1991, Nwankwegu2015, Ogbulogo1995, Amaechi2018}.}

\ea%11
    \label{ex:amaechi:11} 
    Question formation strategies (Igbo)\\
    \ea[]{\label{ex:amaechi:11a}
    \gll    Ònyé hụ̀rụ̀ Àdá  n'-áhíá.\\
            who saw Ada P-market\\
    \glt    `Who saw Àdá at the market?' \hfill{\small wh-\textsc{sbj}, in-situ}}
    \ex[]{\label{ex:amaechi:11b}
    \gll    Òbí hụ̀rụ̀ ònyé n'-áhíá.\\
            Obi saw who P-market\\
    \glt    `Who did Òbí see at the market?' \hfill{\small wh-\textsc{do}, in-situ}}
    \ex[*]{\label{ex:amaechi:11c}
    \gll    Ònyé kà hụ̀rụ̀ Àdá n'-áhíá.\\
            who \textsc{foc} saw Ada P-market\\
    \glt    `Who saw Àdá at the market?' \hfill{\small wh-\textsc{sbj}, ex-situ}}
    \ex[]{\label{ex:amaechi:11d}
    \gll    Ònyé *({kà}) Òbí hụ̀rụ̀ n'-áhíá.\\
            who \textsc{foc} Obi saw P-market\\
    \glt    `Who did Òbí see at the market?' \hfill{\small wh-\textsc{do}, ex-situ}}
    \ex[]{\label{ex:amaechi:11e}
    \gll    Ọ̀ bụ̀ ònyé hụ̀rụ̀0 Àdá.\\
            it \textsc{cop} who saw Ada\\
    \glt    `Who saw Àdá?' \hfill{\small wh-\textsc{sbj}, cleft}}
    \ex[]{\label{ex:amaechi:11f}
    \gll    Ọ̀ bụ̀ ònyé *(kà) Òbí hụ̀rụ̀.\\
            it \textsc{cop} who \textsc{foc} Obi saw\\
    \glt    `Who did Òbí see?' \hfill{\small wh-\textsc{do}, cleft}}
    \z
\z

As in focus constructions, the ex-situ strategy is not available for (local) subjects since they can never co-occur with the morpheme \textit{kà}, see \REF{ex:amaechi:11a}. In contrast to (non-wh) focused subjects, however, the in-situ strategy is available for wh-subjects even without the addition of a focus-sensitive particle, see \REF{ex:amaechi:11c}. Note further that question formation via the ex-situ and the cleft strategy can also apply long-distance, just like in focus constructions (cf. \ref{ex:amaechi:7} and \ref{ex:amaechi:8}).

With this background on the formation of term focus and questions, we can discuss the nature of the morpheme \textit{k\`a} that occurs with non-subjects in the ex-situ and the cleft construction as well as with long-distance ex-situ / clefted subjects. We identify this morpheme as a focus marker (a claim also  made in \citealt{Osuagwu2015}) for the following reasons. It is clear that this marker is related to the expression of focus: first, it occurs in sentences that express focus, viz. focus constructions and questions, but not in other \=A-dependencies such as topicalization or relativization; and second, it is syncretic to the disjunction `or' in Igbo (cf. \citealt{Nwachukwu1987}), viz. it expresses alternatives. Furthermore, we can exclude that \textit{k\`a} is a focus-sensitive particle because it is obligatory in the contexts where it can occur (i.e. with non-subjects), it cannot associate with the focused XP at a distance (see \ref{ex:amaechi:12}, \textit{k\`a} must be left adjacent to the focused constituent), and unlike \textit{k\`a}, focus-sensitive elements like `only' precede their associate (see \ref{ex:amaechi:4}).

\ea[*]{\label{ex:amaechi:12}%12
    \gll    Ònyé Obi (kà) hụ̀rụ̀ (kà) {\longrule}  nà {\`m}gbèdè (kà) n'-áhíá (kà).\\
            who Obi (\textsc{foc}) saw (\textsc{foc}) {} P evening (\textsc{foc}) P-market (\textsc{foc})\\
    \glt    `Who did Òbí see in the evening at the market?'}
\z

We conclude that \textit{k\`a} is a focus marker. Moreover, we also have evidence that it does not realize an inherent focus feature of focused constituents, but rather an element in the left periphery of the clause:  It  cannot attach to in-situ focus / wh-elements, cf. \REF{ex:amaechi:3} and \REF{ex:amaechi:11b}, even though these also bear a focus feature (by assumption). We interpret these results such that \textit{k\`a} is the exponent of a functional head related to focus (= Foc$^{0}$ in the split CP-system, cf. \citealt{Rizzi1997} et seq.). This view is supported by the observation that \textit{k\`a} linearly follows the focused element (occupying SpecFoc) and attaches to whole phrases, not just to single words that are in focus: in \REF{ex:amaechi:13b} only \textit{áhíá} `market' is focused, but \textit{k\`a} cannot  attach to it; rather, it has to follow the pied-pied PP that includes the focused element.

\ea%13
    \label{ex:amaechi:13}
    \ea\label{ex:amaechi:13a}
        Òbí saw Àdá at the old farm.
    \ex\label{ex:amaechi:13b}
        %correction:\\
        \gll    {\`M}bà, [$_{PP}$ N'-áhíá (*kà) ochie ] \textbf{kà} {} Òbí hụ̀rụ̀ Àdá.\\
                no {} P-market (\textsc{foc}) old {} \textsc{foc} {} Obi saw Ada\\
        \glt    `No, Òbí saw Àdá at the {old} \textsc{market}.'% (and not at the old farm)
    \z
\z

\section{The subject / non-subject asymmetry in focus marking}\label{sec:amaechi:4}

Even though the extensive study of focus marking strategies has shown that languages differ remarkably in how exactly focus is encoded, some cross-linguistic  generalizations have  emerged. In a study of about 20 West African languages (Kwa, Gur, Chadic), \citet{FiedlerEtAl2010} find a marking asymmetry between focused subjects and non-subjects in all of the investigated languages;
SF = subject focus, NSF = non-subject focus:

\ea%14
    \label{ex:amaechi:14} 
    Marking asymmetry \citealt[242,~ex. (11)]{FiedlerEtAl2010}\\
    \ea\label{ex:amaechi:14a}
        NSF cannot or need not be marked syntactically.
        \ea\label{ex:amaechi:14ai}
            NSF is restricted to in-situ positions (Bole, Duwai, Bade, Ngamo)
        \ex\label{ex:amaechi:14aii} 
            NSF is not restricted to in-situ positions (Gur; Kwa; Hausa)
        \z
    \ex\label{ex:amaechi:14b}
        SF must be marked.
    \z
\z

In a nutshell, \citet{FiedlerEtAl2010} found that while focus marking for non-subjects is excluded or optional, subject focus must obligatorily be marked by morphological devices (focus markers) and/or syntactic means (displacement, clefting). \citet[171f]{SkopeteasFanselow2010} formulate this as an implicative relation: ``If a non-canonical structure occurs with focus on non-subjects, it is expected to occur with focus on subjects too.''. \citet{FiedlerEtAl2010} also propose an explanation for the observed asymmetry:  They assume that subjects in sentence-initial position are by default interpreted as topics; in order to overwrite this default interpretation in a focus context ``the focused subject will have to be realized in a non-canonical structure, for instance, by means of special morphological markers and/or syntactic reorganization'' (p.249).

Igbo is not included in \citegen{FiedlerEtAl2010} study of focus marking in West African languages, but it is very interesting to consider it in light of their findings since it provides us with new insights into the scope of the generalization. Given that Igbo also exhibits subject / non-subject asymmetries in focus marking, as outlined in the previous sections, it is a typical West African language with respect to \REF{ex:amaechi:14}. As for non-subject focus, Igbo also behaves like other West African languages: focus marking is optional here since focused elements can stay in-situ (no syntactic displacement, no orphological focus marking by \textit{k\`a}; alternatively, morphological and/or syntactic encoding is possible in the ex-situ and the cleft strategies. Subject focus marking does not entirely behave as expected according to \REF{ex:amaechi:14}. Focused (local) subjects actually do not have to be focus marked at all: they can never  co-occur with the focus marker \textit{k\`a}; moreover, at least (local) wh-subjects can occur in-situ without being syntactically displaced in any obvious way, but still the sentence is grammatical. In fact the ex-situ strategy (with displacement and the focus marker) is excluded for (local) subjects. Hence, focus marking on (wh-)subjects is \textit{not} obligatory in Igbo. The only context in which (local) subjects must be \todo{Single quotation markers are for linguistic meaning only}`marked' for focus is when they are not wh-pronouns and they occur in-situ: this is only possible if a  focus-sensitive particle is added, see \REF{ex:amaechi:4}. In any case, (local) focused subjects are incompatible with focus movement (ex-situ strategy, also involved  in cleft formation) and morphological focus marking. The generalization for Igbo seems to be a bit more abstract: focus on subjects needs to be encoded morphosyntactically \textit{in some way} to indicate the difference to an information-structurally neutral affirmative sentence as in \ref{ex:amaechi:1} where the subject is interpreted as the (default) topic.  ``In some way'' includes not only the regular focus marking strategies (not available for local subjects) but also the occurrence of focus-sensitive particles and wh-morphology (the form of the wh-subject pronoun differs from the non-wh person pronouns and the interrogative sentence with a wh-subject thus differs and can be distinguished from an affirmative sentence). If wh-morphology also counts as a focus marking device, we can explain why wh-subjects can occur in the in-situ strategy without further focus marking devices, while focus subjects in the focus construction need to be accompanied by a focus particle to be able to occur in this construction: without the focus-sensitive particle attached to the subject, the sentence would be morphosyntactically indistinguishable from an affirmative sentence as in \REF{ex:amaechi:1}. Thus, \citegen{FiedlerEtAl2010} generalization also holds for Igbo if focus marking comprises more than syntactic displacement and the use of focus markers.\footnote{\citet{Aboh2007} offers a different view on the `exceptionality' of wh-subjects: wh-elements are not necessarily inherently focused. The ex-situ ones are in focus, while the `in-situ' ones (moved to a low focus position) are not focused at all and hence do not receive focus marking.}

\section{On the source of the marking asymmetry in Igbo}\label{sec:amaechi:5}

In the previous section we came to the conclusion that focused subjects in Igbo can occur in the in-situ strategy without any focus marking  (at least there is no regular encoding by syntactic displacement or attachment of a focus marker), even though there should be a pressure to encode especially subjects according to the \citeauthor{FiedlerEtAl2010} generalization. In this section we will briefly outline what the reason for the absence of focus marking with (local) subjects is. For more details, derivations and supporting empirical arguments, the reader is referred to \citet{AmaechiGeorgi2019}, where we develop an optimality-theoretic analysis of the marking asymmetry for questions in Igbo. We have argued above that the focus marker \textit{k\`a} realizes the left-peripheral head Foc$^{0}$. We content that in the ex-situ strategy and in clefts the focused non-subject constituent undergoes syntactic movement to SpecFoc; that the observed displacement involves movement rather than base-generation is supported by the fact that the dependency exhibits the hallmarks of movement (island-sensitivity, reconstruction).

\ea%15
    \label{ex:amaechi:15}
    $[_{FocP}$ XP$_{foc}$ [$_{Foc'}$ Foc$^{0}$ [$_{TP}$ ... [$_{vP}$ ... t$_{XP}$ ]]]]
\z

We can derive the absence of the focus marker \textit{k\`a} in the in-situ strategy by the following assumption: The head Foc$^{0}$ is morphologically realized as \textit{k\`a} only if an overt (phonologically realized) XP occupies SpecFoc, otherwise Foc$^{0}$ remains silent (= contextual allomorphy). Since nothing moves (overtly) to SpecFoc in the in-situ construction, Foc$^{0}$ is not phonologically realized. In the ex-situ (and cleft) strategy where focused non-subjects move to SpecFoc, they surface at the left periphery of the clause and are accompanied by \textit{k\`a} (we will not say more about the structure of clefts here). Since movement for non-subjects is optional, we get optionality in the ex-situ/cleft vs. in-situ strategy. The question that remains is why local focused subjects cannot co-occur with  \textit{k\`a}, not even optionally. We suggest that this is because they have to stay in the canonical subject position SpecT (see \citealt{AmaechiGeorgi2019} for empirical evidence); i.e., unlike focused non-subjects, they cannot undergo movement to the minimal SpecFoc position. And since no XP occupies SpecFoc, the head Foc$^{0}$ has to remain silent. One piece of evidence for this claim is the observation that subject movement in Igbo triggers a tonal reflex on the verb, but constructions with a preverbal focused subject do not exhibit this tonal reflex. That subjects cannot undergo local movement has been claimed for other languages as well (see among many others \citealt{Chomsky1986, Agbayani1997} on the Vacuous Movement Hypothesis in English). A prominent (but not the only) account for this immobility of subjects is that the movement from SpecT (the canonical subject position in Igbo) to the local SpecFoc position would be too short, which is excluded by an anti-locality constraint (see \citealt{Abels2003, Grohmann2003, Erlewine2016} and references cited there for this concept). Long-distance movement of the subject (as well as clause-bound movement of non-subjects) covers a greater distance and does not qualify as too short by the definition of anti-locality. Non-subjects and long-distance moved subjects can thus occur in the ex-situ construction (where they trigger the realization of Foc$^{0}$) as \textit{k\`a} without any problems.

\section{Conclusions}\label{sec:amaechi:6}

We have described the focus marking strategies in Igbo and the pragmatic contexts in which they are used. Igbo exhibits a subject / non-subject split in focus marking; however, this split partially challenges the generalization by \citet{FiedlerEtAl2010} on other West African languages because local focused subjects in Igbo cannot be marked by the regular focus marking devices. We provide an analysis according to which the occurrence of the focus marker \textit{k\`a} is not solely regulated by pragmatic principles, but rather by an interplay of its high syntactic position, morphological realization rules and a ban on too local subject movement.

\section*{Abbreviations}
\begin{tabularx}{.45\textwidth}{lQ}
 \textsc{adj}  &  adjunct\\
 \textsc{do}  &  direct object\\
 \end{tabularx}
 \begin{tabularx}{.45\textwidth}{lQ}
 \textsc{sbj}  &  subject\\
 \textsc{v}  &  (main) verb\\
 \end{tabularx}
\medskip\\
 glosses:\todo{Can be merged}
\medskip\\
 \begin{tabularx}{.45\textwidth}{lQ}
  \textsc{cop}  &  copula\\
 \textsc{dir}  &  directional\\
 \textsc{foc}  &  focus marker\\
 \textsc{ipfv}  &  imperfective\\
 \end{tabularx}
 \begin{tabularx}{.45\textwidth}{lQ}
 \textsc{nmlz}  &  nominalizer\\
 \textsc{p}  &  preposition\\
  \textsc{sg}  &  singular\\
  \\
\end{tabularx}

\todo[inline]{\textsc{impf} changed to \textsc{ipfv} and \textsc{su} to \textsc{sbj} for consistency with Leipzig Glossing Rules.}

\section*{Acknowledgements}
We would like to thank Jeremiah Nwankwegu, Gerald Nweya, Basil Ovu, Chioma Eweama and Francis Umunnakwe for verification of the  data. For valuable comments we are grateful to the audiences at ``Quirks on subject extraction'' (Singapore, August 2017), ACAL 49 (Michigan, March 2018), ``Referential and relational approaches to syntactic asymmetries'' (Stuttgart, March 2018), the syntax colloquium at the University of Frankfurt, and especially to Malte Zimmermann and Katharina Hartmann. This research is funded by the Deutsche Forschungsgemeinschaft (DFG), Collaborative Research Centre SFB 1287, Project C05 (Georgi).

\sloppy
 \printbibliography[heading=subbibliography,notkeyword=this]

\end{document}
