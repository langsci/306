\documentclass[output=paper,colorlinks,citecolor=brown]{langscibook} 
\author{Deo Ngonyani\affiliation{Michigan State University}\and Ann Biersteker\affiliation{Michigan State University}\and Angelina Nduku Kioko\affiliation{US International University-Africa}\lastand  Josephat Rugemalira\affiliation{Tumaini University Dar es Salaam College}}
\title{Proto-Bantu reflexes in Dhaisu}  


\abstract{This paper is a study of Proto-Bantu reflexes in Dhaisu, a highly endangered language also known as Dhaiso, Segeju, Daisu and Kidhaisu (dhs, E56). Dhaisu is  spoken in the East Usambara Mountains in northeastern Tanzania, but its closest relative is Kamba (E55). Seven vowels are reported in this study as has been in other studies /i, ɪ, ɛ, a, ɔ, ʊ, u/. However, no contrast can be established between \textit{ɪ} and \textit{ɛ}, or between u and ʊ. The data shows that Dhaisu vowel system is changing to a  5-vowel system \textit{*i̧, *i, *e, *a, *o, *u, *u}̧  > /i, ɛ, a, ɔ, u/. The most remarkable feature of this change is that unlike other Bantu languages, which merge the mid-high to high vowels, Dhaisu is merging the mid-high vowels to mid-low. The innovation is demonstrated in (a) numerous lexical items in which PB \textit{*i} has become \textit{ɛ}, and PB \textit{*u} has become \textit{ɔ}; (b)  several nominal prefixes that are constructed in PB having mid-high vowels now have mid-low vowels, and (c) the applicative suffix whose PB form was \textit{*id}  is now \textit{-ɛr} in Dhaisu. The fact that the change does not seem to have affected all nominal prefixes with mid-high vowels, and has affected not all verbal derivation with mid-high vowel suggests  an  ongoing  transition. Reflexes of consonants are presented to show that they are not a result of spirantization. }

\IfFileExists{../localcommands.tex}{
  \addbibresource{localbibliography.bib}
  \usepackage{langsci-optional,langsci-branding}
\usepackage{langsci-gb4e}
% \usepackage{langsci-textipa}
% \usepackage{langsci-glyphs}
\usepackage[linguistics]{forest}
\usepackage{tabto}
\usepackage{multirow}
\usepackage{bbding}

\usepackage[normalem]{ulem}

\usepackage{tikz-qtree}

\usepackage{enumitem}

\usepackage{multicol}
\usepackage{stmaryrd} %double brackets

\makeatletter
\let\pgfmathModX=\pgfmathMod@
\usepackage{pgfplots,pgfplotstable}%
\let\pgfmathMod@=\pgfmathModX
\makeatother
\usepgfplotslibrary{colorbrewer}
\usetikzlibrary{fit}

\usepackage{jambox}
\usepackage{tikz-qtree-compat}
\usetikzlibrary{arrows, arrows.meta}
\usepackage{longtable}
\usepackage{subcaption}

  \makeatletter
\let\thetitle\@title
\let\theauthor\@author
\makeatother

\newcommand{\togglepaper}[1][0]{
%   \bibliography{../localbibliography}
  \papernote{\scriptsize\normalfont
    \theauthor.
    \thetitle.
    To appear in:
    Change Volume Editor \& in localcommands.tex
    Change volume title in localcommands.tex
    Berlin: Language Science Press. [preliminary page numbering]
  }
  \pagenumbering{roman}
  \setcounter{chapter}{#1}
  \addtocounter{chapter}{-1}
}

\newcommand{\bari}{\ipabar{\i}{.5ex}{1.1}{}{}}
\newcommand{\notipa}[1]{\textnormal{#1}}

\newcommand{\agre}{\textsc{agr}-\ol{eene}}

\renewcommand{\emph}[1]{\textit{#1}} % resetting a setting from ling-macros-modified (I think?)

% forest settings to make compact but (mostly) straight-spined trees:
\forestset{
fairly nice empty nodes/.style={
            delay={where content={}{shape=coordinate,for parent={
                  for children={anchor=north}}}{}}
, angled/.style={content/.expanded={$<$\forestov{content}$>$}}
}}

\forestset{sn edges/.style={for tree={parent anchor=south, child anchor=north}}}

\newcommand{\bex}{\begin{exe}}
\newcommand{\fex}{\end{exe}}

\newcommand{\bxl}{\begin{exe}}
\newcommand{\fxl}{\end{exe}}

\newcommand{\ix}[1]{\textsubscript{#1}}
\newcommand{\alert}[1]{\textbf{#1}}
\newcommand{\ol}[1]{\textit{#1}}


			\usetikzlibrary{shapes,arrows,positioning,decorations,decorations.pathmorphing,intersections}
\forestset{
nice empty nodes/.style={
    for tree={calign=fixed edge angles},
    delay={where content={}{shape=coordinate,for siblings={anchor=north}}{}}
},
}

\definecolor{dark-gray}{gray}{0.3}

%\usepackage{dingbat,pifont}


%%%%%%%%%%%%For arrows%%%%%%%%%%%%%

\newcommand\Tikzmark[2]{%
  \tikz[remember picture]\node[inner sep=0pt,outer sep=0pt] (#1) {#2};%
}
\NewDocumentCommand\DrawArrow{O{}mmmmO{3}}{
\tikz[remember picture,overlay]
  \draw[->,line width=0.8pt,shorten >= 2pt,shorten <= 2pt,#1]
    (#2) -- ++(0,-#6\ht\strutbox) coordinate (aux) -- node[#4] {#5} (#3|-aux) -- (#3);
}
\NewDocumentCommand\DrawDotted{O{}mmmmO{3}}{
\tikz[remember picture,overlay]
  \draw[->,line width=0.9pt,dotted,shorten >= 2pt,shorten <= 2pt,#1]
    (#2) -- ++(0,-#6\ht\strutbox) coordinate (aux) -- node[#4] {#5} (#3|-aux) -- (#3);
}
\NewDocumentCommand\DrawLine{O{}mmmmO{3}}{
\tikz[remember picture,overlay]
  \draw[line width=0.8pt,shorten >= 2pt,shorten <= 2pt,#1]
    (#2) -- ++(0,-#6\ht\strutbox) coordinate (aux) -- node[#4] {#5} (#3|-aux) -- (#3);
}
%%%%%%%%%%%%%%%%%%%%%%%%%%%%%%%%%%%%%


\newcommand{\baru}{ʉ}
\newcommand{\baruH}{\'\baru}
\newcommand{\baruL}{\`\baru}

\newcommand{\ep}{ε}
\newcommand{\epH}{\'\ep}
\newcommand{\epL}{\`\ep}

\newcommand{\schwa}{ə}
\newcommand{\schwaH}{\'ə}
\newcommand{\schwaL}{\`ə}

\newcommand{\oo}{ɔ}
\newcommand{\ooH}{\'\oo}
\newcommand{\ooL}{\`\oo}

\newcommand{\ds}{\textsuperscript{
	\hspace*{-2pt}\begin{tikzpicture}
		\draw[-{>[scale=0.5]}] (0,0.4) --(0,0.25);
	\end{tikzpicture}}}

\newcommand{\ch}{t͡ʃ}
\newcommand{\dz}{d͡ʒ}

\newcommand{\tgl}{ʔ}

%shortcuts for the complementizers
\newcommand{\mbuL}{mb\baruL}
\newcommand{\mbuHL}{mb\baruH\baruL}
\newcommand{\mbuLH}{mb\baruL\baruH}
\newcommand{\la}{lá}
\newcommand{\nda}{ndà}

\newcommand{\tsc}[1]{\textsc{#1}}
\renewcommand{\textscb}{ʙ}
\newcommand{\ipa}[1]{#1} %disable IPA

\newcommand{\SM}[1]{#1}

\DeclareNewSectionCommand
  [
    counterwithin = chapter,
    afterskip = 2.3ex plus .2ex,
    beforeskip = -3.5ex plus -1ex minus -.2ex,
    indent = 0pt,
    font = \usekomafont{section},
    level = 1,
    tocindent = 1.5em,
    toclevel = 1,
    tocnumwidth = 2.3em,
    tocstyle = section,
    style = section
  ]
  {appendixsection}

\renewcommand*\theappendixsection{\Alph{appendixsection}}
\renewcommand*{\appendixsectionformat}
              {\appendixname~\theappendixsection\autodot\enskip}
\renewcommand*{\appendixsectionmarkformat}
              {\appendixname~\theappendixsection\autodot\enskip}

\renewcommand{\lsChapterFooterSize}{\footnotesize}
 
  %% hyphenation points for line breaks
%% Normally, automatic hyphenation in LaTeX is very good
%% If a word is mis-hyphenated, add it to this file
%%
%% add information to TeX file before \begin{document} with:
%% %% hyphenation points for line breaks
%% Normally, automatic hyphenation in LaTeX is very good
%% If a word is mis-hyphenated, add it to this file
%%
%% add information to TeX file before \begin{document} with:
%% %% hyphenation points for line breaks
%% Normally, automatic hyphenation in LaTeX is very good
%% If a word is mis-hyphenated, add it to this file
%%
%% add information to TeX file before \begin{document} with:
%% \include{localhyphenation}
\hyphenation{
affri-ca-te
affri-ca-tes 
Līk-pāk-páln
pro-sod-ic
phe-nom-e-non
Chi-che-wa
Lu-bu-ku-su
Ngbu-gu
Boyel-dieu
Mat-chi
pho-neme
Mil-em-be
Nyan-chera
Mc-Pher-son
Tsoo-tso
Sku-pin
dis-tin-guishes
con-ser-va-tion
Me-dum-ba
}

\hyphenation{
affri-ca-te
affri-ca-tes 
Līk-pāk-páln
pro-sod-ic
phe-nom-e-non
Chi-che-wa
Lu-bu-ku-su
Ngbu-gu
Boyel-dieu
Mat-chi
pho-neme
Mil-em-be
Nyan-chera
Mc-Pher-son
Tsoo-tso
Sku-pin
dis-tin-guishes
con-ser-va-tion
Me-dum-ba
}

\hyphenation{
affri-ca-te
affri-ca-tes 
Līk-pāk-páln
pro-sod-ic
phe-nom-e-non
Chi-che-wa
Lu-bu-ku-su
Ngbu-gu
Boyel-dieu
Mat-chi
pho-neme
Mil-em-be
Nyan-chera
Mc-Pher-son
Tsoo-tso
Sku-pin
dis-tin-guishes
con-ser-va-tion
Me-dum-ba
}
 
  \togglepaper[1]%%chapternumber
}{}

\begin{document}
\maketitle

\section{Introduction}\label{sec:ngonyani:1}

This paper presents the sound system of Dhaisu (iso 632-2 code \textit{dhs}) and reflexes of Proto-Bantu sounds. Dhaisu is a Bantu language spoken on East Usambara in northeastern Tanzania in Tanga region. \cite{SimonsFennig2017} estimate there are about 5,000 speakers. \cite[17]{Nurse2000} reported between 8,000 and 10,000 speakers. \cite[2]{RugemaliraEtAl2019}, however, estimate there are about 11,000 speakers. The language is also known as Dhaiso, Dhaisu, Daisu, Daiso, Kidaisu and is referred to by outsiders as Segeju. However, Segeju is actually a different language spoken on the coast \citep{Nurse1982}. Guthrie codes Dhaisu E56 \citep{Guthrie1967, Maho2009} in a group that includes Gikuyu (E51), Embu (E52), Meru (E53), Tharaka (E54), Cuka (E541), and Kamba (E55), a group commonly refered to as Thagicu. Dhaisu's closest relative in the Thagicu group is Kamba \citep{Nurse1982, Nurse2000, Nurse1999}. Dhaisu shares with other Thagicu languages features that include an inherited common lexicon, retention of 7 vowels, absence of spirantization in high vowel environments, and fronting of PB palatal \textit{*c} and \textit{*j} \citep{Nurse1982}. The term spirantization in Bantu languages is used to denote the process of forming fricatives from Proto-Bantu stops that appear before high vowels  \citep{Nurse1982, Schadeberg1995}. 

The present study is based on data collected in Dar es Salaam from six  speakers of Dhaisu from Bwiti in July-August, 2017 and in Tanga in July-August 2018. The dictionary that was subsequently published as \cite{RugemaliraEtAl2019} is the main source for this study. The sounds in Dhaisu are compared to the Proto-Bantu forms from Tervuren's Bantu Lexical Reconstruction 3 (BLR3) \citep{BastinEtAl2002}. The paper demonstrates that the Dhaisu vowel system is in a transition from a 7 vowel system to a 5 vowel system. The paper presents more data to confirm this and show that the change is unusual because it is the mid-vowels that are merging rather than an upward merger to high vowels that is observed in other Bantu languages. The lowering of mid-high vowels was first noted by \cite{Nurse2000}. The uneven distribution of the vowel change and the fact that the mid-high vowels are also heard, suggests that the change is ongoing. Furthermore, the data demonstrate that the vowel merger is not associated with spirantization.

These findings are presented in the following 5 sections. \sectref{sec:ngonyani:2} presents the phoneme inventory, while evidence of vowel merger is in \sectref{sec:ngonyani:3} in form of correspondences in Dhaisu, Kamba, Swahili, Sambaa, and Digo. \sectref{sec:ngonyani:4} illustrates the vowel change as attested in nominal affixes and verbal suffixes. \sectref{sec:ngonyani:5} contains examples of consonant reflexes which reveal that previous changes in the consonant were not associated with spirantization. Concluding remarks and directions for further studies appear in \sectref{sec:ngonyani:6}.

\section{Dhaisu phonemes}\label{sec:ngonyani:2}

This section presents the vowel and consonant phonemes of Dhaisu and sets the stage for comparing them to Proto-Bantu forms. The Dhaisu sound system consists of 7 vowels, reflecting Proto-Bantu vowel system. There are three front vowels, three back vowels and a middle open vowel. These appear in four levels 
as shown in \tabref{tab:ngonyani:1}.

\todo[inline]{All the tabular environments formatted as examples have been converted into tables, while tabbing environments have been kept as examples. It might be a good idea to convert them as well.}

\begin{table}
	\caption{Dhaisu vowel inventory \citep[4]{RugemaliraEtAl2019}}
    \label{tab:ngonyani:1}
    \begin{tabular}{ccc}
        i   &     & u\\
        ɪ   &     & ʊ\\
        ɛ   &     & ɔ\\
            & a   & \\
    \end{tabular}
\end{table}
\todo[inline]{Consider changing to figure.}

These seven vowels are recognized by \cite{Nurse2000}  and \cite{RugemaliraEtAl2019}. Furthermore, \cite[20]{Nurse2000} reports on the difficulties researchers faced in identifying distinctive 7 vowels. The current study faced the same problem of being unable to elicit minimal pairs to contrast mid-high and mid-low vowels. Clear pairs were elicited that contrasted the high vowels /i, u/ and mid vowels /ɛ, ɔ/.

\ea%1
    \label{ex:ngonyani:1}
    \begin{tabbing}
        \= rita \quad\= `devine' \quad\= \quad\=  rɛta \quad\= `bring'\\
        \> ita \> `war' \> \> ɛta \> `call'\\
        \> ngurɔ \> `dog' \> \> ngɔrɔ \> `heart' \\
        \> suma \> `trade' \> \> sɔma \> `read'
\end{tabbing}
\z

The sounds /i/ and /ɛ/, are distinct as are /u/ and /ɔ/, as the minimal pairs  in \REF{ex:ngonyani:1} demonstrate. However, the contrast of the mid-vowels  \textit{ɪ} vs \textit{ɛ} and \textit{ʊ} vs \textit{ɔ} is not  easy to establish. The  same words may be pronounced  in two different ways by the same speaker. Whenever  the researchers sought  to get a clearer  repetition,  the speakers would  pronounce  the mid-low vowels. Here are some examples of words pronounced with alternative mid vowels. The  speakers also insisted  on writing  with  the mid-low vowels. The writing  is reflected  in the orthography adopted in \cite{RugemaliraEtAl2019}.

\ea%2
    \label{ex:ngonyani:2}
    \begin{tabbing}
        \= a. \quad\= malɔwa \quad\= malʊwa\quad\= `flowers'\\
        \> \> rɔchi \> rʊchi  \>  `river' \\
        \> \> uvɔngɔ \> uvʊngʊ \> `lies, gossip'\\
        \> b. \>  tumbɛ \> tumbɪ \> `egg' \\
        \> \> kumɛra \> kumɪra \> `to swallow'\\
        \> \> kulɛa \> kulɪa \> `to contradict'
    \end{tabbing}
\z

This is likely why \cite{Nurse2000} notes that there are the 7 as well as 5 vowel transcriptions in previous documentation. Vowel length is not contrastive.

The consonant system consists of 25 consonants as presented in \tabref{tab:ngonyani:2}. 

\begin{table}
	\caption {Consonant inventory \citep[5]{RugemaliraEtAl2019} }
 	\label{tab:ngonyani:2}
    \begin{tabular}{llllllll}
    \lsptoprule 
        & Bilabial & Labio-  & Alveolar & Palatal & Velar & Glottal\\
        & & dental\\
    \midrule
        Stops & p & & t & c & k\\
        & b &  &   d & ɟ & g\\
        Fricatives & & f &   s & ʃ & & h\\
        & & v &   z\\
        Nasals & m & &    n & ɲ & ŋ\\
        Prenasal. stops & mb & &  nd & ɲɟ & ŋg \\
        Liquid & & &    r\\
        Glides & w & &    & j\\
    \lspbottomrule
    \end{tabular}
\end{table}

There are 8 stops, paired into voiced/voiceless in four places of articulation, namely, bilabial, alveolar, palatal and velar. There are two pairs of fricatives \textit{f/v} and \textit{s/z}. In some cases, /β/ is heard as a possible variant of /v/. In addition, there is a palatal fricative /ʃ/ and glottal fricative/h/. The four nasal consonants are bilabial /m/, alveolar /n/, palatal /ɲ/ and velar/ŋ/. These same places are for prenasalized stops /mb, nd, ɲɟ, ŋg/. Both /r/ and /1/ are heard, with /r/ more frequently heard. The distribution of the two non-contrastive liquids is not clear yet.
 
Some sounds appear mixed possibly due to the influence of the Swahili alphabet as speakers insisted on how some sounds should be written even though Swahili and Dhaisu have significant differences. Dhaisu speakers also speak and write Swahili. The idea of Swahili influence on Dhaisu writing, therefore, is not far fetched. \cite{Nurse2000} also notes such mixtures. Some of the vowel features may be influenced by tones, which have not been marked in this study.


\section{Vowel merger}\label{sec:ngonyani:3}

As noted earlier in the foregoing section and as discussed in \cite{Nurse2000}, one can perceive the seven vowels in Dhaisu. But finding contrastive pairs for the intermediate vowels is an elusive endeavour. We conclude that this is due to the current state of transition from a seven vowel system to a five vowel system. In this section, we present data that show that Dhaisu is exhibiting the loss of distinction between /ɪ/ - /ɛ/ and /ʊ/ - /ɔ/. This is very interesting in the light of Schedeberg's observation that the merger of intermediate high vowels i/e and \textit{*u/*o} is unknown \citep[74]{Schadeberg1995}. The baseforms are the Proto-Bantu vowels  shown in \tabref{tab:ngonyani:3}.

\begin{table}
	\caption{Proto-Bantu vowels \citep[82]{Meeussen1967}}
    \label{tab:ngonyani:3}
        \begin{tabular}{cccc}
        i̧ &    &   & u̧\\
        i &    &   & u\\
        e &    & o &\\
          & a  &   &\\
        \end{tabular}
\end{table}

The more widely attested 7 > 5 vowel change merges the so called superhigh vowels \textit{*i̧} and \textit{*u̧} with high vowels \textit{*i} and \textit{*u}, respectively. Data from Dhaisu reveal that intermediate \textit{*i} and \textit{*e }merge to /ɛ/ and \textit{*u} and \textit{*o} merge to /ɔ/. This is summarized in \tabref{tab:ngonyani:4}. The more usual merger, here exemplified by Swahili in \tabref{tab:ngonyani:5}. Swahili merges up, while the unusual merger in Dhaisu merges down.

\begin{table}
	\caption{Dhaisu vowel merger}
	\label{tab:ngonyani:4}
    \begin{tabular}{lcr}
        i               &   & u\\
        ɪ $\downarrow$  &   & $\downarrow$ ʊ\\
        ɛ               &   & ɔ\\
                        & a &\\
    \end{tabular}
\end{table}	

\begin{table}
	\caption{Swahili merger}
    \label{tab:ngonyani:5}
    \begin{tabular}{lcr}
        i               &   & u\\
        ɪ $\uparrow$    &   & $\uparrow$  ʊ\\
        ɛ               &   & ɔ\\
                        & a &\\
    \end{tabular}
\end{table}
\todo[inline]{Consider changing to figure (Table 3--5).}

This downward merger is most likely not due to loans since there are no languages in the region that have had such a merger. 
We compare PB reflexes of Dhaisu and languages that may have had great influence on it. The languages that have most influence on Dhaisu are Kamba (E55), its closest relative, Swahili (G42a), the national language that all Dhaisu speakers are fluent in, Sambaa (G23), a neighboring language, and Digo (E73) another neighboring language. The following sets of examples illustrate that Dhaisu is unlike Kamba's seven-vowel system and Thagicu languages of Kenya, and unlike five-vowel systems like Swahili, Sambaa, and Digo. Data for the comparison are from BLS for Proto-Bantu forms  \citep{BastinEtAl2002}, \cite{NurseHinnebusch1993}  and \cite{TUKI1996} for Swahili, \cite{NursePhilippson1975} for  Sambaa and Kamba, and \cite{MwalonyaEtAl2004} for Digo.

\begin{table}
    \caption{Proto-Bantu *i̧}
    \label{tab:ngonyani:6}
    \begin{tabular}{@{}p{1cm} p{1.4cm} p{1.3cm} p{1.3cm} p{1.3cm} p{1.3cm} p{1.3cm}@{}}
    \lsptoprule
        PB & & Kamba & Dhaisu & Swahili & Sambaa & Digo \\
        \midrule
        *kí̧ŋgó & `neck' & ŋgiŋgɔ & ŋgiŋgɔ  & ʃiŋgɔ & ʃiŋgɔ & siŋgɔ\\
        *pí̧c & `hide' & viθa & vida & fica & fiʃa & fitsa\\
        *dì̧tò & `heavy' & ŋgitɔ & ritɔ & zito &  & \\
        *pí̧gò & `kidney' & mbiɔ & mvijɔ & figɔ & fiɣɔ & figɔ \\
    \lspbottomrule\
    \end{tabular}
\end{table}
\begin{table}
    \caption{Proto-Bantu *i}
    \label{tab:ngonyani:7}
    \begin{tabular}{@{}p{1cm} p{1.4cm} p{1.3cm} p{1.3cm} p{1.3cm} p{1.3cm} p{1.3cm} @{}}
    \lsptoprule
        PB & & Kamba & Dhaisu & Swahili & Sambaa & Digo \\
        \midrule
        *dìm & `cultivate' & ɪma & rɛma  & lima & ima & rima\\ 
        *díd & `cry' &  & rɛra & lia & iya & rira\\
        *kídà & `tail' & & mkɛra & mkia & mkia & \\
        *bícì & `unripe' & bɪθɪ & & bichi & iʃi & itsi \\
        *dímì̧     & `tongue' & uɪmɪ & lulɛmɛ & ulimi & ulimi & \\
    \lspbottomrule
    \end{tabular}
\end{table}

\begin{table}
    \caption{Proto-Bantu *e}
    \label{tab:ngonyani:8}
    \begin{tabular}{@{}p{1cm} p{1.3cm} p{1.3cm} p{1.3cm} p{1.3cm} p{1.3cm} p{1.3cm}@{}}
    \lsptoprule
        PB & & Kamba & Dhaisu & Swahili & Sambaa & Digo \\
        \midrule
        *cèk & `laugh' & θɛka & dɛka  & cɛka &  ʃɛka & tsɛka \\
        *tém & `cut' & tɛma & tɛma & tɛma & & tɛma\\
        *pèep & `blow' &  & mvɛɛvo & upɛpɔ & mpɛhɔ & upɛpɔ \\
        *dèdù̧ & `beard' &  & ɛru & ndɛvu & ndɛzu & \\
    \lspbottomrule
    \end{tabular}
\end{table}

\begin{table}
    \caption{Proto-Bantu *a}
    \label{tab:ngonyani:9}
    \begin{tabular}{@{}p{1cm} p{1.3cm} p{1.3cm} p{1.3cm} p{1.3cm} p{1.3cm} p{1.3cm}@{}}
    \lsptoprule
        PB & & Kamba & Dhaisu & Swahili & Sambaa & Digo \\
        \midrule
        *ɲàmà & `meat' & ɲama & ɲama   &ɲama  & ɲama  & ɲama  \\
        *bàdù̧  & `rib' & wau & rwaru  & ubavu & abazu & ubavu\\
        *jánà & `child' & mwana & mwana & mwana & ɲwana &  mwana\\
        *jókà & `snake' & nzɔka & sɔka & ɲɔka & ɲɔka &  ɲɔka\\
    \lspbottomrule
    \end{tabular}
\end{table}

\begin{table}
    \caption{Proto-Bantu *o}
    \label{tab:ngonyani:10}
    \begin{tabular}{@{}p{1cm} p{1.3cm} p{1.3cm} p{1.3cm} p{1.3cm} p{1.3cm} p{1.3cm}@{}}
    \lsptoprule
        PB & & Kamba & Dhaisu & Swahili & Sambaa & Digo \\
        \midrule
        *gòngò & `back' & muɔŋgɔ & mɔŋgɔ  & mgɔŋgɔ & mɣɔŋgɔ  & mgɔŋgɔ\\
        *jókà & `snake' & nzɔka & sɔka & ɲɔka & ɲɔka & ɲɔka \\
        *dóot & `dream' & ɔta & rɔta & ɔta & & \\
        *cónì & `shame' & nθɔni & ndɔni & sɔni & ʃɔni &  \\
    \lspbottomrule
    \end{tabular}
\end{table}

\begin{table}
    \caption{Proto-Bantu *u}
    \label{tab:ngonyani:11}
    \begin{tabular}{@{}p{1cm} p{1.3cm} p{1.3cm} p{1.3cm} p{1.3cm} p{1.3cm} p{1.3cm}@{}}
    \lsptoprule
        PB & & Kamba & Dhaisu & Swahili & Sambaa & Digo \\
        \midrule
        *búmb & `mould' & ʊmba & ɔmba  & umba &  umba & umba\\
        *gùdù  & `leg' & kʊʊ & kɔrɔ & mguu & & gulu\\
        *gùdùbè & `pig' & ŋgʊlʊɛ & ŋgɔrɔwɛ & ŋguruwɛ & ŋguuwe & nguluwɛ\\
        *júm & `dry' & ʊma & ɔma &  &  & uma\\
    \lspbottomrule
    \end{tabular}
\end{table}

\begin{table}
    \caption{Proto-Bantu *u̧}
    \label{tab:ngonyani:12}
    \begin{tabular}{@{}p{1cm} p{1.3cm} p{1.3cm} p{1.3cm} p{1.3cm} p{1.3cm} p{1.3cm}@{}}
    \lsptoprule
        PB & & Kamba & Dhaisu & Swahili & Sambaa & Digo \\
        \midrule
        *bàdù̧  & `rib' & wau & rwaru  & ubavu & abazu & ubavu \\
        *bù̧ŋk & `stink' &  & ɲuŋga & nuka & nuŋka & ɲusa\\
        *kútà & `oil' & mauta & maguuta & mafuta & mavuta & mafuha\\
        *dèdù̧ & `beard' &  & ɛru & ndɛvu & ndɛzu & \\
    \lspbottomrule
    \end{tabular}
\end{table}

The data show the critical difference between Dhaisu and other 5-vowel systems such as Swahili, Sambaa, and Digo. These attest to \textit{*i} > \textit{ɛ} (\tabref{tab:ngonyani:7}) and  \textit{*u} >  \textit{ɔ} (\tabref{tab:ngonyani:11}). The corresponding innovations in Swahili and Sambaa are \textit{*i > i} (\tabref{tab:ngonyani:7}) and \textit{*u > u} (\tabref{tab:ngonyani:11}).

The examples demonstrate three sets of reflexes of PB vowels:

\begin{enumerate}
    \item[a)] Kamba representing seven vowel systems /i, ɪ, ɛ, a, ɔ, ʊ, u/ retained from Proto-Bantu; 
    \item[b)] Swahili, Sambaa, and Digo representing languages with 5 vowel systems that resulted from the mid-high vowels merging with high vowels,  \textit{ *ɪ/*ʊ >  i/u}.
    \item[c)] Dhaisu with a 5 vowel system in which the mid-low vowels have merged with mid-low vowels, \textit{*ɪ/*ʊ > ɛ/ɔ}.
\end{enumerate}

\noindent These patterns are represented in \tabref{tab:ngonyani:13}.

\begin{table}
	\caption{Vowel correspondences} 
    \label{tab:ngonyani:13}
    \begin{tabular}{llll}
    \lsptoprule
        Proto-Bantu &	7 Vowels	& 5 Vowels & 5 vowels\\
         & Kamba & Swahili & Dhaisu \\
    \midrule
        *i̧ & i & i & i\\
        *i & ɪ & i & ɛ \\
        *e & ɛ & ɛ & ɛ \\
        *a & a & a & a \\
        *o & ɔ & ɔ & ɔ \\
        *u & ʊ & u & ɔ\\
        *u̧ & u & u & u \\
    \lspbottomrule
    \end{tabular}
\end{table}
\cite[20]{Nurse2000} suggests that more than half of Dhaisu vocabulary is from neighboring languages. A possible source of this merger could be loans from those languages. As the  \textit{*ɪ/*ʊ > ɛ/ɔ} is not attested in any other language which may have greatly influenced Dhaisu, this change must be considered unique for this language. The lowering of PB mid–high vowels  PB \textit{*i > ɛ} and PB \textit{*u > ɔ} appears to be in progress and incomplete. This is because the lowering of mid-high vowels appears not to have taken place in some words and some environments. 

This section has presented lexical data that reveals a unique 7 > 5 vowel change in Dhaisu. The merger is evident in other environments too wo which we now turn. 

\section{Evidence from affixes}\label{sec:ngonyani:4}

Another body of data that attests to the merger is affixes. In this section, we highlight (a) nominal affixes and (b) the applicative suffix.

\subsection{Nominal affixes}\label{sec:ngonyani:4.1}

Like other Bantu languages, Dhaisu categorizes nouns into classes. There are 17 classes in Dhaisu. Some of the classes are semantically transparent and others are not. One important feature of the noun classes is the nominal prefixes that characterize each class. The classes are labeled using number as is commonly done in Bantu linguistics and reconstructed PB forms. Dhaisu’s correspondences of class prefixes with mid-high vowels in Proto-Bantu provide useful evidence. These are Classes 1, 3, 4, 7, 11, 13, 14, 15, and 18.  \cite[22]{Nurse2000} suggests that the vowels for the prefixes in Dhaisu are mid-high. The relevant prefixes are shown as bold in \tabref{tab:ngonyani:14}.

\begin{table}[small]
\caption{Class prefixes compared to (\citealt[97]{Meeussen1967}; \citealt[14]{RugemaliraEtAl2019}}
\label{tab:ngonyani:14}
    \begin{tabularx}{\textwidth}{r XXX l}
    \lsptoprule
        Class & PB Prefix & Dhaisu Prefix & Example\\
    \midrule
        1 & \textbf{*mu}-   & \textbf{mɔ}-  & mɔtu              & ‘person’\\
        2 & *ba-            & a-            & atu               & ‘people’\\
        3 & \textbf{*mu}-   & \textbf{mɔ }  & mɔtɛ              & ‘tree’\\
        4 & \textbf{*mi}-   & \textbf{mɛ}-  & mɛtɛ              & ‘trees’\\
        5 & *i̧              & ø             & tumbe             & ‘egg’\\
        6 & *ma-            & ma-           & matumbe           & ‘eggs’\\
        7 & \textbf{*ki}-   & \textbf{ki}-  & kitu              & ‘thing’\\
        8 & *bi̧             & i-            & itu               & ‘things’\\
        9 & *n-             & n-            & mbɔri             & ‘goat’\\
        10 & *n-            & n-            & mbɔri             & ‘goats’\\
        11 & \textbf{*du-}  & \textbf{rɔ}   & rɔsi              & ‘river’\\
        12 & *ka-           & ka-           & kayana            & ‘little child’\\
        13 & \textbf{*tu-}  & tu            & \textbf{tu}yana   & ‘little children’\\
        14 & \textbf{*bu-}  & u-            & urwari            & ‘sickness’\\
        15 & \textbf{*ku-}  & kɔ-           & \textbf{kɔ}rɛma   & 'cultivating'  \\
        16 & *pa-           & va-           & vatu              & 'place’\\
        17 & \textbf{*ku-}  & \textbf{kɔ-}  & kɔyo              & 'there' \\
    \lspbottomrule
    \end{tabularx}
\end{table}

Evidently, not all intermediate high vowels in prefixes have merged to mid-low. Classes 1 and 3 \textit{*mu- > mɔ-}, as Classes 4 and 11 lowered \textit{*mi- > mɛ-} and \textit{*du- > rɔ} respectively. Meanwhile, Classes 7  and 14 have merged higher as \textit{*ki- > ki-} and \textit{*bu- > u-}. \cite[32]{Nurse2000}  indicates Class 15 as \textit{*ku- > ku-}. The nominal prefixes, therefore, provide examples of both downward merger and upward merger of the same vowel. Further investigation should be able to determine the exact patterns of the change and possible factors, such as contact with different languages.

\subsection{Verbal suffix}\label{sec:ngonyani:4.2}

Dhaisu has a system of verbal derivations, a system that characterizes Bantu languages. This includes verbal suffixes for causative, applicative, reciprocal, passive, neuter, separative. These have also been reconstructed to some proto-forms with high vowels, mid-high vowels, and the low vowel \citep{Meeussen1967}. Four verb extensions in Dhaisu may be traced to PB mid-high vowels. They are:

\begin{enumerate}
    \item[a)] Applicative *-id
    \item[b)] Neuter *-ik
    \item[c)] Reversive *-ul
    \item[d)] Passive *-u
\end{enumerate}

Of these, only the applicative appears relatively consistently in Dhaisu as \textit{-ɛr}. The applicative suffix which is reconstructed as \textit{*id} in Proto-Bantu \cite[92]{Meeussen1967} shows evidence of downward merger. The reflex in Dhaisu is \textit{-ɛr-}, as shown in the following examples.

\begin{table}
    \caption{The applicative suffix}
    \label{tab:ngonyani:15}
    \begin{tabular}{@{}l l l l l@{}}
        rita &  `devine' & & ritɛra & 'devine with' \\
        ɛtɛra & `spill' & & ɛtɛrɛra & `spill for'\\
        rɛma & `cultivate' & & rɛmɛra & `cultivate for, with'\\
        duma & `come out' & & dumɛra & `come out from'\\
        ʃɔka & `repeat' & & ʃɔkɛra & `repeat for'\\
        dɔra & `open' & & dɔrɛra & `open for, with'\\
        mama & `lie down, sleep' & & mamɛra & `sleep for,on'\\
        tɛɛja & `set a trap' & & tɛɛjɛra & `trap for someone'\\
    \end{tabular}
\end{table}

In some cases, the applicative may still be pronounced as \textit{-ir} as can be seen in \cite[25]{RugemaliraEtAl2019}. It is quite possible that such forms result from Swahili influence. Unlike other languages in the area, Dhaisu does not exhibit vowel harmony for such an extension. An invariant \textit{-ɛr-} appears even though \cite[40]{Nurse2000} suggests that some are realized as \textit{–ɛl-} and some are \textit{–ir-}. 

The neuter affix, also known as stative, is less consistent in the realization of the vowel, as the following examples show.

\begin{table}
    \caption{The neuter suffix}
    \label{tab:ngonyani:16}
    \begin{tabular}{@{}l l l l@{}}
        dɔra & `open' & dɔrɛka & `can open'  \\
         mada & `gather beans' & madɛka & `can be gathered'\\
         tia & `leave' & tiɛka & `be devorced'\\
         suma & `look for' & sumɛka & `be searchable'\\
         guna & `break' & gunika & `breakable'\\
         mɛra & `swallow' & mɛrɛka & `can be swallowed'\\
         dɔja & `buy' & dɔjika & `can be bought'\\
         keva & `close' & kɛvika & `swallowable'
    \end{tabular}
\end{table}

This set contains words some of which take \textit{-ɛk }and others that take \textit{-ik} as the neuter suffix. They do not appear to be motivated by vowel harmony. Considering that \textit{-ɛk} is not used the same way in any of the languages that seem to influence Dhaisu, it must be an innovation in this language. This innovation is consistent with the ongoing merger of the intermediate vowels.

The reversive, also known as separative, appears as a rather unproductive suffix applying to very few items that cannot lead us to a reasonable generalization. The passive is generally realized as a /w/ followed by the final vowel /a/.

From the evidence presented, we conclude that the vowels are in the process of merging intermediate vowels. The indeterminancy that is sometimes observed is due to the fact that the changes are ongoing and not affecting all cases in the same way. That is why there is a mixture of high-mid and low-mid vowels in the class prefixes and in some cases where both appear as in \textit{*gùdù} > \textit{kɔru} 'leg'. There are many cases that appear to indicate the lowering may not affect vowels in word-final position, as in \textit{*kuku >  ŋgɔku} 'chicken'.

\cite{Schadeberg1995} notes, among other things, the co-occurence of spirantization and 7 > 5 vowel change in Bantu languages and that the 7 > 5 vowel change was invariably preceded by spirantization. He notes also that only a few languages have had spirantization without vowel merger. Most notable is that merger of intermediate high vowels \textit{*i/*e} and \textit{*u/*o} is unknown in the languages studied \citep{Schadeberg1995}. The data in Dhaisu is interesting in at least two respects. One is that the merger is rather recent and no spirantization is co-occurring. The vowel merger that is associated with spirantization involved mid-high and high vowels. Only very few languages attest to 7 > 5 change without spirantization \citep{Bostoen2008}. This leads us to consider consonant reflexes.

\section{Consonant reflexes}\label{sec:ngonyani:5}

In this section, we present Dhaisu's reflexes of Proto-Bantu consonants and examine various processes that have affected the consonant inventory. The reflexes are presented in \citet[204]{Nurse1982} where they are compared to other Thagicu languages. This section provides examples of words with the reflexes. 

All stops  appear to have undergone significant innovations that are not associated with spirantization or fricativization before \textit{*i̧ }and \textit{*u̧} \citep{Nurse1982,Nurse2000}. In the neighboring Seuta languages (Shambala, Bondei, Digo, Segeju) and Sabaki (eg Swahili, Mijikenda), as well as in the entire Northeast Coast Bantu, stops became fricatives before the two high vowels \citep{NurseHinnebusch1993}\todo{The same source was cited twice at the same point. One of them has been removed.}. Again, our reference point is Proto-Bantu as shown in \tabref{tab:ngonyani:17}. 

\begin{table}
	\caption{Proto-Bantu \cite[83]{Meeussen1967}}
	\label{tab:ngonyani:17}
    \begin{tabular}{@{}c c c c@{}}
    	Bilabial & Alveolar & Palatal & Velar\\
    	p & t & c & k\\
    	b & d & j & g\\
    	m & n & ɲ &   \\
    \end{tabular}
\end{table}

The exact nature of the PB voiced stops is a subject of considerable controversy and disagreements \citep{Hyman2019, Mould1972}. Whether they are stop \textit{*b, *d, *g} \citep{Guthrie1967, Meeussen1967, Meinhof1932}  or continuant \textit{*β, *l, *ɣ} \citep{Meinhof1932}  does not affect the findings. We describe (a) frictivization of \textit{*p}, (b) voiceless stops that did not change, (c) various reflexes of voiced consonants, and (d) changes of PB palatal consonants.

\begin{table}
    \caption{PB-Dhaisu consonant reflexes}
    \label{tab:ngonyani:18}
    \begin{tabular}{@{}l r r r r r r r r@{}}
         Proto-Bantu & *p & *b & *t & *d & *c & *j & *k & *g\\
         Dhaisu & v & ø & t & r & d & s & k & ø \\
    \end{tabular}
\end{table}

Proto-Bantu voiceless bilabial stop \textit{*p} became voiced and eventually to a labio-dental fricative /v/ in Dhaisu. Examples in \REF{ex:ngonyani:3} demonstrate this.

\ea%3
	\label{ex:ngonyani:3}
	*p > v\\
    \begin{tabbing} 
        \= PB \quad\= \quad\= \quad\= \quad\= \quad\= \quad\= \quad\= \quad\= Dhaisu \quad\=  \quad\= \\
        \> *kú̧pɪ́ \> \>‘short’ \> \> \>  \> \> \> ŋguvɛ \> `short' \\
        \> *pú̧d \> \>‘froth over’ \>  \> \> \> \> \> vurɔ \> ‘foam’\\
        \> *pì̧k \> \>‘arrive’ \> \>  \> \> \> \> vika \> ‘arrive’\\
        \> *pèep \> \>‘blow’ \> \> \> \> \> \> mvɛɛvo \> ‘wind’\\
        \> *kúápà \> \>‘armpit’ \> \> \>  \> \> \> ŋguava \> ‘armpit’
    \end{tabbing}
\z

The  \textit{*p} > \textit{v}  change is attested in all environments. The change from a stop to fricative is the kind of spirantization that is not associated with super high vowels noted in other Bantu languages.

In Dhaisu, the voiceless alveolar stop and voiceless velar stop have remained stable and not subject to changes observed with other consonants. The two sets in (\ref{ex:ngonyani:4}) and (\ref{ex:ngonyani:5}) provide examples.

\ea%4
    \label{ex:ngonyani:4}
    *t > t\\
    \begin{tabbing} 
        \= PB \quad\= \quad\= \quad\= \quad\= \quad\= \quad\= \quad\= Dhaisu \quad\=  \quad\= \\
        \>  *tí\>  ‘tree’ \> \> \>  \> \> \>  mɔti \>  ‘tree’\\
        \> *dì̧tò\> ‘heavy’ \> \> \>  \> \> \>  ritɔ \>  ‘heavy’\\
        \>  *tég \> ‘set trap’ \> \> \>  \> \> \>  tɛa \>  ‘set a trap’\\
        \>  *tém \>  ‘cut’ \> \> \>  \> \> \>  tɛma \>  ‘cut’ \\
        \> *kít \> ‘do’ \> \> \>  \> \> \> kɛta \> ‘do’    
    \end{tabbing}
\z

\ea%5
    \label{ex:ngonyani:5}
    *k > k\\
    \begin{tabbing} 
        \= PB \quad\= \quad\= \quad\= \quad\= \quad\= \quad\= \quad\= Dhaisu \quad\=  \quad\= \\
        \> *pì̧ka \> ‘arrive’ \> \> \>  \> \> \> vika \> ‘arrive’\\
        \> *kít \> ‘do’ \> \> \>  \> \> \> kɛta \> ‘do’\\
        \> *cèk \> ‘laugh’ \> \> \>  \> \> \> dɛka \> ‘laugh’\\
        \> *kókó \> ‘chicken’ \> \> \>  \> \> \> ɳgɔkɔ \> ‘chicken’\\
        \> *càká \> ‘thicket’\> \> \>  \> \> \> daka \> ‘bush, wilderness’
    \end{tabbing}
\z

The two consonants have remained stable with the reflexes showing \textit{*t > t }and \textit{*k > k } in  all  environments.

Proto-Bantu voiced stops \textit{*b}, \textit{*d}, and \textit{*g} have reflexes that show significant changes before all vowels. The bilabial \textit{*b} has been lost before all vowels except in cases where it appears in post-nasal position.

\ea%6
    \label{ex:ngonyani:6}
    *b > ø\\
    \begin{tabbing} 
        \= PB \quad\= \quad\= \quad\= \quad\= \quad\= \quad\= \quad\= Dhaisu \quad\=  \quad\= \\
        \> *bìdɪ̀
        \> ‘body’
        \> \> \>  \> \> \> mwɛrɛ
        \> ‘body’\\
        \> *bí̧n
        \> ‘sing’
        \> \> \>  \> \> \> wina
        \> ‘to sing’\\
        \> *bèbà
        \> ‘rat’
        \> \> \>  \> \> \> mbɛa
        \> ‘rat’\\
        \> *bón
        \> ‘see’
        \> \> \>  \> \> \>  ɔna
        \> ‘see’
    \end{tabbing}
\z

The plural form \textit{*batu}  'people', for example, is \textit{atu} in Dhaisu. Likewise, \textit{*bona} 'see' has become ɔna. In other environments, this loss was followed by glide formation when a prefix is attached as in \textit{*bidi} > \textit{mwɛrɛ} 'body' where the prefix  \textit{*mu-} has become \textit{mw-} before the now vowel-initial stem. The change \textit{*b} > ø is widely attested in the language. A similar case to this is observed with the reflex of Proto-Bantu voiced velar stop \textit{*g}. Consider the following example reflexes. 

\ea%7
    \label{ex:ngonyani:7}
    *g > ø\\
	\begin{tabbing} 
        \= Proto-Bantu \quad\= \quad\= \quad\= \quad\= \quad\= \quad\= \quad\= Dhaisu \quad\=  \quad\= \\
        \> *gùndà
        \> ‘forest’\> \> \>  \> \> \> 
        mnda
        \> ‘farm’\\
        \> *gòngò
        \> ‘back’ \> \> \>  \> \> \> 
        mɔngɔ
        \> ‘back’\\
        \> *tég
        \> ‘set trap’\> \> \>  \> \> \> 
        tɛa
        \> ‘set a trap’\\
        \> *mbògà
        \> ‘vegetable’\> \> \>  \> \> \> 
        mbɔa
        \> ‘vegetable’\\
        \> *gàmb
        ‘\> speak’\> \> \>  \> \> \> 
        kwamba
        \> ‘to speak’
   \end{tabbing}
\z

In \textit{*tega > tɛa} ‘set a trap,’ the velar stop is deleted as is the case in several other words such as \textit{*mbògà > mboa} ‘vegetable’. The change \textit{*gùndà} ‘forest’ > \textit{mnda} ‘farm’ involved the deletion of the velar stop as well as the vowel \textit{u}. It is important to note, once more, that both \textit{*b > ø} and \textit{*g > ø} are not associated with the so called super high vowels. 

Proto-Bantu alveolar consonant \textit{*d} has the alveolar/r/ as its reflex. This is illustrated in the following examples. 

\ea%8
    \label{ex:ngonyani:8}
    *d > r\\
    \begin{tabbing} 
        \= Proto-Bantu \quad\= \quad\= \quad\= \quad\= \quad\= \quad\= \quad\= Dhaisu \quad\=  \quad\= \\
        \> *dì̧tò
        \> ‘heavy’
        \> \> \>  \> \> \> ritɔ
        \> ‘heavy’\\
        
        \> *bìdì
        \> ‘body’
        \> \> \>  \> \> \> mwɛrɛ
        \> ‘body’\\
        
        \> *dògá
        \> ‘bewitch’
        \> \> \>  \> \> \> urɔi
        \> ‘witchcraft’\\
        
        \> *dúm
        \> ‘bite’
        \> \> \>  \> \> \> rɔma
        \> ‘bite’\\
        
        \> *dob
        \> ‘(to) fish’
        \> \> \>  \> \> \> lɔwa
        \> ‘(to) fish’
   \end{tabbing}
\z

Occasionally, a lateral liquid /I/ appears as the reflex. But the majority of cases are /r/. As with other consonants, there are no effects of the high vowels associated with spirantization.

The reflexes of the palatal consonants \textit{*c} and \textit{*j} reveal more dramatic shift in the consonants. Both are fronted in Dhaisu. Proto-Bantu \textit{*c} became /d/ that can occassionally be heard as a dental \todo{Is this slash intended here?}/ /d̪/.

\ea%9
    \label{ex:ngonyani:9}
    *c > d\\
    \begin{tabbing} 
        \= Proto-Bantu \quad\= \quad\= \quad\= \quad\= \quad\= \quad\= \quad\= Dhaisu \quad\=  \quad\= \quad\= \quad\= \quad\=\\
     
        \> *cùbad
        \> ‘urinate’
        \> \> \>  \> \> \> duwama/d̪uwama
        \> \> \> \> \>  ‘urinate’\\
        
        \> *jícò
        \> ‘eye’
        \> \> \>  \> \> \> ridɔ/ rid̪ɔ
        \> \> \> \> \> ‘eye’\\
        
        \> *cónì
        \> ‘shame’
        \> \> \>  \> \> \> ndɔni/ nd̪̪ɔni
        \> \> \> \> \> ‘shame’\\
        
        \> *cèk
        \> ‘laugh’
        \> \> \>  \> \> \> dɛka/d̪ɛka
        \> \> \> \> \> ‘laugh’\\
        
        \> *càká
        \> ‘thicket’
        \> \> \>  \> \> \> d̪aka
        \> \> \> \> \> ‘bush, wilderness’
    \end{tabbing}
\z

Voiceless \textit{*c} became /d/. The dentalization that is sometimes heard is a feature which identifies Dhaisu with Thagicu languages but also sets it apart from the rest. Thagicu languages have developed interdental fricatives for Proto-Bantu \textit{*c }\citep{Nurse1982}. Dhaisu distinguishes itself from Thagicu in this by not having an interdental fricative. 

Proto-Bantu voiced palatal consonant \textit{*j}  shifted to a voiceless alveolar fricative /s/ in Dhaisu.

\ea%10
    \label{ex:ngonyani:10}
    *j > s\\
	\begin{tabbing} 
        \= Proto-Bantu \quad\= \quad\= \quad\= \quad\= \quad\= \quad\= \quad\= Dhaisu \quad\=  \quad\= \\
        \>  *jókà
        \> ‘snake’
        \> \> \>  \> \> \> sɔka
        \> ‘snake’\\
        
        \> *jí
        \> ‘water’
        \> \> \>  \> \> \> masi
        \> ‘water’\\
        
        \> *jògù
        \> ‘elephant’
        \> \> \>  \> \> \> sɔu
        \> ‘elephant’\\
        
        \> *júkì
        \> ‘bee’
        \> \> \>  \> \> \> sɔki
        \> ‘bee’
    \end{tabbing}
\z

Unlike its voiceless counterpart, the PB voiced \textit{*j} devoiced and the loss of occlusion  created a fricative. Compared to other consonants, the PB reflexes of \textit{*c} and \textit{*j} are relatively few.

To sum up, we have identified various reflexes of PB consonants and noted the absence of patterns of spirantization, a feature that appears prevalent in Thagicu languages. PB bilabial \textit{*b} and velar \textit{*g}  have been lost. A notable feature has been the dramatic shift of palatal \textit{*c }and \textit{*j} to /d/ and /s/ respectively. The sounds /p, b, f, g, h, z/ found in Dhaisu are non-inherited \citep{Nurse2000}  and not a result of the spirantization observed in other Bantu zones. 
     
One question that arises is if the reflexes of \textbf{*b} and \textbf{*g} are ø, then where did /b, g/ in contemporary Dhaisu come from? Likewise, in spite of \textbf{*p > v}, there is /p/ in the language today. There are several sources. One source contemporary forms is loanwords. Recall that \cite[20]{Nurse2000} notes, as much as 60 percent of Dhaiso vocabulary is borrowed. Many words with such consonants are from Swahili and neighboring languages. For example, words like \textit{bahati }‘luck’ and \textit{bei} ‘price’ are words with /b/ found in the language. These words are from Arabic \citep{TUKI1996}, borrowed via Swahili. Another likely source of such sounds is phonological innovations that may have happened. Consider the following words:

\begin{table}
    \caption{Examples of Dahl's Law}
    \label{tab:ngonyani:19}
    \begin{tabular}{@{}l l l l@{}}
        PB & & Dhaisu &  \\
        *tánò & `three' & datɔ & `three'\\
        *kútà & `oil, fat' & guta & `oil'\\
        *pèt & `bend'&  bɛta & `bend, bow'\\
    \end{tabular}
\end{table}
\todo[inline]{Inconsistent use of examples vs tables, see ex 10 vs table 19.}

These are traces of Dahl’s Law, a dissimilation phenomena observed in some Bantu languages in which the voiceless consonant of the first syllable in a sequence of two voiceless onsets becomes voiced. Thus, although PB \textit{*g > ø,} the voicing of \textit{*k} results in /g/, as in \textit{*kú̧tà > maguta} 'oil'. Such processes have contributed to the inventory of phonemes in contemporary Dhaisu.

\section{Concluding remarks and future research}\label{sec:ngonyani:6}

In this paper we set out to describe Proto-Bantu reflexes in Dhaisu. Using lexical items from Bantu Lexical Reconstruction, we have demonstrated one remarkable feature in Dhaisu, namely, *7 > 5 vowel merger which is taking place. Although *7 > 5 vowel merger is widespread in Bantu, Dhaisu is unique in that it is merging mid-high and mid-low vowels rather that merging mid-high and high vowels. Future research calls for an exploration of tone and its possible role in the merger and other processes. The fact that the intermediate vowels are merging raises the question of what are the effects of such merger on the quality of the high vowels. It may also be the case that the mid-high vowels are acoustically closer to mid-low vowels. Acoustic studies will shed some light on this question. The data on which this study are based was collected in elicitation of words. Often times a request to repeat a word with intermediate high vowel brought a different vowel. Studying vowels from texts and narratives rather than elicited wordlists may result in clearer data. The paper has also presented data that show that Dhaisu, like other Thagicu languages, did not undergo spirantization before high vowels, a process that is linked to changes in the Bantu vowel systems. 

\section*{Abbreviations}
\begin{tabular}{@{}ll@{}}
  BLR    &   Bantu Lexical Reconstruction\\
  PB     &   Proto-Bantu\\
  TLS    &   Tanzania Language Survey\\
\end{tabular}

\section*{Acknowledgements}
This work could not have been possible without the generous participation of the native speakers of Dhaisu, particularly Omari Gauwa, Juma Faki, Juma Mwinyihamisi and Fatuma, all of Bwiti. Many thanks to Mohammed Rafiq Yunus whose initial work on the Dhaisu people led to this study. He put us in contact with the Dhaisu speakers and paved the way for much of the project. We appreciate comments from reviewers who helped us clarify many points in the paper. The fieldwork on documentation of Dhaisu, on which this paper is based, was funded by Michigan State University's Alliance for African Partnership.
 
{\sloppy\printbibliography[heading=subbibliography,notkeyword=this]}

\end{document}
