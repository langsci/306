\documentclass[output=paper,colorlinks,citecolor=brown]{langscibook}
\ChapterDOI{10.5281/zenodo.6393750}

\author{Terrance Gatchalian\affiliation{University of British Columbia} and Rachel Lee\affiliation{University of British Columbia} and  Carolin Tyrchan\affiliation{University of Potsdam}}
\title{Propositional attitude verbs and complementizers in Medumba}

\abstract{We present the preliminary results of an investigation on complementizers and their interaction with propositional attitude verbs in Medumba (Grassfields Bantu, Niger-Congo). This initial sketch of the Medumba C-system opens up questions about the syntactic distribution and semantic force of the various Cs. There are two clause-initial Cs, /mb\baru/ and /nd{à}/, of which /mbʉ/ has three syntactically conditioned allomorphs: [\mbuL]-L, [\mbuHL]-HL and [\mbuLH]-LH. The clause-final [\la] obligatorily co-occurs with two of the clause-initial Cs, namely [\mbuHL]-HL and [nd{à}]. Additionally, the inventory of propositional attitude verbs (\textsc{pav}s) is quite small, with only four identified thus far: two are monomorphemic ([l\ep n] `know’, [\ch{ú}p] ‘say’) and two are are bi-morphemic ([kw\epL-d\schwaL] `think-\textsc{iter}’, [b\epH t-t\schwaH] `ask-\textsc{iter}’). We make the case for a syntactically conditioned floating H-tone. Additionally, we propose a basic structure of the Medumba CP and raise questions about the scope of polarity and the nature of the clause-final particle /\la/. \\
\textbf{Keywords:} Medumba, complementizer, propositional attitude, embedding}


\IfFileExists{../localcommands.tex}{%hack to check whether this is being compiled as part of a collection or standalone
  \addbibresource{localbibliography.bib}
  \usepackage{langsci-optional,langsci-branding}
\usepackage{langsci-gb4e}
% \usepackage{langsci-textipa}
% \usepackage{langsci-glyphs}
\usepackage[linguistics]{forest}
\usepackage{tabto}
\usepackage{multirow}
\usepackage{bbding}

\usepackage[normalem]{ulem}

\usepackage{tikz-qtree}

\usepackage{enumitem}

\usepackage{multicol}
\usepackage{stmaryrd} %double brackets

\makeatletter
\let\pgfmathModX=\pgfmathMod@
\usepackage{pgfplots,pgfplotstable}%
\let\pgfmathMod@=\pgfmathModX
\makeatother
\usepgfplotslibrary{colorbrewer}
\usetikzlibrary{fit}

\usepackage{jambox}
\usepackage{tikz-qtree-compat}
\usetikzlibrary{arrows, arrows.meta}
\usepackage{longtable}
\usepackage{subcaption}

  \makeatletter
\let\thetitle\@title
\let\theauthor\@author
\makeatother

\newcommand{\togglepaper}[1][0]{
%   \bibliography{../localbibliography}
  \papernote{\scriptsize\normalfont
    \theauthor.
    \thetitle.
    To appear in:
    Change Volume Editor \& in localcommands.tex
    Change volume title in localcommands.tex
    Berlin: Language Science Press. [preliminary page numbering]
  }
  \pagenumbering{roman}
  \setcounter{chapter}{#1}
  \addtocounter{chapter}{-1}
}

\newcommand{\bari}{\ipabar{\i}{.5ex}{1.1}{}{}}
\newcommand{\notipa}[1]{\textnormal{#1}}

\newcommand{\agre}{\textsc{agr}-\ol{eene}}

\renewcommand{\emph}[1]{\textit{#1}} % resetting a setting from ling-macros-modified (I think?)

% forest settings to make compact but (mostly) straight-spined trees:
\forestset{
fairly nice empty nodes/.style={
            delay={where content={}{shape=coordinate,for parent={
                  for children={anchor=north}}}{}}
, angled/.style={content/.expanded={$<$\forestov{content}$>$}}
}}

\forestset{sn edges/.style={for tree={parent anchor=south, child anchor=north}}}

\newcommand{\bex}{\begin{exe}}
\newcommand{\fex}{\end{exe}}

\newcommand{\bxl}{\begin{exe}}
\newcommand{\fxl}{\end{exe}}

\newcommand{\ix}[1]{\textsubscript{#1}}
\newcommand{\alert}[1]{\textbf{#1}}
\newcommand{\ol}[1]{\textit{#1}}


			\usetikzlibrary{shapes,arrows,positioning,decorations,decorations.pathmorphing,intersections}
\forestset{
nice empty nodes/.style={
    for tree={calign=fixed edge angles},
    delay={where content={}{shape=coordinate,for siblings={anchor=north}}{}}
},
}

\definecolor{dark-gray}{gray}{0.3}

%\usepackage{dingbat,pifont}


%%%%%%%%%%%%For arrows%%%%%%%%%%%%%

\newcommand\Tikzmark[2]{%
  \tikz[remember picture]\node[inner sep=0pt,outer sep=0pt] (#1) {#2};%
}
\NewDocumentCommand\DrawArrow{O{}mmmmO{3}}{
\tikz[remember picture,overlay]
  \draw[->,line width=0.8pt,shorten >= 2pt,shorten <= 2pt,#1]
    (#2) -- ++(0,-#6\ht\strutbox) coordinate (aux) -- node[#4] {#5} (#3|-aux) -- (#3);
}
\NewDocumentCommand\DrawDotted{O{}mmmmO{3}}{
\tikz[remember picture,overlay]
  \draw[->,line width=0.9pt,dotted,shorten >= 2pt,shorten <= 2pt,#1]
    (#2) -- ++(0,-#6\ht\strutbox) coordinate (aux) -- node[#4] {#5} (#3|-aux) -- (#3);
}
\NewDocumentCommand\DrawLine{O{}mmmmO{3}}{
\tikz[remember picture,overlay]
  \draw[line width=0.8pt,shorten >= 2pt,shorten <= 2pt,#1]
    (#2) -- ++(0,-#6\ht\strutbox) coordinate (aux) -- node[#4] {#5} (#3|-aux) -- (#3);
}
%%%%%%%%%%%%%%%%%%%%%%%%%%%%%%%%%%%%%


\newcommand{\baru}{ʉ}
\newcommand{\baruH}{\'\baru}
\newcommand{\baruL}{\`\baru}

\newcommand{\ep}{ε}
\newcommand{\epH}{\'\ep}
\newcommand{\epL}{\`\ep}

\newcommand{\schwa}{ə}
\newcommand{\schwaH}{\'ə}
\newcommand{\schwaL}{\`ə}

\newcommand{\oo}{ɔ}
\newcommand{\ooH}{\'\oo}
\newcommand{\ooL}{\`\oo}

\newcommand{\ds}{\textsuperscript{
	\hspace*{-2pt}\begin{tikzpicture}
		\draw[-{>[scale=0.5]}] (0,0.4) --(0,0.25);
	\end{tikzpicture}}}

\newcommand{\ch}{t͡ʃ}
\newcommand{\dz}{d͡ʒ}

\newcommand{\tgl}{ʔ}

%shortcuts for the complementizers
\newcommand{\mbuL}{mb\baruL}
\newcommand{\mbuHL}{mb\baruH\baruL}
\newcommand{\mbuLH}{mb\baruL\baruH}
\newcommand{\la}{lá}
\newcommand{\nda}{ndà}

\newcommand{\tsc}[1]{\textsc{#1}}
\renewcommand{\textscb}{ʙ}
\newcommand{\ipa}[1]{#1} %disable IPA

\newcommand{\SM}[1]{#1}

\DeclareNewSectionCommand
  [
    counterwithin = chapter,
    afterskip = 2.3ex plus .2ex,
    beforeskip = -3.5ex plus -1ex minus -.2ex,
    indent = 0pt,
    font = \usekomafont{section},
    level = 1,
    tocindent = 1.5em,
    toclevel = 1,
    tocnumwidth = 2.3em,
    tocstyle = section,
    style = section
  ]
  {appendixsection}

\renewcommand*\theappendixsection{\Alph{appendixsection}}
\renewcommand*{\appendixsectionformat}
              {\appendixname~\theappendixsection\autodot\enskip}
\renewcommand*{\appendixsectionmarkformat}
              {\appendixname~\theappendixsection\autodot\enskip}

\renewcommand{\lsChapterFooterSize}{\footnotesize}

\togglepaper[23]
}{}


\begin{document}
\maketitle

\section{Introduction}

Medumba, a Bamileke language of western Cameroon, exhibits a wide use of grammatical tone. While the patterns of Medumba tone have been described in detail by \citet{Voorhoeve1971}, the grammatical functions which are expressed through tonal morphemes have not received the same level of detailed description or analysis, and have often been left out of the discussion of tone in Medumba. In fact, the status of tone as a syntactically conditioned element is not discussed in the current literature on Medumba. Instead, tone in Medumba has been treated as a strictly phonological phenomenon. This, as we will outline below, obscures some important relationships within the grammar of Medumba (see also \citealt{Keupdjio2020}).

This paper has two principle aims. Firstly, it describes both the system of Propositional Attitude Verbs (\textsc{pav}s) and the system of complementizers, which have received little descriptive treatment. Secondly, it presents an argument for tonal morphology specifically within the complementizer system, further suggesting the possibility for syntactic activity of tone elsewhere in the language.

\section{Propositional attitude verbs}

We will be defining \tsc{propositional attitude verbs} (\textsc{pav}s) as verbs conveying mental attitude or communicative verbs, as outlined in \citet{Pearson2021} (see also \citealt{Asher1987}). The inventory of \textsc{pav}s in Medumba is relatively small, consisting of two mono-morphemic forms (\ref{pav1a}, \ref{pav1b}) and two bi-morphemic forms (\ref{pav1c}, \ref{pav1d}).

\TabPositions{.175\textwidth}
\ea \label{PAV1}
    \begin{xlist}
    \ex /l\epL n/                   \tab `know'                  \label{pav1a}
    \ex /\ch{ú}p/                   \tab `say'                   \label{pav1b}
    \ex /kʷ\epL-d\schwaL/           \tab `think-\tsc{iter}'      \label{pav1c}
    \ex /b\epH t-t\schwaH/          \tab `ask-\tsc{iter}'        \label{pav1d}
    \end{xlist}
\z

The unsuffixed, mono-morphemic forms of the bi-morphemic \textsc{pav}s (/kʷ\epL-d\schwaL/ and /b\epH t-t\schwaH/) are unattested. The iterative morpheme is underlyingly toneless, and takes the final tone of the base to which it is affixed. The bi-morphemic \textsc{pav}s appear to be lexicalized iterative forms, as can be seen by productive use of the iterative suffix, such as in (\ref{iterative}).

\ea \label{iterative}
    \begin{xlist}
    \ex {[n\baruL\ \ch {ú}b-\schwaH]}   \tab `to say'
    \ex {[n\baruL\ \ch {ú}p-t\schwaH]}  \tab `to talk'
    \end{xlist}
\z

These four propositional attitude verbs participate in a variety of constructions, taking either a nominal complement, as in (\ref{NomCompl}) or a clausal complement as in (\ref{ClauseCompl}).

\ea
    \begin{xlist}
    \ex \label{NomCompl}
        \gll    w{à}t\epL\epH t l\epL\epH n n{ù}{ŋ}g\epL \\
                Watat           know        Nuga    \\
        \glt    `Watat knows Nuga.'
    \ex \label{ClauseCompl}
        \gll    {m\baruL} l\epH n     {mb\baruL}    nzì kʰ{ú}\tgl{ú} \ch{ʷ}\epL\epH t n\dz\epH        \\
                1\tsc{sg}  know    \tsc{comp}  envy taro \tsc{pres} hurt  \\
        \glt    `I know that Numi is hungry.'
    \end{xlist}
\z

When the complement of the \textsc{pav} is a noun, the bare noun appears. That is, the thematic role assigned to the DPs is not marked by any overt morphosyntactic element. Instead, the semantic interpretation follows from the rigid order of constituents, S V O$_{\text{direct}}$ O$_{\text{indirect}}$. As exhibited in examples (\ref{constituentorder}a, b), the theme constituent must precede the goal constituent.

\ea \label{constituentorder}
{[\ch{ù}p]} selecting for a DP$_{theme}$ and a DP$_{goal}$ \\
    \begin{xlist}\judgewidth{\#}
    \ex[]{
        \gll {w{à}t\epL\epH t} {\ch {ú}p} {n\baruH\ds n\baruH} {n{ù}{ŋ}g\epL}\\
             Watat say truth Nuga \\
        \glt `Watat said the truth to Nuga.'}

    \ex[\#]{
        \gll {w{à}t\epL\epH t} {\ch {ú}p} {n{ù}{ŋ}g\epL} {n\baruH\ds n\baruH} \\
             Watat say Nuga truth \\
        \glt Intended: `Watat said the truth to Nuga.'}
    \end{xlist}
\z

When the \textsc{pav} takes a clausal complement, the left-edge of the clause is delineated by one of four complementizers. The properties of these complementizers and their analysis form the basis for the remaining sections of this paper.

\tabref{tab:subcat} roughly outlines the subcategorization properties of the \textsc{pav}s. All \textsc{pav}s except /kʷ\epL-d\schwaL/ can take a DP as a complement. A more detailed investigation of the subcategorization of \textsc{pav}s is left as an avenue for further research.

\begin{table}
    \caption{Subcategorization properties of \textsc{pav}s (\textit{tentative})}
    \label{tab:subcat}
     \begin{tabular}{lllll}
        \lsptoprule
         \textsc{pav} & {[l\epL n]} & {[\ch {ú}p]} & {[kʷ\epL-d\schwaL]} & {[b\epH t-t\schwaH]} \\
         \midrule
         \multirow{4}{6em}{Possible Complements} & DP$_{\text{theme}}$ & DP$_{\text{theme}}$ & CP & DP \\
         & CP & DP$_{\text{goal}}$ & & CP \\
         &  & DP$_{\text{theme}}$ + DP$_{\text{goal}}$ & & \\
         & & CP & & \\
        \lspbottomrule
     \end{tabular}
\end{table}

We will also include the modal verb [bʰʷ\ooL\ooH] `be.good' in the following discussion. It should be noted that this verb occurs only with the impersonal subject, as can be seen in (\ref{bhoo1}). While it is not considered a \textsc{pav}, it is included here due to its participation in subordinating constructions and its interaction with the complementizers.

\ea
    \begin{xlist}
    \ex[]{ \label{bhoo1}
        \gll {{á}} {bʰʷ\ooL\ooH} {n{ù}m{í}} {\ch {ú}p} {n\baruH\ds n\baruH n\schwaH} \\
             {3.\tsc{sg}} {be.good} {Numi} {say} {truth} \\
        \glt `Numi should say the truth.' (lit. `It is good that Numi says the truth.')}
    \ex[*]{
        \gll {n{ù}m{í}} {bʰʷ\ooL\ooH} {\ch {ú}p} {n\baruH\ds n\baruH n\schwaH} \\
             {Numi} {be.good} {say} {truth} \\
        \glt Intended: `Numi should say the truth.'}
    \end{xlist}
\z

\section{Complementizers and clausal complements}
\subsection{Introduction}\largerpage

There are four complementizer forms in Medumba, each varying in the semantic contribution it gives to the subordinate clause. It will be argued that their co-occurrence restrictions arise principally from incompatibility between the semantics of the complementizer and the semantics of the subsequent selecting verb. We will also argue for a compositional account regarding the semantics of the complementizers responsible for this incompatibility with certain \textsc{pav}s.

\begin{sloppypar}
These four complementizers can be divided into two different underlying forms based on their segmental content; one being referred to as the /nd{à}/-form complementizer, and the other as the /mbʉ/-form complementizer, which has three tonal surface forms.
\end{sloppypar}

\subsection{/nd{à}/-form complementizers}

To begin with the simplest case, the clause-initial /nd{à}/ complementizer, in (\ref{nda1}), is uniformly low-tone (L). Furthermore, as in (\ref{ndaobligla}), it obligatorily co-occurs with the clause-final /\la/ particle.

\ea \label{nda1}
    \begin{xlist}
    \ex[]{ \label{ndaobligla}
        \gll {{á}} {bʰʷ\ooL\ooH}     {\nda} {n{ù}m{í}} {ʒ\baruH\baruL} {ʒ{ú}} {\la} \\
             {3.\tsc{sg}} {be.good} {\tsc{comp}} {Numi} {eat} {thing} {\tsc{la}} \\
        \glt `It is good that Numi ate something.'}
    \ex[*]{
        \gll {{á}} {bʰʷ\ooL\ooH} {\nda} {n{ù}m{í}} {ʒ\baruH\baruL} {ʒ{ú}} \\
             {3.\tsc{sg}} {be.good} {\tsc{comp}} {Numi} {eat} {thing} \\
        \glt Intended: `It is good that Numi ate something.'}
    \end{xlist}
\z

The syntactic and semantic properties of this complementizer are subject to further investigation and will not be discussed in great detail in this paper. However, for the purposes of our discussion, it is relevant that this complementizer does not appear to have surface tonal allomorphy as does the complementizer that will be discussed immediately below.

\subsection{/mb\baru/-form complementizers}

\subsubsection{L-tone complementizer}

There are three complementizers which all carry the segmental content /mb\baru/ but have distinct tonal melodies: [\mbuL]-L, [\mbuHL]-HL, and [\mbuLH]-LH. These complementizer forms each have a unique distribution under the four \textsc{pav}s and [bʰʷ\ooL\ooH]. The possible co-occurrence patterns are given in \tabref{tab:cooc} below. All logically possible combinations of a \textsc{pav} and a C with or without [\la] that are not depicted here were tested as well, but were judged as infelicitous by our consultant.


\begin{table}%2
    \begin{tabularx}{.8\textwidth}{Xcccc}
    \lsptoprule
         \textsc{pav}/Comp & [\mbuL] & [\mbuHL + \la] & [\mbuLH] & [nd{á} + \la] \\ \midrule
         {[l\epH n]} `know' & \langscicheckmark (\ref{mbuLexa}) & \langscicheckmark (\ref{mbuHLexa}) & * (\ref{mbuLHexa}) & \langscicheckmark \\
         {[\ch{ú}p]} `say' & \langscicheckmark (\ref{mbuLexb}) & \langscicheckmark (\ref{mbuHLexb}) & * (\ref{mbuLHexb}) & \langscicheckmark \\
         {[kʷ\epL-d\schwaL]} `think' & \langscicheckmark (\ref{mbuLexc}) & \langscicheckmark (\ref{mbuHLexc}) & * (\ref{mbuLHexc}) & \langscicheckmark \\
         {[b\epH t-t\schwaH]} `ask' & * (\ref{mbuLexd}) & \langscicheckmark (\ref{mbuHLexd}) & * (\ref{mbuLHexd}) & \langscicheckmark \\[5pt]
         %\midrule
         {[bʰʷ\ooL]} `be.good' & * (\ref{mbuLexe}) & * & \langscicheckmark (\ref{mbuLHexe}) & \langscicheckmark (\ref{ndaobligla}) \\
    \lspbottomrule
    \end{tabularx}
    \caption{Co-occurence of complementizers with \textsc{pav}s}
    \label{tab:cooc}
\end{table}

The L-tone allomorph [\mbuL] can introduce embedded indirect speech under three \textsc{pav}s, as seen in (\ref{mbuLex}a--c). The verb [b\epH t-t\schwaH] is incompatible with [\mbuL], as in (\ref{mbuLexd}).

\ea \label{mbuLex}
    \judgewidth{\#}
    \begin{xlist}
    \ex[]{ \label{mbuLexa}
        \gll {m\baruL} {l\epH n} {\mbuL} {nzì} {kʰ{ú}\tgl{ú}} {\ch{ʷ}\epL\epH t} {n\dz\epH} {n{ù}m{í}} \\
             {1.\tsc{sg}} {know} {\tsc{comp}} {envy} {taro} {\tsc{pres}} {hurt} {Numi} \\
        \glt `I know that Numi is hungry.'}
    \ex[]{ \label{mbuLexb}
        \gll {m\baruL} {\ch{ú}p} {n{ù}m{í}} {\mbuL} {\ch\schwaL\schwaH{ŋ}} {\ds\textscb\schwaH} \\
             {1.\tsc{sg}} {say} {Numi} {\tsc{comp}} {food} {be.cooked} \\
        \glt `I say to Numi that the food is ready.'}
    \ex[]{ \label{mbuLexc}
        \gll {m\baruL} {kʷ\epL d\schwaL} {\mbuL} {nzì} {kʰ{ú}\tgl{ú}} {\ch{ʷ}\epL\epH t} {n\dz\epH} {n{ù}m{í}} \\
             {1.\tsc{sg}} {think} {\tsc{comp}} {envy} {taro} {\tsc{pres}} {hurt} {Numi} \\
        \glt `I think that Numi is hungry.'}
    \ex[*]{ \label{mbuLexd}
        \gll {m\baruL} {b\epH tt\schwaH} {n{ù}m{í}} {\mbuL} {\ch\schwaL\schwaH{ŋ}} {\ds\textscb\schwaH} \\
             {1.\tsc{sg}} {ask} {Numi} {\tsc{comp}} {food} {be.cooked} \\
        \glt Intended: `I ask Numi if the food is cooked.'}
    \ex[\#]{ \label{mbuLexe}
        \gll {{á}} {bʰʷ\ooL} {\mbuL} {n{ù}m{í}} {ʒ\baruH\baruL} {ʒ{ú}} \\
             {3.\tsc{sg}} {be.good} {\tsc{comp}} {Numi} {eat} {something} \\
        \glt Intended: `It is good that Numi eats something.'}
    \end{xlist}
\z

As can be seen in example (\ref{mbuLexe}), the combination of [bʰʷ\ooL] and [\mbuL] was also judged as infelicitous by our consultant.\footnote{Note here that the tone on the modal verb differs from that given in \REF{nda1}. A discussion of this follows at the end of \sectref{secLH}.} As [\mbuL] introduces embedded speech and appears to simply introduce an embedded clause, it could be that [bʰʷ\ooL] is simply incompatible with the semantics of the embedded clauses tested. This observation leads to the assumption that the incompatibilities of certain \textsc{pav}s and Cs actually only arises on a semantic level, thus suggesting that the composed meanings that arise from their combination have to be accounted for at LF. As this paper aims at describing and explaining the phenomenon on a morphosyntactic level, this semantic interaction after narrow syntax is left to further investigation.

\subsubsection{LH-tone complementizer} \label{secLH}

The LH-allomorph [\mbuLH] seems to be well-formed only in contexts that are compatible with a modal context, exhibited in (\ref{mbuLHex}a--e).

\ea \label{mbuLHex}
    \begin{xlist}
    \ex[*]{ \label{mbuLHexa}
        \gll {m\baruL} {l\epH n} {\mbuLH} {nzì} {kʰ{ú}\tgl{ú}} {\ch{ʷ}\epL\epH t} {n\dz\epH} {n{ù}m{í}} \\
             {1.\tsc{sg}} {know} {\tsc{comp}} {envy} {taro} {\tsc{pres}} {hurt} {Numi} \\
        \glt Intended: `I know that Numi is/should be hungry.'}
    \ex[]{ \label{mbuLHexb}
        \gll {n{ù}m{í}} {\ds\ch{ú}p} {\mbuLH} {b{ù}} {bʰ{ú}{ù}m-\ds nd\schwaH} \\
             {Numi} {say} {\tsc{comp}} {3.\tsc{pl}} {meet-\tsc{recp}} \\
        \glt `Numi said that they should meet.'}
    \ex[*]{ \label{mbuLHexc}
        \gll {m\baruL} {kʷ\epL d\schwaL} {\mbuLH} {nzì} {kʰ{ú}\tgl{ú}} {\ch{ʷ}\epL\epH t} {n\dz\epH} {n{ù}m{í}} \\
             {1.\tsc{sg}} {think} {\tsc{comp}} {envy} {taro} {\tsc{pres}} {hurt} {Numi} \\
        \glt Intended: `I think Numi is/should be hungry.'}
    \ex[*]{ \label{mbuLHexd}
        \gll {m\baruL} {b\epH tt\schwaH} {n{ù}m{í}} {\mbuLH} {\ch\schwaL\schwaH{ŋ}} {\ds\textscb\schwaH} \\
             {1.\tsc{sg}} {ask} {Numi} {\tsc{comp}} {food} {be.cooked} \\
        \glt Intended: `I ask Numi if the food is/should be cooked.'}
    \ex[]{ \label{mbuLHexe}
        \gll {{á}} {bʰʷ\ooL} {\mbuLH} {n{ù}m{í}} {{\ds}ʒ\baruH\baruL} {{\ds}ʒ{ú}} \\
             {3.\tsc{sg}} {be.good} {\tsc{comp}} {Numi} {eat} {something} \\
        \glt `Numi should eat.'}
    \end{xlist}
\z

\begin{sloppypar}
The segmental similarity between the LH-allomorph and the L-allomorph, despite their semantic and distributional heterogeneity, raises the question of wheth\-er these forms are related to each other. By assuming a positive answer to this question, we will argue that we receive an insightful understanding of how tone can be syntactically conditioned in the complementizer system, and by consequence, opens up the possibility of understanding tone elsewhere in the grammar as an integral part of syntax-proper.
\end{sloppypar}

The core of the present discussion rests on the assumption that the complementizer can be meaningfully decomposed into an underlying /\mbuL/-form of the complementizer and a syntactic H-tone. There are both empirical and theoretical consequences for such an analytical assumption, so it is worth decomposing the assumption itself.

Consider the [\mbuLH]-LH complementizer, which, as was illustrated above, is interpreted as having deontic meaning. A decompositional analysis of this complementizer takes this form as the combination of the low-toned /\mbuL/-form, which introduces the embedded clause, with a syntactic H-tone, which provides the deontic force whose effects are seen from its distributional restrictions under the \textsc{pav}s.

The first possible objection to this analysis is the empirical evidence for the decomposition. To answer this, we can consider cases of deontic force without the presence of a complementizer. Consider the data in (\ref{NoComp}) below, where (\ref{NoCompa}) shows an embedded clause introduced by the LH-complementizer as expected. In (\ref{NoCompb}), however, the complementizer is not present; instead, the verb surfaces with an additional obligatory H-tone.

\ea \label{NoComp}
    \begin{xlist}
    \ex \label{NoCompa}
        \gll {{á}} {bʰʷ\ooL} {\mbuLH} {n{ù}m{í}} {\ch{ú}p} {n\baruH\ds n\baruH n\schwaH} \\
             {3.\tsc{sg}} {be.good} {\tsc{comp}} {Numi} {say} {truth} \\
        \glt `Numi should say the truth.'
    \ex \label{NoCompb}
        \gll {{á}} {bʰʷ\ooL\ooH} {n{ù}m{í}} {\ch{ú}p} {n\baruH\ds n\baruH n\schwaH} \\
             {3.\tsc{sg}} {be.good} {Numi} {say} {truth} \\
        \glt `Numi should say the truth.'
    \end{xlist}
\z

When the complementizer is not pronounced as in (\ref{NoCompb}) (whether it is a null element in C or syntactically absent), we see that the H-tone is still present, this time surfacing on the verb [b{ʰʷ}\ooL\ooH], which appears as L-toned in (\ref{NoCompa}). This demonstrates the presence of a floating H-tone, which cannot be taken as an inherent part of a simplex [\mbuLH]-LH. Rather, the persistence of the H-tone in this construction is a direct result of its morphosyntactic independence from the rest of the complementizer.

\subsubsection{HL-tone complementizer} \label{secHL}

The HL allomorph [\mbuHL] introduces embedded polar statements, and requires the presence of the clause-final [\la], (\ref{mbuHLex}a--e).

\ea \label{mbuHLex}
    \begin{xlist}
    \ex \label{mbuHLexa}
        \gll {m\baruL} {l\epH n} {\mbuHL} {nzì} {kʰ{ú}\tgl{ú}} {\ch{ʷ}\epL\epH t} {n\dz\epH} {n{ù}m{í}} {\ds\la} \\
             {1.\tsc{sg}} {know} {\tsc{comp}} {envy} {taro} {\tsc{pres}} {hurt} {Numi} {\tsc{la}} \\
        \glt `I know if Numi is hungry.' (I can tell whether Numi is hungry (or not))
    \ex \label{mbuHLexb}
        \gll {m\baruL} {\ch{ú}p} {n{ù}m{í}} {\mbuHL} {\ch\schwaL\schwaH{ŋ}} {\ds\textscb\schwaH} {\ds\la}\\
             {1.\tsc{sg}} {say} {Numi} {\tsc{comp}} {food} {be.cooked} {\tsc{la}}\\
        \glt `I wonder (to Numi) if the food is ready.'
    \ex \label{mbuHLexc}
        \gll {m\baruL} {kʷ\epL d\schwaL} {\mbuL} {nzì} {kʰ{ú}\tgl{ú}} {\ch{ʷ}\epL\epH t} {n\dz\epH} {n{ù}m{í}} {\ds\la} \\
             {1.\tsc{sg}} {think} {\tsc{comp}} {envy} {taro} {\tsc{pres}} {hurt} {Numi} {\tsc{la}} \\
        \glt `I think (about) if Numi is hungry.' (Whether Numi is hungry or not by now, it's something that I thought about.')
    \ex \label{mbuHLexd}
        \gll {m\baruL} {b\epH tt\schwaH} {n{ù}m{í}} {\mbuL} {\ch\schwaL\schwaH{ŋ}} {\ds\textscb\schwaH} {\ds\la} \\
             {1.\tsc{sg}} {ask} {Numi} {\tsc{comp}} {food} {be.cooked} {\tsc{la}} \\
        \glt `I ask Numi if the food is ready.'
    \end{xlist}
\z

The final [\la] element is obligatory with the clause-initial HL-complementizer. This appears to parallel two other instances: the floating H-tone present in the LH-complementizer, and the /nd{à}/-form complementizer /\la/. Given that there are two elements in all these cases, that these elements appear to delineate the embedded clause, and further that they interact directly in the case of the floating H tone, we will assume that these elements are all local to each other. That is, syntactically, they are all articulations of the CP-domain. Such an approach might invoke the work done in the Cartographic approach, which assumes multiple functional projections within the CP-domain, each with a dedicated function \citep{Rizzi1997}. While our proposal is not immediately incompatible with the details of this approach, the Cartographic CP is decomposible into multiple functional projections, which are mappings between syntactic position and function. We will put aside the question of how this analysis might translate into a Cartographic approach for further research.

This assumption of having two CP-elements is further motivated by the presence of these two elements in other semantically distinct contexts, such as relative clauses \citep{Kouankem2011} and in other syntactically distinct contexts, such as the DP-domain and its articulation \citep{Kouankem2011, Kouankem2012}.

\subsection{Clause-final /\la/}

As described earlier, the clause-final element /\la/ is obligatory at the end of CPs introduced by [\mbuHL]-HL and [\nda]-L. It uniformly carries H-tone. The /\la/ particle also appears in several other environments and is not exclusive to complementizer contexts. When /\la/ occurs, it is always obligatory. That is, the co-occurrence with the /mb\baru/-form complementizers is strict: When the particle is grammatical, it must be present and in constructions where it is not present, it is ungrammatical. The nature and function of the /\la/ particle need to be further investigated.

In questions that contain the [\mbuHL]-HL C, /\la/ is in complementary distribution with the Q-particle /k{í}/. This supports our previous assumption that the H-tone /\la/ and /k{í}/ contribute to the syntax of the CP. As the two particles are in complementary distribution with the floating H-tone as we showed earlier, we assume that the H-tone carried by /\la/ and /k{í}/ is inherent to them. As further evidence, there are no instances of a L-tone /\la/ or /k{í}/ in our research.

As mentioned in \sectref{secHL}, [\mbuHL]-HL can be understood as introducing embedded polar statements. Consider the data in (\ref{laex}a--d), where (\ref{laexa}) contains a statement while (\ref{laexb}) outlines a direct question. As can be seen in the cited examples, the clause-final particle /\la/ and the polar question marker /k{í}/ are in complementary distribution.

\ea \label{laex}
    \begin{xlist}
    \ex[]{ \label{laexa}
        \gll {m\baruH} {l\epL n} {\mbuHL} {{á}} {l\epL gd\schwaL\schwaH} {bʰ{ú}\tgl{ŋ}w{à}nì} {\ds\la} \\
             {1.\tsc{sg}} {know} {\tsc{comp}} {3.\tsc{sg}} {forget} {packet.school} {\tsc{la}} \\
        \glt `I know if he forgot the book.'}
    \ex[]{ \label{laexb}
        \gll {{ú}} {l\epL n} {\mbuHL} {{á}} {l\epL gd\schwaL\schwaH} {bʰ{ú}\tgl{ŋ}w{à}nì} {k{í}} \\
             {1.\tsc{sg}} {know} {\tsc{comp}} {3.\tsc{sg}} {forget} {packet.school} {\tsc{q}} \\
        \glt `Would you know if he forgot the book?' (lit. `Would you know (or not) if he forgot the book (or not)?')}
    \ex[*]{ \label{laexc}
        \gll {{ú}} {l\epL n} {\mbuHL} {{á}} {l\epL gd\schwaL\schwaH} {bʰ{ú}\tgl{ŋ}w{à}nì} {\la} {k{í}} \\
             {1.\tsc{sg}} {know} {\tsc{comp}} {3.\tsc{sg}} {forget} {packet.school} {\tsc{la}} {\tsc{q}} \\
        \glt Intended: `Would you know (or not) if he forgot the book (or not)?'}
    \ex[*]{ \label{laexd}
        \gll {{ú}} {l\epL n} {\mbuHL} {{á}} {l\epL gd\schwaL\schwaH} {bʰ{ú}\tgl{ŋ}w{à}nì} {k{í}} {\la} \\
             {1.\tsc{sg}} {know} {\tsc{comp}} {3.\tsc{sg}} {forget} {packet.school} {\tsc{q}} {\tsc{la}} \\
        \glt Intended: `Would you know (or not) if he forgot the book (or not)?'}
    \end{xlist}
\z

While /\la/ appears to only interact with the C of the embedded clause, the presence of /k{í}/ seems to also affect the matrix clause. Although polarity is indicated through the embedded clause, it may also be found within the matrix clause. The question particle induces a polarity reading in the matrix clause. Lacking the question particle, the embedded clause is still polar, if it contains [\mbuHL]-HL. This /k{í}/-independent polarity is likely linked to the HL-form complementizer, as /\la/ is not limited to these contexts. Due to our lack of data concerning the scoping behaviour of /k{í}/, this topic should be investigated further with new data. Additionally, it might also be fruitful to further investigate if [\mbuHL]-HL is composed of an underlying [mbʉ̀]-L plus a floating tone like its deontic counterpart. In this scenario, which we will follow further in section 4, the floating H tone would realize polarity. As a result, it would be necessary to investigate how the polarity on the complementizer and the question particle interact, if this could explain the unusual scope of polarity in (\ref{laex}) and what implications it would have to assume multiple, phonologically similar, but syntactically and semantically distinct floating tones for modality and polarity.

\section{Deriving the CP}

Taking into account the above discussion, this section aims to provide a sketch for the derivation of the Medumba CP. The presence of (at least) two projections will be taken for granted on the basis of the discussion in the previous sections. Additionally, the distributional clues of the elements within those projections provides us with their potential syntactic positions.

First, consider the floating H-tone of the LH-complementizer. This, assuming that it linearly \textit{follows} the underlying L-complementizer and that linear order is a heuristic for syntactic position, gives us the following syntactic structure in (\ref{struc1}). Furthermore, we will assume that Medumba is uniformly head-initial and that all linearization that deviates from this is a result of movement \citep{Kayne1994}.

\ea \label{struc1}
    {[$_{\text{CP}}$ \text{ {\mbuL} } [$_{\text{CP}}$ \text{ H TP }]]}
\z

The above derivation shows two available positions within the CP.\footnote{We would like to thank an anonymous reviewer for pointing out a bracketing issue that arises here: As the deontic H-tone is below the [\mbuL] complementizer, it should not be available for selection by a \textsc{pav}. This remains a problem, which the phonological process of tone docking itself is unable to solve. One possible stipulative solution is to treat the H-tone as an affix, moving it and adjoining its formal features to the upper CP.} Naturally then, the /\la/ particle, which delineates the right edge of embedded clauses, might head this lower CP projection. This is further suggested by the fact that it is in complementary distribution with the deontic H-tone of the LH-complementizer, as discussed previously. Under the model in (\ref{struc1}) we have, as desired, a local relationship between the deontic H-tone and the complementizer [\mbuL]. In the case of /\la/ heading the lower CP, this requires movement of the TP rightward past /\la/ as /\la/ is a clause-final element. The exact mechanism responsible is a question for future research.

This raises questions about the status of the polarity H-tone of the HL complementizer. Since this co-occurs with the /\la/ complementizer and its surface realization suggests that it precedes the underlying L-complementizer, we might suggest an additional position above the structure, as in (\ref{struc2}).

\ea \label{struc2}
    {[$_{\text{XP}}$ \text{ (H) } [$_{\text{CP}}$ \text{ {\mbuL} } [$_{\text{CP}}$ \text{ {\la} TP }]]]}
\z

Given its semantic force, we assume that this upper position occupied by the polarity-inducing H-tone might be an optional PolP head. However, such detailed questions about the CP-structure are subject to further investigation.

\section{Concluding remarks and remaining questions}

While this paper is far from presenting a complete analysis of the complementizer system in Medumba, we hope this discussion has provided a motivation for the presence of syntactically conditioned tone by focusing on the CP domain. By analyzing the impact of \textsc{pav}s and how they interact with  complementizers, we realize that there is much to be gained from looking at some tones in Medumba as properly syntactic rather than pushing off all tonal alternations to the phonology, which can possibly be extended to other domains of the syntax. Furthermore, the interaction of complementizers, grammatical tone and a small inventory of \textsc{pav}s proves to be a successful strategy to bridge the gap to the semantic possibilities that are lexical in languages with a richer \textsc{pav} inventory.

Major questions that have arisen during our research include understanding the presence of /\la/ and how the scope of polarity works in Medumba. /\la/ holds an important clause-final position, but at this time we cannot conclusively say whether or not /\la/ holds a definitive role in relation with any of the complementizers. Despite tone acting as a main divisor between meanings and allomorphs, we cannot definitively state whether or not it holds any power as to what determines the selection of what gets read as a propositional attitude reading. Additionally, as the scopal behaviour of polarity is still diffuse and we cannot define a proper position for the PolP so far, we suggest further investigation on this topic.

\section*{Abbreviations}
\begin{tabular}[t]{@{}ll@{}}
\textsc{iter} & iterative \\
\textsc{recp} & reciprocal
\end{tabular}

\section*{Acknowledgements}

We would like to thank Hermann Keupdjio for sharing his language with us, Rose-Marie Déchaine for her guidance, the UBC LING 431/432/531/532 class of 2017\slash 2018 for their comments and discussions, and the attendees of ACAL\,49 for their comments. All remaining errors of data and interpretation are the authors'.

{\printbibliography[heading=subbibliography,notkeyword=this]}

\end{document}
