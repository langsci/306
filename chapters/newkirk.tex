\documentclass[output=paper,colorlinks,citecolor=brown]{langscibook}

\title{Unifying prolepsis and cross-clausal cliticization in Lubukusu}

\author{Lydia Newkirk\affiliation{Rutgers University}}

\abstract{This paper examines proleptic constructions in Lubukusu (Bantu, Western Kenya). I find that Lubukusu has two distinct strategies for prolepsis: one where the extra matrix nominal is base-generated high, and one where the nominal moves from the embedded clause to the matrix position. The latter is subject to island effects, whereas the former is not. I propose an analysis for these two kinds of prolepsis based on these facts, dependant on the particularities of what nominals can be licensed in what syntactic positions in Lubukusu, and explore the cross-linguistic implications of this analysis.}

\IfFileExists{../localcommands.tex}{%hack to check whether this is being compiled as part of a collection or standalone
  \addbibresource{localbibliography.bib}
  \usepackage{langsci-optional,langsci-branding}
\usepackage{langsci-gb4e}
% \usepackage{langsci-textipa}
% \usepackage{langsci-glyphs}
\usepackage[linguistics]{forest}
\usepackage{tabto}
\usepackage{multirow}
\usepackage{bbding}

\usepackage[normalem]{ulem}

\usepackage{tikz-qtree}

\usepackage{enumitem}

\usepackage{multicol}
\usepackage{stmaryrd} %double brackets

\makeatletter
\let\pgfmathModX=\pgfmathMod@
\usepackage{pgfplots,pgfplotstable}%
\let\pgfmathMod@=\pgfmathModX
\makeatother
\usepgfplotslibrary{colorbrewer}
\usetikzlibrary{fit}

\usepackage{jambox}
\usepackage{tikz-qtree-compat}
\usetikzlibrary{arrows, arrows.meta}
\usepackage{longtable}
\usepackage{subcaption}

  \makeatletter
\let\thetitle\@title
\let\theauthor\@author
\makeatother

\newcommand{\togglepaper}[1][0]{
%   \bibliography{../localbibliography}
  \papernote{\scriptsize\normalfont
    \theauthor.
    \thetitle.
    To appear in:
    Change Volume Editor \& in localcommands.tex
    Change volume title in localcommands.tex
    Berlin: Language Science Press. [preliminary page numbering]
  }
  \pagenumbering{roman}
  \setcounter{chapter}{#1}
  \addtocounter{chapter}{-1}
}

\newcommand{\bari}{\ipabar{\i}{.5ex}{1.1}{}{}}
\newcommand{\notipa}[1]{\textnormal{#1}}

\newcommand{\agre}{\textsc{agr}-\ol{eene}}

\renewcommand{\emph}[1]{\textit{#1}} % resetting a setting from ling-macros-modified (I think?)

% forest settings to make compact but (mostly) straight-spined trees:
\forestset{
fairly nice empty nodes/.style={
            delay={where content={}{shape=coordinate,for parent={
                  for children={anchor=north}}}{}}
, angled/.style={content/.expanded={$<$\forestov{content}$>$}}
}}

\forestset{sn edges/.style={for tree={parent anchor=south, child anchor=north}}}

\newcommand{\bex}{\begin{exe}}
\newcommand{\fex}{\end{exe}}

\newcommand{\bxl}{\begin{exe}}
\newcommand{\fxl}{\end{exe}}

\newcommand{\ix}[1]{\textsubscript{#1}}
\newcommand{\alert}[1]{\textbf{#1}}
\newcommand{\ol}[1]{\textit{#1}}


			\usetikzlibrary{shapes,arrows,positioning,decorations,decorations.pathmorphing,intersections}
\forestset{
nice empty nodes/.style={
    for tree={calign=fixed edge angles},
    delay={where content={}{shape=coordinate,for siblings={anchor=north}}{}}
},
}

\definecolor{dark-gray}{gray}{0.3}

%\usepackage{dingbat,pifont}


%%%%%%%%%%%%For arrows%%%%%%%%%%%%%

\newcommand\Tikzmark[2]{%
  \tikz[remember picture]\node[inner sep=0pt,outer sep=0pt] (#1) {#2};%
}
\NewDocumentCommand\DrawArrow{O{}mmmmO{3}}{
\tikz[remember picture,overlay]
  \draw[->,line width=0.8pt,shorten >= 2pt,shorten <= 2pt,#1]
    (#2) -- ++(0,-#6\ht\strutbox) coordinate (aux) -- node[#4] {#5} (#3|-aux) -- (#3);
}
\NewDocumentCommand\DrawDotted{O{}mmmmO{3}}{
\tikz[remember picture,overlay]
  \draw[->,line width=0.9pt,dotted,shorten >= 2pt,shorten <= 2pt,#1]
    (#2) -- ++(0,-#6\ht\strutbox) coordinate (aux) -- node[#4] {#5} (#3|-aux) -- (#3);
}
\NewDocumentCommand\DrawLine{O{}mmmmO{3}}{
\tikz[remember picture,overlay]
  \draw[line width=0.8pt,shorten >= 2pt,shorten <= 2pt,#1]
    (#2) -- ++(0,-#6\ht\strutbox) coordinate (aux) -- node[#4] {#5} (#3|-aux) -- (#3);
}
%%%%%%%%%%%%%%%%%%%%%%%%%%%%%%%%%%%%%


\newcommand{\baru}{ʉ}
\newcommand{\baruH}{\'\baru}
\newcommand{\baruL}{\`\baru}

\newcommand{\ep}{ε}
\newcommand{\epH}{\'\ep}
\newcommand{\epL}{\`\ep}

\newcommand{\schwa}{ə}
\newcommand{\schwaH}{\'ə}
\newcommand{\schwaL}{\`ə}

\newcommand{\oo}{ɔ}
\newcommand{\ooH}{\'\oo}
\newcommand{\ooL}{\`\oo}

\newcommand{\ds}{\textsuperscript{
	\hspace*{-2pt}\begin{tikzpicture}
		\draw[-{>[scale=0.5]}] (0,0.4) --(0,0.25);
	\end{tikzpicture}}}

\newcommand{\ch}{t͡ʃ}
\newcommand{\dz}{d͡ʒ}

\newcommand{\tgl}{ʔ}

%shortcuts for the complementizers
\newcommand{\mbuL}{mb\baruL}
\newcommand{\mbuHL}{mb\baruH\baruL}
\newcommand{\mbuLH}{mb\baruL\baruH}
\newcommand{\la}{lá}
\newcommand{\nda}{ndà}

\newcommand{\tsc}[1]{\textsc{#1}}
\renewcommand{\textscb}{ʙ}
\newcommand{\ipa}[1]{#1} %disable IPA

\newcommand{\SM}[1]{#1}

\DeclareNewSectionCommand
  [
    counterwithin = chapter,
    afterskip = 2.3ex plus .2ex,
    beforeskip = -3.5ex plus -1ex minus -.2ex,
    indent = 0pt,
    font = \usekomafont{section},
    level = 1,
    tocindent = 1.5em,
    toclevel = 1,
    tocnumwidth = 2.3em,
    tocstyle = section,
    style = section
  ]
  {appendixsection}

\renewcommand*\theappendixsection{\Alph{appendixsection}}
\renewcommand*{\appendixsectionformat}
              {\appendixname~\theappendixsection\autodot\enskip}
\renewcommand*{\appendixsectionmarkformat}
              {\appendixname~\theappendixsection\autodot\enskip}

\renewcommand{\lsChapterFooterSize}{\footnotesize}

\togglepaper[23]
}{}

\begin{document}
\SetupAffiliations{mark style=none}
\maketitle

\section{Background}\label{sec:newkirk:1}

Before proceeding to the main description and analysis of Lubukusu prolepsis, it will be useful to briefly introduce both prolepsis as a phenomenon, and provide some preliminary background on Lubukusu as a language.

\subsection{Prolepsis}\label{sec:newkirk:1.1}

Prolepsis, as characterized in \citet{Salzmann2017Chapter}, is a multiclausal construction in which a verb that normally takes an embedded finite clause apparently takes an additional nominal argument (the proleptic object) in the matrix clause, often accompanied by a preposition as in German \REF{ex:newkirk:1}, but sometimes licensed with case marking, as in Middle Dutch \REF{ex:newkirk:2}.\pagebreak

\ea%1
    \label{ex:newkirk:1}
    German \citep{Salzmann2017Chapter}\\
	\gll    Ich glaube \alert{von} \alert{ihm}\ix{i}, dass \alert{er}\ix{i} ein ganz guter Trainer ist.\\
			I believe.\textsc{1sg} of he.\textsc{dat} that he a quite good coach be.\textsc{3sg}\\
	\glt    `I believe of him\ix{i} that he\ix{i} is a pretty good coach.'
\ex%2
    \label{ex:newkirk:2}
    Middle Dutch \citep{VanKoppenEtAl2016}\\
	\gll    Maer die serjanten sijn kenden [\alert{den} \alert{coninc}                   \alert{van} \alert{Isra\"{e}l}]\ix{i}, dat \alert{hi}\ix{i} niet was         harde fel.\\
	       but the sergeants his knew \hphantom{[}the.\textsc{acc} king of Israel that he.\textsc{nom} not was very fierce\\
	\glt   \textit{lit.} `But his sergeants knew the king of Israel that he was          not very fierce.'\\
	        `But his sergeants knew about [the king of Israel]\ix{i} that he wasn't very fierce.'
\z

Proleptic objects have most typically been analyzed as base-generation of the proleptic object either in the matrix clause (as in \citealt{Salzmann2017Chapter}) or in the left periphery of the embedded clause (as in \citealt{VanKoppenEtAl2016}), with an obligatory coreference requirement between the proleptic object and some pronominal in the embedded clause. Prolepsis without such an embedded pronominal is degraded, as in the English example below:

\ea[??]{%3
    \textit{Mary thinks of dinner that John will cook fish tonight.}}
\z

\subsection{Lubukusu}\label{sec:newkirk:1.2}

Lubukusu (Bantu J, Western Kenya) utilizes a set of prefixes on its verbs to indicate the noun class of its subject \REF{ex:newkirk:4}.\footnote{Many of the Lubukusu examples in this paper is from the Afranaph Project. For those sentences I have marked their sentence ID for lookup in the Afranaph database. Other examples I have drawn from the Lubukusu literature, and are marked accordingly. Examples without an accompanying citation are from my own field work. I am indebted to Dr.\ Justine Sikuku for his patience and assistance by providing me with the data. \nocite{SafirSikuku2011}}

\ea%4
    \label{ex:newkirk:4}
    \gll    Wekesa \alert{a}-a-kul-a sii-tabu.\\
			Wekesa \textsc{sm.c1-pst-}buy\textsc{-fv} \textsc{c7-}book\\
	\glt    `Wekesa bought a book.'\hfill \citep[ex. 11a]{Wasike2006}
\z

Lubukusu also has a set of object-marking prefixes, but in neutral contexts they cannot cooccur with an overt object, unless that object is a pronoun. This has led \citet{DiercksSikuku2015, SikukuEtAl2018} to analyze the object marker as an incorporated pronoun/clitic, rather than an agreement morpheme.

\ea%5
    \label{ex:newkirk:5}
    \gll    N-a-\alert{mu}\ix{i}-bon-a (\#Wekesa\ix{i}).\\
			\textsc{1sg.s-pst-om.c1-}see\textsc{-fv} \hphantom{(\#}Wekesa\\
	\glt    `I saw him.' \hfill \citep[ex. 2]{DiercksSikuku2015}
\z

The third person pronominal \ol{niye} \emph{can} cooccur with verbal object marking, however:

\ea%6
    \label{ex:newkirk:6}
    \gll    Wekesa a-a-\alert{mu}\ix{i}-p-a (\alert{niye}\ix{i}). \\
			Wekesa \textsc{sm.c1-pst-\alert{om.c1}-}beat\textsc{-fv} \alert{him} \\
	\glt    `Wekesa beat him.' \hfill (Afranaph ID: 3734/5039)
\z

This is in line with the generalizations in \citet{Anagnostopoulou2016, Anagnostopoulou2017}, in that even languages which do not allow clitics to double full DP objects allow doubling for overt object pronouns.

Lubukusu also marks reflexivity on the verb, where an invariant reflexive marker (\textsc{refl}) occurs in the same position as the OM. A pronoun which takes noun class agreement matching its antecedent may also cooccur with the \textsc{refl}, which surfaces as \ol{i-} regardless of the noun class of its antecedent. The \textsc{refl} alone is sufficient to establish reflexivity, so the agreeing anaphor is optional. The Lubukusu \textsc{refl} is also analyzed as an incorporated pronoun in line with Lubukusu's object markers, given its similar syntactic behavior.

\ea%7
    \label{ex:newkirk:7}
    \gll    Y\`oh\'an\'a\ix{i} \'a-\'a-\alert{i}\ix{i}-bon-a (o-mu-eene\ix{i}). \\
			Yohana \textsc{sm.c1-pst-\alert{refl}-}see\textsc{-fv} \textsc{c1-c1-}own \\
	\glt    `John\ix{i} saw himself\ix{i}.' \hfill (Afranaph ID:1248/1249)
\z

The \agre\ pronoun can also occur without an accompanying \textsc{refl}, in which case it cannot take a local antecedent, but is allowed to take a discourse antecedent:

\ea%8
    \label{ex:newkirk:8}
    \gll    Billi\ix{i} a-a-bon-a o-mu-eene\ix{k/\**i}. \\
			Billi \textsc{sm.c1-pst}-see\textsc{-fv} \textsc{c1-c1-}own \\
	\glt    `Bill\ix{i} saw him\ix{k/\**i}.' \hfill (ID: 1367)
\z

\ea%9
    \label{ex:newkirk:9}
    \gll    Jack\ix{i} a-many-il-e a-li George\ix{j} a-\alert{mu}\ix{i/k}-siim-a \alert{o-mu-eene}\ix{i/k}. \\
		    Jack \textsc{sm.c1-}knows\textsc{-tns-fv} \textsc{c1-}that George \textsc{sm.c1-\alert{om.c1}-}like\textsc{-fv} \alert{\textsc{c1-c1-}own} \\
	\glt    `Jack\ix{i} knows that George\ix{j} likes him\ix{i/k}.'
\z

These pieces in place, I now proceed to give a description of prolepsis in Lubukusu.

\section{Prolepsis in Lubukusu}\label{sec:newkirk:2}

In Lubukusu there are three ways to license a proleptic object: first, a proleptic object can be introduced with a preposition \REF{ex:newkirk:10}, as is the case in English. Second, there is an equivalent construction with an applicative morpheme \REF{ex:newkirk:11}.

\ea%10
    \label{ex:newkirk:10}
	\gll    John a-subil-a \alert{khu} \alert{Bill}\ix{i} a-li \alert{o-mu-eene}\ix{i}/\alert{niye}\ix{i} a-li o-mu-miliyu.\\
			John \textsc{sm.c1-}believe\textsc{-fv} \textsc{prep} Bill \textsc{c1-}that \textsc{c1-c1-}own/him \textsc{c1-}be \textsc{c1-c1-}smart\\
	\glt    `John believes of Bill\ix{i} that he\ix{i} is smart/clean.' 
	        \hfill (Lubukusu)
\z

\ea%11
    \label{ex:newkirk:11}
    \gll    John a-kanakan-\alert{il}-a \alert{Jane}\ix{i} a-li Bill a-mu-siim-a \alert{o-mu-eene}\ix{i}/\alert{niye}\ix{i}.\\
			John \textsc{sm.c1-}think\textsc{-appl-fv} Jane \textsc{c1-}that Bill \textsc{sm.c1-om.c1-}like\textsc{-fv} \textsc{c1-c1-}own/her\\
	\glt `John thinks of Jane\ix{i} that Bill likes her\ix{i}.'
\z

Third, it is also possible for a proleptic object to be a reflexive pronoun, in German, English, and Lubukusu, but crucially the Germanic cases still require that a preposition introduce the proleptic object, while in Lubukusu the preposition is optional (\ref{ex:newkirk:14}):

\ea%12
    \label{ex:newkirk:12}
    German \citep[ex. 12a]{Salzmann2017Chapter}\\
    \gll    dass Peter\ix{i} von \alert{sich}\ix{i} denkt, dass \alert{er}\ix{i} der Gr\"o\ss te ist.\\
	        that Peter of self thinks that he the greatest be.\textsc{3sg}\\
	\glt    `that Peter\ix{i} thinks of himself\ix{i} that he is the greatest.'
\z

\ea%13
    \label{ex:newkirk:13}
    \gll    John\ix{i} a-lom-a \alert{khu-mu-eene}\ix{i} a-li Bill a-khaenj-a [o-mu-undu o-wa-mu-lip-a \alert{o-mu-eene}\ix{i}].\\
			John \textsc{sm.c1-}say\textsc{-fv} \textsc{prep-c1-}own \textsc{c1-}that Bill \textsc{sm.c1-}look.for\textsc{-fv} \textsc{c1-c1-}person \textsc{wh-c1-om.c1-pst-}pay\textsc{-fv} \textsc{c1-c1-}own\\
	\glt    `John\ix{i} said about himself\ix{i} that Bill is looking for the person who paid himself\ix{i}.'
\z

\ea%14
    \label{ex:newkirk:14}
    \gll    Jack\ix{i} a-\alert{i}\ix{i}-many-il-e a-li George a-mu\ix{i}-siim-a \alert{o-mu-eene}\ix{i}.\\
		    Jack \textsc{sm.c1-refl-}knows\textsc{-tns-fv} \textsc{c1-}that George \textsc{sm.c1-om.c1-}like\textsc{-fv} \textsc{c1-c1-}own\\
	\glt    `Jack\ix{i} knows that George likes him\ix{i}.' 
	        \hfill (Afranaph ID 3759)
\z

\ea%15
    \label{ex:newkirk:15}
    \gll    Jack\ix{i} a-\alert{i}\ix{i}-kanakan-il-a \alert{o-mu-eene}\ix{i} a-li Lisa a-many-il-e a-li Wendy a-mu\ix{i}-siim-a \alert{o-mu-eene}\ix{i}. \\
			Jack \textsc{sm.c1-refl-}think\textsc{-appl-fv} \textsc{c1-c1-}own \textsc{c1-}that Lisa \textsc{sm.c1-}know\textsc{-tns-fv} \textsc{c1-}that Wendy \textsc{sm.c1-om.c1-}like\textsc{-fv} \textsc{c1-c1-}own \\
	\glt    `Jack\ix{i} thought for himself\ix{i} that Lisa thinks that Wendy likes him\ix{i}.'
\z

In \REF{ex:newkirk:14} there is no \agre\ in the matrix clause, as the invariant \textsc{refl} suffices to mark reflexivity, though \REF{ex:newkirk:15} demonstrates that \agre\ can occur both in the embedded clause and in the matrix clause. In \REF{ex:newkirk:13} however, there is no \textsc{refl} on the matrix verb, and instead there is an overt proleptic object in the matrix clause, which does not participate in clitic doubling on the matrix verb, and has an (optional) embedded resumptive pronoun. Similar constructions are possible with a matrix (third person, non-reflexive) object marker rather than the reflexive marker, although it is degraded when the embedded object marker is in object position:

\ea[]{%16
    \label{ex:newkirk:16}
    \gll    John a-a-mu-lom-a a-li o-mu-eene a-a-siim-a Mary.\\
			John \textsc{sm.c1-pst-om.c1-}say\textsc{-fv} \textsc{c1-}that \textsc{c1-c1-}own \textsc{sm.c1-pst-}like\textsc{-fv} Mary\\
	\glt    `John said about him\ix{i} that he\ix{i} likes Mary.'}
\z

\ea[]{%17
    \label{ex:newkirk:17}
    \gll    John a-a-mu-lom-a a-li Mary a-a-lom-a khu o-mu-eene.\\
			John \textsc{sm.c1-pst-om.c1-}say\textsc{-fv} \textsc{c1-}that Mary \textsc{sm.c1-pst-}say\textsc{-fv} of \textsc{c1-c1-}own\\
	\glt    `John said about him\ix{i} that Mary speaks of him\ix{i}.'}
\z

\ea[?]{%18
    \label{ex:newkirk:18}
    \gll    John a-a-mu-lom-a a-li George a-mu-siima.\\
			John \textsc{sm.c1-pst-om.c1-}say\textsc{-fv} \textsc{c1-}that George \textsc{sm.c1-om.c1-}like\textsc{-fv}\\
	\glt    `John say of him\ix{i} that George likes him\ix{i}.'}
\z

Constructions with \ol{khu-mu-eene} in the matrix clause are insensitive to locality, whereas the construction with the \textsc{refl}/\textsc{om} cliticized to the matrix verb is sensitive to island boundaries, here shown with the \textsc{refl} on matrix verb:

\ea[]{%19
    \label{ex:newkirk:19}
    \gll    John\ix{i} a-lom-a \alert{khu-mu-eene}\ix{i} a-li Bill a-khaenj-a [o-mu-undu o-wa-mu-lip-a \alert{o-mu-eene}\ix{i}].\\
			John \textsc{sm.c1-}say\textsc{-fv} \textsc{prep-c1-}own \textsc{c1-}that Bill \textsc{sm.c1-}look.for\textsc{-fv} \textsc{c1-c1-}person \textsc{wh-c1-om.c1-pst-}pay\textsc{-fv} \textsc{c1-c1-}own\\
	\glt    `John\ix{i} said about himself\ix{i} that Bill is looking for the person who paid himself\ix{i}.'}
\z			
			
\ea[*]{%20
    \label{ex:newkirk:20}
    \gll    John\ix{i} a-\alert{i}\ix{i}-lom-a a-li Bill a-khaenj-a [o-mu-undu o-w-a-mu-lip-a \alert{o-mu-eene}\ix{i}].\\
			John \textsc{sm.c1-refl-}say\textsc{-fv} \textsc{c1-}that Bill \textsc{sm.c1-}look.for\textsc{-fv} \textsc{c1-c1-}person \textsc{wh-c1-om.c1-pst-}pay\textsc{-fv} \textsc{c1-c1-}own\\
	\glt    `John\ix{i} said that Bill is looking for [the person who paid himself\ix{i}].'}
\z

\ea[*]{%21
    \label{ex:newkirk:21}
    \gll    John\ix{i} a-\alert{i}\ix{i}-lom-a a-li o-mu-eene\ix{i} a-rekukh-a [{paata ya} Mary khu-mu-khuu-p-a \alert{o-mu-eene}\ix{i}].\\
	        John \textsc{sm.c1-refl-}say\textsc{-fv} \textsc{c1-}that \textsc{c1-c1-}own \textsc{sm.c1-}leave\textsc{-fv} after Mary \textsc{c15-om.c1-c15?-}hit\textsc{-fv} \textsc{c1-c1-}own\\
	\glt    `John\ix{i} said that he\ix{i} left [after Mary hit him\ix{i}].'             \hfill (Adjunct island)}
\z

\ea[*]{%22
    \label{ex:newkirk:22}
    \gll    Jack a-i-many-il-e a-li George a-ch-a nge a-mu-bon-a o-mu-eene.\\
			Jack \textsc{sm.c1-refl-}know\textsc{-appl?-fv} \textsc{c1-}that George \textsc{sm.c1-}leave\textsc{-fv} when \textsc{sm.c1-om.c1-}see\textsc{-fv} \textsc{c1-c1-}own\\
	\glt    `Jack\ix{i} knows that George left when he saw himself\ix{i}.'
	        \hfill (Adjunct island)}
\z

\ea[*]{%23
    \label{ex:newkirk:23}
    \gll    Bill\ix{i} a-\alert{i}\ix{i}-nyol-a [chilomo mbo John a-mu-lip-a \alert{o-mu-eene}\ix{i}].\\
			Bill \textsc{sm.c1-}receive\textsc{-fv} information that John \textsc{sm.c1-om.c1-}pay\textsc{-fv} \textsc{c1-c1-}own\\
	\glt    `Bill\ix{i} heard [a rumor (about himself\ix{i}) that John paid him\ix{i}].' 
	        \hfill (CNPC)}
\z

\ea[*]{%24
    \label{ex:newkirk:24}
    \gll    John\ix{i} a-\alert{i}\ix{i}-subil-a [likhuwa mbo Bill a-mu-bon-a \alert{o-mu-eene}\ix{i}].\\
			John \textsc{sm.c1-refl-}hope\textsc{-fv} claim that Bill \textsc{sm.c1-om.c1-}see\textsc{-fv} \textsc{c1-c1-}own\\
	\glt    `John\ix{i} believes [the claim that Bill saw himself\ix{i}].' 
	        \hfill (CNPC)}
\z

And similarly with the OM on the verb, embedding \agre\ inside of an island is degraded:

\ea[*]{%25
    \label{ex:newkirk:25}
    \gll    John a-a-mu\ix{i}-lom-a a-li George a-khaenj-a [o-muu-ndu o-w-a-mu-lip-a o-mu-eene].\\
			John \textsc{sm.c1-pst-om.c1-}say\textsc{-fv} \textsc{c1-}that George \textsc{sm.c1-}look.for \textsc{c1-c1-}person \textsc{wh-c1-pst-}pay\textsc{-fv} \textsc{c1-c1-}own\\
	\glt    `John said of him\ix{i} that George is looking for [the person who paid him\ix{i}].'}
\z

\ea[?]{%26
    \label{ex:newkirk:26}
    \gll    John a-\alert{mu}\ix{i}-lom-a a-li o-mu-eene\ix{i} a-rekukh-a [{paata ya} Mary khu-mu-khuu-p-a \alert{o-mu-eene}\ix{i}].\\
            John \textsc{sm.c1-om.c1-}say\textsc{-fv} \textsc{c1-}that \textsc{c1-c1-}own \textsc{sm.c1-}leave\textsc{-fv} after Mary \textsc{c15-om.c1-c15?-}hit\textsc{-fv} \textsc{c1-c1-}own\\
	\glt    John said of him\ix{i} that he\ix{i} left after Mary hit him\ix{i}.'}
\z

These correlate with the island/locality constraints for \ol{wh}-movement in Lubukusu. The following are the corresponding island examples from \citet{Wasike2006}:

\ea[*]{%27
    \label{ex:newkirk:27}
    \gll    Naanu ni-y-e Wafula a-kha-enj-a [o-muu-ndu o-w-a-kul-a].\\
			who \textsc{pred-c1-pron} Wafula \textsc{c1-prs-}look.for\textsc{-fv} \textsc{c1-c1-}person \textsc{wh-c1-pst-}buy\textsc{-fv}\\
	\glt    `What is it that Wafula is looking for [the person who bought]?'}
\z

\ea[*]{%28
    \label{ex:newkirk:28}
    \gll    Naanu ni-y-e Nasike a-a-rekukh-a [paata ye $t$ khu-khuup-a Nanjala].\\
	        who \textsc{pred-c1-pron} Nasike \textsc{c1-pst-}leave\textsc{-fv} after of {} inf-beat\textsc{-fv} Nanjala\\
	\glt `Who is that Nasike left [after $t$ hitting Nanjala]?'}
\z

\ea[??]{%29
    \label{ex:newkirk:29}
    \gll    [Chi-lomo mbo Wafula a-a-ib-a si(ina) cha-a-chun-i-a] Nafula ku-mw-oyo?\\
			\textsc{c7-}report that Wafula \textsc{c1-pst-}steal\textsc{-fv} what \textsc{c7-pst-}hurt\textsc{-caus-fv} Nafula \textsc{pp-3-}heart\\
    \glt    `What did [the report that Wafula stole] hurt Nafula?'}
\z

Based on the demonstrated island restrictions, I take the cliticization strategy to be movement of a pronoun from its argument position in the embedded clause to the matrix clause, and the applicative and prepositional phrase strategies to be base-generation of a pronoun or DP in the matrix clause.
These same sentences are illicit without the appropriate embedded object marking, however:

\ea[*]{%30
    \label{ex:newkirk:30}
    \gll    John\ix{i} a-i\ix{i}-lom-a a-li Mary a-siim-a o-mu-eene\ix{i}.\\
			John \textsc{sm.c1-refl-}say\textsc{-fv} \textsc{c1-}that Mary \textsc{sm.c1-}like\textsc{-fv} \textsc{c1-c1-}own\\
	\glt    `John\ix{i} said that Mary likes him\ix{i}.'}
\z

\ea[*]{%31
    \label{ex:newkirk:31}
    \gll    John\ix{i} a-i\ix{i}-lom-a a-li George a-khaeknj-a o-muu-ndu o-wa-lip-a o-mu-eene\ix{i}.\\
	        John \textsc{sm.c1-refl-}say\textsc{-fv} \textsc{c1-}that George \textsc{sm.c1-}look.for\textsc{-fv} \textsc{c1-c1-}person \textsc{wh-c1-}pay\textsc{-fv} \textsc{c1-c1-}own\\
	\glt    `John\ix{i} said that George is looking for the person who paid him\ix{i}.'}
\z

The ungrammaticality of \REF{ex:newkirk:31} is unsurprising, given the general island sensitivity of this construction. \REF{ex:newkirk:30} shows that the embedded object marker is obligatory, a fact I will return to later. If the cliticization strategy is movement from the embedded clause to the matrix clause, I will have to explain why the embedded OM remains obligatory.

In summary, Lubukusu has three kinds of proleptic strategies:\footnote{An anonymous reviewer astutely observes that there are also a variety of embedded-clause strategies as well. These appear to be subject to the general constraints on object marking and pronominals in Lubukusu, which for reasons of space I will not explore here. The reader is referred to \citet{SikukuEtAl2018} for more in-depth discussion of Lubukusu object marking.}

\begin{itemize}
	\item The proleptic object accompanied by a preposition
	\item The proleptic object accompanied by an applicative marker
	\item The proleptic object as a reflexive marker (\textsc{refl}) without an accompanying applicative or preposition
\end{itemize}

Three main characteristics that are common across these constructions:

\begin{enumerate}
	\item An ``extra" nominal argument in the matrix clause, which the matrix verb does not ordinarily take
	\item A aboutness relation between the extra argument and the embedded predicate
	\item A specific (\ol{de se}-like) acquaintence relation between the extra argument and the matrix attitude holder
\end{enumerate}

I will conclude that characteristics 2 and 3 come about by the same process, and so I will consider them together. Characteristic 1 is a separate concern, so I will address it first.

\section{Nominal licensing}\label{sec:newkirk:3}

In analyzing the island-sensitive clitic-licensed prolepsis, I generally follow analyses of cross-clausal agreement in \citet{PolinskyPotsdam2001, Bruening2001, BraniganMacKenzie2002}. The embedded DP A$'$-moves to to the embedded left periphery. In Lubukusu, that pronoun can then undergo further A$'$-movement to cliticize to the matrix verb. I follow the analysis of clitics as incorporated pronouns from \citet{Matushansky2006, BakerKramer2016}, more specifically implemented in Lubukusu as in \citet{SikukuEtAl2018}.

On this analysis, \REF{ex:newkirk:14} has the preliminary structure in \REF{ex:newkirk:32}.

\begin{exe}
\exr{ex:newkirk:14}{
    \gll    Jack\ix{i} a-\alert{i}\ix{i}-many-il-e a-li George a-mu\ix{i}-siim-a \alert{o-mu-eene}\ix{i}.\\
		    Jack \textsc{sm.c1-refl-}knows\textsc{-tns-fv} \textsc{c1-}that George \textsc{sm.c1-om.c1-}like\textsc{-fv} \textsc{c1-c1-}own\\
	\glt    `Jack\ix{i} knows that George likes him\ix{i}.'}
\end{exe}

\ea%32
    \label{ex:newkirk:32}
    \scalebox{.75}{
\begin{forest}
fairly nice empty nodes
[TP
	[DP,name=specTP [Jack,name=Jack,roof] ]
	[{}
		[T]
		[$v$P
			[\sout{DP},name=specvP [\sout{Jack},name=vPJack,roof]]
			 [{}
				[$v$ [\ol{-i-}\\\textsc{refl},name=reduceme3] [\ol{a-i-many-il-e}\\know,name=v1]]
				[VP
					[V [\sout{\ol{many-il-e}}\\\sout{know},name=V1]]
					[CP
						[\sout{DP},name=specCP [\sout{\ol{o-mu-eene}}\\\sout{\agre},name=CPagre,roof] { \draw[->,semithick] () -| (reduceme3); }]
						[{}
							[C [\ol{a-li}]]
							[TP
								[\ol{George a-mu-siim-a}\\George likes \agre{},roof] { \draw[->,semithick] (.south) -| ($(.south)+(0,-1em)$) -| (CPagre); }
								% [DP,name=specTP2 [George,name=George,roof]]
								% [
									% [T]
									% [$v$P
										% [\sout{DP},name=specvP2 [\sout{George},name=George2,roof]]
										% [
												% [$v$ [OM,name=reduceme2] [likes,name=v2]]
												% [VP
													% [V [\sout{likes},name=V2]]
													% [DP,name=compV [\agre,name=agre3]]
												% ]
										% ]
									% ]
								% ]
							]
						]
					]
				 ]
			 ]
		]
	]
]
% \draw[->,semithick] (compV)..controls +(south east:5) and +(south west:3)..(reduceme2);
% \draw[->,semithick] (reduceme2)..controls +(south west:3) and +(south west:3)..(specCP);
% \draw[->,semithick] (specCP)..controls +(south west:3) and +(south west:3)..(reduceme3);
% % \draw[->,semithick,dashed] (v2)..controls +(south west:2) and +(west:2) ..(reduceme2)
% % node[very near end,left] {\scriptsize m-merger};
\draw[->,semithick,dashed] (v1.south)..controls +(south west:3) and +(west:2.5) ..($(reduceme3.west)+(0,0.5em)$)
node[midway,below left]{\scriptsize m-merger};
\end{forest}
    }
\z

The preposition-licensed and applicative-licensed cases, on the other hand, have a proleptic object that is base-generated in the matrix clause, introduced by a preposition or applicative, and then are related to the embedded pronoun by binding.

\begin{exe}
\exr{ex:newkirk:11}{
    \gll    John a-kanakan-\alert{il}-a \alert{Jane}\ix{i} a-li Bill a-mu-siim-a \alert{o-mu-eene}\ix{i}/\alert{niye}\ix{i}.\\
            John \textsc{sm.c1-}think\textsc{-appl-fv} Jane \textsc{c1-}that Bill \textsc{sm.c1-om.c1-}like\textsc{-fv} \textsc{c1-c1-}own/her\\
    \glt    `John thinks of Jane\ix{i} that Bill likes her\ix{i}.'}
\end{exe}

\ea%33
    \label{ex:newkirk:33}
    \scalebox{.75}{
    \begin{forest}
    fairly nice empty nodes
    [TP
    	[DP,name=specTP [John,name=Jack,roof] ]
    	[{}
    		[T]
    		[$v$P
    			[\sout{DP},name=specvP [\sout{John},name=vPJack,roof]]
    			 [{}
    				[$v$ [\ol{a-kana-kan-il-a}\\think-\textsc{appl},name=v1]]
    				[ApplP
    				[DP [Jane,roof]]
    				[
    				[Appl [\sout{\ol{-il-}}\\\sout{\textsc{appl}},name=appl]]
    				[VP
    					[V [\sout{\ol{a-kana-kan-il-a}}\\\sout{think},name=V1]]
    					[CP
    							[C]
    							[TP
    								[\ol{Bill a-mu-siim-a}\\Bill likes \agre{},roof]
    								% [DP,name=specTP2 [George,name=George,roof]]
    								% [
    									% [T]
    									% [$v$P
    										% [\sout{DP},name=specvP2 [\sout{George},name=George2,roof]]
    										% [
    												% [$v$ [OM,name=reduceme2] [likes,name=v2]]
    												% [VP
    													% [V [\sout{likes},name=V2]]
    													% [DP,name=compV [\agre,name=agre3]]
    												% ]
    										% ]
    									% ]
    								% ]
    							]
    					] % CP
    				 ]
    			 ] ] ]
    		]
    	]
    ]
    \draw[->,semithick] (V1)..controls +(south west:2) and +(south:1)..(appl);
    \draw[->,semithick] (appl)..controls +(south west:2) and +(south:2)..(v1);
    % \draw[->,semithick] (compV)..controls +(south east:5) and +(south west:3)..(reduceme2);
    % \draw[->,semithick] (reduceme2)..controls +(south west:3) and +(south west:3)..(specCP);
    % \draw[->,semithick] (specCP)..controls +(south west:3) and +(south west:3)..(reduceme3);
    % % \draw[->,semithick,dashed] (v2)..controls +(south west:2) and +(west:2) ..(reduceme2)
    % % node[very near end,left] {\scriptsize m-merger};
    % \draw[->,semithick,dashed] (v1)..controls +(south west:2) and +(west:2) ..(reduceme3)
    % node[midway,left] {\scriptsize m-merger};
    \end{forest}
    	}
\z

The movement strategy is restricted to pronouns due to independent facts about Lubukusu object marking. The object markers are clitics, and these clitics can only be doubled by pronouns, and not by full DPs:

\ea[]{%34
    \label{ex:newkirk:34}
    \gll    N-a-\alert{mu}\ix{i}-bon-a (\#Wekesa\ix{i}).\\
            \textsc{1sg.s-pst-om.c1-}see\textsc{-fv} Wekesa\\
    \glt    `I saw him.' 
            \hfill \citep[2]{SikukuEtAl2018}}
\z

\ea[]{%35
    \label{ex:newkirk:35}
    \gll    Wekesa a-a-\alert{mu}\ix{i}-p-a (\alert{niye}\ix{i}). \\
			Wekesa \textsc{sm.c1-pst-\alert{om.c1}-}beat\textsc{-fv} \alert{him} \\
	\glt    `Wekesa beat him.' 
            \hfill (Afranaph ID: 3734/5039)}
\z

\ea[]{%36
    \label{ex:newkirk:36}
    \gll    Y\`oh\'an\'a\ix{i} \'a-\'a-\alert{i}\ix{i}-bon-a (o-mu-eene\ix{i}). \\
            Yohana \textsc{sm.c1-pst-\alert{refl}-}see\textsc{-fv} \textsc{c1-c1-}own \\
    \glt    `John\ix{i} saw himself\ix{i}.' 
            \hfill (Afranaph ID:1248/1249)}
\z

In principle, a full DP could undergo movement to the matrix clause, but Lubukusu has no way of licensing it there by cliticization. There is no position for it to move to.\footnote{There may also be licensing concerns in terms of } At the same time, although prepositions can provide licensing to an additional matrix argument, they are not viable landing sites for movement, and so preclude movement of an embedded argument into their complement. The specifier of an applicative phrase is an eligible landing site for movement, but also for base-generation of a prolpetic object, so island effects are obviated in the presence of an applicative morpheme.

I can now offer a tentative explanation for why the embedded OM remains obligatory even in the movement cases. The embedded pronoun begins by receiving a theta role in the embedded clause, but while it is then syntactically licensed in the matrix clause via cliticization, it is not semantically licensed there. So the embedded clitic contains information about where (and from what) the embedded pronoun received semantic licensing, while the matrix clitic contains information about its syntactic licensing in the proleptic construction. Since the two copies contain different information, they both must be pronounced.

Since the distinction between movement-based and base-generated prolepsis ultimately rests on the particular nominal licensing strategies in Lubukusu, we should expect cross-linguistic variation along the lines of what types of nominals can be licensed in what position, and what that licensing strategy is: that is, what provides a syntactically appropriate place for the proleptic object to inhabit.

\section{Acquaintance relations}\label{sec:newkirk:4}

There are still several questions left to address, however. The obligatory binding relationship between base-generated proleptic objects and the embedded pronoun is so far unexplained, as is the topic-like interpretation found for all three types of prolepsis.

An important fact on the way to addressing these issues is that proleptic objects must always be read transparently \citep{Salzmann2006,Salzmann2017Chapter}.

\ea%37
    \label{ex:newkirk:37} \begin{context}
    Bill is walking down the street. He glances down a dark alley and sees a man in a trench coat talking into his watch. Bill, who reads too many thrillers, immediately thinks to himself ``That man is a spy." In reality, the man in the alley is Bill's friend Wayne, although Bill didn't recognize him.
    \end{context}
	\ea[\#]{\label{ex:newkirk:37a}
	    Bill thinks of Wayne\ix{i} that he\ix{i} is a spy.}
	\ex[]{\label{ex:newkirk:37b}
	    Bill thinks that Wayne is a spy.}
	\z
\z

Saying that the embedded clause is ``about'' the proleptic object is not sufficient to account for this data. The matrix attitude holder has to \emph{knowingly ascribe} the embedded predicate to the proleptic object, and properly identify the proleptic object as well.

The framework I will use as a starting point for these facts is from \citet{SpeasTenny2003}. They propose a set of projections in the left periphery to account for various perspectival phenomena. The projections include a Speech-Act Phrase (SAP), Evaluative Phrase (EvalP), and Evidential Phrase (EvidP). The projections host various null nominals that have a perspectival semantics, and can both bind embedded pronouns and be bound by higher nominals to force coreference. A sketch of their left periphery is in \REF{ex:newkirk:38}.\footnote{The multiple instances of \ol{sa}(\**) in the tree below are derived via head-movement.}

\ea%38
    \label{ex:newkirk:38}
    \scalebox{.75}{
	\begin{forest} fairly nice empty nodes
	[SAP
		[Speaker]
		[sa
			[sa]
			[sa\**
				[\alert{EvalP}
        [Seat of Knowledge]
        [Eval$'$
          [Eval]
          [EvidP
            [Evidence]
            [Evid$'$
              [Evid]
              [TP]
            ]
          ]
        ]
        ]
				[sa\**
					[sa\**]
					[Hearer]
				]
			]
		]
	]
	\end{forest}
	}
\z


All of these positions are inherently perspectival, however. Accordingly, they won't work for a proleptic object (which doesn't even have to be sentient, much less a perspective-holder). But within their system, there is space to add one more position, for an \emph{evaluated object}. Speas \& Tenny derive an extended SAP by head movement of the speech act head. The same movement can apply to the evaluative head, creating an additional position for the evaluated object. Rather than having a perspective-taking semantics, the evaluated object can be non-sentient, so long as it is the object perceived by the seat of knowledge evaluating the embedded propositional content. This projection is parallel to the Hearer in the speech act projection, but for the lower EvalP head.

% \begin{exe}
% \ex{\scalebox{.75}{
%   	\begin{forest} fairly nice empty nodes
%   		[EvalP
%   			[Seat of Knowledge]
%   			[Eval$'$
%   				[Eval]
%   				[\alert{Eval\**}
%   					[\alert{Evaluated}]
%   					[\alert{Eval\**}
%   					[\alert{Eval\**}]
%   				[EvidP
%   					[Evidence]
%   					[Evid$'$
%   						[Evid]
%   						[TP]
%   					]
%   				]
%   			] ] ]
%   		]
%   	\end{forest}
%   	}}\label{evalpadj}
% \end{exe}

In base-generated prolepsis, the evaluated object binds the embedded \agre, and in turn the evaluated object is bound by the proleptic object in the matrix clause. Therefore the modified tree for \REF{ex:newkirk:11} is in (\ref{ex:newkirk:39}).

\begin{exe}
	\exr{ex:newkirk:11}{
	\gll    John a-kanakan-\alert{il}-a \alert{Jane}\ix{i} a-li Bill a-mu-siim-a \alert{o-mu-eene}\ix{i}/\alert{niye}\ix{i}.\\
			John \textsc{sm.c1-}think\textsc{-appl-fv} Jane \textsc{c1-}that Bill \textsc{sm.c1-om.c1-}like\textsc{-fv} \textsc{c1-c1-}own/her\\
	\glt    `John thinks of Jane\ix{i} that Bill likes her\ix{i}.'}
\end{exe}

\ea%39
    \label{ex:newkirk:39}
    \scalebox{.75}{
\begin{forest}
fairly nice empty nodes
% [TP
	% [DP,name=specTP [Jack,name=Jack,roof] ]
	% [{}
		% [T]
		% [$v$P
			% [\sout{DP},name=specvP [\sout{Jack},name=vPJack,roof]]
			 % [{}
				% [$v$ [think-\textsc{appl},name=v1]]
				[ApplP
				[DP\ix{i} [Jane\ix{i},roof]]
				[
				[Appl [\ol{-il-}\\\textsc{appl},name=appl]]
				[VP
					[V [\ol{a-kanakan-il-a}\\think,name=V1]]
					[SAP
							[Speaker]
							[\ldots
								[EvalP
									[Seat of Knowledge]
									[Eval$'$
										[Eval]
										[Eval\**
											[Evaluated\ix{i}]
											% [Eval\**
												% [Eval\**]
												[\ldots %EvidP
													[Evidence]
													% [Evid$'$
													% 	[Evid]
															[TP
								[\ol{Bill a-mu-siim-a}\\Bill likes \agre{}\ix{i},roof]
								% [DP,name=specTP2 [George,name=George,roof]]
								% [
									% [T]
									% [$v$P
										% [\sout{DP},name=specvP2 [\sout{George},name=George2,roof]]
										% [
												% [$v$ [OM,name=reduceme2] [likes,name=v2]]
												% [VP
													% [V [\sout{likes},name=V2]]
													% [DP,name=compV [\agre,name=agre3]]
												% ]
										% ]
									% ]
								% ]
							] % TP
					% ] % Evid'
				] % EvidP
			% ] % Eval*
			 ] % Eval*
			  ] % Eval'
								] % EvalP
								[sa\**]
					] ] % CP/SAP
				 ]
			 ]
			 ]
			 % ]
		% ]
	% ]
% ]
% \draw[->,semithick] (V1)..controls +(south west:1) and +(south:1)..(appl);
% \draw[->,semithick] (appl)..controls +(south west:1) and +(south:1)..(v1);
% \draw[->,semithick] (compV)..controls +(south east:5) and +(south west:3)..(reduceme2);
% \draw[->,semithick] (reduceme2)..controls +(south west:3) and +(south west:3)..(specCP);
% \draw[->,semithick] (specCP)..controls +(south west:3) and +(south west:3)..(reduceme3);
% % \draw[->,semithick,dashed] (v2)..controls +(south west:2) and +(west:2) ..(reduceme2)
% % node[very near end,left] {\scriptsize m-merger};
% \draw[->,semithick,dashed] (v1)..controls +(south west:2) and +(west:2) ..(reduceme3)
% node[midway,left] {\scriptsize m-merger};
\end{forest}
	}
\z

The movement-based prolepsis construction is much as it was before, but now we can pinpoint the left-peripheral location that serves as an escape hatch for the moved pronoun: it passes through the site of the evaluated object, and thereby receives its proleptic semantics. Then \agre\ moves further upward to cliticize to the matrix verb for its syntactic licensing.

Since both constructions involve the same projection in the left periphery, they get the same interpretation from the Eval head. Despite their disparate syntax, a common left periphery allows them to get the same semantics, one similar to topichood, though the proleptic object is not in a Topic projection in either case.

\ea%40
    \label{ex:newkirk:40}
    \scalebox{.75}{
    \hspace{-1cm}
\begin{forest}
fairly nice empty nodes
% [TP
	% [DP,name=specTP [Jack,name=Jack,roof] ]
	% [{}
		% [T]
		[$v$P
			[\sout{DP},name=specvP [John,name=vPJack,roof]]
			 [{}
				[$v$ [\ol{-i-}\\\textsc{refl},name=reduceme3] [\ol{many-il-e}\\know,name=v1]]
				[VP
					[V [\sout{\ol{many-il-e}}\\\sout{know},name=V1]]
					[SAP
							[Speaker]
							[\ldots
								[EvalP
									[Seat of Knowledge]
									[Eval$'$
										[Eval]
										[Eval\**
											[\sout{DP},name=specCP [\sout{\ol{o-mu-eene}}\\\sout{\agre},name=CPagre,roof] { \draw[->,semithick] () -| (reduceme3); }]
											[Eval\**
												[Eval\**]
												[\ldots %EvidP
													[Evidence]
													% [Evid$'$
													% 	[Evid]
															[TP
								[\ol{George a-mu-siim-a o-mu-eene}\\George likes \agre{},roof] { \draw[->,semithick] (.south) -| ($(.south)+(0,-1em)$) -| (CPagre); }
								% [DP,name=specTP2 [George,name=George,roof]]
								% [
									% [T]
									% [$v$P
										% [\sout{DP},name=specvP2 [\sout{George},name=George2,roof]]
										% [
												% [$v$ [OM,name=reduceme2] [likes,name=v2]]
												% [VP
													% [V [\sout{likes},name=V2]]
													% [DP,name=compV [\agre,name=agre3]]
												% ]
										% ]
									% ]
								% ]
							] % TP
					%] % Evid'
				] % Evid P
			] ] ]
								]
								[sa\**]
					] ] % CP/SAP
				 ]
			 ]
		]
% 	] % T'
% ] % TP
% \draw[->,semithick] (compV)..controls +(south east:5) and +(south west:3)..(reduceme2);
% \draw[->,semithick] (reduceme2)..controls +(south west:3) and +(south west:3)..(specCP);
% \draw[->,semithick] (specCP)..controls +(south west:3) and +(south west:3)..(reduceme3);
% % \draw[->,semithick,dashed] (v2)..controls +(south west:2) and +(west:2) ..(reduceme2)
% % node[very near end,left] {\scriptsize m-merger};
\draw[->,semithick,dashed] (v1)..controls +(south west:3) and +(west:2.5) ..($(reduceme3)+(-1em,0.5em)$)
node[pos=0.35,below left] {\scriptsize m-merger};
\end{forest}
}
\z

\section{Cross-linguistic predictions}\label{sec:newkirk:5}

Turning our attention to other languages, we can see that the difference between movement-based and base-generated prolepsis is how the nominal in the matrix clause is syntactically licensed, and whether that licensing position is eligible for movement or base-generation. For Passamaquoddy \citep{Bruening2001},  Innu-Aimûn \citep{BraniganMacKenzie2002}, and Tsez \citep{PolinskyPotsdam2001}, agreement can reach to the CP domain and license the nominal there. But the nominal can only surface in the matrix clause if it is licensed by an agreeing matrix verb. If the verb surfaces in the non-agreeing (TI) voice, the nominal must stay in-situ, and there is no topicality:

\ea%41
	\label{ex:newkirk:41} 
	Innu-Aimûn \citep{BraniganMacKenzie2002}: 
	\ea[]{\label{ex:newkirk:41a}
	\gll    N-uî-tshissenit-\alert{en} tshetshî mûpishtâshkuenit \alert{kassinu} \alert{kâuâpikueshit}.\\
		    1-want-know-\textsc{ti} if visited-2/\textsc{inv} every priest\\
	\glt    `I want to know if every priest visited you.'}
    \ex[*]{\label{ex:newkirk:41b}
            N-uî-tshissenit-\alert{en}\ix{i} [\alert{kassinu} \alert{kâuâpikueshit}]\ix{i} tshetshî mûpishtâshkuenit.}
	\z
\z

For Middle Dutch the matrix nominal is licensed by case marking, but on the analysis \citep{VanKoppenEtAl2016} it's in spec,CP, although it hasn't been moved there. In German, prolepsis often feeds further movement that would otherwise be degraded \citep{Salzmann2017Chapter}.

% \bex
% 	\ex[]{\gll Von [\alert{welchem} \alert{Maler}]\ix{i} glaubst du, dass Maria \alert{ihn}\ix{i} mag?\\
% 	of which.\textsc{dat} painter think.\textsc{2sg} you that Mary him like.\textsc{3sg}\\
% 	\glt `Of which painter\ix{i} do you think that Mary likes him\ix{i}?'}
% \fex

If prolepsis is used when A$'$-movement is degraded, then it comes as no surprise that the proleptic object in those constructions would not be moved into that position, since movement out of the embedded clause is impossible in the first place. And similar to the base-generation strategy in Lubukusu, the complement of a preposition is not an eligible landing site for A'-movement. If German only licenses extra matrix clause nominals with a preposition, then those extra nominals will necessarily be base-generated there. Once again, the particulars of a given language condition which of the movement and base-generation strategies are available, and under which circumstances.

These considerations bring to the fore an important distinction between semantic and syntactic licensing. Semantically, the evaluated object head provides a viable semantic interpretation for the extra matrix nominal, so long as the context supports that interpretation. Thus the left periphery is identical in both types of structure. The syntactic licensing requirements, however, differ by construction (and by language), as independently established. It is precisely these syntactic facts that derive the differences between prolepsis types.

\section*{Abbreviations}
\begin{tabular}{@{}ll@{}}
{\textsc{appl}}  &  applicative\\
 \multicolumn{2}{@{}l}{{c} followed by a number \hspace{2\tabcolsep}  noun class marker}\\
 {\textsc{fv}}  &  Final vowel\\
 {\textsc{om}}  &  Object Marker (typically followed by noun class number)\\
 {\textsc{prep}}  &  Preposition\\
 {\textsc{refl}}  &  Reflexive Marker\\
 {\textsc{sm}}  &  Subject Marker (typically followed by noun class number)\\
 {\textsc{tns}}  &  Tense
\end{tabular}

\section*{Acknowledgements}
Special thanks to Justine Sikuku, Ken Safir, Mark Baker, Vivian Deprez, and the attendees of Rutgers ST@R and SURGE reading groups. Some of the data for this project are from the Afranaph Project (NSF BCS 1324404). I am indebted to two anonymous reviewers for their helpful comments. Any remaining errors are my own.% Those sentences are marked with an Afranaph ID for access on the Afranaph database (\url{http://www.africananaphora.rutgers.edu/}).

{\sloppy\printbibliography[heading=subbibliography,notkeyword=this]}

\end{document}
