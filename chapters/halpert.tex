\documentclass[output=paper,colorlinks,citecolor=brown]{langscibook} 
\author{Claire Halpert\affiliation{University of Minnesota}}
\title{Overt subjects and agreement in Zulu infinitives}  
\abstract{This paper explores a surprising interaction of agreement and concord inside infinitive clauses in Zulu.  In Zulu, as in many Bantu languages, infinitive verbs are marked with noun class 15/17 morphology. Internal arguments of infinitives are typically unmarked, while the external argument  must receive so-called associative morphology and must precede internal arguments. I argue that the external argument in these constructions is realized in Spec,vP, a finding that has a number of consequences for our understanding of clause structure and agreement in Zulu and related languages.}
\IfFileExists{../localcommands.tex}{
  \addbibresource{localbibliography.bib}
  \usepackage{langsci-optional,langsci-branding}
\usepackage{langsci-gb4e}
% \usepackage{langsci-textipa}
% \usepackage{langsci-glyphs}
\usepackage[linguistics]{forest}
\usepackage{tabto}
\usepackage{multirow}
\usepackage{bbding}

\usepackage[normalem]{ulem}

\usepackage{tikz-qtree}

\usepackage{enumitem}

\usepackage{multicol}
\usepackage{stmaryrd} %double brackets

\makeatletter
\let\pgfmathModX=\pgfmathMod@
\usepackage{pgfplots,pgfplotstable}%
\let\pgfmathMod@=\pgfmathModX
\makeatother
\usepgfplotslibrary{colorbrewer}
\usetikzlibrary{fit}

\usepackage{jambox}
\usepackage{tikz-qtree-compat}
\usetikzlibrary{arrows, arrows.meta}
\usepackage{longtable}
\usepackage{subcaption}

  \makeatletter
\let\thetitle\@title
\let\theauthor\@author
\makeatother

\newcommand{\togglepaper}[1][0]{
%   \bibliography{../localbibliography}
  \papernote{\scriptsize\normalfont
    \theauthor.
    \thetitle.
    To appear in:
    Change Volume Editor \& in localcommands.tex
    Change volume title in localcommands.tex
    Berlin: Language Science Press. [preliminary page numbering]
  }
  \pagenumbering{roman}
  \setcounter{chapter}{#1}
  \addtocounter{chapter}{-1}
}

\newcommand{\bari}{\ipabar{\i}{.5ex}{1.1}{}{}}
\newcommand{\notipa}[1]{\textnormal{#1}}

\newcommand{\agre}{\textsc{agr}-\ol{eene}}

\renewcommand{\emph}[1]{\textit{#1}} % resetting a setting from ling-macros-modified (I think?)

% forest settings to make compact but (mostly) straight-spined trees:
\forestset{
fairly nice empty nodes/.style={
            delay={where content={}{shape=coordinate,for parent={
                  for children={anchor=north}}}{}}
, angled/.style={content/.expanded={$<$\forestov{content}$>$}}
}}

\forestset{sn edges/.style={for tree={parent anchor=south, child anchor=north}}}

\newcommand{\bex}{\begin{exe}}
\newcommand{\fex}{\end{exe}}

\newcommand{\bxl}{\begin{exe}}
\newcommand{\fxl}{\end{exe}}

\newcommand{\ix}[1]{\textsubscript{#1}}
\newcommand{\alert}[1]{\textbf{#1}}
\newcommand{\ol}[1]{\textit{#1}}


			\usetikzlibrary{shapes,arrows,positioning,decorations,decorations.pathmorphing,intersections}
\forestset{
nice empty nodes/.style={
    for tree={calign=fixed edge angles},
    delay={where content={}{shape=coordinate,for siblings={anchor=north}}{}}
},
}

\definecolor{dark-gray}{gray}{0.3}

%\usepackage{dingbat,pifont}


%%%%%%%%%%%%For arrows%%%%%%%%%%%%%

\newcommand\Tikzmark[2]{%
  \tikz[remember picture]\node[inner sep=0pt,outer sep=0pt] (#1) {#2};%
}
\NewDocumentCommand\DrawArrow{O{}mmmmO{3}}{
\tikz[remember picture,overlay]
  \draw[->,line width=0.8pt,shorten >= 2pt,shorten <= 2pt,#1]
    (#2) -- ++(0,-#6\ht\strutbox) coordinate (aux) -- node[#4] {#5} (#3|-aux) -- (#3);
}
\NewDocumentCommand\DrawDotted{O{}mmmmO{3}}{
\tikz[remember picture,overlay]
  \draw[->,line width=0.9pt,dotted,shorten >= 2pt,shorten <= 2pt,#1]
    (#2) -- ++(0,-#6\ht\strutbox) coordinate (aux) -- node[#4] {#5} (#3|-aux) -- (#3);
}
\NewDocumentCommand\DrawLine{O{}mmmmO{3}}{
\tikz[remember picture,overlay]
  \draw[line width=0.8pt,shorten >= 2pt,shorten <= 2pt,#1]
    (#2) -- ++(0,-#6\ht\strutbox) coordinate (aux) -- node[#4] {#5} (#3|-aux) -- (#3);
}
%%%%%%%%%%%%%%%%%%%%%%%%%%%%%%%%%%%%%


\newcommand{\baru}{ʉ}
\newcommand{\baruH}{\'\baru}
\newcommand{\baruL}{\`\baru}

\newcommand{\ep}{ε}
\newcommand{\epH}{\'\ep}
\newcommand{\epL}{\`\ep}

\newcommand{\schwa}{ə}
\newcommand{\schwaH}{\'ə}
\newcommand{\schwaL}{\`ə}

\newcommand{\oo}{ɔ}
\newcommand{\ooH}{\'\oo}
\newcommand{\ooL}{\`\oo}

\newcommand{\ds}{\textsuperscript{
	\hspace*{-2pt}\begin{tikzpicture}
		\draw[-{>[scale=0.5]}] (0,0.4) --(0,0.25);
	\end{tikzpicture}}}

\newcommand{\ch}{t͡ʃ}
\newcommand{\dz}{d͡ʒ}

\newcommand{\tgl}{ʔ}

%shortcuts for the complementizers
\newcommand{\mbuL}{mb\baruL}
\newcommand{\mbuHL}{mb\baruH\baruL}
\newcommand{\mbuLH}{mb\baruL\baruH}
\newcommand{\la}{lá}
\newcommand{\nda}{ndà}

\newcommand{\tsc}[1]{\textsc{#1}}
\renewcommand{\textscb}{ʙ}
\newcommand{\ipa}[1]{#1} %disable IPA

\newcommand{\SM}[1]{#1}

\DeclareNewSectionCommand
  [
    counterwithin = chapter,
    afterskip = 2.3ex plus .2ex,
    beforeskip = -3.5ex plus -1ex minus -.2ex,
    indent = 0pt,
    font = \usekomafont{section},
    level = 1,
    tocindent = 1.5em,
    toclevel = 1,
    tocnumwidth = 2.3em,
    tocstyle = section,
    style = section
  ]
  {appendixsection}

\renewcommand*\theappendixsection{\Alph{appendixsection}}
\renewcommand*{\appendixsectionformat}
              {\appendixname~\theappendixsection\autodot\enskip}
\renewcommand*{\appendixsectionmarkformat}
              {\appendixname~\theappendixsection\autodot\enskip}

\renewcommand{\lsChapterFooterSize}{\footnotesize}
 
  %% hyphenation points for line breaks
%% Normally, automatic hyphenation in LaTeX is very good
%% If a word is mis-hyphenated, add it to this file
%%
%% add information to TeX file before \begin{document} with:
%% %% hyphenation points for line breaks
%% Normally, automatic hyphenation in LaTeX is very good
%% If a word is mis-hyphenated, add it to this file
%%
%% add information to TeX file before \begin{document} with:
%% %% hyphenation points for line breaks
%% Normally, automatic hyphenation in LaTeX is very good
%% If a word is mis-hyphenated, add it to this file
%%
%% add information to TeX file before \begin{document} with:
%% \include{localhyphenation}
\hyphenation{
affri-ca-te
affri-ca-tes 
Līk-pāk-páln
pro-sod-ic
phe-nom-e-non
Chi-che-wa
Lu-bu-ku-su
Ngbu-gu
Boyel-dieu
Mat-chi
pho-neme
Mil-em-be
Nyan-chera
Mc-Pher-son
Tsoo-tso
Sku-pin
dis-tin-guishes
con-ser-va-tion
Me-dum-ba
}

\hyphenation{
affri-ca-te
affri-ca-tes 
Līk-pāk-páln
pro-sod-ic
phe-nom-e-non
Chi-che-wa
Lu-bu-ku-su
Ngbu-gu
Boyel-dieu
Mat-chi
pho-neme
Mil-em-be
Nyan-chera
Mc-Pher-son
Tsoo-tso
Sku-pin
dis-tin-guishes
con-ser-va-tion
Me-dum-ba
}

\hyphenation{
affri-ca-te
affri-ca-tes 
Līk-pāk-páln
pro-sod-ic
phe-nom-e-non
Chi-che-wa
Lu-bu-ku-su
Ngbu-gu
Boyel-dieu
Mat-chi
pho-neme
Mil-em-be
Nyan-chera
Mc-Pher-son
Tsoo-tso
Sku-pin
dis-tin-guishes
con-ser-va-tion
Me-dum-ba
}
 
  \togglepaper[1]%%chapternumber
}{}

\begin{document}
\maketitle 

\section{Introduction}\label{sec:halpert:1}

This paper investigates infinitive clauses in the Bantu language Zulu that have overt agents. As illustrated below in \REF{ex:halpert:1}, agents of Zulu infinitives must precede internal arguments \REF{ex:halpert:1a} and cannot follow them \REF{ex:halpert:1b}.\footnote{All examples in this paper are from Zulu, unless otherwise noted. Unsourced Zulu examples are taken from my own fieldwork.}

\ea%1
    \label{ex:halpert:1}
    \ea[]{
    \label{ex:halpert:1a}
    \gll    [u-ku-nikeza kwa-khe izingane amavuvuzela] ku-ya-ngi-casula\\
            \textsc{aug}-15-give 15.\textsc{assoc-1pro} \textsc{aug}.10child \textsc{aug}.6vuvuzela \textsc{sm}15-\textsc{dj}-1\textsc{sg.om}-annoy \\
    \glt    `His giving the children vuvuzelas annoys me.'}
    \ex[*]{
    \label{ex:halpert:1b}
    \gll    [u-ku-nikeza izingane amavuvuzela kwa-khe] ku-ya-ngi-casula\\
            \textsc{aug}-15-give  \textsc{aug}.10child \textsc{aug}.6vuvuzela  15.\textsc{assoc-1pro} \textsc{sm}15-\textsc{dj}-1\textsc{sg.om}-annoy \\
    \glt    `His giving the children vuvuzelas annoys me.'}
    \z
\z

These constructions exhibit a puzzling constellation of properties. As the examples above illustrate, overt agents in infinitives must be marked with so-called associative morphology, which typically mark adnominal adjuncts \citep[e.g.][]{Sabelo1990,Halpert2015,Pietraszko2019}.  At the same time, they require VSO word order, placing the associative-marked subject in a position that is otherwise unusual for adjuncts in the language but typical for in situ subjects. I will argue, using evidence from binding, that the overt subject in these constructions is truly in an argument position, in Spec,vP, despite the appearance of associative morphology.  This conclusion raises an additional puzzle: as we will see in Section \ref{sec:halpert:3.2}, these overt subjects do not block object agreement from appearing inside the infinitive, unlike vP-internal subjects of finite clauses in the language. 

How can we reconcile this mix of properties? I will suggest two instructive parallels: Linker Phrases in Kinande \citep{BakerCollins2006, Schneider-Zioga2015ACAL, Schneider-Zioga2015WCCFL} and external arguments of passives in Zulu. If we treat the associative morphology that appears on subjects in infinitives as a head in the clausal spine, akin to the Kinande Linker, then the patterns found in these infinitival constructions with respect to agreement are analogous to those found in Zulu passives, as I will discuss in Section \ref{sec:halpert:4}. 

\section{Background: a subject syntax baseline}\label{sec:halpert:2}

Zulu is a Bantu language (S42) spoken primarily in South Africa.  In this section, I will lay out some of the basic properties of Zulu that will allow us to understand the puzzles posed by the subjects of infinitives.  In particular, we will need to establish the expected patterns of agreement and word order, the basic properties of infinitives, and the basic properties of so-called associative constructions.

\subsection{Agreement and word order}\label{sec:halpert:2.1}

Zulu nouns are divided into 14 noun classes that are notated by number. Agreement and concord processes are glossed using  noun class numbers – a number that matches the number on a noun agrees with the noun. Like most Bantu languages, Zulu has obligatory subject agreement morphology and optional object agreement morphology on verbs. In Zulu, predicates agree with vP-external arguments only: subject agreement tracks the highest vP-external (or pro-dropped) argument, while object agreement appears when a lower argument is vP-external or pro-dropped. In situations when there is no vP-external argument, an expletive agreement \textit{ku-} (class 15/17) appears in the subject agreement spot.  The verb in Zulu undergoes head movement to a vP-external position, so any preverbal arguments are outside of vP \citep{Buell2005,Halpert2015}.

In \REF{ex:halpert:2} below, we can see subject agreement tracking a pre-verbal/pro-dropped subject.  The postverbal object is inside vP, so no object agreement appears.\footnote{We can determine the position of postverbal material using the distribution of the present tense \textit{disjoint} morpheme \textit{-ya-}, which appears on the verb just in case vP is empty \citep{Buell2005, Halpert2015, Halpert2017}.  In the examples in \REF{ex:halpert:2}, there is no disjoint morpheme (the verb appears in its bare \textit{conjoint} form), so the postverbal object must be inside vP.}

\ea%2
    \label{ex:halpert:2}
    \ea{%2a
    \label{ex:halpert:2a}
    \gll    (\textbf{uZinhle}) \textbf{u}- xova ujeqe\\
            \textsc{aug}.1Zinhle \textsc{sm}1- make \textsc{aug}.1steamed.bread\\
    \glt    `Zinhle is making steamed bread.'}
    \ex{%2b
    \label{ex:halpert:2b}
    \gll    (\textbf{omakhelwane}) \textbf{ba}- xova ujeqe\\
            \textsc{aug}.2neighbor \textsc{sm}1- make \textsc{aug}.1steamed.bread\\
    \glt    `The neighbors are making steamed bread.'}
\z 
\z 

When the subject remains inside vP, we get default agreement: class 17 \textit{ku-}.\footnote{As \citet{BuellDreu2013} note, in modern Zulu, classes 15 and 17 have become indistinguishable. For clarity here, I follow the convention of marking default agreement as class 17, but infinitives as class 15.}  In the examples in \REF{ex:halpert:3} below, the post-verbal subject is followed by a low adverb, \textit{kahle}, `well,' which must appear inside vP \citep{Buell2005}.

\ea%3
    \label{ex:halpert:3}
    \ea[*]{%3a
    \label{ex:halpert:3a}
    \gll    \textbf{u}- pheka uZinhle kahle\\
            \textsc{sm}1- cook 1Zinhle well\\}
    \ex[]{%3b
    \label{ex:halpert:3b}
    \gll    \textbf{ku}- pheka uZinhle kahle\\
            \textsc{sm}17- cook \textsc{aug}.1Zinhle well\\
    \glt    `Zinhle cooks well.'}
    \z 
\z 

When objects remain in situ, no object agreement appears, as we saw in \REF{ex:halpert:2}. When an object appears outside of vP, it controls object agreement:\footnote{Object agreement is typically \textit{required} for vP-external objects, with limited exceptions in the case of double dislocation constructions \cite[e.g.][]{Adams2010,Zeller2012}.  The placement of the low adverb \textit{kahle} and the appearance of the  disjoint  (\textit{-ya-}) here indicate that the object is outside of vP \citep{Buell2005,Halpert2015}.}

\ea%4
    \label{ex:halpert:4}
    \gll    uZinhle u-ya-\textbf{m}-xova kahle  \textbf{ujeqe}\\
            \textsc{aug}.1Zinhle \textsc{sm}1-\textsc{dj}-\textsc{om}1-make well  \textsc{aug}.1steamed.bread\\ 
    \glt    `Zinhle makes steamed bread well.'
\z 

We saw in \REF{ex:halpert:3b} that subjects can remain inside vP and cannot control subject agreement from this position.  When the subject remains low in a finite clause, all lower arguments must also remain vP-internal.\footnote{There are a few limited cases in Zulu where a locative or instrumental argument can control subject agreement while the external argument remains in vP \citep{Buell2007,Zeller2013}. \citet{Zeller2013} argues that these cases involve introduction of the instrument or locative in a position structurally higher than vP, which would make them non-exceptions to this generalization.}
As expected, these trapped internal arguments cannot be pro-dropped to control either subject or object agreement: 

\ea%5
    \label{ex:halpert:5}
    \ea[]{%5a
    \label{ex:halpert:5a}
    \gll    ku-phek-e uSipho amaqanda\\
            \textsc{sm}17-cook-{\textsc{pst}} \textsc{aug}.1Sipho \textsc{aug}.6egg\\
    \glt    `\textsc{Sipho} cooked eggs.'}
    \ex[*]{%5b
    \label{ex:halpert:5b}
    \gll    a-phek-e uSipho (amaqanda)\\
            \textsc{sm}6-cook-{\textsc{pst}} \textsc{aug}.1Sipho \textsc{aug}.6egg\\
    \glt    intended: `\textsc{Sipho} cooked them.'}
    \ex[*]{%5c
    \label{ex:halpert:5c}
    \gll    kw-a-phek-e uSipho (amaqanda)\\
            \textsc{sm}17-\textsc{om}6-cook-{\textsc{pst}} \textsc{aug}.1Sipho \textsc{aug}.6egg\\
    \glt    intended: `\textsc{Sipho} cooked them.'}
    \z 
\z 
 
\ea%6
    \label{ex:halpert:6}
    \ea[]{%6
    \label{ex:halpert:6a}
    \gll    kw-a-nikeza uMfundo izingane amavuvuzela\\
            \textsc{sm}17-\textsc{pst}-give \textsc{aug}.1Mfundo \textsc{aug}.10child \textsc{aug}.6vuvuzela\\
    \glt    `\textsc{Mfundo} gave the children vuvuzelas.'}
    \ex[*]{%6b
    \label{ex:halpert:6b}
    \gll    kw-a-zi-nikeza uMfundo amavuvuzela (izingane)\\
            \textsc{sm}17-\textsc{pst}-\textsc{om}10-give \textsc{aug}.1Mfundo \textsc{aug}.6vuvuzela \textsc{aug}.10child \\
    \glt    intended: `\textsc{Mfundo} gave them vuvuzelas.'}
    \ex[*]{%6c
    \label{ex:halpert:6c}
    \gll    kw-a-wa-nikeza uMfundo  izingane (amavuvuzela)\\
            \textsc{sm}17-\textsc{pst}-\textsc{om}6-give \textsc{aug}.1Mfundo  \textsc{aug}.10child \textsc{aug}.6vuvuzela \\
    \glt    intended: `\textsc{Mfundo} gave them to the children.'}
    \z 
\z 
 
Word order in these transitive expletive constructions is completely rigid: V S (IO) DO, which I have argued reflects the base positions of the arguments \citep{Halpert2015}. To summarize the basic picture of agreement and word order in finite clauses, we have seen in this section that agreement in Zulu corresponds with movement out of vP and that low subjects block other arguments inside vP from moving or agreeing.\footnote{\citet{Zeller2015} argues that in Zulu, T – the host of subject agreement – must probe before other heads in the same phase, including the host of object agreement.  If the non-agreeing subject is a defective intervener, it would necessarily block both subject and object agreement on this view.} 
 
\subsection{Infinitives}\label{sec:halpert:2.2}
 
Infinitives in Bantu languages often look like verbs that bear noun class morphology \citep{Schadeberg2003}. In Zulu, verbs that have the typical distribution of infinitives are marked with noun class 15(/17) \textit{uku-}:

\ea%7
    \label{ex:halpert:7}
    \ea%7a
    \label{ex:halpert:7a}
    \gll    ngi-funa [uku-xova ujeqe]\\
            1\textsc{sg}-want  \textsc{aug}.15-make \textsc{aug}.1bread\\
    \glt    `I want to make steamed bread.'
    \ex%7b
    \gll    ngi-yethemba [uku-ni-bona]\\
            1\textsc{sg}-hope  \textsc{aug}.15-2\textsc{pl}.\textsc{om}-see\\
    \glt    `I hope to see you all.'
    \z 
\z 

The \textit{uku-} prefix can attach above a variety of verbal inflectional morphology, including object agreement, negation, mood, and aspect, as \REF{ex:halpert:8} below illustrates.  The basic generalization is that \textit{uku-} can combine with morphology that would follow subject agreement in a finite clause.

\ea%8
    \label{ex:halpert:8}
    \ea%8a
    \label{ex:halpert:8a}
    \gll    uku-nga-zi-bon-i\\
            \textsc{aug}.15-\textsc{neg}-\textsc{refl}-see-\textsc{neg}\\
    \glt    `to not see oneself'
    \ex%8b
    \label{ex:halpert:8b}
    \gll    uku-sa-m-thanda kabi uSipho\\
            \textsc{aug}.15-\textsc{dur}-\textsc{om}1-love badly \textsc{aug}.1Sipho\\
    \glt    `to still really love Sipho'
    \z 
\z 

As \REF{ex:halpert:7} and \REF{ex:halpert:8} show, Zulu infinitives can involve quite a bit of clausal structure above the verb root and seem to preserve the internal argument structure of the verb. As the \textit{uku-} infinitive morphology suggests, from the outside, infinitives look just like nominals: as \REF{ex:halpert:9} illustrates, they can control subject and object agreement under the same circumstances that nominal arguments do:

% \ea%8.5
%     \label{ex:halpert:8.5}
%     \ea%8.5a
%     \label{ex:halpert:8.5a}
%     \gll    ngi-ya-\textbf{ku-}funa ukudla\\
%             1\textsc{sg}.\textsc{sm}-\textsc{dj}-15\textsc{om}-want \textsc{aug}.15food\\
%     \glt    `I want food.'
%     \ex%8.5b
%     \label{ex:halpert:8.5b}
%     \gll    ukudla \textbf{ku}-mnandi\\
%             \textsc{aug}.15food 15\textsc{sm}-nice\\
%     \glt    `Food is nice.'
%     \z 
% \z 

\ea%9
    \label{ex:halpert:9}
    \ea%9a
    \label{ex:halpert:9a}
    \gll    ngi-ya-\textbf{ku-}funa [ uku-xova ujeqe ]\\
            1\textsc{sg}.\textsc{sm}-\textsc{dj}-15\textsc{om}-want {} \textsc{aug}.15-make \textsc{aug}.1bread\\
    \glt    `I want to make steamed bread.'
    \ex%9b
    \label{ex:halpert:9b}
    \gll    uku-xova ujeqe \textbf{ku}-mnandi\\
            \textsc{aug}.15-make \textsc{aug}.1bread 15\textsc{sm}-nice\\
    \glt    `Making steamed bread is nice.'
    \z 
\z 

The main takeaways about Zulu infinitives, then, are that they have an internal structure (below the position of subject agreement) that looks similar to finite clauses but an external structure that looks similar to nominals.

\subsection{Associative: adnominal modification}\label{sec:halpert:2.3}

The final piece that we need in order to return to our puzzle is the so-called \textit{associative} construction \citep{Sabelo1990,Halpert2015,Jones2018}. Zulu marks a variety of adnominal dependents with a complex prefix consisting of two parts: a nominal concord that matches the head noun and a fixed \textit{a} vowel that predictably coalesces with the initial vowel of the noun it marks.  In \REF{ex:halpert:10a}, we can see the associative marking a possessor; in \REF{ex:halpert:10b}, it marks a nominal modifier, and in \REF{ex:halpert:10c}, it marks the internal argument of a low nomimalization, where the root \textit{cabang} `think' has been nominalized as a class 3 noun:

\ea%10
    \label{ex:halpert:10}
    \ea%10a
    \label{ex:halpert:10a}
    \gll    umkhovu \textbf{wo}-mthakathi\\
            \textsc{aug}.3zombie 3\textsc{assoc}.\textsc{aug}-1wizard\\
    \glt    `the wizard's zombie'%\footnote{Zulu zombies are  corpses reanimated by practitioners of malicious magic ({\it abathakathi}) and kept under the control of a particular person.  Throughout this handout, solitary zombies are of the Zulu type, while pluralities of zombies are American. \hfill u+a+u \ra\ wo
    \ex%10b
    \label{ex:halpert:10b}
    \gll    isiminyaminya \textbf{se}-mikhovu\\
            \textsc{aug}.7swarm 7\textsc{assoc}.\textsc{aug}-4zombie\\
    \glt    `a horde of zombies' %\hfill si + a + i \ra\ se
    \ex%10c
    \label{ex:halpert:10c}
    \gll    um-cabango \textbf{we}-mikhovu\\
            \textsc{aug}.3thought 3\textsc{assoc}.\textsc{aug}-4zombie\\
    \glt    `the thought of zombies' %\hfill u + a + i \ra\ we
    \z 
\z 

Multiple nominal modifiers can appear in the same noun phrase, each marked by a separate associative morpheme:

\ea%11
    \label{ex:halpert:11}
    \gll    isiminyaminya \textbf{se}-mikhovu \textbf{so}-mthakathi\\
            \textsc{aug}.7swarm 7\textsc{assoc}-4zombie 7\textsc{assoc}-1wizard\\
    \glt    `the wizard's horde of zombies'
\z 

To summarize, the associative marks nominal adjuncts to a nominal, can occur multiple times within a single noun phrase, and is compatible with a range of semantic relationships. \citet{Pietraszko2019} treats the associative in closely-related Ndebele as a nominal adjunct with concord. She analyzes the \textit{a} morpheme as a Linker head that takes the modifying nominal (or CP) as its complement and receives a copy of the  phi (noun class) features of the head noun via a DP-internal concord process.  On this view, multiple associative-marked nominals can easily modify a single head noun, with each attaching as a right-adjoined adjunct.\footnote{See \citet{Jones2018}, though, for an analysis of Zulu associative as a D head.  It's not clear how such an analysis would account for the cases of multiple associative-marked modifiers.} 

%\ea \textbf{Pietraszko (2017): associative as Linker}

%{\small

% \ea \textbf{Pietraszko (2017): associative as Linker}

% \begin{forest} 
% [DP [D [i- [7] ] ] [NP  [N' [N [siminyaminya] ] ] [LkP [Lk [si+a] ] [DP [imikhovu] ] ] ] ] 
% \end{forest}
% \z 
%\z  
%{\makedash{3pt}
%         
%\anodecurve[l]{d}[b]{a}{-.3in}\lput{:U}{X}}
%\anodecurve[br]{a}[b]{b}{-1in}


%\ea \Tree [.DP D\\i-\\{\node{a}7}  [.NP [.N' [.N'  [.N' N\\siminyaminya ]  [.LkP Lk\\{\node{b}si+a} \qroof{imikhovu}.DP  ] ] [.LkP Lk\\{\node{c}si+a} \qroof{umthakathi}.DP ] ] ] ] 
%\z 
%\z 

\subsection{Interim summary}\label{sec:halpert:2.4}

To summarize what we have seen in this section, subject and object agreement have a tight correlation with word order: non-agreeing arguments remain in their base position inside vP, while agreement is required to track vP-external arguments. Movement of the external argument out of vP is \textit{required} in order for internal arguments to be available for movement and agreement. Subject and object agreement contrasts with associative marking, which appears to be a concord process internal to the noun phrase that can mark multiple adnominal adjuncts. In the next section, we will return to the initial puzzle and see various ways in which these baseline expectations are not met in infinitives with overt agents. In the next section, we'll see some surprising ways in which overt subjects of infinitives depart from these baseline expectations.

\section{The puzzle: subjects in infinitives}\label{sec:halpert:3}

In the introduction, we saw that Zulu infinitives with overt subjects have two basic properties: rigid VSO word order and obligatory associative morphology on the subject.  Given the baseline behaviors that we observed in Section \ref{sec:halpert:2}, these properties alone raise questions about the underlying structure of infinitives with subjects.  In this section, I will unpack these puzzles and discuss an additional puzzle raised by the behavior of object agreement in these constructions.

\subsection{Locating the associative-marked subject}\label{sec:halpert:3.1}

As we've seen in previous sections, class 15/17 \textit{uku-} nominalizations have the distribution of infinitives and preserve internal argument structure, as illustrated in \REF{ex:halpert:12} below:\footnote{I've seen limited cases where an internal argument can be marked with an associative, as in \REF{ex:halpert:fn1a} but more often, the associative forces an external argument interpretation, as in \REF{ex:halpert:fn1b}:

\ea%fn1
    \label{ex:halpert:fn1}
    \ea%fn1a
    \label{ex:halpert:fn1a}
    \gll    uku-bhubha kwe-zwe\\
            \textsc{aug}.15-destroy 15\textsc{assoc}.\textsc{aug}-5country\\
    \glt    `the destruction of the country/to destroy the country'
    \ex%fn1b
    \label{ex:halpert:fn1b}
    \gll    u-no-ku-saba kwe-gundane\\
            1\textsc{sm}-with.\textsc{aug}-15-fear 15\textsc{assoc}.\textsc{aug}-5mouse\\
    \glt    `S/he has a mouse's fear.' (same fears as a mouse, NOT a fear of mice)
    \z 
\z 
}

\ea%12
    \label{ex:halpert:12} 
    \gll    u-ku-saba igundane\\
            \textsc{aug}-15-fear \textsc{aug}.5mouse\\
    \glt    `to fear mice/a fear of mice'
\z 

When an overt external argument is present (here an experiencer, rather than an agent), it \textit{must} be marked with associative morphology:

\ea%13
    \label{ex:halpert:13}
    \gll    uku-saba \textbf{kwa}-mi ku-khulu\\
            \textsc{aug}.15-fear 15\textsc{assoc}-1\textsc{sg}.\textsc{pro} 15\textsc{sm}-big\\
    \glt    `My fear is big.'
\z 

This behavior of class 15 infinitives contrasts with nominalizations that involve other noun classes and that typically do not permit any preverbal morphology between the root and nominal prefix.  In these low nominalizations, \textit{all} arguments of the verb, including the external argument, must be marked with associative, as \REF{ex:halpert:14b} shows:

\ea%14
    \label{ex:halpert:14}
    \ea%14a
    \label{ex:halpert:14a}
    \gll    ngi-fisa uku-thola iziqu\\
            1\textsc{sg}.\textsc{sm}-wish \textsc{aug}.15-get \textsc{aug}.8degree\\
    \glt    `I wish to get a degree.' 
    \ex%14b
    \label{ex:halpert:14b}
    \gll    isi-fiso \textbf{sa}-kho \textbf{so}-ku-thola iziqu si-zo-fezeka\\
            \textsc{aug}.7-wish 7\textsc{assoc}-2\textsc{sg}.\textsc{pro} 7\textsc{assoc}.\textsc{aug}-15-get \textsc{aug}.8degree \textsc{sm}7-\textsc{fut}-come.true\\
    \glt    `Your wish to get a degree will come true.'
    \z 
\z 

In these low nominalizations where the nominals that correspond to the arguments of the root verb are marked with associative morphology, we see no evidence of c-command. For example, the external argument is unable to bind a pronoun inside the internal argument in \REF{ex:halpert:15}:

\ea%15
    \label{ex:halpert:15}
    \gll    isi-fiso sa-wo wonke umtwana so-ku-bona uma wa-khe si-zo-fezeka\\
            \textsc{aug}.7-wish 7\textsc{assoc}-1\textsc{dem} 1.every \textsc{aug}.1person 7\textsc{assoc}.\textsc{aug}-15-see \textsc{aug}.1mom 1\textsc{assoc}-1\textsc{pro} \textsc{sm}7-\textsc{fut}-come.true\\
    \glt    `Every child's$_k$ wish to see her$_m$ mother will come true.' (non-bound reading salient, speakers find bound reading difficult)
\z 

This lack of a bound reading is unsurprising on a view of associative adjunction like that of \citet{Pietraszko2019}, discussed in the previous section. If associative-marked nominals are always adjoined within the nominal phrase of the head that they modify, then both of the \todo{Single quotation markers are for linguistic meaning only}`arguments' in \REF{ex:halpert:15} would be right-adjoined and we would not expect the first to c-command the second.  

In an infinitive with an associative-marked subject, we might expect the structure to similarly involve adjunction within the nominal domain, above the level of verbal structure associated with the root.  If so, we first make a prediction about word order that we have already seen does not hold: an associative-marked subject should \textit{follow} any unmarked internal arguments that are introduced in the verbal domain, contrary to what \REF{ex:halpert:16} shows:

\ea%16
    \label{ex:halpert:16}
    \ea[]{%16a
    \label{ex:halpert:16a}
    \gll    [u-ku-nikeza \textbf{kwa-khe} izingane amavuvuzela] ku-ya-ngi-casula\\
            \textsc{aug}-15-give 15.\textsc{assoc-1pro} \textsc{aug}.10child \textsc{aug}.6vuvuzela 15\textsc{sm}-\textsc{dj}-1\textsc{sg.om}-annoy \\
    \glt    `His giving the children vuvuzelas annoys me.'}
    \ex[*]{%16b
    \label{ex:halpert:16b}
    \gll    [u-ku-nikeza izingane amavuvuzela \textbf{kwa-khe}] ku-ya-ngi-casula\\
            \textsc{aug}-15-give  \textsc{aug}.10child \textsc{aug}.6vuvuzela  15.\textsc{assoc-1pro} \textsc{sm}15-\textsc{dj}-1\textsc{sg.om}-annoy \\
    \glt    Intended: `His giving the children vuvuzelas annoys me.'}
    \z 
\z 

If we maintain the assumption that the associative-marked subject involves nominal adjunction, then the word order illustrated by \REF{ex:halpert:16}, where internal arguments must appear to the right of the subject, would have to involve right-adjunction of these internal arguments in the nominal domain as well, similar to the low nominalization cases in \REF{ex:halpert:14} and \REF{ex:halpert:15}.  Given the lack of c-command in \REF{ex:halpert:15}, we would predict that an internal argument that \textit{follows} the subject in an infinitive would also not be c-commanded by it. Again, the prediction of the adjunction hypothesis is not met.  As \REF{ex:halpert:17} illustrates, the associative-marked subject of an infinitive \textit{can} bind into the following (non-marked) internal argument:

\ea%17
    \label{ex:halpert:17}
    \gll    uku-nikeza kwa-wo wonke umuntu intombi isithombe sa-khe ku-thatha isikhathi\\
            \textsc{aug}.15-give 15\textsc{assoc}-1\textsc{dem} 1.every \textsc{aug}.1person \textsc{aug}.9girl \textsc{aug}.7picture 7\textsc{assoc}-1\textsc{pro} \textsc{sm}15-take \textsc{aug}.7time\\
    \glt    `For everyone$_k$ to give the girl$_m$ his$_k$ picture takes a long time.'
\z 

The basic conundrum: neither the word order nor the binding facts fits with an adjunction picture for the subjects of infinitives.  Instead, what we've seen in this section is that in infinitives, we find rigid VSO word order and evidence that S c-commands O.  As we saw in Section \ref{sec:halpert:2.1}, those are precisely the structural properties of in situ arguments in finite clauses.  Based on what we've learned in this section, then, I will suggest that, despite the presence of associative morphology, the overt subject in infinitives is simply in Spec,vP.

\subsection{Puzzling object agreement}\label{sec:halpert:3.2}

The hypothesis that the overt subject in an infinitive is in Spec,vP brings with it additional predictions.  Recall from Section \ref{sec:halpert:2.1} that in finite clauses in Zulu, in situ subjects block objects from moving and controlling object agreement. This agreement blocking effect contrasts with the availability of object agreement in both finite clauses with agreeing subjects \textit{and}  infinitives with no subject (as we saw in Section \ref{sec:halpert:2.2}). If overt subjects in infinitives are in Spec,vP, we expect a similar object agreement blocking effect.  Unlike finite clauses with low subjects, however, infinitives with overt subjects permit object agreement, as \REF{ex:halpert:18} below illustrates:

\ea%18
    \label{ex:halpert:18}
    \ea%18a
    \label{ex:halpert:18a}
    \gll    [uku-zi-nikeza kwakhe amavuvuzela] ku-ya-ngi-casula\\
            \textsc{aug}15-10\textsc{om}-give 15.\textsc{assoc-1pro} \textsc{aug}.6vuvuzela \textsc{sm}15-\textsc{dj}-1\textsc{sg.om}-annoy \\
    \glt    `His giving them vuvuzelas annoys me.'
    \ex%18b
    \label{ex:halpert:18b}
    \gll    [uku-wa-nikeza kwakhe izingane] ku-ya-ngi-casula\\
            \textsc{aug}15-6\textsc{om}-give 15.\textsc{assoc-1pro} \textsc{aug}.10child \textsc{sm}15-\textsc{dj}-1\textsc{sg.om}-annoy \\
    \glt    `His giving them to the children annoys me.'
    \z 
\z 

The full puzzle, then, involves not only the presence of associative morphology despite the VSO word order and c-command relationship between arguments, but also the availability of object agreement despite the presence of the overt postverbal subject.  In the next section, I will explore a solution that links all of these properties.

\section{Toward an analysis}\label{sec:halpert:4}

The word order and binding facts from the previous section suggest that the external argument in Zulu infinitives is \textit{inside} the verbal part of the infinitival clause. If infinitives involve enough clausal structure to include the external argument, why is special associative marking required on that argument, but not on the low external argument of a finite clause? Furthermore, why does the associative marker not signal the type of adjunction structure that it appears to create when marking nominal modifiers?  I'll turn first to this second question, arguing that the associative here is plausibly a head in the verbal extended projection, along the lines of what has been argued for in Kinande by \citet{BakerCollins2006}.

As \citet{BakerCollins2006} discuss, when multiple nominals appear in the postverbal field in Kinande, they must be separated by a so-called Linker, as illustrated in \REF{ex:halpert:19}:

\ea%19
    \label{ex:halpert:19}
    \langinfo{Kinande}{Bantu}{\citealt[ex. 1]{BakerCollins2006}}\\
    \gll    mo-n-a-h-ere omukali \textbf{y'-} eritunda\\
            \textsc{Aff-1sg.S-T-}give-\textsc{Ext} 1woman Lk.1- 5fruit.\\
    \glt    `I gave a fruit to a woman.' %\citep[(1)]{BakerCollins2006}
\z 

The linker matches in noun class with the \textit{preceeding} nominal but cliticizes to the following nominal. \citet{BakerCollins2006} argue that the Linker (Lk) is a head in the clausal spine between V and v that is involved in case-licensing.  On their analysis, the Linker attracts either internal argument in a ditransitive like \REF{ex:halpert:19} to its specifier and agrees with that nominal.  The verb undergoes head-movement to a position above the Linker head but does not need to move through Lk  because Lk itself is not verbal (violating the Head Movement Constraint).

The possibility of Lk before a low subject suggests that LkP is perhaps a bit higher than \citet{BakerCollins2006} posit, at least above vP, in such cases, as illustrated by \REF{ex:halpert:20}:

\ea%20
    \label{ex:halpert:20}
    \langinfo{Kinande}{Bantu}{Pierre Mujomba, p.c.}\\
    \gll    Esy\'ongw\'e si-k\'a-seny-ere omo-musitu \textbf{mo} bakali\\
            \textsc{aug}.9wood \textsc{sm}9-T-chop-\textsc{appl.pfv} \textsc{aug}.18-3village Lk.18 2women\\
    \glt    `\textsc{Women} chop wood in the village.' (Pierre Mujomba, p.c.)
\z 

For \citet{Schneider-Zioga2015ACAL, Schneider-Zioga2015WCCFL}, the Kinande Linker is not a case-licenser, but rather a copula that can be used to mediate predication relations within a verb phrase.  She argues that it appears as a last-resort mechanism when multiple arguments remain in the post-verbal field.  

I believe that certain insights of these accounts can apply to the puzzle of agents in Zulu infinitives.  If the associative in Zulu infinitives is a Linker-like element, following \citet{Pietraszko2019} for Ndebele, that marks true arguments of infinitives, as the binding data in \sectref{sec:halpert:3.1} shows, then as in Kinande, it could be a head in the verbal extended projection that is ``skipped'' by head movement, along the lines of \citet{BakerCollins2006}. Also like the Kinande Linker, it appears only when it is needed to ``license'' an external argument in an infinitive.

Why would the associative be required in infinitives with a low subject but not their finite counterpart? Here is a sketch of a potential analysis: suppose that Zulu infinitives lack some head that helps to license the subject of a finite clause; this would be a typologically common property of infinitives (vs. finite clauses).  Likely candidates for such a head in Zulu could be Voice (or possibly Pred\footnote{See \citet{Zeller2013}.} or the locus of the conjoint/disjoint alternation, which I argue in \citet{Halpert2015} helps to license the subject in a finite clause, but which does not appear in infinitives. In the absence of this relevant category, Zulu uses a Linker to mediate predication involving the external argument. As a copular element, Lk is not involved in verbal head movement (like Kinande).
Unlike in Kinande, Lk in Zulu doesn't attract a specifier. Once the infinitive is constructed, it undergoes concord with the head, as \citet{Pietraszko2019} argues for Ndebele.  The presence of the Linker head on the external argument prevents it from being a phi-goal, which allows object marking to target lower arguments.

One reason to think that Voice might be relevant in licensing subjects in infinitives comes from parallels to passive constructions with overt agents: overt external arguments in Zulu passives appear in Spec,vP,  marked with the \textbf{copula} \citep{HalpertZeller2016}.

\ea%21
    \label{ex:halpert:21}
    \langinfo{Zulu}{}{\citealt[ex. 3]{HalpertZeller2016}}\\
    \gll    USipho 		w-a-nikez-w-a 		\textbf{w-uMary} 			incwadi\\
			\textsc{aug}.1Sipho	\textsc{sm}1-\textsc{pst}-give-\textsc{pass}	\textsc{cop}-\textsc{aug}.1Mary	\textsc{aug}.9book\\
    \glt    `Sipho was given a book by Mary.'
\z 

In both passives and infinitives, the subject appears immediately after the verb and before other vP-internal arguments. In both, the overt subject does not block lower arguments from controlling agreement, unlike in active or finite clauses. Both constructions  morphologically mark the subject by something that normally looks like a head (copula, linker).

\citet{HalpertZeller2016} hypothesize that the copula in these constructions is a head in the clausal spine that gets skipped by head-movement of the verb.  We hypothesize that the appearance of this morpheme on the subject renders the subject a non-intervener for object agreement.  Given the findings of \citet{Schneider-Zioga2015WCCFL, Schneider-Zioga2015ACAL} that the Kinande Linker is a copula, the parallels between Kinande Linkers, Zulu passive subjects, and Zulu infinitive subjects seem even more striking.  While Kinande realizes Linkers and copulas with the same morphology in a variety of situations, it is possible that the difference between morphological marking of the subject in Zulu passives and infinitives could depend on the ultimate category of the clause: in a clause that is ultimately verbal (passive), the copula appears; when the clause is ultimately nominal (infinitive), the associative marker and concord obtain.

\section{Conclusion}\label{sec:halpert:5}

This paper presents an initial description and investigation of the syntactic properties of overt subjects of infinitive clauses in Zulu.  I show that these constructions require VSO word order and associative marking on the subject and demonstrate that c-command relationships hold between the subject and lower arguments. I also show that the presence of an overt associate-marked subject does not prevent an object from controlling object agreement.  I argue that the subject in these constructions is expressed in Spec,vP and sketch a proposal that might account for its puzzling properties. 

One intriguing issue that any treatment of this phenomenon must contend with is the question of argument licensing: in a language (and broader language family) that doesn't show typical properties of (nominative) case licensing associated with finite T \citep{Diercks2012,Halpert2015}, what syntactic role does the associative marker play in this construction? What can we learn about argument licensing in Bantu languages from the parallels between associative, copula, and linkers discussed in Section \ref{sec:halpert:4} and cross-Bantu variation in how overt subjects of infinitives are expressed? The data and approach sketched in this paper lay out an avenue for systematic investigation of subject expression in infinitives and the behavior of copula and linker particles both within and across Bantu languages that will advance our understanding of the role that argument licensing plays in these languages.

% \begin{table}
%     \begin{tabular}{lllll} 
%     \lsptoprule
%             & nouns & verbs & adjectives & adverbs\\ 
%     \midrule
%             absolute  &   12 &    34  &    23     & 13\\
%             relative  &   3.1 &   8.9 &    5.7    & 3.2\\
%     \lspbottomrule
%     \end{tabular}
%     \caption{Frequencies of word classes}
%     \label{tab:halpert:1}
% \end{table}

\section*{Abbreviations}

Glossing follows the Leipzig Glossing Conventions, with the following departures: noun classes are glossed with cardinal numbers (1,2,3,\ldots); first and second person with ordinal numbers; \textsc{assoc} associative, \textsc{aug} augment, \textsc{dj} disjoint, \textsc{ext} extension, Lk linker, \textsc{om} object marker, \textsc{sm} subject marker.

\section*{Acknowledgements}

I am grateful to Mthuli Percival Buthlezi, Monwabisi Mhlophe, Mfundo Didi, Menzi Komo, and Thandeka Maphumulo  for their assistance with Zulu data and grammaticality judgments and to Pierre Mujomba for sharing his Kinande judgments. I am also grateful to Jochen Zeller for our ongoing discussions about passives that have informed this research and to the audience at ACAL 49 for useful feedback and discussion.

\printbibliography[heading=subbibliography,notkeyword=this]

\end{document}
