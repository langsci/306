\documentclass[output=paper,colorlinks,citecolor=brown]{langscibook} 
\author{Abdul-Razak Sulemana\affiliation{Massachusetts Institute of Technology}}
\title{Obligatory controlled subjects in Bùlì}  
\abstract{The paper argues that despite the lack of morphological marking to distinguish between  niteness and non niteness, such a distinction does exist in Bùlì. It also argues that unlike the non niteness of the English type languages where non nite clauses take a null subject (\textsc{pro}), the non nite clauses of Bùlì obligatorily take overt pronominals.  e fact that the controlled element is overt in the language, I argue, shows that phonetic nullness is not an inherent property of the controlled element.}
\IfFileExists{../localcommands.tex}{
  \addbibresource{localbibliography.bib}
  \usepackage{langsci-optional,langsci-branding}
\usepackage{langsci-gb4e}
% \usepackage{langsci-textipa}
% \usepackage{langsci-glyphs}
\usepackage[linguistics]{forest}
\usepackage{tabto}
\usepackage{multirow}
\usepackage{bbding}

\usepackage[normalem]{ulem}

\usepackage{tikz-qtree}

\usepackage{enumitem}

\usepackage{multicol}
\usepackage{stmaryrd} %double brackets

\makeatletter
\let\pgfmathModX=\pgfmathMod@
\usepackage{pgfplots,pgfplotstable}%
\let\pgfmathMod@=\pgfmathModX
\makeatother
\usepgfplotslibrary{colorbrewer}
\usetikzlibrary{fit}

\usepackage{jambox}
\usepackage{tikz-qtree-compat}
\usetikzlibrary{arrows, arrows.meta}
\usepackage{longtable}
\usepackage{subcaption}

  \makeatletter
\let\thetitle\@title
\let\theauthor\@author
\makeatother

\newcommand{\togglepaper}[1][0]{
%   \bibliography{../localbibliography}
  \papernote{\scriptsize\normalfont
    \theauthor.
    \thetitle.
    To appear in:
    Change Volume Editor \& in localcommands.tex
    Change volume title in localcommands.tex
    Berlin: Language Science Press. [preliminary page numbering]
  }
  \pagenumbering{roman}
  \setcounter{chapter}{#1}
  \addtocounter{chapter}{-1}
}

\newcommand{\bari}{\ipabar{\i}{.5ex}{1.1}{}{}}
\newcommand{\notipa}[1]{\textnormal{#1}}

\newcommand{\agre}{\textsc{agr}-\ol{eene}}

\renewcommand{\emph}[1]{\textit{#1}} % resetting a setting from ling-macros-modified (I think?)

% forest settings to make compact but (mostly) straight-spined trees:
\forestset{
fairly nice empty nodes/.style={
            delay={where content={}{shape=coordinate,for parent={
                  for children={anchor=north}}}{}}
, angled/.style={content/.expanded={$<$\forestov{content}$>$}}
}}

\forestset{sn edges/.style={for tree={parent anchor=south, child anchor=north}}}

\newcommand{\bex}{\begin{exe}}
\newcommand{\fex}{\end{exe}}

\newcommand{\bxl}{\begin{exe}}
\newcommand{\fxl}{\end{exe}}

\newcommand{\ix}[1]{\textsubscript{#1}}
\newcommand{\alert}[1]{\textbf{#1}}
\newcommand{\ol}[1]{\textit{#1}}


			\usetikzlibrary{shapes,arrows,positioning,decorations,decorations.pathmorphing,intersections}
\forestset{
nice empty nodes/.style={
    for tree={calign=fixed edge angles},
    delay={where content={}{shape=coordinate,for siblings={anchor=north}}{}}
},
}

\definecolor{dark-gray}{gray}{0.3}

%\usepackage{dingbat,pifont}


%%%%%%%%%%%%For arrows%%%%%%%%%%%%%

\newcommand\Tikzmark[2]{%
  \tikz[remember picture]\node[inner sep=0pt,outer sep=0pt] (#1) {#2};%
}
\NewDocumentCommand\DrawArrow{O{}mmmmO{3}}{
\tikz[remember picture,overlay]
  \draw[->,line width=0.8pt,shorten >= 2pt,shorten <= 2pt,#1]
    (#2) -- ++(0,-#6\ht\strutbox) coordinate (aux) -- node[#4] {#5} (#3|-aux) -- (#3);
}
\NewDocumentCommand\DrawDotted{O{}mmmmO{3}}{
\tikz[remember picture,overlay]
  \draw[->,line width=0.9pt,dotted,shorten >= 2pt,shorten <= 2pt,#1]
    (#2) -- ++(0,-#6\ht\strutbox) coordinate (aux) -- node[#4] {#5} (#3|-aux) -- (#3);
}
\NewDocumentCommand\DrawLine{O{}mmmmO{3}}{
\tikz[remember picture,overlay]
  \draw[line width=0.8pt,shorten >= 2pt,shorten <= 2pt,#1]
    (#2) -- ++(0,-#6\ht\strutbox) coordinate (aux) -- node[#4] {#5} (#3|-aux) -- (#3);
}
%%%%%%%%%%%%%%%%%%%%%%%%%%%%%%%%%%%%%


\newcommand{\baru}{ʉ}
\newcommand{\baruH}{\'\baru}
\newcommand{\baruL}{\`\baru}

\newcommand{\ep}{ε}
\newcommand{\epH}{\'\ep}
\newcommand{\epL}{\`\ep}

\newcommand{\schwa}{ə}
\newcommand{\schwaH}{\'ə}
\newcommand{\schwaL}{\`ə}

\newcommand{\oo}{ɔ}
\newcommand{\ooH}{\'\oo}
\newcommand{\ooL}{\`\oo}

\newcommand{\ds}{\textsuperscript{
	\hspace*{-2pt}\begin{tikzpicture}
		\draw[-{>[scale=0.5]}] (0,0.4) --(0,0.25);
	\end{tikzpicture}}}

\newcommand{\ch}{t͡ʃ}
\newcommand{\dz}{d͡ʒ}

\newcommand{\tgl}{ʔ}

%shortcuts for the complementizers
\newcommand{\mbuL}{mb\baruL}
\newcommand{\mbuHL}{mb\baruH\baruL}
\newcommand{\mbuLH}{mb\baruL\baruH}
\newcommand{\la}{lá}
\newcommand{\nda}{ndà}

\newcommand{\tsc}[1]{\textsc{#1}}
\renewcommand{\textscb}{ʙ}
\newcommand{\ipa}[1]{#1} %disable IPA

\newcommand{\SM}[1]{#1}

\DeclareNewSectionCommand
  [
    counterwithin = chapter,
    afterskip = 2.3ex plus .2ex,
    beforeskip = -3.5ex plus -1ex minus -.2ex,
    indent = 0pt,
    font = \usekomafont{section},
    level = 1,
    tocindent = 1.5em,
    toclevel = 1,
    tocnumwidth = 2.3em,
    tocstyle = section,
    style = section
  ]
  {appendixsection}

\renewcommand*\theappendixsection{\Alph{appendixsection}}
\renewcommand*{\appendixsectionformat}
              {\appendixname~\theappendixsection\autodot\enskip}
\renewcommand*{\appendixsectionmarkformat}
              {\appendixname~\theappendixsection\autodot\enskip}

\renewcommand{\lsChapterFooterSize}{\footnotesize}
 
  %% hyphenation points for line breaks
%% Normally, automatic hyphenation in LaTeX is very good
%% If a word is mis-hyphenated, add it to this file
%%
%% add information to TeX file before \begin{document} with:
%% %% hyphenation points for line breaks
%% Normally, automatic hyphenation in LaTeX is very good
%% If a word is mis-hyphenated, add it to this file
%%
%% add information to TeX file before \begin{document} with:
%% %% hyphenation points for line breaks
%% Normally, automatic hyphenation in LaTeX is very good
%% If a word is mis-hyphenated, add it to this file
%%
%% add information to TeX file before \begin{document} with:
%% \include{localhyphenation}
\hyphenation{
affri-ca-te
affri-ca-tes 
Līk-pāk-páln
pro-sod-ic
phe-nom-e-non
Chi-che-wa
Lu-bu-ku-su
Ngbu-gu
Boyel-dieu
Mat-chi
pho-neme
Mil-em-be
Nyan-chera
Mc-Pher-son
Tsoo-tso
Sku-pin
dis-tin-guishes
con-ser-va-tion
Me-dum-ba
}

\hyphenation{
affri-ca-te
affri-ca-tes 
Līk-pāk-páln
pro-sod-ic
phe-nom-e-non
Chi-che-wa
Lu-bu-ku-su
Ngbu-gu
Boyel-dieu
Mat-chi
pho-neme
Mil-em-be
Nyan-chera
Mc-Pher-son
Tsoo-tso
Sku-pin
dis-tin-guishes
con-ser-va-tion
Me-dum-ba
}

\hyphenation{
affri-ca-te
affri-ca-tes 
Līk-pāk-páln
pro-sod-ic
phe-nom-e-non
Chi-che-wa
Lu-bu-ku-su
Ngbu-gu
Boyel-dieu
Mat-chi
pho-neme
Mil-em-be
Nyan-chera
Mc-Pher-son
Tsoo-tso
Sku-pin
dis-tin-guishes
con-ser-va-tion
Me-dum-ba
}
 
  \togglepaper[1]%%chapternumber
}{}

\begin{document}
\maketitle 

\section{Introduction}\label{sec:sulemana:1}

Bùlì does not have overt morphological marking to systematically distinguish finite clauses from nonfinite clauses. As such, notions like these will appear not to be useful descriptive labels in the Syntax and Semantics of the language. This finite-nonfinite distinction is often manifested differently including the distri-bution of overt DPs and empty categories: finite verbs license overt DPs while nonfinite verbs cannot without special mechanisms. As an illustration, consider the paradigm in \REF{ex:sulemana:1} from English. The external arguments of the nonfinite com-plements which are coindexed with a matrix argument has to be null.

\ea%1
    \label{ex:sulemana:1}
    \ea%1a
    \label{ex:sulemana:1a}
    Mary remembered [*she/pro to buy a book]
    \ex%1b
    \label{ex:sulemana:1b}
    Mary persuaded John [*he/pro to buy a book]
    \z
\z

The goal of this paper is twofold: first to argue that despite the lack of morpho-logical marking to distinguish between finiteness and nonfiniteness, such a dis-tinction exists in the language. Second, I argue that unlike the nonfiniteness of the English type languages, the nonfinite clauses of Bùlì obligatorily take overt pronominals which must be coindexed with a matrix argument. The rest of the paper is organized as follows: In \sectref{sec:sulemana:2}, I present a brief background to this language. In \sectref{sec:sulemana:3}, I present a discussion of the finite-nonfinite distinction in the language. \sectref{sec:sulemana:4} argues that with the exception of its overtness, the pronominal in the subject of the nonfinite clause must be controlled. Section 6 discusses and concludes the paper.

\section{Bùlì}\label{sec:sulemana:2}

It is a Mabia (Gur) language spoken in Sandema in the Upper East Region of Ghana. It has three dialects: Central, Northern and Southern. This paper concentrates on the Central dialect. It is a tone language with three contrastive tones: Low, Mid and High. It is also a noun class language with five singular classes and four plural classes build around the pronouns. Its basic clause structure is SVO. Temporal interpretation of a predicate is sensitive to the eventive/stative distinc-tion in the language (tenseless). Unmarked eventive predicates have default past interpretation while their stative counterparts have present interpretation,\footnote{This is related to what is sometimes called the factitive constructions which are attested crosslin-guistically Haitian \citep{Dechaine1991}, and Fɔ̀n-gbè \citep{Avolonto1992} among others. \citet{Stowell1991} also observes that bare eventive verbs have only a past reading while bare stative verbs are interpreted as non-past in what is called headlinese.} \REF{ex:sulemana:2}. \todo{\citet{Stowell1991} is missing from the bib file}

\ea%2
    \label{ex:sulemana:2}
    \ea%2a
    \label{ex:sulemana:2a}
    \gll    Asibi  dà gbǎŋ.\\
            Asibi buy book\\
    \glt    `Asibi bought a book.' 
    \ex%2b
    \label{ex:sulemana:2b}
    \gll    Asouk sèbì Ajohn.\\
            Asouk know John\\
    \glt    `Asouk knows John.'
    \z 
\z

The data in \REF{ex:sulemana:2} also eliminates the potential for analyzing the low tone on the verb as the past tense morpheme, since both predicates are marked with a low tone. I will therefore consider the low tone as a form of 3rd person agreement. In the next section, I will present various arguments to show that Bùlì meets the general conditions on the finite vs. nonfinite distinction since an adequate classification of some syntactic structures would not be achieved if such a distinction is not assumed. 

\section{The finite-nonfinite distinction}\label{sec:sulemana:3}

Since Bùlì is a tenseless (factitive) language, notions like finite-nonfinite will appear not to be  useful descriptions in the syntax and semantics of the language. Contrary to this, I present four arguments/diagnostics that will distinguish between them. These diagnostics, I argue, bring out two different kinds of nonfinite clauses: The first kind, which I call nonfinite obligatory control  complement (nonfinite-OC) illustrated in (\ref{ex:sulemana:3}-\ref{ex:sulemana:4}), the pronominal subject of the embedded clause  must be co-indexed with a matrix argument. 

\ea%3
    \label{ex:sulemana:3}
    {\bf{Nonfinite-OC: Subject-coindexation}}
    \ea%3a
    \label{ex:sulemana:3a}
    \gll    Asouk$_i$ tìerì [*(wà$_i$/$_*${}$_j$) dā gbáŋ]. \\
            Asouk remember \textsc{3sg}. buy book\\
    \glt    `Asouk remembered to buy a book.'
    \ex%3b
    \label{ex:sulemana:3b}
    \gll    Núrmà$_i$  zèrì [*(bà$_i$/$_*${}$_j$) dā gbáŋ]. \\
            people.\textsc{def.pl} refuse \textsc{3pl}. buy book\\
    \glt    `The people refused to buy a book.'\todo{Full stops in non-portmanteau morphemes throughout the file.}
    \z
\z

\ea%4
    \label{ex:sulemana:4}
    {\bf{Nonfinite-OC: Object-coindexation}}
    \ea%4a
    \label{ex:sulemana:4a}
    \gll    Mí túlím Asouk$_i$ zúk [*(wà$_i$/$_*${}$_j$) dā gbáŋ]. \\
            \textsc{1sg}. turn Asouk head \textsc{3sg}. buy book\\
    \glt    `I convinced Asouk to buy a book.'
    \ex%4b
    \label{ex:sulemana:4b}
    \gll    Mí túlím núrmà$_i$ zúk [*(bà$_i$/$_*${}$_j$) dā gbáŋ]. \\
            \textsc{1sg}. turn people.\textsc{def.pl} head \textsc{3pl}. buy book\\
    \glt    `I convinced the people to buy a book.' 
    \z
\z

In the second kind, which I call the nonfinite non-obligatory control complement (nonfinite-NOC) as illustrated in (\ref{ex:sulemana:5}-\ref{ex:sulemana:6}),  allows a full DP in the subject of the embedded clause. These further distinguishes between those that are not introduced by complementizers \REF{ex:sulemana:5} and those requiring complementizers \REF{ex:sulemana:6}. The other differences between these constructions will be made clear as the discussion proceeds as the main  reason for this section is to defend the finite-nonfinite distinction in the language. 

\ea%5
    \label{ex:sulemana:5}
    {\bf{Nonfinite-NOC without COMP}}
    \ea%5a
    \label{ex:sulemana:5a}
    \gll    Mí à-yā: Asouk dā gbáŋ.\\
            \textsc{1sg}. \textsc{asp}-want Asouk buy book\\
    \glt    `I want Asouk to buy a book.' 
    \ex%5b
    \label{ex:sulemana:5b}
    \gll    Nà:wǎ tè síuk Asouk dā gbáŋ. \\
            chief.\textsc{def} give path Asouk buy book\\
    \glt    `The chief gave permission for Asouk to buy a book.' 
    % \ex%5c
    % \label{ex:sulemana:5c}
    % \gll    Kù màgsì Asouk dā gbáŋ\\
    %         \textsc{3sg}. right asouk buy book\\
    % \glt    `It is right for Asouk to buy a book.'
    \z
\z

\ea%6
    \label{ex:sulemana:6}
    {\bf{Nonfinite-NOC with COMP}}
    \ea%6a
    \label{ex:sulemana:6a}
    \gll    Kù à-fɛ̄ ātī Asouk dā gbáŋ.\\
            \textsc{3sg}. \textsc{asp}-necessary \textsc{c} Asouk buy book\\
    \glt    `It is necessary for Asouk to buy a book.' 
    \ex%6b
    \label{ex:sulemana:6b}
    \gll    Kù nālā ātī Asouk dā gbáŋ.\\
            it good \textsc{c} Asouk buy book\\
    \glt    `It is good for Asouk to buy a book.' 
    \z
\z

The first argument to consider for the finite-nonfinite distinction comes from the Low-tone (Agreement) on the verb. In finite clauses, a third person subject triggers a low tone (agreement) on the verb when there are no preverbal particles intervening between the subject and the verb \REF{ex:sulemana:7}. This is the case for all 3rd person arguments in matrix as well as embedded clauses for different DP including r-expressions and pronouns and regardless of the tone on the argument. Note that the embedded clauses of the nonfinite clauses bear mid tones (see examples (\ref{ex:sulemana:3}-\ref{ex:sulemana:6}). 

\ea%7
    \label{ex:sulemana:7}
    \ea%7a
    \label{ex:sulemana:7a}
    \gll    Wà dà gbǎŋ.  \\
            \textsc{3sg}.  buy book\\
    \glt    `S/he bought a book.'
    % \ex%7?
    % \gll    Asibi  dà gbǎŋ  \\
    %         Asibi buy book\\
    % \glt    `Asibi bought a book.' 
    \ex%7b
    \label{ex:sulemana:7b}
    \gll    Bí:ká dà gbǎŋ.  \\
            child.\textsc{def} \textsc{3sg}.  buy book\\
    \glt    `S/he bought a book.' 
    % \ex%7?
    % \gll    Asouk pàchìm Asibi dà gbǎŋ \\
    %         Asouk think Asibi   buy book\\
    % \glt    `Asouk thought Asibi bought a book.' 
    \ex%7c
    \label{ex:sulemana:7c}
    \gll    Asouk pàchìm wà dà gbǎŋ.  \\
            Asouk think \textsc{3sg}. buy book\\
    \glt    `Asouk thought he bought a book.' 
    \z
\z

The second argument for treating the embedded clauses above as nonfinite clauses is based on the distribution of the future marker. In finite clauses, both matrix and embedded, the future marker is required for future interpretations. This is illustrated in \REF{ex:sulemana:8}.

\ea%8
    \label{ex:sulemana:8}
    {\bf{Future marker required in finite clauses}}
    \ea%8a
    \label{ex:sulemana:8a}
    \gll    Asibi àlí dā gbáŋ.\\
            Asibi \textsc{fut} buy book\\
    \glt    `Asibi will buy a book.'
    \ex%8b
    \label{ex:sulemana:8b}
    \gll    Asouk pàchìm Asibi chūm *(àlí) dā gbáŋ.\\
            Asouk think Asibi tomorrow \textsc{fut}  buy book\\
    \glt    `Asouk thought Asibi will buy a book tomorrow.' 
    \z
\z

In contrast, the future marker is excluded from all the nonfinite clauses. The examples in \REF{ex:sulemana:9} illustrate this point. The inability of the future marker to appear in nonfinite clauses reminds us of nonfinite clauses in Chinese which cannot take modals like {\it{hui}} `will' Huang (1998).\footnote{Whether the future marker {\it{àlí}} in Bùl`i is a modal or a tense marker is beyond the focus of this paper, however.} 

\ea%9
    \label{ex:sulemana:9}
    {\bf{Future marker excluded from nonfinite clauses}}
    \ea%9a
    \label{ex:sulemana:9a}
    \gll    Asouk sìak *(wà$_i$/$_*${}$_j$) chūm (*àlí) dā gbáŋ.\\
            Asouk agree \textsc{3sg}. tomorrow \textsc{fut} buy book\\
    \glt    `Asouk agreed to buy a book tomorrow.'
    % \ex[*]{%9?
    % \gll    Asouk$_i$ tìerì *(wà$_i$/$_*${}$_j$) àlí dā gbáŋ.\\
    %         Asouk remember  \textsc{3sg}. \textsc{fut} buy book\\}
    \ex%9b
    \label{ex:sulemana:9b}
    \gll    Mí$_i$ à-yā: Asouk chūm (*àlí) dā gbáŋ.\\
            \textsc{1sg}. \textsc{asp}-want Asouk  tomorrow \textsc{fut}  buy book\\
    \glt    `I want Asouk to buy a book tomorrow.' 
    % \ex[*]{%9?
    % \gll    Nà:wà tè síuk Asouk àlí dā gbáŋ. \\
    %         chief.\textsc{def} give path Asouk \textsc{fut} buy book\\}
    % \ex[*]{%9?
    % \gll    Kù à-fɛ̄ ātī Asouk àlí dā gbáŋ.\\
    %         \textsc{3sg}. \textsc{asp}-necessary \textsc{c} Asouk \textsc{fut} buy book\\}
    \z
\z

The third argument for the finite-nonfinite distinction comes from subject questions. In-situ subject wh-questions in finite clauses require the obligatory presence of {àlì-\textsc{ali}} in the clausal spine \REF{ex:sulemana:10}. 

\ea%10
    \label{ex:sulemana:10} 
    {\bf{Finite clauses: In-situ subject wh-questions require \textsc{ali}}}
    \ea%10a
    \label{ex:sulemana:10a}
    \gll    Ká wānā *(àlì) dā gbáŋ a\\
            \textsc{q} who \textsc{ali} buy book \textsc{prt} \\
    \glt    `Who bought a book?' 
    \ex%10b
    \gll    Asouk pàchìm ka wana  *(àlì) dā gbáŋ a\\
            Asouk think \textsc{q} who \textsc{ali}  buy book \textsc{prt} \\
    \glt    `Who does Asouk think bought  a book?' 
    \z
\z

Although it is generally possible to question the subject of a nonfinite-NOC complement (\ref{ex:sulemana:11a}-\ref{ex:sulemana:11b}), questioning the subject requires the obligatory absence of {\it{àlí}}. 
The ungrammaticality of example \REF{ex:sulemana:11c} shows that it is not possible to question the controlled subject of the nonfinite-OC complement. Hence another difference between finite and nonfinite clauses. %the nonfinite-OC complement and the nonfinite-NOC complement.  

%11
\ea%11
    \label{ex:sulemana:11} 
    {\bf{Nonfinite clauses: In-situ subject wh-questions doesn't require \textsc{ali}}}
    \ea%11a
    \label{ex:sulemana:11a}
    \gll    Mí$_i$ à-yā: ka wana (*àlì) dā gbáŋ a \\
            \textsc{1sg}. \textsc{asp}-want \textsc{q} who  \textsc{ali}  buy book \textsc{prt}  \\
    \glt    `Who do I want for him to buy a book?' 
    \ex%11b
    \label{ex:sulemana:11b}
    \gll    Nà:wà tè síuk  ka wānā (*àlì) dā gbáŋ a \\
            chief.\textsc{def} give path \textsc{q} who  \textsc{ali} buy book \textsc{prt} \\
    \glt    `Who did the chief give permission to buy a book?' 
    \ex[*]{%11c
    \label{ex:sulemana:11c}
    \gll    Asouk$_i$ tìerì ka wana (*àlì) dā gbáŋ.  \\
            Asouk remember \textsc{q} who  \textsc{ali} buy book\\}
    % \ex%11?
    % \gll    Kù à-fɛ̄ ātī ká wānā (*àlì) dā gbáŋ a \\
    %         \textsc{3sg}. \textsc{asp}-necessary \textsc{c} \textsc{q} who  \textsc{ali} buy book \textsc{prt}  \\
    % \glt    `Who is it necessary for him to buy a book?'
    \z
\z

Is it possible that what we are questioning in (\ref{ex:sulemana:11a}-\ref{ex:sulemana:11b}) are arguments of the matrix predicates rather than subjects of the complement clauses as a result {\it{àlí}} is not required, since nonsubjects don't require an {\it{àlí}}. This is indeed a possible analysis especially for \REF{ex:sulemana:11a}, however, there is evidence that these arguments are subjects of the complement clauses and as such the absence of {\it{àlí}} cannot be attributed to questioning a nonsubject argument. 

Bùlì employs resumptive pronouns in long distance extraction of a subject, \REF{ex:sulemana:12a} but not an object, \REF{ex:sulemana:12b}.  

\ea%12
    \label{ex:sulemana:12}
    \ea%12a
    \label{ex:sulemana:12a}
    \gll    (Ká) wānā *(ātì) fì pá:-chīm *(wà) àlì dīg lāmmú: \\
            \textsc{q} who \textsc{ati}   \textsc{2sg}.  think   \textsc{3sg}.  \textsc{ali} cook  meat.\textsc{def}\\
    \glt    `Who do you think cooked the meat?'
    \ex%12b
    \label{ex:sulemana:12b}
    \gll    (Ká) bʷā *(ātì) fì pá:-chīm Asouk dìgì: (*bu)  \\
            \textsc{q} what \textsc{ati}  \textsc{2sg}. think Asouk  cook   \textsc{3sg}. \\
    \glt    `What do you think Asouk cooked?'
    \z
\z

If the questioned arguments in \REF{ex:sulemana:11} above are objects, they should pattern with object extraction and if they are subjects they should pattern with long distance subject extraction. As shown in \REF{ex:sulemana:13} they pattern with long distance subject extraction by requiring a resumptive pronoun. 

\ea%13
    \label{ex:sulemana:13}
    \ea%13a
    \label{ex:sulemana:13a}
    \gll    (Ká) wānā *(ātì) mi$_i$ à-yā: *(wà)  (*àlì) dā gbáŋ a \\
            \textsc{q} who \textsc{ati}   \textsc{1sg}. \textsc{asp}-want \textsc{3sg}.   \textsc{ali}  buy book \textsc{prt} \\
    \glt    `Who do I want  to buy a book?'
    \ex%13b
    \label{ex:sulemana:13b}
    \gll    (Ká) wānā *(ātì) nà:wà tè síuk  *(wà) (*àlì) dā gbáŋ a \\
            \textsc{q} who \textsc{ati}  chief.\textsc{def} give path  \textsc{3sg}.  \textsc{ali} buy book \textsc{prt}\\
    \glt    `Who did the chief give permission to buy a book?'
    % \ex%13?
    % \gll    (Ká) wānā *(ātì) kù {à-fɛ̄ ātī  *(wà) (*àlì) dā gbáŋ a \\
    %         \textsc{q} who \textsc{ati}  \textsc{3sg}.  \textsc{asp}-necessary  \textsc{c}.  \textsc{3sg}. \textsc{ali} buy book \textsc{prt} \\
    % \glt    `For whom is it necessary to buy a book?'
    \z
\z

The final argument for the finite-nonfinite distinction comes from N-word licensing.\footnote{For more on NPIs see \citet{Zeijlstra2017}}\todo{Missing from the bib file. \citet{GiannakidouZeijlstra2017}?} 
It has been noted that NPIs and n-words differ in that NPIs can be licensed across the border of a clause, but n-words cannot. 
N-words in Bùlì are formed by reduplicating indefinite nouns, and they must always occur with negation regardless of their position and number.

%\ea  \label{O}
%\ea\label{Oa}
%\gll *Gianni non ha dichiarato che ha visto niente \\
%Gianni NEG has declared that has seen n-word\\
%\glt `Gianni did not declare that he saw anything' 
%
%\ex\label{Ob}
%\gll Gianni non ha dichiarato che ha visto alcunché \\
%Gianni NEG has declared that has seen anything\\
%\glt `Gianni did not declare that he saw anything'   (Zeijlstra 2007:512)
%
%\ex\label{Oc}
%\gll Gianni non ha dichiarato di aver visto niente \footnote{I thank Stanislao Zompí for this data.}  \\
%Gianni NEG have.3sg declared of to.have  seen n-word\\
%\glt `Gianni has not declared to have seen anything'  (Italian)
%\z\z
%A closer examination, however, reveal  that n-words can be licensed across the border of nonfinite clauses. 
%
%\ea  \label{P}
%\gll Gianni non ha dichiarato di aver visto niente \footnote{I thank Stanislao Zompí for this data.}  \\
%Gianni NEG have.3sg declared of to.have  seen n-word\\
%\glt `Gianni has not declared to have seen anything'  (Italian)
%\z
%%
%A similar observation is made for Hebrew.%\footnote{I thank Moysh Bar-Lev for this data.}
%%
%%\ex.  \label{Q}
%%\ag. Dani lo zaxar liknot  {\v sum-davar}  \\
%%Dani NEG remember to-buy n-word\\
%%`Dani didn't remember to buy anything'
%%\bg. *Dani lo zaxar se-hu kana \v sum-davar\\
%%Dani NEG remember that-he bought n-word\\
%%`Dany didn't remember that he bought anything'
%

\ea%14
    \label{ex:sulemana:14}
    \ea%14a
    \label{ex:sulemana:14a}
    \gll    Asouk  *(àn)  dīg jāab-jāab  *(ā). \\
            Asouk  \textsc{neg1} cook thing-thing  \textsc{neg2}  \\
    \glt    `Asouk didn't cook anything.' 
    \ex%14b
    \label{ex:sulemana:14b}
    \gll    Wāi-wāi  *(àn)  dīg lām  *(ā). \\
            someone-someone   \textsc{neg1}  cook meat  \textsc{neg2}  \\
    \glt    `Nobody cooked meat.' 
    \ex%14c
    \label{ex:sulemana:14c}
    \gll    Wāi-wāi  *(àn)  dīg jāab-jāab  *(a). \\
            someone-someone  \textsc{neg1}  cook thing-thing  \textsc{neg2}  \\
    \glt    `Nobody cooked anything.' 
    \z
\z

In Bùlì and other languages, including Italian and Hebrew, n-words can be licensed across the border of nonfinite clauses in but not finite ones, \REF{ex:sulemana:15}. 

%15
\ea%15
    \label{ex:sulemana:15}
    \ea%15a
    \label{ex:sulemana:15a}
    \gll    Asouk àn tīeri  wà dīg jāab-jāab  *(ā).\\
            Asouk  \textsc{neg1}  remember \textsc{3sg}. cook thing-thing  \textsc{neg2} \\
    \glt    `Asouk didn't remember to cook anything.'
    \ex[*]{%15b
    \gll    Asouk àn tīeri āsī wà dìg jāab-jāab  *(ā).\\
            Asouk  \textsc{neg1}  remember \textsc{c} \textsc{3sg}. cook thing-thing  \textsc{neg2} \\}
    \glt    `Asouk didn't remember that he cooked anything.' 
    \z
\z

%\subsection{Interim Summary}
I have shown in this section that the distinction between finite and nonfinite clauses hold in the language and that the complement clauses in (\ref{ex:sulemana:3}-\ref{ex:sulemana:6})  are indeed nonfinite. 
%The arguments for this conclusion came from the low tone (agreement), future marker, in-situ subject wh-questions, and n-word licensing.
In the next section, I  argue that the nonfinite clauses in Bùlì require pronominal subjects which covaries with the number and class of the matrix argument that it is coindexed with, and as such, despite its overtness, this pronominal shares all the properties of \textsc{pro}. 

\section{Obligatory controlled subjects}\label{sec:sulemana:4}

In the previous section, I argued that certain clauses in the language are nonfinite. However, unlike the \todo{Single quotation markers are for linguistic meaning only}`regular' nonfinite clauses, the nonfinite clauses of Bùlì require an overt pronominal. In this section, I will argue that the pronominal in the embedded clauses of nonfinite-OC clauses is a subject and must be controlled by a matrix argument.  As noted, the subjects of the nonfinite-OC clauses must be co-indexed with a matrix argument. In \REF{ex:sulemana:16} the co-indexation is with a matrix subject and in \REF{ex:sulemana:17}, it is with a matrix object. Note that the pronominal also covaries with the number and class of the matrix argument it is coindexed with. 

\ea%16
    \label{ex:sulemana:16}
    \ea%16a
    \label{ex:sulemana:16a}
    \gll    Asouk$_i$ tìerì *(wà$_i$/$_*${}$_j$) dā gbáŋ.\\
            Asouk remember  \textsc{3sg}. buy book\\
    \glt    `Asouk remembered to buy a book.'
    \ex%16b
    \label{ex:sulemana:16b}
    \gll    Núrmà$_i$ bàŋ *(bà$_i$/$_*${}$_j$) kpārī tóukú.\\
            people.\textsc{def.pl} forget \textsc{3pl} lock door\\
    \glt    `The people forgot to lock the door.' 
    \z
\z

%17
\ea%17
    \label{ex:sulemana:17}
    \ea%17a
    \label{ex:sulemana:17a}
    \gll    Mì túlím Asouk$_i$ zuk *(wà$_i$/$_*${}$_j$) bāsī dēlā.\\
            \textsc{1sg}. turn Asouk head \textsc{3sg}. leave here\\
    \glt    `I convinced Asouk to leave.'
    \ex%17b
    \label{ex:sulemana:17b}
    \gll    Núr-wá fὲ  bísáŋá$_i$  *(bà$_i$/$_*${}$_j$) bāsī dēlā. \\
            man.\textsc{def} force children \textsc{3pl}. leave here\\
    \glt    `The man forced the children to leave.' 
    \z
\z

Although the subject of these nonfinite clauses are overt, applying the diagnostics from  \citet{Hornstein1999, Landau2013}; and \citet{Williams1980} the often called  signature properties of \textsc{pro}, suggest that the overt pronominal behaves like \todo{Single quotation markers are for linguistic meaning only}`\textsc{pro}' except for its overtness.\todo{Anything missing from this sentence?}

First, like \textsc{pro}, and unlike pronouns, the subjects of these clauses must pick up their antecedent in the immediately preceding clause, \REF{ex:sulemana:18}. That is, just like \textsc{pro}, and unlike a pronominal subject of a finite clause, the pronominal subject of the most embedded clause can only  be {\it{núrmà}} `the people' which is the subject of the immediately preceding clause, It cannot refer to the singular subject of the matrix clause, \REF{ex:sulemana:18a}. The referential facts are different when the most embedded clause is a finite clause. As shown in \REF{ex:sulemana:18b}, the pronominal subject can freely refer to subject of the matrix clause. 

\ea%18
    \label{ex:sulemana:18}
    {\bf{Long-distance binding of this pronominal is not possible}}.
    \ea%18a
    \label{ex:sulemana:18a}
    \gll    Asouk$_i$  nỳa  āsī núrmà$_j$ tìeri  *wà$_i${}$_/$bà$_j$ dā gbáŋ.\\
            Asouk realize \textsc{c} people.\textsc{def.pl} remember \textsc{3sg/3pl}.  buy book\\
    \glt    `Asouk realized that the people remembered to buy a book.'
    \ex%18b
    \label{ex:sulemana:18b}
    \gll    Asouk$_i$ nyà āsī núrmà$_j$ wèin āyīn  wà$_i${}$_/$bà$_j$ dà gbáŋ.\\
            Asouk realize \textsc{c} people.\textsc{def.pl}  say \textsc{c} \textsc{3sg/3pl}.  buy book\\
    \glt    `Asouk realized that the people say that he bought a book.' 
    \z
\z

Second, Non c-command coreference of this pronominal is not possible, \REF{ex:sulemana:19}. The antecedent of a pronominal subject in nonfinite clauses must c-command it, just like \textsc{pro} \REF{ex:sulemana:19}. In \REF{ex:sulemana:19a}, {\it{Asouk}} cannot be the antecedent of the pronominal subject because it doesn't c-command it. On the contrary, in finite clauses this restriction does not hold \REF{ex:sulemana:19b}. 

\ea%19
    \label{ex:sulemana:19}
    {\bf{The pronominal must be c-commanded by its antecedent}}
    \ea%19a
    \label{ex:sulemana:19a}
    \gll    Asouk$_i$  dóamà$_j$ bàŋ *wà$_i${}$_/$bà$_j$ kpārī tóukú.\\
            Asouk friend.\textsc{def.pl} forget \textsc{3sg/3pl}. lock door\\
    \glt    `Asouk's friends forgot to lock the door.' 
    \ex%19b
    \label{ex:sulemana:19b}
    \gll    Asouk$_i$  dóamà$_j$ pàchìm  wa$_i${}$_/$bà$_j$ kpàrì tóukú.\\
            Asouk friend.\textsc{def.pl} think \textsc{3sg/3pl}. lock door\\
    \glt    `Asouk's friends thought he locked the door.' 
    \z
\z

In ellipsis contexts, the pronominal must be construed sloppily. In example \REF{ex:sulemana:20} which involves a finite complement, the pronominal could be construed strictly or sloppily. In the strict reading, Asouk was the first to say that he bought a book before Asibi said he (Asouk) bought a book. 

%\ea \label{} {\bf{Finite clause: the pronominal is ambiguous: strict or sloppy}}\\
%\gll Asouk wien wa da gbang alege Asibi wien wa da gbang.\\
%Asouk say 3sg buy book before Asibi say 3sg buy book\\
%\glt `Asouk said he bought a book before Asibi.' 
%\z

%20
\ea%20
    \label{ex:sulemana:20}
    {\bf{Finite clause: the pronominal is ambiguous: strict or sloppy}}\\
    \gll    Asouk$_i$ wìen wà  dà gbáŋ àlēgē Asibi$_j$  {\sout{wìen wà$_i${}$_/${}$_j$  dà gbáŋ}} \\
            Asouk say \textsc{3sg}. buy book before Asibi {\sout{say \textsc{3sg}. buy book}}\\
    \glt    `Asouk said he bought a book before Asibi {\sout{said that he bought a book.'}} 
\z

In contrast, in the nonfinite case \REF{ex:sulemana:21}, the pronominal must be construed sloppily. In \REF{ex:sulemana:21}, Asouk was the first to agree to buy the book before Asibi also agreed to buy a book.

%21
\ea%21
    \label{ex:sulemana:21}
    {\bf{Non-finte clause: the pronominal must be construed as sloppy}}\\
    \gll    Asouk$_i$ sìak wa$_i$ dā gbáŋ àlēgē Asibi$_j$ {\sout{sìak wa$_{*i}${}$_/${}$_j$ dā gbáŋ}} \\
            Asouk agree \textsc{3sg}. buy book before Asibi {\sout{agree \textsc{3sg}. buy book}}\\
    \glt    `Asouk agreed to buy a book before Asibi {\sout{agreed to buy a book.'}}
\z

Another observation is that \textsc{pro} in OC environments is interpreted as a bound variable i.e it must be bound by the controller. This results in the difference in interpretation between \REF{ex:sulemana:22a} and \REF{ex:sulemana:22b}. While the pronoun in the nonfinite complement is limited to the bound variable reading in \REF{ex:sulemana:22a}, the pronoun in \REF{ex:sulemana:22b} is not. 

\ea%22
    \label{ex:sulemana:22}
    {\bf{ The pronominal is interpreted as a bound variable.} }
    \ea%22a
    \label{ex:sulemana:22a}
    \gll    Wā:-wāi$_i$ àn tīeri  wà$_{i/}${}$_*${}$_j$ dā gbáŋ a. \\
            someone-someone  \textsc{neg1} remember \textsc{3sg}. buy book  \textsc{neg2} \\
    \glt    `No one remembered to buy a book.' 
    \ex%22b
    \label{ex:sulemana:22b}
    \gll    Wā:-wāi$_i$ àn wēn  wà$_{i/}${}$_j$ dā gbáŋ a. \\
            someone-someone \textsc{neg1} say \textsc{3sg}. buy book  \textsc{neg2}\\
    \glt    `No one said that he bought a book.' 
    \z
\z

Finally, as observed by \citet{Chierchia1989} infinitival controlled constructions are always {\it{de se}}. The pronominal subject in these complements must be {\it{de se.}} This reading arises when the controller/antecedent is the subject of an attitude predicate and is aware that the complement proposition pertains to him/herself. In any situation where the attitude holder mistakes the embedded subject as someone other him/herself, the pronominal cannot be truthfully used.\\

{\it{Consider the following Scenario: An old man (Asouk) is listening to the credentials of three people being considered for a chieftaincy title. Not knowing that the credentials of the second person mentioned refers to him (because he hardly remembers anything), he says to his wife 'this person should be given the title'.}}\\

In this scenario \REF{ex:sulemana:23} is false. An outcome expected if the pronominal is an instance of a lexicalised \textsc{pro}. 

%23
\ea%23
    \label{ex:sulemana:23}
    \gll    Asouk$_i$ à-zīentī wà$_i$ chīm nà:b.  \\
            Asouk eager \textsc{3sg}. become chief \\
    \glt    `Asouk is eager to become a chief.'
\z

It is important to note here that there have been reports in the literature where overt pronominal subjects are possible in controlled infinitives when they are focused \citep{Szabolcsi2009}.\footnote{See also \citet{Barbosa2009} and  \citet{Madigan2008}.}

%\ea 
%\gll   Senki nem Akart csak \H{o} le ül-ni.\\
%       Nobody not wanted-\textsc{3sg}. only he/she sit-\textsc{inf}\\
%\glt   `Nobody wanted it to be the case that only he/she takes a seat' (Hungarian: \citealt{Szabolcsi2009})
%\z

There is, however, solid evidence that the controlled pronominal subjects in Bùlì are not focused marked, thus making it distinct from all the other cases identified where `\textsc{pro}' is overt. Bùlì makes a distinction between weak and strong pronouns, with strong pronouns sometimes associated with focus. Weak pronouns usually have low tones. In controlled constructions, only the weak pronouns are acceptable \REF{ex:sulemana:24a}. The strong pronouns are grammatical only when they are modified by scope bearing element like {\it{also/too}} similar to what \citet{Szabolcsi2009}  identified \REF{ex:sulemana:24b}.

\ea%24
    \label{ex:sulemana:24}
    \ea%24a
    \label{ex:sulemana:24a} 
    \gll    Asouk$_i$ sàik *(wà$_i$/*wá$_i$) dā gbáŋ. \\
            Asouk agree \textsc{3sg}.  buy book\\
    \glt    `Asouk agreed to buy a book.'
    \ex%24b
    \label{ex:sulemana:24b} 
    \gll    Asouk$_i$ sàik *(*wà$_i$/wá$_i$) mɛ̄ dā gbá.ŋ \\
            Asouk agree \textsc{3sg}. also  buy book\\
    \glt    `Asouk agreed to also buy a book.'
    \z
\z

Crucially, focus is not required to overtly expressed the subject. This indicates that overtness of the infinitival subject does not depend on focus in this language. Thus what we uncover here is not identical to the ones identified by \citet{Szabolcsi2009} and others. 

\section{The pronominal is a subject}\label{sec:sulemana:5}

In the previous section, I have established that the overt pronominal in the nonfinite complement clause must be controlled. An alternative view is that \textsc{pro} is actually null as in other languages, and that this pronominal is an agreement marker found in nonfinite clauses similar to what we see in languages like Brazilian Portuguese. This alternative view though attractive faces a number of challenges.
First, analogous agreement marking is conspicuously lacking in both finite and other nonfinite clauses \REF{ex:sulemana:25}. In finite clauses in both matrix and embedded contexts, repeating the pronominal as an agreement marker results in ungrammaticality (\ref{ex:sulemana:25a}-\ref{ex:sulemana:25b}). 
Similarly, repeating the pronominal in the nonfinite clauses that permit referential DPs as in (\ref{ex:sulemana:25c}-\ref{ex:sulemana:25d}) is also ungrammatical. 

\ea%25
    \label{ex:sulemana:25}
    \ea%25a
    \label{ex:sulemana:25a}
    \gll    Asibi$_i$  (*wà$_i$) dà gbáŋ. \\
            Asibi  \textsc{3sg}. buy book\\
    \glt    `Asibi bought a book.'
    \ex%25b
    \label{ex:sulemana:25b}
    \gll    Asouk pàchìm Asibi$_i$ (*wà$_i$) dà gbáŋ. \\
            Asouk think Asibi   \textsc{3sg}. buy book\\
    \glt    `Asouk thought Asibi bought a book.'
    \ex%25c
    \label{ex:sulemana:25c}
    \gll    Mí à-yā: Asouk$_i$ (*wà$_i$) dā gbáŋ. \\
            \textsc{1sg}. \textsc{asp}-want Asouk\textsc{3sg}. buy book\\
    \glt    `I want Asouk to buy a book.'
    \ex%25d
    \label{ex:sulemana:25d}
    \gll    Kù à-fɛ̄ ātī Asouk$_i$ (*wà$_i$) dā gbáŋ. \\
            \textsc{3sg} \textsc{asp}-necessary \textsc{c} Asouk \textsc{3sg}. buy book\\
    \glt    `It is necessary for Asouk to buy a book.' 
    \z
\z

Second, claiming that this pronominal is agreement suggests that it is not in Spec of the embedded clause. However, the placement of adverbials in both kinds of clauses places the pronominal in the same location as matrix and embedded subjects, Spec,TP. The adverb, tomorrow, follows the subject in matrix clauses whether they are referential \REF{ex:sulemana:26a} or pronominal \REF{ex:sulemana:26b}. 

\ea%26
    \label{ex:sulemana:26}
    \ea%26a
    \label{ex:sulemana:26a}
    \gll    Asibi chúm àlí dā gbáŋ. \\
            Asibi tomorrow \textsc{fut} buy book\\
    \glt    `Asibi will buy a book tomorrow.' 
    \ex%26b
    \label{ex:sulemana:26b}
    \gll    Wà chúm àlí dā gbáŋ. \\
            \textsc{3sg}. tomorrow buy book\\
    \glt    `He will buy a book tomorrow.' 
    \z
\z

In nonfinite clauses too, the adverb follows the pronominal \REF{ex:sulemana:27}. This shows that the pronominal is in Spec, TP just as in matrix subjects and that is not a clitic on the verb as one might assume. 

\ea%27
    \label{ex:sulemana:27}
    \ea%27a
    \label{ex:sulemana:27a}
    \gll    Asouk$_i$ sàik *(wà$_i$/$_*${}$_j$) chúm dā gbáŋ.\\
            Asouk agree \textsc{3sg}. tomorrow buy book\\
    \glt    `Asouk agreed to buy a book tomorrow.'
    \ex%27b
    \label{ex:sulemana:27b}
    \gll    Asouk$_i$ à-yā:  *(wà$_i$/$_*${}$_j$) chúm dā gbáŋ.\\
            Asouk  \textsc{asp}-want \textsc{3sg}. tomorrow buy book\\
    \glt    `Asouk wants to buy a book tomorrow.'  
    \z
\z

Finally, the pronominal in the nonfinite clauses can be modified just as any subject, \REF{ex:sulemana:28}.

\ea%28
    \label{ex:sulemana:28}
    \ea%28a
    \label{ex:sulemana:28a}
    \gll    Asouk mɛ̄ dà gbáŋ.\\
            Asouk also buy book\\
    \glt    `Asibi also bought a book.' 
    \ex%28b
    \label{ex:sulemana:28b}
    \gll    Asouk$_i$ sàik *(wá$_i$/$_*${}$_j$) mɛ̄ dā gbáŋ.\\
            Asouk agree \textsc{3sg}. also also buy book\\
    \glt    `Asouk agreed to also buy a book.'
    \ex%28c
    \label{ex:sulemana:28c}
    \gll    Asouk$_i$ à-yā:  *(wá$_i$/$_*${}$_j$) mɛ̄ dā gbáŋ.\\
            Asouk \textsc{asp}-want \textsc{3sg}. also buy book\\
    \glt    `Asouk wants to also buy a book.'  
    \z
\z

All these facts put together suggest that the pronominal is not an agreement marker or a clitic on the verb, but a real subject in Spec, TP. 
%The arguments for this come from the lack of agreement marking in analogous environments, the pronominal precedes the adverb just as subjects in matrix clauses do and the fact that they can be modified by scope bearing elements. 

%{\textbf{Summary}}
This section has shown that the overt pronominal subject of the nonfinite complement is a subject and must be controlled by a matrix argument. Except for its overtness  this pronominal shares the properties of \textsc{pro}, distinguishing it from the pronouns. 
%\item{The critical arguments come from the so-called  signature properties of  OC \textsc{pro} (Williams 1980, Hornstein 1999, and Landau 2013): long-distance control of \textsc{pro} is not possible, non c-command of control is not possible, \textsc{pro} must be de se, only bound variable reading of \textsc{pro} is possible, and \textsc{pro} under ellipsis must be construed sloppily.} 
%\item{Secondly, the facts of \ref{P} also eliminates the possibility of analyzing them as focused pronouns. These facts show that the Bùlì control constructions have a lexical instance of \textsc{pro} not induced by focus.} 
%\end{itemize}
%

\section{Discussions and conclusion}\label{sec:sulemana:6}

The previous sections have established that Buli makes a distinction between finite and nonfinite clauses. Secondly these nonfinite clauses requires overt DPs in their specifier position. This conclusion raises a number of interesting questions for the various approaches to Control. I highlight these approaches and argue that the subjectless-based approach to  control cannot be extended to Buli for obvious reasons. I will, however, leave open the option between the Agree-based model and the movement based model for future studies. 

I group the approaches to Control into two: (1) Subject-based Accounts: (i)Agree-based accounts \citet{Landau2001, Landau2013} in which the relation between the matrix argument and the embedded subject, \textsc{pro} (a null nominal element distinct from a trace or copy) is established via an agree operation. On this view, \textsc{pro} is inherently null because of its association with infinitival T, which only assigns null Case \citep{ChomskyLasnik1993}, and the (ii) Movement-based account \citep{Hornstein1999} which considers the relation between \textsc{pro} and the matrix argument as involving movement. This approach accounts for the nullness of the subject by considering it as an unpronounced copy of the matrix argument. (2) Subjectless-based Accounts: these approaches take the lack of overt subjects in the embedded complements as evidence for the lack of a subject \citep{Bresnan1982, Dowty1985, JackendoffCulicover2003, Wurmbrand1998, Wurmbrand2004, Chierchia1989} essentially arguing that there is no \textsc{pro}. \citet{Wurmbrand1998, Wurmbrand2004} for example considers infinitival complements as instances of restructuring where the matrix verb selects a VP complement.  
In the previous sections, I have argued that the overt pronominal found in nonfinite complements under control shares all the properties of \textsc{pro}. The clear fact that nonfinite controlled complements surface with overt subjects raises interesting questions for theories of Control which denies the syntactic presence of a subject. Thus approaches to Control, which take the lack of an overt subject in control complements as evidence for the lack of a subject essentially arguing against the existence of \textsc{pro}, cannot be extended to Bùlì for obvious reasons. The fact that the controlled element is overt in Bùlì, I argue, shows that phonetic nullness is not an inherent property of the controlled element. Hence any approach to control that necessarily requires controlled elements to be null cannot be universal. 
The present data also presents a challenge for standard theories of DP  distribution based on abstract Case. It has been standardly assumed that DPs are licensed in structural positions where Case assignment is possible. Subject DPs are assumed to get nominative Case from  finite clauses. Since the complement clauses are nonfinite, the prediction of abstract Case theory is either their subjects be null or the DPs should be getting Case from elsewhere. An open question is thus how the overt pronominal is licensed.\\



\section*{Abbreviations}
I use the following abbreviations in this paper
1$=$ First person, 
2$=$ Second  person, 
3$=$ Third  person, 
\textsc{asp}$=$ Aspect
\textsc{c/comp} $=$ Complementizer, 
\textsc{def}$=$ Definite,
\textsc{fut}$=$ Future,
\textsc{neg}$=$ Negation,
\textsc{pl}$=$ Plural,
\textsc{sg} $=$ Singular. 

\section*{Acknowledgements}
For helpful discussions, I thank İsa Kerem Bayırlı, Kenyon Branan, Suzana Fong, Sabine Iatridou, David Pesetsky, Norvin Richards, Michelle Sheehan, the audience of ACAL49, and two anonymous reviewers. Any and all errors are my own.

\printbibliography[heading=subbibliography,notkeyword=this]

\end{document}
