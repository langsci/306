\documentclass[output=paper,colorlinks,citecolor=brown]{langscibook}
\ChapterDOI{10.5281/zenodo.6393746}

\title{Edges and extraction: Evidence from Chichewa}

\author{Kenyon Branan\affiliation{MIT} and Colin Davis\affiliation{MIT}}

\abstract{A growing body of work argues that Agree has the effect of “unlocking” certain domains (\textit{phases}) such that otherwise illicit extraction from them becomes permitted. However, existing proposals disagree on whether Agree is in fact always required to Unlock phases for extraction, or only required for extractions that would otherwise bypass the phase edge. We argue that constraints on extraction from DP in Chichewa reported in \citet{Mchombo2004, Mchombo2006} provide evidence for the latter theory, which privileges phase edges. We go on to show that this theory makes correct predictions about Dinka, the language \citet{VanUrkRichards2015} originally argued provides evidence for the opposite conclusion.}
\IfFileExists{../localcommands.tex}{
  \addbibresource{localbibliography.bib}
  \usepackage{langsci-optional,langsci-branding}
\usepackage{langsci-gb4e}
% \usepackage{langsci-textipa}
% \usepackage{langsci-glyphs}
\usepackage[linguistics]{forest}
\usepackage{tabto}
\usepackage{multirow}
\usepackage{bbding}

\usepackage[normalem]{ulem}

\usepackage{tikz-qtree}

\usepackage{enumitem}

\usepackage{multicol}
\usepackage{stmaryrd} %double brackets

\makeatletter
\let\pgfmathModX=\pgfmathMod@
\usepackage{pgfplots,pgfplotstable}%
\let\pgfmathMod@=\pgfmathModX
\makeatother
\usepgfplotslibrary{colorbrewer}
\usetikzlibrary{fit}

\usepackage{jambox}
\usepackage{tikz-qtree-compat}
\usetikzlibrary{arrows, arrows.meta}
\usepackage{longtable}
\usepackage{subcaption}

  \makeatletter
\let\thetitle\@title
\let\theauthor\@author
\makeatother

\newcommand{\togglepaper}[1][0]{
%   \bibliography{../localbibliography}
  \papernote{\scriptsize\normalfont
    \theauthor.
    \thetitle.
    To appear in:
    Change Volume Editor \& in localcommands.tex
    Change volume title in localcommands.tex
    Berlin: Language Science Press. [preliminary page numbering]
  }
  \pagenumbering{roman}
  \setcounter{chapter}{#1}
  \addtocounter{chapter}{-1}
}

\newcommand{\bari}{\ipabar{\i}{.5ex}{1.1}{}{}}
\newcommand{\notipa}[1]{\textnormal{#1}}

\newcommand{\agre}{\textsc{agr}-\ol{eene}}

\renewcommand{\emph}[1]{\textit{#1}} % resetting a setting from ling-macros-modified (I think?)

% forest settings to make compact but (mostly) straight-spined trees:
\forestset{
fairly nice empty nodes/.style={
            delay={where content={}{shape=coordinate,for parent={
                  for children={anchor=north}}}{}}
, angled/.style={content/.expanded={$<$\forestov{content}$>$}}
}}

\forestset{sn edges/.style={for tree={parent anchor=south, child anchor=north}}}

\newcommand{\bex}{\begin{exe}}
\newcommand{\fex}{\end{exe}}

\newcommand{\bxl}{\begin{exe}}
\newcommand{\fxl}{\end{exe}}

\newcommand{\ix}[1]{\textsubscript{#1}}
\newcommand{\alert}[1]{\textbf{#1}}
\newcommand{\ol}[1]{\textit{#1}}


			\usetikzlibrary{shapes,arrows,positioning,decorations,decorations.pathmorphing,intersections}
\forestset{
nice empty nodes/.style={
    for tree={calign=fixed edge angles},
    delay={where content={}{shape=coordinate,for siblings={anchor=north}}{}}
},
}

\definecolor{dark-gray}{gray}{0.3}

%\usepackage{dingbat,pifont}


%%%%%%%%%%%%For arrows%%%%%%%%%%%%%

\newcommand\Tikzmark[2]{%
  \tikz[remember picture]\node[inner sep=0pt,outer sep=0pt] (#1) {#2};%
}
\NewDocumentCommand\DrawArrow{O{}mmmmO{3}}{
\tikz[remember picture,overlay]
  \draw[->,line width=0.8pt,shorten >= 2pt,shorten <= 2pt,#1]
    (#2) -- ++(0,-#6\ht\strutbox) coordinate (aux) -- node[#4] {#5} (#3|-aux) -- (#3);
}
\NewDocumentCommand\DrawDotted{O{}mmmmO{3}}{
\tikz[remember picture,overlay]
  \draw[->,line width=0.9pt,dotted,shorten >= 2pt,shorten <= 2pt,#1]
    (#2) -- ++(0,-#6\ht\strutbox) coordinate (aux) -- node[#4] {#5} (#3|-aux) -- (#3);
}
\NewDocumentCommand\DrawLine{O{}mmmmO{3}}{
\tikz[remember picture,overlay]
  \draw[line width=0.8pt,shorten >= 2pt,shorten <= 2pt,#1]
    (#2) -- ++(0,-#6\ht\strutbox) coordinate (aux) -- node[#4] {#5} (#3|-aux) -- (#3);
}
%%%%%%%%%%%%%%%%%%%%%%%%%%%%%%%%%%%%%


\newcommand{\baru}{ʉ}
\newcommand{\baruH}{\'\baru}
\newcommand{\baruL}{\`\baru}

\newcommand{\ep}{ε}
\newcommand{\epH}{\'\ep}
\newcommand{\epL}{\`\ep}

\newcommand{\schwa}{ə}
\newcommand{\schwaH}{\'ə}
\newcommand{\schwaL}{\`ə}

\newcommand{\oo}{ɔ}
\newcommand{\ooH}{\'\oo}
\newcommand{\ooL}{\`\oo}

\newcommand{\ds}{\textsuperscript{
	\hspace*{-2pt}\begin{tikzpicture}
		\draw[-{>[scale=0.5]}] (0,0.4) --(0,0.25);
	\end{tikzpicture}}}

\newcommand{\ch}{t͡ʃ}
\newcommand{\dz}{d͡ʒ}

\newcommand{\tgl}{ʔ}

%shortcuts for the complementizers
\newcommand{\mbuL}{mb\baruL}
\newcommand{\mbuHL}{mb\baruH\baruL}
\newcommand{\mbuLH}{mb\baruL\baruH}
\newcommand{\la}{lá}
\newcommand{\nda}{ndà}

\newcommand{\tsc}[1]{\textsc{#1}}
\renewcommand{\textscb}{ʙ}
\newcommand{\ipa}[1]{#1} %disable IPA

\newcommand{\SM}[1]{#1}

\DeclareNewSectionCommand
  [
    counterwithin = chapter,
    afterskip = 2.3ex plus .2ex,
    beforeskip = -3.5ex plus -1ex minus -.2ex,
    indent = 0pt,
    font = \usekomafont{section},
    level = 1,
    tocindent = 1.5em,
    toclevel = 1,
    tocnumwidth = 2.3em,
    tocstyle = section,
    style = section
  ]
  {appendixsection}

\renewcommand*\theappendixsection{\Alph{appendixsection}}
\renewcommand*{\appendixsectionformat}
              {\appendixname~\theappendixsection\autodot\enskip}
\renewcommand*{\appendixsectionmarkformat}
              {\appendixname~\theappendixsection\autodot\enskip}

\renewcommand{\lsChapterFooterSize}{\footnotesize}
 
  %% hyphenation points for line breaks
%% Normally, automatic hyphenation in LaTeX is very good
%% If a word is mis-hyphenated, add it to this file
%%
%% add information to TeX file before \begin{document} with:
%% %% hyphenation points for line breaks
%% Normally, automatic hyphenation in LaTeX is very good
%% If a word is mis-hyphenated, add it to this file
%%
%% add information to TeX file before \begin{document} with:
%% %% hyphenation points for line breaks
%% Normally, automatic hyphenation in LaTeX is very good
%% If a word is mis-hyphenated, add it to this file
%%
%% add information to TeX file before \begin{document} with:
%% \include{localhyphenation}
\hyphenation{
affri-ca-te
affri-ca-tes 
Līk-pāk-páln
pro-sod-ic
phe-nom-e-non
Chi-che-wa
Lu-bu-ku-su
Ngbu-gu
Boyel-dieu
Mat-chi
pho-neme
Mil-em-be
Nyan-chera
Mc-Pher-son
Tsoo-tso
Sku-pin
dis-tin-guishes
con-ser-va-tion
Me-dum-ba
}

\hyphenation{
affri-ca-te
affri-ca-tes 
Līk-pāk-páln
pro-sod-ic
phe-nom-e-non
Chi-che-wa
Lu-bu-ku-su
Ngbu-gu
Boyel-dieu
Mat-chi
pho-neme
Mil-em-be
Nyan-chera
Mc-Pher-son
Tsoo-tso
Sku-pin
dis-tin-guishes
con-ser-va-tion
Me-dum-ba
}

\hyphenation{
affri-ca-te
affri-ca-tes 
Līk-pāk-páln
pro-sod-ic
phe-nom-e-non
Chi-che-wa
Lu-bu-ku-su
Ngbu-gu
Boyel-dieu
Mat-chi
pho-neme
Mil-em-be
Nyan-chera
Mc-Pher-son
Tsoo-tso
Sku-pin
dis-tin-guishes
con-ser-va-tion
Me-dum-ba
}
 
  \togglepaper[1]%%chapternumber
}{}

\begin{document}
\SetupAffiliations{mark style=none}
\maketitle 

\section{Introduction}\label{sec:branan:1}

In this paper, we argue that Chichewa (Bantu) informs us about the conditions on movement in syntax. A small body of work has argued that one such condition involves Agree: that is, certain domains (\textit{phases}) must be targeted by Agree and hence \textit{Unlocked} before a movement operation can extract something from them, at least some of the time. As it stands, there are two theories about when Unlocking is necessary for extraction:

\begin{enumerate}
\item \textit{either/or} (\citealt{RackowskiRichards2005, Halpert2016, Halpert2019, Branan2018}):

    Extraction from a phase requires Agree to Unlock that phase, unless the element to be extracted can move via the phase edge.

\item \textit{both/and} (\citealt{VanUrkRichards2015} on Dinka):

    Extraction from a phase always requires the containing phase to be Unlocked through Agree, and for extraction to pass through the edge.
\end{enumerate}

While theory \#1, the \emph{either/or} theory, predicts extraction to be permitted either by moving through the phase edge or by Unlocking the phase, theory \#2, the \emph{both/and} view, predicts that both are prerequisites for extraction.

In this work, we argue that patterns of extraction from DP in Chichewa (Bantu) provide evidence for theory \#1, which privileges edges. As we will see, the Chichewa patterns make it evident that extraction really is possible from the edge of a “locked” phase, which has not been agreed with.
% % \subsection{Roadmap of the paper}
\sectref{sec:branan:2} provides more background on theories of extraction from phases and Unlocking. \sectref{sec:branan:3} provides the relevant facts from Chichewa, which in \sectref{sec:branan:4} we apply to the question of determining which Unlocking theory is correct. In \sectref{sec:branan:5} we show in more detail how this account captures the facts in Chichewa and extends to correct predictions about Dinka (Nilo-Saharan), which \citeauthor{VanUrkRichards2015} originally argued mandates in favor of theory \#2.

\section{Background: How to escape a phase}\label{sec:branan:2}

\subsection{Phases and the edge escape hatch}\label{sec:branan:2.1}

An influential idea in contemporary syntactic theory is that the derivation proceeds in phases (\citealt{Chomsky2000, Chomsky2001}; a.o). The set of phases is at least CP, vP, and likely DP as well, DP being our focus in this paper. In essence, Chomsky argues that phases constrain the syntactic derivation because operations outside of a phase cannot target elements within that phase, with one exception: phrases in the edge (specifier) of a phase remain accessible.

We see this system schematized in \REF{ex:branan:1}, where movement of XP directly out of the phase's complement is impossible. However, XP can escape the phase if it stops in the edge of the phase first:\largerpage[-1]

%1
\ea\label{ex:branan:1} 
    Must exit a phase via its edge\\\vskip\baselineskip
{[} \Tikzmark{A}{\uline{\phantom{xp}}}  \ldots \ldots \,  [$_{PhaseP}$ \Tikzmark{B}{\uline{\phantom{xp}}}  \ldots \ldots \,  \Tikzmark{C}{XP} ] ]
\DrawArrow{C}{A}{}{✘}[-1.2]
\DrawArrow{C}{B}{}{}[1.2]
\DrawArrow{B}{A}{}{}[1.25]
\z

This is one hypothesis about the conditions on escaping phases, which explains why syntactic operations seem to have a cyclic, local, punctuated character.\footnote{For Chomsky, the accessibility of edges is attributed to the nature of spellout: a phase head spells-out its complement, thus sending it to PF and LF, and out of the syntactic derivation. Because the specifier and head of a phase are outside of the phase's complement, they remain accessible. The view of phases for which we present evidence suggests a different reason for the accessibility of phase edges, which predicts that material deeper within a phase is not inaccessible, but rather more difficult to access due to locality constraints on probing.}

\subsection{Unlocking by Agree}\label{sec:branan:2.2}\largerpage[-2]

A distinct method of phase escape is proposed by \citet{RackowskiRichards2005, VanUrkRichards2015, Halpert2016, Halpert2019}, and \citet{Branan2018}. These works argue that Agreeing with a phase Unlocks it, making it transparent for extraction. For proponents of Unlocking theory \#1 previewed in the introduction, this Unlocking allows extraction to bypass the phase edge, which would otherwise be impossible. This is illustrated in \REF{ex:branan:2}, where the probe P Agrees with PhaseP, permitting XP to escape PhaseP without stopping in its edge.

%2
\ea\label{ex:branan:2}
    Agree with a phase allows bypassing of the phase edge\\\vskip1.5\baselineskip
{[} \Tikzmark{A}{\uline{\phantom{xp}}}  \ldots \Tikzmark{P}{P} \ldots \,  [$_{PhaseP}$ \Tikzmark{B}{\uline{\phantom{xp}}}  \ldots\Tikzmark{G}{\vphantom{1em}}\ldots \,  \Tikzmark{C}{XP} ] ]
\DrawArrow{C}{A}{}{}[1.2]
\DrawLine{P}{G}{above}{\textsl{\footnotesize Agree $\rightarrow$ PhaseP}}[-1.2]

\z

\citet{Branan2018} presents evidence for this theory based on cross-linguistic restrictions on extraction from DP, where extraction from nominals seems to be contingent on whether or not the nominal controls agreement morphology. An example of this comes from Northern Ostyak (Uralic, \citealt{Nikolaeva1999}). Northern Ostyak has obligatory agreement with subjects, but agreement with objects is optional. In this language extraction of possessors from subjects is always possible. However, extraction of possessors from objects is possible only when the object is agreed with, as the contrast between \REF{ex:branan:3} and \REF{ex:branan:4} shows.

\ea%3
    \label{ex:branan:3}
    Possessor extraction from object is impossible without agreement\\
    \gll    *\textbf{Juwan}$_{k}$ motta [ xot-əl \_$_{k}$ ] k\v{a}\'{s}alə-s-əm.\\
            \hphantom{*}John before {} house-\textsc{3sg} {} {} see-\textsc{t-1sg} \\
    \glt    `I saw John's house before.'
\z

\ea 
    \label{ex:branan:4}
    Agreement with object permits extraction of possessor\\
    \gll    \textbf{Juwan}$_{k}$ motta [ xot-əl \_$_{k}$ ] k\v{a}\'{s}alə-s-e:m.  \\
            John before {} house-\textsc{3sg} {} {} see-\textsc{t-sg.\textbf{3obj}} \\
    \glt    `I saw John's house before.'
\z

In \REF{ex:branan:3} we see that possessor extraction from a direct object is not possible on its own. Rather, as \REF{ex:branan:4} shows, such extraction requires the direct object to be agreed with. Under the theory we argue for today, the extraction in \REF{ex:branan:3} would have been grammatical if the possessor could extract via the DP edge, but evidently such a derivation is not available. (See \citealt{Branan2018} for details about how such extraction is constrained). Consequently, Agree with the object is necessary to permit extraction. This is because Agree has the effect of Unlocking DPs, as this pattern in Northern Ostyak and similar ones across other languages indicate.

\subsection{Two versions of Unlocking theory}\label{sec:branan:2.3}

As previewed in \sectref{sec:branan:1} above, two variants of Unlocking theory have been proposed. \citeauthor{RackowskiRichards2005} initially proposed Unlocking theory \#1, which affords “escape hatch” status to phase edges, in order to connect Unlocking to an understanding of locality and successive-cyclic movement through phase edges. Later evidence for Unlocking has been indirect enough to be compatible with a reformulation eventually offered by \citeauthor{VanUrkRichards2015}' (\citeyear{VanUrkRichards2015}) work on Dinka, which we've introduced as theory \#2. We argue that Chichewa verifies the predictions of theory $\#$1, and we go on to show that Dinka actually behaves as we would expect if theory \#1 is really the correct choice.

\section{Extraction and agreement in Chichewa}\label{sec:branan:3}

\citet{Mchombo2004, Mchombo2006} discusses how the formation of discontinuous DPs in Chichewa is constrained by the distribution of the Object Marker (OM), a morpheme in the verbal complex which expresses the $\phi$-features (in particular for Chichewa, the noun-class features, here represented numerically) of an internal argument. We hypothesize that these discontinuous DPs are derived by movement, and we will argue in this section that the OM involves a probe that Agrees with the $\phi$-features of an internal argument DP. Importantly for our proposal, certain correlations hold between the possibility of extraction from DP, and whether or not the OM Agrees with that DP.

\subsection{When extraction and agreement go together}\label{sec:branan:3.1}

Mchombo shows that in the basic case, the OM is optional in Chichewa. However, under certain circumstances the OM becomes necessary. For instance, the OM is required when extracting an adjective, as the contrast between \REF{ex:branan:5} and \REF{ex:branan:6} shows. In \REF{ex:branan:5}, the OM reflects the features of the class 4 object DP (\textit{lions}) out of which the adjective (\textit{aged}) has been fronted:

\ea%5
    \label{ex:branan:5}
    Adjective extraction possible with OM \hfill{(\citealt[ex. 21b]{Mchombo2004})}\\
    \gll    \textbf{Y\'{o}k\'{a}lamba}$_{k}$ any\'{a}n\'{i} a-na-\textbf{\'{i}}-g\'{u}l-\'{i}l\'{a} mak\'{a}s\'{u} awa \'{o}b\'{u}ntha [ mik\'{a}ngo  \_$_{k}$ ].   \\
             \textbf{4\textsc{sm}-aged} 2-baboons 2\textsc{sm-pst}-\underline{\textbf{4\textsc{om}}}-buy-\textsc{appl-fv} 6-hoes 6-these 6\textsc{sm}-blunt {}  \underline{\textbf{4}}-lions \\
    \glt    `The baboons  bought the aged lions these blunt hoes.'
\z

However, in \REF{ex:branan:6} the OM is absent, and such extraction is blocked:

\ea%6
    \label{ex:branan:6}
    Adjective extraction impossible without OM  \hfill{(\citealt[ex. 4a]{Mchombo2006})}\\
    \gll    * \textbf{Zakuda}$_{k}$ ats\'{i}k\'{a}n\'{a} \'{a} mf\'{u}mu a-a-gul-\'{a} [ mb\^{u}zi   \_$_{k}$].   \\
            {} \textbf{10\textsc{sm}-black} 2-girls 2\textsc{assoc} 9-chief 2\textsc{sm-pfv}-buy-\textsc{fv} {} \textbf{10}.goats {} \\
    \glt    `The chief's girls have bought black goats.'
\z

In just the same way, extraction of a demonstrative requires the OM:

\ea%7
    \label{ex:branan:7}
    No demonstrative extraction without OM \hfill{(\citealt[ex. 2b–c]{Mchombo2006})}\\
    \gll    \textbf{Awa}$_{k}$ njuchi izi zi-n\'{a}-\textbf{*(w\'{a})}-l\^{u}m-\'{a} [ alenje \_$_{k}$ \'{o}p\'{u}sa  ].\\
            \textbf{2}.\textsc{prox} 10.bees 10\textsc{prox} 10\textsc{sm-pst}-\underline{\textbf{2\textsc{om}}}-bite-\textsc{fv} {} \underline{\textbf{2.}}hunters {} 2\textsc{sm}.foolish {}\\
    \glt    `These bees bit the foolish hunters.'
\z

Note that both adjectives and demonstratives in Chichewa originate on the right side of N(P), which sits at the left edge of the DP.

\ea%8
    \label{ex:branan:8}
    The Chichewa DP: N $<$ Dem $<$ Adj  \hfill{(from \citealt[ex. 2a]{Mchombo2006})}\\
    \gll    alenje awa \'{o}p\'{u}sa \\
            hunter(N) these(Dem) foolish(Adj) \\
\z

This pattern is comparable to that of Northern Ostyak: Certain elements in DP can only be extracted when the DP that contains them controls agreement morphology. Importantly, in contrast, certain other elements are extractable whether or not the OM agrees with the containing DP, as the next section shows.

\subsection{When extraction and agreement come apart}\label{sec:branan:3.2}

So far, we've established that in Chichewa extraction of adjectives and demonstratives requires the OM to agree with the containing DP. As mentioned, in Chichewa the left edge of DP is occupied by N(P). As discussed further in \sectref{sec:branan:4}, we take this as evidence that NP moves to the left edge of the Chichewa DP. Furthermore, NP is the only element of the Chichewa DP that can always be extracted, whether or not the containing DP controls the OM, as \REF{ex:branan:9} and \REF{ex:branan:10} below show. In \REF{ex:branan:9} the OM is present, while in \REF{ex:branan:10} it is absent, but both of these examples involve acceptable extraction of NP:

\ea%9
    \label{ex:branan:9}
    Extraction of NP possible with OM \hfill{(\citealt[50, ex. 16c]{Mchombo2004})}\\
    \gll    \textbf{Alenje}$_{k}$  zi-n\'{a}-\textbf{w\'{a}}-l\'{u}m-a [ nj\'{u}ch\'{i} izi ] [ \_$_{k}$  awa \'{o}p\'{u}sa ]. \\
            \textbf{2-hunters} 10\textsc{sm-pst}-\textbf{\underline{2\textsc{om}}}-bite-\textsc{fv} {} 10-bees 10-\textsc{prox} {} {} \, \underline{\textbf{2}}-\textsc{prox} \underline{\textbf{2}}\textsc{sm}-foolish \\
    \glt    `These bees bit these foolish hunters.'
\ex%10
    \label{ex:branan:10}
    Extraction of NP possible without OM \hfill{(\citealt[ex. 4]{Mchombo2006})}\\
    \gll    \textbf{Mb\^{u}zi}$_{k}$ ats\'{i}k\'{a}n\'{a} \'{a} mf\'{u}mu a-a-gul-\'{a} [ \_$_{k}$ z\'{a}k\'{u}da ]. \\
            \textbf{10-goats} 2-girls 2-\textsc{assoc} 9chief 2\textsc{s-pfv}-buy-\textsc{fv} {} {} \textbf{10\textsc{sm}}-black   \\
    \glt    `Goats, the chief's girls have bought black (ones).'
\z

This fact is significant: It is unclear, given the theory of Unlocking presented in \citet{VanUrkRichards2015}, why extraction of NP is permitted even in cases where the containing object DP is not Agreed with by the OM. Under such a theory, we expect extraction of NP to require the DP which originally contained it to always be agreed with, and thereby Unlocked -- contrary to fact.

However, these Chichewa patterns are expected under a theory of Unlocking like that of \citet{RackowskiRichards2005} and \citet{Branan2018}, in which extraction is permitted by moving via the phase edge, or by Agree with the phase itself. In the previous subsection, we saw that extraction of adjectives and demonstratives, which originate at a non-edge position in DP, requires the OM to Agree with the containing DP. However, as we've just seen, agreement is not required when extracting NP, the element occupying the edge of the Chichewa DP.

We discuss a more specific model of extraction from DP in Chichewa in \sectref{sec:branan:5}. Before that, in the next subsection, we present some evidence that the OM does indeed involve an Agree relation with the DP that extraction exits.

\subsection{The OM involves an Agree dependency}\label{sec:branan:3.3}\largerpage

As \citet{Baker2016} overviews, whether the OM in Bantu languages is a pure agreement marker or a (doubled) pronominal clitic is a subject of debate. For the account proposed in this paper, the presence of the OM must be contingent on some sort of probe-goal Agree relationship, even if the morphological effect of that syntactic relationship is a pronominal clitic rather than a mere agreement marker.\footnote{See \citet{Preminger2015} for an argument that some sorts of clitic doubling involve an Agree relation.} We argue that the dependency between the OM and the DP it cross-references indeed shows locality effects characteristic of such probe-goal relations.

While the OM in Chichewa can normally target a direct object, certain other nominals interrupt a potential relation between the OM probe and the direct object. For example, if a benefactive indirect object is present, it must be targeted by the OM rather than the direct object, as shown in \REF{ex:branan:11}.\footnote{As a reviewer points out, this effect has independent precedent in \citet{Kramer2014}, who argues that the presence of IO blocks agreement (and hence clitic doubling) with DO in Amharic.}

\ea%11
    \label{ex:branan:11}
    IO blocks agreement with DO \hfill{(\citealt[101, ex. 41a-b]{Mchombo2004})}\\
    \gll    Alenje a-ku-\textbf{wá/*zí}-phík-il-á zítúmbûwa$_{DO}$  \textbf{anyáni}$_{IO}$. \\
            2-hunters 2\textsc{sm-prs}-\underline{\textbf{\textsc{2om/*8om}}}-cook-\textsc{appl-fv} 8-pancakes \textbf{\underline{2-}baboons}  \\
    \glt    `The hunters are cooking the baboons some pancakes.'
\z

\citet{Mchombo2004} discusses how similar considerations apply to raised possessors, which the OM must target rather than a direct object.\footnote{\citeauthor{Mchombo2004} states that the raised possessor behaves like the object of the clause in other ways, such as accessibility to passivization. This is also expected if the raised possessor is more local than the direct object to higher probes, thus passivization (presumably involving probing by T and subsequent A-movement) targets the raised possessor rather than the lower direct object.} As expected, an adjective or demonstrative can only be extracted from the raised possessor, and then, only when it is Agreed with:\footnote{The benefactive arguments and raised possessors that intervene between the OM probe and the direct object nevertheless appear to the right of that object. What is necessary for our account is that these DPs are structurally above the object at the relevant point in the derivation where the OM Agrees with its goal, regardless of their final linear position.}

\ea%12
    \label{ex:branan:12}
    No extraction from DO with raised possessor \hfill{(\citealt[55, ex. 22e]{Mchombo2004})}\\
    \gll    * \textbf{\'{A}ákûlu}$_{k}$ mkángó u-ku-wá-dy-él-á [ maûngu \_$_{k}$ ] [ amalinyé ógúnata ]. \\
            {} 6\textsc{sm}-big 3-lion 3\textsc{sm-prs}-\textbf{\underline{2\textsc{om}}}-eat-\textsc{appl-fv} {} 6-pumpkins {} {} {} \textbf{\underline{2}}-sailors 2\textsc{assoc}-foolish {} \\
    \glt    `The lion is eating for the foolish sailors (their) big pumpkins.'
\ex%13
    \label{ex:branan:13}
    Extraction possible from raised possessor \hfill{(\citealt[56, ex. 22g]{Mchombo2004})}\\
    \gll    \textbf{\'{O}gúnata}$_{k}$ mkángó u-ku-wá-dy-él-á [ maúngú áákûlu ] [ amalinyêlo \_$_{k}$ ]. \\
            2\textsc{assoc}-foolish 3-lion 3\textsc{sm-prs}-\textbf{\underline{2\textsc{om}}}-eat-\textsc{appl-fv} {} 6-pumpkins 6\textsc{sm}-big  {} {} \textbf{\underline{2}}-sailors  \\
\glt        `The lion is eating for the foolish sailors (their) big pumpkins.'
\z

Further locality effects emerge from the interaction of extraction with recursive possession. Example \REF{ex:branan:14} shows a recursive possession configuration:

\ea%14
    \label{ex:branan:14}
    Recursive possession \hfill{(\citealt[60, ex. 29]{Mchombo2004})}\\
    \gll    Anyaní á mísala a-ku-chí-phwány-a [$_{DP3}$ \textbf{chipanda}  [$_{DP2}$ \textbf{chá}  \textbf{kazitápé} [$_{DP1}$ \textbf{wá} \textbf{alenje} ]]]. \\
            2-baboons 2\textsc{assoc} 4-madness 2\textsc{sm-prs}-7\textsc{om}-smash-\textsc{fv} {} 7-calabash {} 7\textsc{assoc} 1\textsc{a}-spy {} 1\textsc{assoc} 2-hunters  \\
    \glt    `The mad baboons are smashing the calabash of the hunters’ spy.'
\z

There are restrictions on what can be extracted from recursive possession structures. As we see in \REF{ex:branan:15}, it is possible to extract the possessor (DP2) of the direct object (DP3) which controls OM, in this case \emph{chipanda} (`calabash').\largerpage

\ea%15
    \label{ex:branan:15}
    The “outermost” possessor may be extracted \hfill{(\citealt[60, ex. 30a]{Mchombo2004})}\\
    \gll    [$_{DP2}$ \textbf{Chá}  \textbf{kazitápé} [$_{DP1}$ \textbf{wá} \textbf{alenje} ]]$_{k}$ Anyaní á mísala a-ku-chí-phwány-a [$_{DP3}$ \textbf{chipanda}   \_$_{k}$ ].  \\
            {} 7\textsc{assoc} 1\textsc{a}.spy {} 1\textsc{assoc} 2-hunters {}  2-baboons 2\textsc{assoc} 4-madness 2\textsc{sm-prs}-\textbf{\underline{7\textsc{om}}}-smash-\textsc{fv} {} \textbf{\underline{7}}-calabash   \\
    \glt    `The mad baboons are smashing the calabash of the hunters’ spy.'
\z

In contrast, as we see in \REF{ex:branan:16}, it is not possible to extract DP1, the possessor of the possessor (DP2) of the element that controls the OM (DP3).

\ea%16
    \label{ex:branan:16}
    The “innermost” possessor may not extract \hfill{(\citealt[61, ex. 32]{Mchombo2004})}\\
    \gll    * [$_{DP1}$ \textbf{Chá} \textbf{alenje}]$_{k}$  Anyaní á mísala a-ku-chí-phwány-a [$_{DP3}$ \textbf{chipanda}  [$_{DP2}$ \textbf{chá}  \textbf{chiphadz\'{u}w\'{a}} \_$_{k}$]].\\
    {} {} 7\textsc{assoc} 2-hunters   2-baboons 2\textsc{assoc} 4-madness 2\textsc{sm-prs}-\textbf{\underline{7\textsc{om}}}-smash-\textsc{fv} {} \textbf{\underline{7}}-calabash {} 7Assoc {\underline{7}-beauty queen}\\
    \glt    `The mad baboons are smashing the calabash of the hunters’ beauty queen.'
\z

We've seen that extraction from the non-edge of the Chichewa DP requires the containing DP to be agreed with by the OM. In \REF{ex:branan:16}, notice that the OM has class 7 $\phi$-features that could have been received via probing of the direct object \textit{calabash}, or the possessor of the direct object, \textit{beauty queen}, both of which are of noun class 7. Thus \REF{ex:branan:16} should be able to represent a structure where the OM agreed with the possessor of the direct object, if it were possible to do so. Such agreement should Unlock the possessor of the direct object, thus permitting extraction of the possessor of the possessor. However, such extraction is impossible.


This fact suggests that in \REF{ex:branan:16} only the direct object can have been agreed with. Thus the direct object is Unlocked and its possessor can be extracted, as in \REF{ex:branan:15}. However, this possessor was not agreed with and thus remains locked, so the possessor of the possessor remains un-extractable, as \REF{ex:branan:16} has shown. This is as expected if locality requirements only allow the OM to agree with and hence Unlock the structurally closest DP, the direct object, which the OM probe will necessarily encounter before any DPs which are its sub-constituents.\footnote{The same asymmetries hold for left branch extraction in Russian (p.c. Tanya Bondarenko). This is as expected if the mechanisms we see evidence for on the surface of Chichewa are in fact more general aspects of syntax, and not mere idiosyncrasies of Chichewa.}

The restrictions we've seen in this subsection are expected if the OM involves a probe-goal Agree relation, which is constrained by locality considerations like the Minimal Link Condition \citep{Chomsky1995, Chomsky2000} or Relativized Minimality \citep{Rizzi1990} which force Agree operations to target the closest possible goal.

\section{Deciding between Unlocking theories}\label{sec:branan:4}\largerpage

 Recall that we are comparing two theories regarding Unlocking:

\begin{enumerate}
	\item \textit{either/or} (\citealt{RackowskiRichards2005, Halpert2016, Halpert2019, Branan2018}): Extraction from a phase requires Agree to Unlock that phase, unless the element to be extracted can move via the phase edge.

    \item \textit{both/and} (\citealt{VanUrkRichards2015} on Dinka): Extraction from a phase always requires the containing phase to be Unlocked through Agree, and for extraction to pass through the edge.
\end{enumerate}

In the previous section, we saw that extraction from DP in Chichewa is subject to a particular requirement: Extraction from DP requires that DP to be agreed with by the OM, unless the extracted element is N(P), which normally appears at the left edge of DP. While the \emph{both/and} theory of Unlocking (\#2) does not accommodate this absence of agreement in the latter case, the \emph{either/or} version of Unlocking theory (\#1) does.

Theory \#1 leads us to expect the Chichewa facts, given an analysis in which the NP-initial order of the Chichewa DP is derived by movement of NP to the edge of DP. Such movement is proposed for independent reasons in \citet{Cinque2005}, and for the Kordofanian language Moro in \citet{Jenks2010}, whose DP structure is analogous to that of Chichewa.

\ea%17
    \label{ex:branan:17}
    NP movement to spec-DP\footnote{A reviewer mentions that cardinal numerals might be expected to be pre-nominal, in which case, they might pose an issue for this analysis. \citet{Mchombo2004} in fact shows that cardinal numerals are post-nominal, as usual for adjectives in this language, hence there is no problem here. A reviewer also asks whether there are any adjective ordering constraints in Chichewa, mentioning that the analysis presented here predicts the relative ordering of adjectives to be essentially the same as in a language like English. At the time of writing, we do not have access to information about this, so we leave this question aside for now.}\\
    {[$_{DP}$} \textbf{NP}$_{k}$ D Dem Adj \_$_{k}$ ]
\z

Because NP occupies the edge of the Chichewa DP, agreement is not required for its extraction. NP can simply be freely extracted even when DP is not Unlocked, as \REF{ex:branan:10} showed. In contrast, we saw that the extraction of elements from the non-edge of DP (adjectives, demonstratives) requires the Unlocking effect of agreement. The next section discusses these derivations involved in detail.

\section{Predicting the facts}\label{sec:branan:5}

\subsection{Chichewa}\label{sec:branan:5.1}

We have seen that agreement with DP is necessary for extraction from the non-edge of the Chichewa DP. As mentioned, we argue that this is because Agree with the DP Unlocks it for further probing. That Unlocking allows an A$'$-probe to subsequently search past the edge of DP, as the tree in \figref{fig-fromex:branan:18} shows.

\begin{figure}
    \caption{\label{fig-fromex:branan:18}Deep A$'$-extraction fed by $\phi$-Agreement with DP}
\begin{forest}
    [,baseline,nice empty nodes [~,name=landing] [[$v$$_{[u\phi],[uA']}$,name=probe ] [\dots [\dots ] [$\textbf{DP}_{[\phi]}$,name=goal1 [\dots ] [\dots [$X_{[A']}$,name=goal2 ] [\dots] ] ] ] ]]
    % 1:phi-probe
    \draw[<->] (probe.285) to[out=270, in=240,looseness=1.5] node [midway,sloped,below] {\footnotesize \textcircled{1}:$\phi$-probe} ($(goal1.south west)+(0.225,0.2)$);% 2:[A']-probe
    \draw[<->,overlay] (probe.215) to[out=270,in=120] node [midway,below,sloped] {\footnotesize \textcircled{2}:$[A']$-probe} (goal2);% 3:move
    \draw[|->,overlay] (goal2) to[out=180,in=250,looseness=1.2]  node [midway,below,sloped] {\footnotesize \textcircled{3}:move} (landing.south);
\end{forest}
\end{figure}

If an extracting adjective or demonstrative were able to pass through spec-DP, the edge of this phase, Agreement with DP should not be necessary for extraction. However, this position is occupied by NP in Chichewa, precluding movement to this “escape hatch” position.\footnote{Implicit here is the claim that Chichewa permits only one spec-DP. Alternatively, we could claim that Chichewa D lacks an A$'$-probe, or that anti-locality constraints block the relevant elements from moving through the DP edge. Additionally, this ban on exiting a phase which something has already moved into the specifier of is analogous to \textit{wh}-island effects, where an initial \textit{wh}-movement into spec-CP blocks extraction of a second \textit{wh}-phrase.} As a result, the Chichewa DP must be Unlocked through Agree if anything but NP is to be extracted from it.\largerpage

In contrast, by being at the DP edge, agreement with DP is not a prerequisite for extraction of NP, as we saw in \REF{ex:branan:10}. This is what we expect in a theory like that of \citet{RackowskiRichards2005}, who argue that while locality considerations require probes to target the closest available goal first, a phase label and its highest specifier (the edge) are equidistant with respect to higher probes. This means that higher A$'$-movement probes can freely target either the phase itself, or material that may happen to be present in its edge. Thus in Chichewa, an A$'$-probe can extract NP from DP even when DP is not Unlocked. However, no harm is done if the containing DP happens to be agreed with and consequently superfluously Unlocked, as in \REF{ex:branan:9}, where both agreement and NP extraction occur.

\subsection{Resolving Dinka}\label{sec:branan:5.2}

\citet{VanUrkRichards2015} proposed that extraction from phases requires Unlocking in addition to moving via the phase edge, based on the interaction of extraction out of embedded CPs and EPP effects in Dinka. This result is in conflict with the theory we have argued for based on Chichewa. However, further examination reveals a resolution to this conflict: extraction from CP in Dinka always requires Unlocking to take place, because elements that undergo extraction never reach the edge of the CP phase in this language.

\citeauthor{VanUrkRichards2015} show that Dinka has two positions in the clause which, in the basic case, must be filled. These are claimed to be spec-CP and spec-vP. Spec-vP must be filled by some internal argument, as we see in (\ref{ex:branan:19}--\ref{ex:branan:21}).

\ea[]{%19
    \label{ex:branan:19}
    Dinka spec-vP filled by DO \hfill{(\citealt[ex. 33b]{VanUrkRichards2015})}\\
    \gll    B\`{o}l a-c\'{i} [$_{vP}$ alk\'{o}k\^{o}l$_{k}$  lɛ̤́k  D\`{ɛ}ŋ \_$_{k}$. \\
            Bol \textsc{3sg-pfv} \,  story tell Deng \\
    \glt    `Bol told Deng a story.'}
\ex[]{%20
    \label{ex:branan:20}
    Dinka spec-vP filled by IO \hfill{(\citealt[ex. 33a]{VanUrkRichards2015})}\\
    \gll    B\`{o}l a-c\'{i} [$_{vP}$ D\`{ɛ}ŋ$_{k}$ lɛ̤́k \_$_{k}$ alk\'{o}k\^{o}l ]. \\
            Bol \textsc{3sg-pfv} \,  Deng tell \,  story \\
    \glt    `Bol told Deng a story.'}
\ex[*]{%21
    \label{ex:branan:21}
    Dinka spec-vP unfilled \hfill{(\citealt[ex. 33c]{VanUrkRichards2015})}\\
    \gll    B\`{o}l a-c\'{i} [$_{vP}$ *\_ lɛ̤́k  D\`{ɛ}ŋ alk\'{o}k\^{o}l ].  \\
            Bol \textsc{3sg-pfv} \, \,  tell Deng story \\
    \glt    `Bol told Deng a story.'}
\z

The subject of the sentence -- the element which controls agreement morphology -- must occupy spec-CP. We see evidence of this in \REF{ex:branan:22}, where unlike the above examples, the subject remains in situ in vP, resulting in ungrammaticality:

\ea[*]{%22
    \label{ex:branan:22}
    Spec-CP unfilled in Dinka \hfill{(\citealt[ex. 33d]{VanUrkRichards2015})}\\
    \gll    *[$_{CP}$ \_  a-c\'{i}i  [$_{vP}$ B\`{o}l   lɛ̤́k  D\`{ɛ}ŋ alk\'{o}k\^{o}l ] ]. \\
            {} {} \textsc{3sg-pfv} {}  Bol tell Deng story \\
    \glt    `Bol told Deng a story.'}
\z

\citeauthor{VanUrkRichards2015} further observe that extraction out of an embedded clause appears to satisfy all such EPP positions passed by that movement, which consequently end up unfilled on the surface. We see this in \REF{ex:branan:23}, where the EPP positions identified in (\ref{ex:branan:19}--\ref{ex:branan:22}) are empty, having been crossed by \textit{wh}-movement:\footnote{A reviewer asks whether the subject DPs in \REF{ex:branan:23} might inhabit spec-vP, given that they do not move into spec-CP in this derivation. \citeauthor{VanUrkRichards2015} are not fully explicit about the position of the subject in these cases, but it appears implicit that subjects occupy a position above spec-vP, presumably spec-TP.}

\ea%23
    \label{ex:branan:23}\textit{wh}-movement satisfies EPP positions passed (\citealt[ex. 37]{VanUrkRichards2015})\\
    \gll    \textbf{Yeŋ\`{a}}$_{k}$ c\'{i}i  Yâ̤a̤r \_ lɛ̤́k  D\`{ɛ}ŋ, [$_{CP}$ y\'{e} \_  c\'{i}i  B\^{o}l \_ tu\`{ɔ}ɔc \_$_{k}$ wṳ́ṳt ]. \\
            Who \textsc{pfv.ns}   Yaar.\textsc{gen} \,  tell Deng \, C \, \textsc{pfv.ns}  Bol.\textsc{gen} \, send {} cattle.camp.\textsc{loc}  \\
    \glt    `Who did Yaar tell Deng that Bol sent to the cattle camp?'
\z

\citeauthor{VanUrkRichards2015} argue that the extracting \textit{wh}-phrase passes through and satisfies the EPP requirement of the spec-vP and spec-CP of the embedded clause. However, they argue that the embedded CP itself satisfies the EPP for the matrix v. They suggest that the embedded CP moves to spec-vP (subsequently extraposing to the right) due to being Agreed with by v in order to Unlock that CP for extraction. If v had to Agree with CP to Unlock it even though \textit{wh}-extraction passed through the CP edge as \REF{ex:branan:23} indicates, it suggests that both movement via the phase edge and Unlocking are required for extraction out of a phase. This finding is contrary to the theory of Unlocking we have argued for here.

To maintain \citeauthor{VanUrkRichards2015}' analysis of the Dinka derivation and keep Dinka consistent with the theory we argue for here, we might hypothesize that elements extracted from an embedded CP in Dinka do not actually pass through the true CP edge. If this is the case, CP will need to be Unlocked before a moving phrase can exit it. There is in fact evidence that there is more structure above the high EPP position in CP that moving phrases pass through: Namely, this position can be preceded by an overt complementizer. We saw this in \REF{ex:branan:23} above, where the gap in the edge of the embedded CP is preceded by the C \textit{y\'{e}}. We can independently see this post-C position filled by the subject in non-extraction contexts, as in (\ref{ex:branan:24}--\ref{ex:branan:25}):

\ea%24
    \label{ex:branan:24}
    EPP position in CP preceded by C \textit{ke}
    \hfill{(\citealt[ex. 4a]{VanUrkRichards2015})}\\
    \gll    A-c\'{a} t\'{a}ak, \textbf{ke}  \underline{Cà̤n} b\'{i}  w\'{i}t t\'{i}aam. \\
            3\textsc{sg-pfv.1sg} think C Can \textsc{fut} wrestling win.\textsc{tr} \\
    \glt    `I think that Can will win the wrestling.'
\ex%25
    \label{ex:branan:25}
    EPP position in CP preceded by C \textit{ye}
    \hfill{(\citealt[ex. 4b]{VanUrkRichards2015})}\\
    \gll    A-c\'{a} lu\'{e}el, \textbf{ye}  \underline{Cà̤n} b\'{i}  w\'{i}t t\'{i}aam. \\
            3\textsc{sg-pfv.1sg} say C Can \textsc{fut} wrestling win.\textsc{tr}  \\
    \glt    `I said that Can will win the wrestling.'
\z

This is the very position that can be unfilled in extraction contexts, which as \citeauthor{VanUrkRichards2015} argue, is because a phrase being extracted from CP passes through it. But this peripheral position in the Dinka CP is evidently not at the very edge of CP. Therefore Unlocking of CP is still required for extraction.

In sum, Dinka as analyzed by \citeauthor{VanUrkRichards2015} in fact behaves as the account of Unlocking that we have argued for predicts. In Dinka, Unlocking is required for all extraction from CP, because there is no escape hatch at the edge of CP for extraction to pass through. Rather, there is only only an EPP position that is not at the true edge.\footnote{A reviewer asks why \citeauthor{VanUrkRichards2015} take a position below the overt complementizer to be spec-CP. \citeauthor{VanUrkRichards2015} propose an extended left periphery in Dinka with at least two CP layers, in which only the lower CP counts as a phase. For them, the EPP position in the left periphery that we have discussed is therefore the specifier of the relevant phase. In this paper, we posit that it is in fact the higher CP layer that is a phase, and that there is no successive-cyclic movement through that upper CP, thus unlocking is required to permit extraction from this domain. We argue that movement through the specifier of the higher CP is banned because movement to this position from the EPP position in the lower CP would be too short, following the formulation of anti-locality in \citeauthor{Erlewine2016} (\citeyear{Erlewine2016}, a.o.).}

\section{Conclusion}\label{sec:branan:6}\largerpage

Chichewa's typically optional object agreement gives suggestive evidence for a particular view of the constraints on cross-phasal extraction -- in particular, one in which Agree allows extraction to bypass the edge of a phase, but in which movement to the edge of a phase is also sufficient to escape the phase. A potential contradiction of this theory presented by Dinka proves to be un-problematic: Agree with embedded CPs is required for extraction from them in Dinka because the peripheral position in CP available for phrasal movement is not at the true edge of the embedded clause. These results are consistent with a theory in which Unlocking is only required for “deep” extraction out of phases.

\section*{Abbreviations}\largerpage
{\multicolsep=0pt
\begin{multicols}{2}
\begin{tabbing}
\textsc{assoc} \hspace{1ex} \=  associative marker\kill
    \textsc{appl}  \>  applicative suffix\\
    \textsc{assoc}  \>  associative marker\\
    \textsc{do}  \>  direct object\\
    \textsc{fut}  \>  future\\
    \textsc{fv}  \>  final vowel\\
    \textsc{gen}  \>  genitive\\
    \textsc{io}  \>  indirect object\\
    \textsc{ns}  \>  non-subject\\
    \textsc{om}  \>  object marker\\
    \textsc{prox}  \>  proximate demonstrative\\
    \textsc{prs}  \>  present tense\\
    \textsc{pfv}  \>  perfective\\
    \textsc{pst}  \>  past\\
    \textsc{sm}  \>  subject marker
\end{tabbing}
\end{multicols}}

\section*{Acknowledgements}

Authors listed alphabetically. Thanks to helpful comments from Norvin Richards, Sabine Iatridou, David Pesetsky, and the audience of ACAL\,49 and WCCFL\,46. All errors are each other's.

% \citet{RackowskiRichards2005, Halpert2016}; and \citet{Branan2018} argue that there are two ways that extraction out of a phase might take place: either by movement of the extracted element to the phase edge, or through Agree with the phase itself 'unlocking' it for further operations and allowing the extracted element to bypass the phase edge. \citet{VanUrkRichards2015} argue, in contrast, that facts from Dinka suggest that extraction via the edge doesn't remove the need for unlocking -- Both must occur. In this work, we argue that correlations between agreement with an object and the possibility out of extracting from that object in Chichewa reported in \citet{Mchombo2004, Mchombo2006} adjudicate between these theories, providing new evidence that extraction is possible via the edge of a 'locked' phase. We show that this theory makes a correct prediction about Dinka as reported by \citeauthor{VanUrkRichards2015} as well.

\printbibliography[heading=subbibliography,notkeyword=this]
\end{document}
