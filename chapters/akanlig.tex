\documentclass[output=paper,colorlinks,citecolor=brown]{langscibook}

\title{Unary/binary-NEG structures of NPIs and reduplication in Bùlì}
\author{%
George Akanlig-Pare\affiliation{University of Ghana, Legon}\lastand
Ken Hiraiwa\affiliation{Meiji Gakuin University}
}

\abstract{Negative-sensitive items in Bùlì, a Mabia (Gur) language spoken in Ghana, exhibit a mixed behavior between NCIs and NPIs. Thus, Bùlì presents a counterexample to Vallduví's generalization. Adopting and extending the framework of \cite{CollinsPostal2014} and \cite{CollinsEtAl2017},  we will show that the apparently mysterious mixed behavior of negative-sensitive items in Bùlì can be explained by articulating an unary-NEG structure and  syntax of reduplication.}

\IfFileExists{../localcommands.tex}{
  \addbibresource{localbibliography.bib}
  \usepackage{langsci-optional,langsci-branding}
\usepackage{langsci-gb4e}
% \usepackage{langsci-textipa}
% \usepackage{langsci-glyphs}
\usepackage[linguistics]{forest}
\usepackage{tabto}
\usepackage{multirow}
\usepackage{bbding}

\usepackage[normalem]{ulem}

\usepackage{tikz-qtree}

\usepackage{enumitem}

\usepackage{multicol}
\usepackage{stmaryrd} %double brackets

\makeatletter
\let\pgfmathModX=\pgfmathMod@
\usepackage{pgfplots,pgfplotstable}%
\let\pgfmathMod@=\pgfmathModX
\makeatother
\usepgfplotslibrary{colorbrewer}
\usetikzlibrary{fit}

\usepackage{jambox}
\usepackage{tikz-qtree-compat}
\usetikzlibrary{arrows, arrows.meta}
\usepackage{longtable}
\usepackage{subcaption}

  \makeatletter
\let\thetitle\@title
\let\theauthor\@author
\makeatother

\newcommand{\togglepaper}[1][0]{
%   \bibliography{../localbibliography}
  \papernote{\scriptsize\normalfont
    \theauthor.
    \thetitle.
    To appear in:
    Change Volume Editor \& in localcommands.tex
    Change volume title in localcommands.tex
    Berlin: Language Science Press. [preliminary page numbering]
  }
  \pagenumbering{roman}
  \setcounter{chapter}{#1}
  \addtocounter{chapter}{-1}
}

\newcommand{\bari}{\ipabar{\i}{.5ex}{1.1}{}{}}
\newcommand{\notipa}[1]{\textnormal{#1}}

\newcommand{\agre}{\textsc{agr}-\ol{eene}}

\renewcommand{\emph}[1]{\textit{#1}} % resetting a setting from ling-macros-modified (I think?)

% forest settings to make compact but (mostly) straight-spined trees:
\forestset{
fairly nice empty nodes/.style={
            delay={where content={}{shape=coordinate,for parent={
                  for children={anchor=north}}}{}}
, angled/.style={content/.expanded={$<$\forestov{content}$>$}}
}}

\forestset{sn edges/.style={for tree={parent anchor=south, child anchor=north}}}

\newcommand{\bex}{\begin{exe}}
\newcommand{\fex}{\end{exe}}

\newcommand{\bxl}{\begin{exe}}
\newcommand{\fxl}{\end{exe}}

\newcommand{\ix}[1]{\textsubscript{#1}}
\newcommand{\alert}[1]{\textbf{#1}}
\newcommand{\ol}[1]{\textit{#1}}


			\usetikzlibrary{shapes,arrows,positioning,decorations,decorations.pathmorphing,intersections}
\forestset{
nice empty nodes/.style={
    for tree={calign=fixed edge angles},
    delay={where content={}{shape=coordinate,for siblings={anchor=north}}{}}
},
}

\definecolor{dark-gray}{gray}{0.3}

%\usepackage{dingbat,pifont}


%%%%%%%%%%%%For arrows%%%%%%%%%%%%%

\newcommand\Tikzmark[2]{%
  \tikz[remember picture]\node[inner sep=0pt,outer sep=0pt] (#1) {#2};%
}
\NewDocumentCommand\DrawArrow{O{}mmmmO{3}}{
\tikz[remember picture,overlay]
  \draw[->,line width=0.8pt,shorten >= 2pt,shorten <= 2pt,#1]
    (#2) -- ++(0,-#6\ht\strutbox) coordinate (aux) -- node[#4] {#5} (#3|-aux) -- (#3);
}
\NewDocumentCommand\DrawDotted{O{}mmmmO{3}}{
\tikz[remember picture,overlay]
  \draw[->,line width=0.9pt,dotted,shorten >= 2pt,shorten <= 2pt,#1]
    (#2) -- ++(0,-#6\ht\strutbox) coordinate (aux) -- node[#4] {#5} (#3|-aux) -- (#3);
}
\NewDocumentCommand\DrawLine{O{}mmmmO{3}}{
\tikz[remember picture,overlay]
  \draw[line width=0.8pt,shorten >= 2pt,shorten <= 2pt,#1]
    (#2) -- ++(0,-#6\ht\strutbox) coordinate (aux) -- node[#4] {#5} (#3|-aux) -- (#3);
}
%%%%%%%%%%%%%%%%%%%%%%%%%%%%%%%%%%%%%


\newcommand{\baru}{ʉ}
\newcommand{\baruH}{\'\baru}
\newcommand{\baruL}{\`\baru}

\newcommand{\ep}{ε}
\newcommand{\epH}{\'\ep}
\newcommand{\epL}{\`\ep}

\newcommand{\schwa}{ə}
\newcommand{\schwaH}{\'ə}
\newcommand{\schwaL}{\`ə}

\newcommand{\oo}{ɔ}
\newcommand{\ooH}{\'\oo}
\newcommand{\ooL}{\`\oo}

\newcommand{\ds}{\textsuperscript{
	\hspace*{-2pt}\begin{tikzpicture}
		\draw[-{>[scale=0.5]}] (0,0.4) --(0,0.25);
	\end{tikzpicture}}}

\newcommand{\ch}{t͡ʃ}
\newcommand{\dz}{d͡ʒ}

\newcommand{\tgl}{ʔ}

%shortcuts for the complementizers
\newcommand{\mbuL}{mb\baruL}
\newcommand{\mbuHL}{mb\baruH\baruL}
\newcommand{\mbuLH}{mb\baruL\baruH}
\newcommand{\la}{lá}
\newcommand{\nda}{ndà}

\newcommand{\tsc}[1]{\textsc{#1}}
\renewcommand{\textscb}{ʙ}
\newcommand{\ipa}[1]{#1} %disable IPA

\newcommand{\SM}[1]{#1}

\DeclareNewSectionCommand
  [
    counterwithin = chapter,
    afterskip = 2.3ex plus .2ex,
    beforeskip = -3.5ex plus -1ex minus -.2ex,
    indent = 0pt,
    font = \usekomafont{section},
    level = 1,
    tocindent = 1.5em,
    toclevel = 1,
    tocnumwidth = 2.3em,
    tocstyle = section,
    style = section
  ]
  {appendixsection}

\renewcommand*\theappendixsection{\Alph{appendixsection}}
\renewcommand*{\appendixsectionformat}
              {\appendixname~\theappendixsection\autodot\enskip}
\renewcommand*{\appendixsectionmarkformat}
              {\appendixname~\theappendixsection\autodot\enskip}

\renewcommand{\lsChapterFooterSize}{\footnotesize}
 
  %% hyphenation points for line breaks
%% Normally, automatic hyphenation in LaTeX is very good
%% If a word is mis-hyphenated, add it to this file
%%
%% add information to TeX file before \begin{document} with:
%% %% hyphenation points for line breaks
%% Normally, automatic hyphenation in LaTeX is very good
%% If a word is mis-hyphenated, add it to this file
%%
%% add information to TeX file before \begin{document} with:
%% %% hyphenation points for line breaks
%% Normally, automatic hyphenation in LaTeX is very good
%% If a word is mis-hyphenated, add it to this file
%%
%% add information to TeX file before \begin{document} with:
%% \include{localhyphenation}
\hyphenation{
affri-ca-te
affri-ca-tes 
Līk-pāk-páln
pro-sod-ic
phe-nom-e-non
Chi-che-wa
Lu-bu-ku-su
Ngbu-gu
Boyel-dieu
Mat-chi
pho-neme
Mil-em-be
Nyan-chera
Mc-Pher-son
Tsoo-tso
Sku-pin
dis-tin-guishes
con-ser-va-tion
Me-dum-ba
}

\hyphenation{
affri-ca-te
affri-ca-tes 
Līk-pāk-páln
pro-sod-ic
phe-nom-e-non
Chi-che-wa
Lu-bu-ku-su
Ngbu-gu
Boyel-dieu
Mat-chi
pho-neme
Mil-em-be
Nyan-chera
Mc-Pher-son
Tsoo-tso
Sku-pin
dis-tin-guishes
con-ser-va-tion
Me-dum-ba
}

\hyphenation{
affri-ca-te
affri-ca-tes 
Līk-pāk-páln
pro-sod-ic
phe-nom-e-non
Chi-che-wa
Lu-bu-ku-su
Ngbu-gu
Boyel-dieu
Mat-chi
pho-neme
Mil-em-be
Nyan-chera
Mc-Pher-son
Tsoo-tso
Sku-pin
dis-tin-guishes
con-ser-va-tion
Me-dum-ba
}
 
  \togglepaper[1]%%chapternumber
}{}

\begin{document}
\maketitle 

\section{Diagnosing for NCIs and NPIs}

\cite{Vallduvi1994} examined behaviors of negative-sensitive items such as \textit{n}-words and minimizers in Catalan and Spanish and proposed four diagnostic tests (1a)--(1d) to distinguish Negative Concord Items (NCIs) and Negative Polarity Items (NPIs). \cite{Giannakidou2000} further added clause-boundedness to the list. For example, \textit{any} in (1) in English is diagnosed as NPI, while \textit{n}-words in (2) in Spanish are considered to be NCI (see \citealt{Ladusaw1979, HaegemanZanuttini1996, Haegeman1995, Watanabe2004, Giannakidou2006, GiannakidouZeijlstra2017}, among others).

\begin{exe}
\ex English
\begin{xlist}
    \ex Did you see anyone?           \hfill{(Question)}
    \ex *Anyone did not come.               \hfill{(*Pre-negative subject position)}
    \ex *I didn't see almost anyone.           \hfill{(*Modification by \textit{almost})}
    \ex *Anyone \sout{(I didn't see)}. (as an answer to (1a))      \hfill{(*Fragment answer)}
    \ex I don't say I saw anyone.            \hfill{(Long-distance licensing)}
\end{xlist}
\end{exe}

\begin{exe}

\ex Spanish (\citealt{Vallduvi1994}, \citealt{Penka2011})
\begin{xlist}
   \ex [*]{\gll{\textquestiondown}Quieres nada? \\
2\textsc{sg}-want nothing \\
\glt`Do you want anything?'    \hfill{(*Question)}}
\ex [] {\gll Nada  funciona.   \\
nothing 3\textsc{sg}-work \\
\glt `Nothing works.'  \hfill{(Pre-negative subject position)}}
\ex [] {\gll {\textquestiondown}A  quien has    visto?  \\
 a  who  2\textsc{sg}-\textsc{perf} see    \\
\glt `Who'd you see?'\\
\gll A (casi) nandie.\\
 a almost no.one  \\
\glt `(Almost) no one.' \hfill{(Modification by \textit{almost}/Fragment answer)}}
\ex[*] {\gll No dije que había nada en el frogprífico. \\
Neg said.1\textsc{sg} \textsc{c} there.was.\textsc{ind} n-thing in the fridge \\
\glt `I didn't say that there was anything in the fridge.'
\phantom{xxxxxxxxxxxxxxxxxxxx}  \hfill{(*Long-distance licensing)}}
\end{xlist}
\end{exe}


\begin{table}
  \begin{tabular}{lccl}
\lspbottomrule
 & NCI & NPI    \\
   \midrule
 Non-negative & *  & $\surd$   \\
Pre-negative &  $\surd$ & *   \\
Modifiability & $\surd$  & *    \\
Fragment Answer & $\surd$ & *   \\
Long-distance & *  & $\surd$    \\
\lspbottomrule
  \end{tabular}
  \caption{NCI vs. NPI (\citealt{Vallduvi1994}, \citealt{Giannakidou2000})}
  \label{nci-npia}
\end{table}
In this paper, we examine negative-sensitive items in Bùlì, a Mabia (Gur) language spoken in Ghana, and show that they exhibit a mixed behavior between NCIs and NPIs, resisting the dichotomy in Table 1.


\begin{table}
  \begin{tabular}{lccl}
\lspbottomrule
 & NCI & NPI    \\
   \midrule
 Non-negative & *  & $\surd$   \\
Pre-negative &  $\surd$ & *   \\
Modifiability & $\surd$  & *    \\
Fragment Answer & $\surd$ & *   \\
Long-distance & *  & $\surd$    \\
\lspbottomrule
  \end{tabular}
  \caption{NCI vs. NPI (\citealt{Vallduvi1994}, \citealt{Giannakidou2000})}
  \label{nci-npib}
\end{table}
In this paper, we examine negative-sensitive items in Bùlì, a Mabia (Gur) language spoken in Ghana, and show that they exhibit a mixed behavior between NCIs and NPIs, resisting the dichotomy in Table 1.

\section{The NCI-NPI dichotomy revisited}

Contrary to the generalization in \tabref{nci-npib}, NCIs and NPIs do not always show systematic behaviors cross-linguistically.  For example, \cite{KorsahMurphy2017Ga} show that negative-sensitive items in G\~a behave like NCIs, but cannot be used as fragment answers (see also \citealt{CollinsEtAl2017} on Ewe and \cite{Hiraiwa2019} on Okinawan).

Negative-sensitive items in Bùlì, a Mabia (Gur) language spoken in Ghana, also exhibit a mixed behavior, but their distribution is different from those in G\~a and Ewe, as shown in  \tabref{nci-npi2}\todo{Fix!}. They can appear in pre-negative subject position, patterning with NCIs in this respect, but they pattern with NPIs in that they are allowed in non-negative contexts, cannot be used as fragment answers, and are licensed long-distance (note that Bùlì lacks a straightforward counterpart of \textit{almost} and hence  modifiability of negative-sensitive items cannot be tested).  Thus, Bùlì presents yet another counterexample to Vallduví's generalization.


\begin{table}
  \begin{tabularx}{\textwidth}{lccccccl}
\lspbottomrule
 & NCI & NPI  &  Spanish &  English  & G\~a &  Bùlì    \\
   \midrule
 Non-negative & *  & $\surd$ & * &  $\surd$ &   *  & $\surd$  \\
Pre-negative &  $\surd$ & * & $\surd$ &   * &  $\surd$ &   $\surd$ \\
Modifiability & $\surd$  & *  & $\surd$ & *    &   * &  N/A \\
Fragment Answer & $\surd$ & *  & $\surd$ & * &  * &  *  \\
Long-distance & *  & $\surd$  & * & $\surd$   &  * & $\surd$  \\
\lspbottomrule
  \end{tabularx}
  \caption{Cross-linguistic Variations} \label{nci-npi2}
\end{table}


Adopting and extending the framework of \cite{CollinsPostal2014} and \cite{CollinsEtAl2017},  we will show that the apparently mysterious mixed behavior of negative-sensitive items in Bùlì can be explained by articulating an unary-NEG structure and  syntax of reduplication.


\section{Negative-sensitive items in Bùlì}

\subsection{Morphosyntax of negative-sensitive items in Bùlì}

Before examining each diagnostic test in Bùlì, let us first detail morphosyntactic properties of negative-sensitive items in Bùlì (see \citealt{Akanlig-Pare2005, Akanlig-Pare2014}).

Negative-sensitive items in Bùlì come in two varieties, a form based on reduplication of a noun class (NC) pronoun\textit{ wa/di/ka/ku/bu} and a suffix \textit{-i} (e.g. \textit{wāā-(ī )wāā-ī} `anyone') (hereafter \textit{i}-suffixed reduplicated noun class pronouns: Table 3) and one based on reduplication of an indefinite ordinary noun (e.g. \textit{nur nur} `anyone/any person') (hereafter reduplicated indefinite nominals: Table 4)\footnote{We assume that a long vowel is purely phonological, but it could be an independent morpheme with its own function. We also leave open the source of the mid-tones for these items, as independent noun class pronouns have low tones.}. In nominal reduplication, just as in all compounds, coda of the preposed stem is deleted and hence partial reduplication results.

\begin{table}
  \begin{tabular}{cccccccc}
\lspbottomrule
NC  &  Wh  &  NSI  &  Universal/FC & Existential & Relativizer    \\
  \midrule
wa & wà-nà &   wāā(-ī) wāā-ī &    wāā-ī mééna & wāā-ī & -wāā-ī  \\
di &  d\`{\i}-nà & dīī(-ī)  dīī-ī &  dīī-ī mééna & dīī-ī & -dīī-ī    \\
ka &  kà-nà &  kāā(-ī)  kāā-ī  & kāā-ī mééna &  kāā-ī & -kāā-ī   \\
ku &  kù-nà &  k\=u\=u(-ī)  k\=u\=u-ī  & k\=u\=u-ī mééna & k\=u\=u-ī & -k\=u\=u-ī  \\
bu &  bù-nà &  b\=u\=u(-ī)  b\=u\=u-ī  &  b\=u\=u-ī mééna  & b\=u\=u-ī &  -b\=u\=u-ī \\
\lspbottomrule
  \end{tabular}
  \caption{Negative-sensitive Items in Bùlì}
\end{table}
\begin{table}
  \begin{tabular}{lcc}
\lspbottomrule
  & Indefinite   & NSI     \\
  \midrule
\textit{person} & n\'u\'r &  n\'u\'r n\'u\'r \\
\textit{child} &  bíík &  bíí bíík \\
\textit{hen} &  kpīāk &  kpīā kpīāk \\
\textit{house} &  yé\'r &  yé\'r yé\'r \\
\lspbottomrule
  \end{tabular}
  \caption{Reduplicated Indefinite Nominals in Bùlì}
\end{table}

Examples (3) illustrate the use of these expressions with negation.\footnote{The sentence-final particle \textit{ā} (also realized as \textit{yā} or \textit{wā} phonetically) appears post-object position in negative and question sentences. But it never appears when the object is followed by an adjunct.
\ea[]{Bùlì \\
\gll Mí àn  ny\v{a} wāā(-ī) wāā-ī (*ā) dèlā (*ā).   \\
1\textsc{sg} \textsc{neg}  see \textsc{nc}-i here \textsc{nc}-i   here \textsc{sfp} \\
\glt `I did not see anyone here.' }
\z

We assume it is more like a conjunct/disjunct marker and not a negative particle of a bipartite negation structure, unlike Ewe (see \citealt{CollinsEtAl2017}). See Section 4.  }


\ea Bùlì
\ea[]{
\gll Mí àn  ny\v{a} wāā(-ī) wāā-ī ā. \\
1\textsc{sg} \textsc{neg}  see \textsc{nc}-i \textsc{nc}-i  \textsc{sfp} \\
\glt `I did not see anyone.' }
\ex[] {
\gll Mí àn  nyà n\'ur n\'u\'r   ā.   \\
1\textsc{sg} \textsc{neg}  see person person  \textsc{sfp} \\
\glt `I did not see anyone.' }
\z
\z


Each of the morphemes (\textit{wa/di/ka/ku/bu}) in Table 3 is a noun class pronoun. This is illustrated in the following example with \textit{wa}.\footnote{We leave it open whether the pronoun \textit{wa} itself in (\ref{pronoun1}) is a noun class pronoun or a noun class marker with a null pronoun [$_{\textup{NCP}}$ \textit{wa} [$_{\textup{NP}}$ $\emptyset$ ]].}

%\ea
\ea[]{Bùlì  \label{pronoun1} \\
\gll àtìm nyà àm\`oàk, àlēgè, mí  àn ny\v{a} wà.   \\
Atim see.\textsc{perf} Amoak but 1\textsc{sg} \textsc{neg}  see \textsc{nc} \\
\glt `Atim saw Amoak, but I did not see him.' }
\z

Following \cite{HiraiwaEtAl2017}, we assume that actual (in)definiteness of noun class pronouns is marked by tone and that the suffix \textit{-i} is for marking indefiniteness SOME. This makes sense because \textit{-i} also appears in existential quantifiers and relativizers.

\ea[] {Bùlì \\
\gll Mí   nyá wāā-ī.   \\
1\textsc{sg}   see \textsc{nc}-i   \\
\glt `I saw someone.' }
\z

\ea[] {Bùlì \\
\gll àtìm nyà [nùr\'u-wāā-ī àlē dà máng\`o l\v{a}]. \\
Atim see person-\textsc{nc}-i \textsc{c} buy mango Dem \\
\glt `Atim saw the person who bought a mango.'    }
\z

Reduplication is obligatory for negative-sensitive items (\ref{waai}a)--(\ref{waai}b). Without reduplication, \textit{waa-i} functions as an existential quantifier that is not negative-sensitive (\ref{waai}c)--(\ref{waai}d).


\ea Bùlì \label{waai}
\ea [*] {
\gll Mí  nyá wāā(-ī) wāā-ī. \\
1\textsc{sg}  see \textsc{nc}-i \textsc{nc}-i \\
\glt `I saw someone.'}
\ex[] {
\gll Mí  àn ny\v{a} wāā(-ī) wāā-ī ā. \\
1\textsc{sg} \textsc{neg}  see \textsc{nc}-i \textsc{nc}-i \textsc{sfp} \\
\glt `I din't see anyone.'}
\ex[] {
\gll Mí  nyá wāā-ī. \\
1\textsc{sg}  see someone \\
\glt `I saw someone.'}
\ex[] {
\gll Mí àn ny\v{a} wāā-ī ā. \\
1\textsc{sg} \textsc{neg} see someone \textsc{sfp} \\
\glt `I didn't see someone/anyone.'}
\z
\z

With this in mind, let us examine syntactic distribution of negative-sensitive items in Bùlì.\footnote{There is another form \textit{bà bāā-ī}, which can only be used in plural noun class pronouns and expresses a partitive meaning `some of X'.


\ea Bùlì
\ea[] {
\gll Mí nyá bà/*wà bāā-ī. \\
1\textsc{sg} see \textsc{ng}.\textsc{pl} \textsc{nc}.\textsc{pl}-i \\
\glt `I saw some of them.'}
\ex[*] {
\gll Mí nyá wà/bà wāā-ī. \\
1\textsc{sg} see \textsc{nc} \textsc{nc}-i \\
\glt `I saw some of them.'}
\z
\z
}


Each of the morphemes (\textit{wa/di/ka/ku/bu}) in Table 3 is a noun class pronoun. This is illustrated in the following example with \textit{wa}.\footnote{We leave it open whether the pronoun \textit{wa} itself in (\ref{pronoun2}) is a noun class pronoun or a noun class marker with a null pronoun [$_{\textup{NCP}}$ \textit{wa} [$_{\textup{NP}}$ $\emptyset$ ]].}

%\ea
\ea[]{Bùlì  \label{pronoun2} \\
\gll àtìm nyà àm\`oàk, àlēgè, mí  àn ny\v{a} wà.   \\
Atim see.\textsc{perf} Amoak but 1\textsc{sg} \textsc{neg}  see \textsc{nc} \\
\glt `Atim saw Amoak, but I did not see him.' }
\z

Following \cite{HiraiwaEtAl2017}, we assume that actual (in)definiteness of noun class pronouns is marked by tone and that the suffix \textit{-i} is for marking indefiniteness SOME. This makes sense because \textit{-i} also appears in existential quantifiers and relativizers.

\ea[] {Bùlì \\
\gll Mí   nyá wāā-ī.   \\
1\textsc{sg}   see \textsc{nc}-i   \\
\glt `I saw someone.' }
\z

\ea[] {Bùlì \\
\gll àtìm nyà [nùr\'u-wāā-ī àlē dà máng\`o l\v{a}]. \\
Atim see person-\textsc{nc}-i \textsc{c} buy mango Dem \\
\glt `Atim saw the person who bought a mango.'    }
\z

Reduplication is obligatory for negative-sensitive items (\ref{waai}a)--(\ref{waai}b). Without reduplication, \textit{waa-i} functions as an existential quantifier that is not negative-sensitive (\ref{waai}c)--(\ref{waai}d).






\ea Bùlì \label{waai}
\ea [*] {
\gll Mí  nyá wāā(-ī) wāā-ī. \\
1\textsc{sg}  see \textsc{nc}-i \textsc{nc}-i \\
\glt `I saw someone.'}
\ex[] {
\gll Mí  àn ny\v{a} wāā(-ī) wāā-ī ā. \\
1\textsc{sg} \textsc{neg}  see \textsc{nc}-i \textsc{nc}-i \textsc{sfp} \\
\glt `I din't see anyone.'}
\ex[] {
\gll Mí  nyá wāā-ī. \\
1\textsc{sg}  see someone \\
\glt `I saw someone.'}
\ex[] {
\gll Mí àn ny\v{a} wāā-ī ā. \\
1\textsc{sg} \textsc{neg} see someone \textsc{sfp} \\
\glt `I didn't see someone/anyone.'}
\z
\z

With this in mind, let us examine syntactic distribution of negative-sensitive items in Bùlì.\footnote{There is another form \textit{bà bāā-ī}, which can only be used in plural noun class pronouns and expresses a partitive meaning `some of X'.


\ea Bùlì
\ea[] {
\gll Mí nyá bà/*wà bāā-ī. \\
1\textsc{sg} see \textsc{ng}.\textsc{pl} \textsc{nc}.\textsc{pl}-i \\
\glt `I saw some of them.'}
\ex[*] {
\gll Mí nyá wà/bà wāā-ī. \\
1\textsc{sg} see \textsc{nc} \textsc{nc}-i \\
\glt `I saw some of them.'}
\z
\z
}

\todo{Is the text being reduplicated here?}


\subsection{Are negative-sensitive items in Bùlì NCIs or NPIs?}

Both \textit{i}-suffixed reduplicated noun class pronouns  and reduplicated indefinite nominals in Bùlì are negative-sensitive items because they cannot appear in positive declarative clauses. On the other hand, the fact that they can be licensed in questions indicate that they are so-called weak NPIs (\citealt{VanDerWouden1997}).



\ea Bùlì
\ea[] {
\gll Mí *(àn)  ny\v{a} wāā(-ī) wāā-ī   ā.   \\
1\textsc{sg} \textsc{neg}  see \textsc{nc}-i \textsc{nc}-i  \textsc{sfp} \\
\glt `I did not see anyone.' \hfill{(*Positive declaratives)}}
\ex[] {
\gll Fí  nyá wāā(-ī) wāā-ī â:? \\
2\textsc{sg} see \textsc{nc}-i \textsc{nc}-i \textsc{sfp}  \\
\glt `Did you see anyone?' \hfill{(Question)}}
\z
\z




\ea Bùlì
\ea[] {
\gll Mí *(àn)  nyà n\'u\'r n\'u\'r   ā.   \\
1\textsc{sg} \textsc{neg}  see person person  \textsc{sfp} \\
\glt `I did not see anyone.' \hfill{(*Positive declaratives)}}
\ex[] {
\gll Fí  nyá n\'u\'r  n\'u\'r â:? \\
2\textsc{sg} see person person \textsc{sfp}  \\
\glt `Did you see anyone?' \hfill{(Question)}}
\z
\z

As expected, being weak NPIs, they cannot appear as fragment answer, in response to a question like \textit{Who did you see?} or \textit{Did you see anyone?}\footnote{There are some speakers who allow for such fragment answers in Bùlì (Abdul-Razak Sulemana p.c.). Those speakers may be interpreting the NPIs in (10)--(11) as NCIs (see Section 4). They also tend to require the sentence-final particle \textit{a} in fragment answers, while the first author and other speakers find fragment answers ungrammatical, irrespective of \textit{a}, as  (10--(11) show. We leave the issue for future research.}

\ea {Bùlì \\
\gll * Wāā(-ī) wāā-ī (ā).  \\
\phantom{*} \textsc{nc}-i \textsc{nc}-i \textsc{sfp} \\
\glt `No one.' \hfill{(*Fragment Answer)}}
\z

\ea {Bùlì \\
\gll * N\'u\'r  n\'u\'r (ā).  \\
\phantom{*} person  person \textsc{sfp} \\
\glt `No one.' \hfill{(*Fragment Answer)}}
\z

Finally, they can be licensed by a non-clausemate negation.

\ea[] {Bùlì \\
\gll Mí  kàn pōlī/wēēnī [ày\v{i}n mí  nyá wāā(-ī) wāā-ī  ā]. \\
1\textsc{sg}  \textsc{neg} think/say \textsc{c}  1\textsc{sg} saw \textsc{nc}-i \textsc{nc}-i \textsc{sfp} \\
\glt `I don't think/will not say that I saw anyone.' \hfill{(Long-distance licensing)}}

\z

\ea[] {Bùlì \\
\gll Mí  kàn pōlī/wēēnī [ày\v{i}n mí  nyá n\'u\'r  n\'u\'r  ā]. \\
1\textsc{sg}  \textsc{neg} think/say \textsc{c}  1\textsc{sg} saw person person \textsc{sfp} \\
\glt `I don't think/will not say that I saw any person.' \phantom{xxxxxxxxxxxxxxxxxxxxxxxx} \hfill{(Long-distance licensing)}}
\z


All of these diagnostics corroborate evidence that negative-sensitive items in Bùlì are weak NPIs. However, they also pattern with NCIs in that they can appear in subject position (followed by sentential negation).

\ea[] {Bùlì \\
\gll  Wāā(-ī) wāā-ī  àn  ch\v{e}ng Wīāgā.  \\
\textsc{nc}-i \textsc{nc}-i \textsc{neg} go   Wiaga \\
\glt `No one went to Wiaga.'   \hfill{(Pre-negative subject position)}}
\z

\ea[] {Bùlì\\
\gll  N\'u\'r  n\'u\'r  àn  ch\v{e}ng Wīāgā.  \\
person  person \textsc{neg} go   Wiaga \\
\glt `No one went to Wiaga.'   \hfill{(Pre-negative subject position)}}
\z


In the discussion below, we offer a theoretical analysis for the apparently mixed behaviors in the framework of \cite{CollinsPostal2014}. We focus on \textit{i}-suffixed reduplicated noun class pronouns, as basically the same analysis is available for reduplicated indefinite nominals, too.


\section{Syntax of pre-negative subject NPIs and (partial) reduplication}

\cite{CollinsPostal2014} and \cite{CollinsEtAl2017} propose that NPIs come in two varieties. One has a unary-NEG structure and the other has a binary-NEG structure.

\ea
\ea[] {[[NEG1 SOME] NP] \hfill{(strong NPIs/NCIs)}}
\ex[] {[[NEG1 [NEG2 SOME]] NP] \hfill{(weak NPIs)}}
\z
\z

They observe that \textit{ke}-NPIs in Ewe are NCIs/strong NPIs because they cannot be licensed in questions or at long-distance. However, unlike \textit{any}-NPIs in English, they can appear in subject position (with sentential negation).

\ea
\ea  {Ewe (\citealt{CollinsEtAl2017}) \label{ewe1} \\
\gll  Ame á{\textrtaild}éké mé-vá ny\v{e}-a\textit{f}é-me o. \\
person any \textsc{neg1}-come 1\textsc{sg}-house-inside \textsc{neg2} \\
\glt `Nobody came to my house.' \hfill{(Pre-negative subject position)}
\ex {English (\citealt{CollinsEtAl2017}) \label{english1} \\
* Anybody didn't come to my house. \hfill{(Pre-negative subject position)}}}
\z
\z

They argue that the reason why the English sentence in (\ref{english1}) is ungrammatical, is because it has the derivation in (\ref{deri1}). NEG1 raises to sentential negation position out of the NPI, while the remnant NPI raises to subject position.

\ea[] {[$_{\textup{DP2}}$ [<NEG1> SOME] body] did NEG1 [$_{\textup{VP}}$ come <DP2> to my house] \label{deri1}
}
\z

Here, the higher occurrence of  NPI DP2 c-commands the raised copy of NEG1. Thus, they propose condition (\ref{rrc}), which prohibits such a structural relation.

\ea[] {
The Remnant Raising Condition (\citealt{CollinsEtAl2017}) \label{rrc} \\
If M = [$_{\textup{DP}}$ [$_{\textup{D}}$<NEGx> SOME] NP], then no occurrence of M c-commands an occurrence of NEGx.}
\z

In contrast, they propose that \textit{ke} is a copy of the original NEG and argue that the Ewe example in (\ref{ewe1}) does not violate (\ref{rrc}) ``since a copy NEG, cNEG1 rather than <NEG1> fills the original position of NEG1 in DP2. In effect, the copy NEG allows the structure to avoid a violation of (\ref{rrc}), just as resumptive pronouns in certain English cases allow a structure to avoid a violation of island constraints.'' (\citealt{CollinsEtAl2017}). The derivation of (\ref{ewe1}) is given in (\ref{ewe2}).

\ea[] {Ewe (\citealt{CollinsEtAl2017}) \label{ewe2} \\
\gll  [Ame á{\textrtaild}é-ké] mé-vá <DP2> ny\v{e}-a\textit{f}é-me o. \\
person SOME-c\textsc{neg1} \textsc{neg1}-come {} 1\textsc{sg}-house-inside \textsc{neg2} \\
\glt `Nobody came to my house.' \hfill{(Pre-negative subject position)}}
\z



But their analysis is challenged by Bùlì. This is because there is no copy NEG, cNEG1 in the original position of NEG1 in DP2 that avoids a violation of (\ref{rrc}).


\ea[] {Bùlì \\
\gll  Wāā(-ī) wāā-ī  àn  ch\v{e}ng Wīāgā. (=(14)) \\
\textsc{nc}-i \textsc{nc}-i \textsc{neg} go   Wiaga \\
\glt `No one went to Wiaga.'   \hfill{(Pre-negative subject position)}}
\z

Nevertheless, we argue that \citeauthor{CollinsEtAl2017}'s analysis, once we articulate substructures that reduplication targets in Bùlì, provides a principle account for the apparently mysterious behavior. To see this, first recall that NPIs in Bùlì are necessarily reduplicated.

\ea[]{Bùlì \\
\gll Mí àn  ny\v{a} wāā(-ī) wāā-ī   ā.   \\
1\textsc{sg} \textsc{neg}  see \textsc{nc}-i \textsc{nc}-i  \textsc{sfp} \\
\glt `I did not see anyone.' }
\z

\ea[] {Bùlì \\
\gll Mí   nyá wāā-ī.   \\
1\textsc{sg}   see \textsc{nc}-i   \\
\glt `I saw someone.' }
\z

Without reduplication, \textit{NC-i} only has an existential quantifier reading. We assume the following structure for existential quantifiers such as `somebody'. Bùlì is head-initial and NCP raises to the specifier of DP (see \citealt{HiraiwaEtAl2017} for relevant discussion on DP-internal word order).

\ea[] { [$_{\textup{DP}}$ [$_{\textup{NCP}}$ [$_{\textup{NC}}$ wāā]]  [$_{\textup{D}}$ -ī]  t$_{\textup{NCP}}$] \hfill{(Existential quantifier)}}
\z


This, in turn,  suggests that reduplicated indefinite noun class pronouns in Bùlì contain an odd number of NEGs when they behave as an NCI,  co-occurring with sentential negation, while they contain an even number of NEGs when they behave as a weak NPI. Suppose, then, that an unary-NEG structure can be articulated as in \tabref{str1}.  Note that NEG structurally c-commands SOME. We assume that NEG is a null affix and undergoes affix-hopping to \textit{waa}.


\begin{table}
\begin{forest}
  [NegP [Neg [NEG] ] [DP  [NCP [NC [wāā] ]]  [D\textsubscript{1}  [D [-ī$_{\textup{SOME}}$] ] [t$_{\textup{NCP}}$] ] ]  ]
\end{forest}\todo[inline]{check subscript in tree}
  \caption{A binary-NEG structure in Bùlì}\label{str1}
\end{table}

Now, suppose reduplication targets  NCP before affix-hopping. Then we get the partially reduplicated structure in (\ref{str2}).

\ea[] {[[$_{\textup{NCP}}$ wāā] [$_{\textup{NegP}}$ NEG wāā -ī]]} \label{str2}
\z

If reduplication targets DP \textit{before} affix-hopping, then we get the partially reduplicated structure in (\ref{str3}).

\ea[] {[[$_{\textup{DP}}$  wāā -ī] [$_{\textup{NegP}}$ NEG wāā -ī]]} \label{str3}
\z

Both (\ref{str2}) and (\ref{str3}) have an unary-NEG structure. This makes them an NCI and explains why they can appear in pre-negative subject position in Bùlì ((14)--(15)) This is because NEG is deeply embedded within the reduplicated indefinite noun class pronouns, just as such subject NPIs are grammatical in English (the data cited from \citealt[362]{Boeckx2000} and \citealt[179]{Boskovic2002AMove}).
\ea
\ea[] {[Pictures of anyone/*someone] did not seem [ \textit{t} to be available].}
\ex[] {[Pictures of any linguist] seem to no psychologist [ \textit{t} to be pretty].}
\z
\z

On the other hand, suppose that reduplication \textit{after} affix-hopping of Neg to NC. Then we get  partially reduplicated structure (\ref{str4}) or (\ref{str5}), depending on whether NCP or DP is targeted by reduplication.

\ea[] {[[$_{\textup{NCP}}$ wāā-NEG ] [$_{\textup{NegP}}$ NEG wāā -\={i}]]} \label{str4}
\z

\ea[] {[[$_{\textup{DP}}$  wāā-NEG -\={i}] [$_{\textup{NegP}}$ NEG wāā -\={i}]]} \label{str5}
\z

As a result, both (\ref{str4}) and (\ref{str5}) have a binary-NEG structure. Assuming that the two NEGs cancel out each other (\citealt{CollinsEtAl2017}, \citealt{Watanabe2004}),  they behave as weak NPIs. Thus, interaction of reduplication and affix-hopping of NEG correctly explains why NPIs in Bùlì can be licensed by questions ((8)--(9)) and non-clause-mate negation ((12)--(13)), but are unable to occur as fragment answer ((10)--(11)).

One piece of supporting evidence that \textit{i}-suffixed reduplicated noun class pronouns have a binary-NEG structure comes from the fact that they can be replaced by an existential quantifier \textit{waa-i}.


\ea[] {Bùlì \\
\gll Fí  nyá  wāā-ī â:? \\
2\textsc{sg} see  \textsc{nc}-i \textsc{sfp}  \\
\glt `Did you see anyone?' \hfill{(Question)}}
\z

\ea[] {Bùlì\\
\gll Mí  kàn pōlī/wēēnī ày\v{i}n mí  nyá wāā(-ī) wāā-ī  ā. \\
1\textsc{sg}  \textsc{neg} think/say \textsc{c}  1\textsc{sg} saw \textsc{nc}-i \textsc{nc}-i \textsc{sfp} \\
\glt `I don't think/will not say that I saw anyone.' \phantom{xxxxxxxxxxxxxxxxxxxxxxxxxxxx}
\hfill{(Long-distance licensing)}}
\z

\ea[] {Bùlì\\
\gll  Wāā(-ī) wāā-ī  àn  ch\v{e}ng Wīāgā.  \\
\textsc{nc}-i \textsc{nc}-i \textsc{neg} go   Wiaga \\
\glt `No one went to Wiaga.'   \hfill{(Pre-negative subject position)}}
\z


\section{Parameter}

\cite{CollinsEtAl2017} propose a parameter that allows/disallows binary-NEG (Type 2) nominal NPIs. English allows, but Ewe doesn't allow, a binary-NEG structure. They also suggest an implicational relation that a language having binary-NEG nominal NPIs will also have unary-NEG nominal NPIs.

 Bùlì allows an unary-NEG structure underlyingly, and a binary-NEG structure is derived through reduplication by copying NEG. Thus, our analysis provides support for the implicational relation in that a binary-NEG structure is structurally based on a unary-NEG structure. We leave it for future research to investigate whether there are other types of NPIs in Bùlì.
Reduplication is not an unfamiliar strategy for forming negative-sensitive items cross-linguistically. It is observed in Vietnamese, Ainu, Yoruba, Malagasy, etc. (\citealt{Haspelmath1997}). According to \cite[179]{Haspelmath1997}, however, no languages that use partial reduplication were in his sample. Bùlì, in this respect, offers a case of NPIs with partial reduplication.

One remaining important question is what reduplication is employed for.  For one thing, reduplication is required in order to obtain two copies of NEG required for weak NPIs. From this viewpoint, it is important to point out that universal quantifiers/free choice items in Bùlì do not rely on reduplication (see Table 4). For another thing, \cite{Haspelmath1997} conjectures that reduplication is used to express distributive plurality. We concur with his conjecture. In Bùlì, the form \textit{waa-i} consists of a singular noun class pronoun and an indefiniteness marker. This is necessarily interpreted as singular. But NPIs are often considered to yield distributivity. It is not unnatural to think that reduplication is used (as a means of domain widening (\citealt{KadmonLandman1993}) to yield required distributivity.

To summarize, we have proposed the following ingredients that compose NPIs in Bùlì.

\ea Ingredients of NPIs in Bùlì
\ea Indefiniteness SOME: -\textit{i} (with noun class pronouns) or -$\emptyset$ (with ordinary nouns)
\ex Negation NEG: -$\emptyset$
\ex Neg-copying: reduplication
\ex Distributivity/Pluraity: reduplication
\z
\z

\section{Conclusion}

Reduplicated indefinite noun class pronouns and reduplicated nominals in Bùlì behave as NPIs in that they can appear in non-negative contexts and can be licensed at a long-distance. On the other hand, they behave as if they were NCIs in that they can appear in pre-negative subject position. We have argued that the mixed behavior can be explained if we articulate an unary/binary-NEG structure proposed in \cite{CollinsPostal2014} and \cite{CollinsEtAl2017} and understand which part of the structure reduplication targets.

There are remaining questions that must be answered in future studies. First, the existential quantifier \textit{wāā-ī} cannot come in subject position in the absence of sentential negation (cf. (7c)). Instead, it is necessary to nominalize \textit{waa-i} by prefixing \textit{a-}.

\ea Bùlì\\
\ea[]{
%\ea [*] {Bùlì\\
%\gll Wāā-ī chèng Wīāgā. \\
%\textsc{nc}-i go Wiaga \\
%\glt `Someone went to Wiaga.'}
%\ex[] {
\gll Wāā-ī *(àn) ch\v{e}ng Wīāgā. \\
\textsc{nc}-i \textsc{neg} go Wiaga \\
\glt `Someone didn't go to Wiaga/No one went to Wiaga.'}
\ex [] {
\gll ā-wāā-ī chèng Wīāgā. \\
\textsc{hm}-\textsc{nc}-i go Wiaga \\
\glt `Someone went to Wiaga.'}
\z
\z

It is also interesting and puzzling that the sentence-final particle \textit{a} is missing when the NPI appears in subject position. We don't have a good account for these at this moment.

Second, NPIs in Bùlì, while they can be licensed in questions, are not licensed in conditionals (see \citealt{CollinsEtAl2017} for a similar observation in Ewe).

\ea [] {Bùlì \\
\gll * Fì dàn nyà wāā(-ī) wāā-ī, wēēnī àtè ǹ wōm. \\
\phantom{*} 2\textsc{sg} \textsc{cond} see \textsc{nc}-i \textsc{nc}-i tell C  1\textsc{sg} hear  \\
\glt `If you see anyone, please let me know.' \hfill{(Conditional)}}
\z
%\ex[*] {
%\gll Fì dàn nyà nù\`r n\'u\'r, wēēnī àtè ǹ wōm. \\
%2\textsc{sg} \textsc{cond} see \textsc{nc}-i \textsc{nc}-i tell C  1\textsc{sg} hear  \\
%\glt `If you see anyone, please let me know.' \hfill{(Conditional)}}
%\z
In contrast, an existential quantifier \textit{waa-i} can also appear in conditionals. This discrepancy needs to be explained, too.


\ea[] {Bùlì \\
\gll Fì dàn nyà wāā-ī, wēēnī àtè ǹ wōm. \\
2\textsc{sg} \textsc{cond} see  \textsc{nc}-i tell C  1\textsc{sg} hear \\
\glt `If you see anyone, please let me know.' \hfill{(Conditional)}}
\z


\section*{Abbreviations}
\begin{tabularx}{.45\textwidth}{lQ}
\textsc{c} & complementizer\\
 \textsc{fc} & free choice\\
 \textsc{hm} & human\\
 \textsc{nc} & noun class pronoun\\
 NEG & negation
 \end{tabularx}
\begin{tabularx}{.45\textwidth}{lQ}
 \textsc{perf} & perfective\\
 \textsc{pl} & plural\\
 \textsc{sfp} & sentence-final particle\\
 \textsc{sg} & singular\\
 \\
 \end{tabularx}

\section*{Acknowledgements}
Acknowledgements to come here. We are grateful to Chris Collins, Kimiko Nakanishi,  Abdul-Razak Sulemana, and two anonymous reviewers for helpful discussions.
Ken Hiraiwa's research is funded by the JSPS Grant-in-Aid for Scientific Research (C) (No. 16K02645).

\printbibliography[heading=subbibliography,notkeyword=this]

\end{document}
