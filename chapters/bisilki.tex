\documentclass[output=paper,colorlinks,citecolor=brown]{langscibook}


\author{Abraham Kwesi Bisilki\affiliation{University of Education, Winneba, Ghana}}
\title{Focus marking and dialect divergence in Līkpākpáln (Konkomba)}
\abstract{In this paper, I discuss some salient aspects of focus marking in Līkpākpáln (Mabia (Gur), Niger-Congo) spoken mainly in the northern parts of Ghana. I compare focus marking in two dialects of Līkpākpáln, namely, Līnàjùúl and Līchábͻ́l. I treat the notion of focus from the angle of \citet{Dik1981}. Data draws from a multi-source corpora digitally recorded from stimuli-based elicitations and other natural discourse settings. Following the analysis of data, the study reveals that the use of focus particles constitutes the significant means of focus marking in Līkpākpáln as that focus strategy is shared by both Līnàjùúl and Līchábͻ́l. Again, a common feature for both Līnàjùúl and Līchábͻ́l is that there are syntactic restrictions for the distribution of various focus particles in the sentence. On the question of divergences, I note that sentence final vowel lengthening also assumes a focus function with respect to Līnàjùúl. Also, the focus markers in Līnàjùúl (ń, ńká and a sentence final focus particle of varied phonological shapes) differ in form from the focus markers, lé and lá in Līchábͻ́l. Finally, I suggest that the focus marking differences between Līnàjùúl and Līchábͻ́l possibly stems from the fact that Līnàjùúl appears to have innovated a complex focus system vis-à-vis focus marking in the Mabia languages of Ghana. However, more thorough investigation into focus marking in other dialects of Likpakpaln and Mabia is recommended. This will help establish whether the Līnàjùúl case is really an isolate system or not. }

\IfFileExists{../localcommands.tex}{%hack to check whether this is being compiled as part of a collection or standalone
  \addbibresource{localbibliography.bib}
  \usepackage{langsci-optional,langsci-branding}
\usepackage{langsci-gb4e}
% \usepackage{langsci-textipa}
% \usepackage{langsci-glyphs}
\usepackage[linguistics]{forest}
\usepackage{tabto}
\usepackage{multirow}
\usepackage{bbding}

\usepackage[normalem]{ulem}

\usepackage{tikz-qtree}

\usepackage{enumitem}

\usepackage{multicol}
\usepackage{stmaryrd} %double brackets

\makeatletter
\let\pgfmathModX=\pgfmathMod@
\usepackage{pgfplots,pgfplotstable}%
\let\pgfmathMod@=\pgfmathModX
\makeatother
\usepgfplotslibrary{colorbrewer}
\usetikzlibrary{fit}

\usepackage{jambox}
\usepackage{tikz-qtree-compat}
\usetikzlibrary{arrows, arrows.meta}
\usepackage{longtable}
\usepackage{subcaption}

  \makeatletter
\let\thetitle\@title
\let\theauthor\@author
\makeatother

\newcommand{\togglepaper}[1][0]{
%   \bibliography{../localbibliography}
  \papernote{\scriptsize\normalfont
    \theauthor.
    \thetitle.
    To appear in:
    Change Volume Editor \& in localcommands.tex
    Change volume title in localcommands.tex
    Berlin: Language Science Press. [preliminary page numbering]
  }
  \pagenumbering{roman}
  \setcounter{chapter}{#1}
  \addtocounter{chapter}{-1}
}

\newcommand{\bari}{\ipabar{\i}{.5ex}{1.1}{}{}}
\newcommand{\notipa}[1]{\textnormal{#1}}

\newcommand{\agre}{\textsc{agr}-\ol{eene}}

\renewcommand{\emph}[1]{\textit{#1}} % resetting a setting from ling-macros-modified (I think?)

% forest settings to make compact but (mostly) straight-spined trees:
\forestset{
fairly nice empty nodes/.style={
            delay={where content={}{shape=coordinate,for parent={
                  for children={anchor=north}}}{}}
, angled/.style={content/.expanded={$<$\forestov{content}$>$}}
}}

\forestset{sn edges/.style={for tree={parent anchor=south, child anchor=north}}}

\newcommand{\bex}{\begin{exe}}
\newcommand{\fex}{\end{exe}}

\newcommand{\bxl}{\begin{exe}}
\newcommand{\fxl}{\end{exe}}

\newcommand{\ix}[1]{\textsubscript{#1}}
\newcommand{\alert}[1]{\textbf{#1}}
\newcommand{\ol}[1]{\textit{#1}}


			\usetikzlibrary{shapes,arrows,positioning,decorations,decorations.pathmorphing,intersections}
\forestset{
nice empty nodes/.style={
    for tree={calign=fixed edge angles},
    delay={where content={}{shape=coordinate,for siblings={anchor=north}}{}}
},
}

\definecolor{dark-gray}{gray}{0.3}

%\usepackage{dingbat,pifont}


%%%%%%%%%%%%For arrows%%%%%%%%%%%%%

\newcommand\Tikzmark[2]{%
  \tikz[remember picture]\node[inner sep=0pt,outer sep=0pt] (#1) {#2};%
}
\NewDocumentCommand\DrawArrow{O{}mmmmO{3}}{
\tikz[remember picture,overlay]
  \draw[->,line width=0.8pt,shorten >= 2pt,shorten <= 2pt,#1]
    (#2) -- ++(0,-#6\ht\strutbox) coordinate (aux) -- node[#4] {#5} (#3|-aux) -- (#3);
}
\NewDocumentCommand\DrawDotted{O{}mmmmO{3}}{
\tikz[remember picture,overlay]
  \draw[->,line width=0.9pt,dotted,shorten >= 2pt,shorten <= 2pt,#1]
    (#2) -- ++(0,-#6\ht\strutbox) coordinate (aux) -- node[#4] {#5} (#3|-aux) -- (#3);
}
\NewDocumentCommand\DrawLine{O{}mmmmO{3}}{
\tikz[remember picture,overlay]
  \draw[line width=0.8pt,shorten >= 2pt,shorten <= 2pt,#1]
    (#2) -- ++(0,-#6\ht\strutbox) coordinate (aux) -- node[#4] {#5} (#3|-aux) -- (#3);
}
%%%%%%%%%%%%%%%%%%%%%%%%%%%%%%%%%%%%%


\newcommand{\baru}{ʉ}
\newcommand{\baruH}{\'\baru}
\newcommand{\baruL}{\`\baru}

\newcommand{\ep}{ε}
\newcommand{\epH}{\'\ep}
\newcommand{\epL}{\`\ep}

\newcommand{\schwa}{ə}
\newcommand{\schwaH}{\'ə}
\newcommand{\schwaL}{\`ə}

\newcommand{\oo}{ɔ}
\newcommand{\ooH}{\'\oo}
\newcommand{\ooL}{\`\oo}

\newcommand{\ds}{\textsuperscript{
	\hspace*{-2pt}\begin{tikzpicture}
		\draw[-{>[scale=0.5]}] (0,0.4) --(0,0.25);
	\end{tikzpicture}}}

\newcommand{\ch}{t͡ʃ}
\newcommand{\dz}{d͡ʒ}

\newcommand{\tgl}{ʔ}

%shortcuts for the complementizers
\newcommand{\mbuL}{mb\baruL}
\newcommand{\mbuHL}{mb\baruH\baruL}
\newcommand{\mbuLH}{mb\baruL\baruH}
\newcommand{\la}{lá}
\newcommand{\nda}{ndà}

\newcommand{\tsc}[1]{\textsc{#1}}
\renewcommand{\textscb}{ʙ}
\newcommand{\ipa}[1]{#1} %disable IPA

\newcommand{\SM}[1]{#1}

\DeclareNewSectionCommand
  [
    counterwithin = chapter,
    afterskip = 2.3ex plus .2ex,
    beforeskip = -3.5ex plus -1ex minus -.2ex,
    indent = 0pt,
    font = \usekomafont{section},
    level = 1,
    tocindent = 1.5em,
    toclevel = 1,
    tocnumwidth = 2.3em,
    tocstyle = section,
    style = section
  ]
  {appendixsection}

\renewcommand*\theappendixsection{\Alph{appendixsection}}
\renewcommand*{\appendixsectionformat}
              {\appendixname~\theappendixsection\autodot\enskip}
\renewcommand*{\appendixsectionmarkformat}
              {\appendixname~\theappendixsection\autodot\enskip}

\renewcommand{\lsChapterFooterSize}{\footnotesize}

\togglepaper[23]
}{}

\begin{document}
\maketitle

\section{Introduction}\label{sec:bisilki:1}

The phenomenon of information structure (IS) and packaging is a sub-domain of linguistics that has received a generous scale of attention from linguists globally. This is as exemplified in works such as \citet{Lambrecht1994, Krifka2007} \citet{Krifka2007}\todo{\citet{Krifka2008}, perhaps?}; \citet{SchwabeWinkler2007, Ameka2010, ZimmermannOnea2011}; and \citet{VanPutten2014}. Often central in studies of information structure and packaging is the subject of focus. Paradoxically, the more that linguists try to put questions to rest regarding focus phenomena in languages, the more insatiable this topic area becomes. This observation is accentuated by the ever-increasing volumes of focus-related analyses and counter-analyses that continue to delve into the topic. For instance, the focus status of the post verbal la in Dagbani (a Gur, Niger-Congo language spoken in the Northern Region of Ghana) has been the source of a series of somewhat varying analyses as reflected in \citep{Olawsky1999}\todo{Second Olawsky 1999 reference deleted. They appear to be the same work}, \citet{Hudu2012} and \citet{Issah2013Focus}. \citet[94]{VanPutten2016}, similarly, notes the difficulty in attempting to find exhaustive explanations to questions bothering on focus phenomena. She makes this observation in relation to the elusive task that linguists face in trying to determine, for instance, when and why focus marking is resorted to in non-obligatory focus languages.

What the foregoing situation clearly suggests is that the need for systematic investigations into the attributes of focus will continually remain relevant for linguists, even with regard to the so-called well-researched languages. Līkpākpáln is a Gur (or Mabia) language, whose speakers are mostly found in the Northern Sectors of Ghana. Speakers of Līkpākpáln natively term themselves as Bikpakpaam, instead of the exonym, Konkomba, which has often been used as a shared tag for both the people and their language. Some specific areas of their location include Saboba District in the Northern Region, Nkwanta South and Nkwanta North Districts in (Northern) Volta (see map in appendix I for the distribution of Līkpākpáln speakers in Ghana). In other contexts, these areas of the Bikpakpaam location are alternatively termed as the North-Eastern parts of Ghana \citep[182]{Schwarz2009}. \citet{SimonsFennig2017, SimonsFennig2018}, in Ethnologue: Languages of the world, estimate the Līkpākpáln speaker population in Ghana at 831000, besides other speakers reported of in the Republic of Togo.

Līkpākpáln has a significant native speaker population, yet one of the very little-researched languages of Ghana. In the view of \citep{Schwarz2009}, the need for basic grammatical descriptions of Līkpākpáln is still very high. This paper contributes to filling the basic knowledge gap on Līkpākpáln by investigating some aspects of focus marking in the language. The study introduces into the literature new data on focus constructions in Mabia. It does so from a comparative perspective by drawing data from two clan dialects of Līkpākpáln, namely, Līnàjùúl and Līchábͻ́l, respectively. The following questions form the crux of this article:

\begin{itemize}
    \item[(i)]      What are the linguistic strategies for coding focus in Likpakpaln?
    \item[(ii)]     What are the functions of focus in Likpakpaln?
    \item[(iii)]    What are the focus marking similarities and differences between Līnàjùúl and Līchábͻ́l?
    \item[(iv)]     To what extent does the focus marking system of Līkpākpáln characterise the focus typology in the Mabia family?
\end{itemize}

\section{Some basic grammatical features of Līkpākpáln}\label{sec:bisilki:2}

As already indicated, Līkpākpáln is a Mabia language. It is further defined as belonging to the Gurma sub-cluster of the Oti Volta family \citep{Naden1988, SteeleWeed1966}. This section briefly explicates some linguistic features of Līkpākpáln. This will provide a prerequisite for understanding discussions on focus marking in subsequent sections of the article.

Līkpākpáln is a word order language\todo{Is something missing in this sentence?}, with the SVO typology as generally known of Mabia and Kwa languages \citep{Schwarz2009}. A simple sentence in Līkpākpáln can have the pattern SVO, SV or SVA, depending on whether the verb is transitive or intransitive. See the sentences in \REF{ex:bisilki:1} below:

\todo[inline]{The combining character for acute accent is not being displayed properly with some glyphs (e.g. velar nasal /ŋ/ or open-mid back rounded /ɔ/).}

\ea%1
    \label{ex:bisilki:1}
    \ea{\label{ex:bisilki:1a}
    \gll    Ú-ŋóó		yá	júú		bī-sáá.\\
            \textsc{cl-}goat	\textsc{def}	bite\textsc{.pfv}	\textsc{cl-}food\\}\jambox*{[SVO]}
    \glt    ‘The goat bit the food.’
    \ex{\label{ex:bisilki:1b}
    \gll    Ú-bú		mↄ՛r.\\
            \textsc{cl-}child	cry\textsc{.ipfv}\\}\jambox*{[SV]}
	\glt    ‘The child is crying.’
    \ex{\label{ex:bisilki:1c}
	\gll    Nákújà	wáá		ḿbámↄ՛m.\\
            Nakuja	dance\textsc{.pfv}	well\\}\jambox*{[SVA]}
    \glt    ‘Nakuja danced well.’
    \z
\z

For purposes of focus, a non-subject constituent can be placed sentence initially, but the subject and the verb will remain in the fixed order of SV in the base clause as in \REF{ex:bisilki:2b} below:

\ea%2
    \label{ex:bisilki:2}
    \ea{\label{ex:bisilki:2a}
    \gll    Mánótī		kↄ՛r			ú-kúló.\\
            Name		slaughter\textsc{.prf}		\textsc{cl-}fowl\\}\jambox*{[canonical clause]}
    \glt    ‘Manoti has slaughtered a fowl.’
    \ex\label{ex:bisilki:2b}
    \gll    Ú-kúló	ńká	Mánótī	kↄ՛r.\\
            \textsc{cl-}fowl	\textsc{foc}	Manoti	slaughter\textsc{.prf}\\
    \glt    ‘It is a fowl that Manoti slaughtered.’
    \z
\z

In a ditransitive construction, the indirect object precedes the direct object as can be seen in (\ref{ex:bisilki:3a}-b) below:

\todo[inline]{Please identify \textit{ti} in \REF{ex:bisilki:3a}}

\ea%3
    \label{ex:bisilki:3}
    \ea\label{ex:bisilki:3a}
    \gll    Nákújà	tí		Mánótī	ú-kúló.\\
            give\textsc{.prf} {}	Manoti	\textsc{cl-}fowl\\
    \glt    ‘Nakuja has given Manoti a fowl.’
    \ex\label{ex:bisilki:3b}
    \gll    Ú-nìmpū	wár		mē		bī-sáá.\\
            \textsc{cl-}woman	cut\textsc{.prf}	\textsc{1sg.obj}	\textsc{cl-}food\\
    \glt    ‘A woman has served me food.’
    \z
\z

Līkpākpáln has an embryonic noun class system based predominantly on class affixes, which also bear number semantic. Although prefixes dominate the class markers, some nouns have obligatory prefix-suffix pairs. Fewer nouns take only suffixes, which become the basis for their class assignment. Prefixes have corresponding class pronouns.

The morphology of Līkpākpáln nouns is basically agglutinating. Verbs, on the other hand, have a poor morphology as there are only a handful of aspectual markers on the verb. Tense is a function of preverbal particles. Three tones, high ( ́  ), mid ( ̄  ) and low ( ՝ ) are identified in Līkpākpáln \citep[16]{SteeleWeed1966}.

The language has an initial orthography (which is reasonably phonemic) that was fashioned based on the Līchábͻ́l dialect. Any sequence of two vowels (whether representing a long vowel or not) in a word is treated as two syllable nuclei that may have the same or varying pitch levels \citep{BisilkiAkpanglo-Nartey2017}. Tone is generally not marked in the orthography of Līkpākpáln. I, nonetheless, mark tone in this study as tone has both lexical and grammatical functions in the language and is quite relevant in an analysis such as this.

\section{The current study in perspective}\label{sec:bisilki:3}

The notion of focus has been defined in many, related ways. \citet[92]{VanPutten2016} maintains that focus is the part of a sentence that carries the common-ground update. The information that is a shared knowledge between both the speaker and the listener in an interlocution constitutes the common-ground. As speakers communicate, they try to increase their common-ground or shared knowledge by introducing and linking new information to this common ground. The new information that is introduced becomes an update to the common-ground and, for that matter, the focal point. My reservation(s) with \citeauthor{VanPutten2016}’s definition, though, is whether or not focus is always solely underpinned by what is necessarily new information. Indeed, \citet[59]{Dik1981} argue that for the purpose of stressing the importance of a certain information or reactivating it in the addressee’s memory, a speaker may place focus on such information s/he (the speaker) knows is not new to the addressee. Similarly, \citet[3]{SkopeteasEtAl2006} hold that a given element may be focused. A given element, in the view of \citet[2]{SkopeteasEtAl2006} refers to information that the speaker believes the addressee already knows.

Consequently, in the present analysis, I treat the notion of focus from the point of view of \citet{Dik1981} and as subsequently in \citet{Dik1997}. Focus represents what is relatively the most important or salient piece of information in a given discourse-context \citep[42]{Dik1981}. Relatedly, a constituent of focus function is assumed to present information bearing upon the pragmatic information difference between the speaker and the addressee as perceived by the speaker. The foregoing conceptualisation of focus replays in \citet[326]{Dik1997} when he sees the focal information in a linguistic expression to be the most essential or salient in a given communicative context and considered by the speaker to be the most relevant for the listener to integrate into his/her pragmatic information. From this point of view, one can say further that a focus construction is a type of sentence or utterance in which a particular constituent (i.e. the focal constituent) is placed in relative prominence or saliency by setting it off from the rest of the sentence or utterance in one way or another (\citealt{Boadi1974, DrubigSchaffer2001}; \citealt[185]{MarfoBodomo2005}). In terms of the expressive devices or strategies that languages deploy in marking focus, \citet[43]{Dik1981} stipulates four ways:

\begin{itemize}
    \item[i.] Intonational prominence- extra stress or higher tone
    \item[ii.] Constituent order
    \item [iii.] Special focus markers
    \item [iv.] Special focus constructions
\end{itemize}

Focus operation in different languages may use some or all of these devices in different combinations. Along a functional line, \citet[60]{Dik1981} typologises focus broadly as either -contrast or +contrast. -Contrast focus is also termed as completive or informative focus whereas +contrast focus, which is also known as contrastive focus. \citet{Akrofi-Ansah2014, Schwarz2009, SkopeteasEtAl2006} further delineate into finer-grained focus types such as selective, expanding, restricting, replacive and parallel. Focus is completive (-contrast) when it serves merely to emphasize (or make prominent) a particular constituent, but contrastive when it contrasts the information of a constituent with that of another. The Līkpākpáln (Līnàjùúl) sentences in \REF{ex:bisilki:4b} and \REF{ex:bisilki:5b} illustrate completive and contrastive focus respectively.

\ea%4
    \label{ex:bisilki:4}
    \ea\label{ex:bisilki:4a}
    \gll    Ú-nìnjà	wé	bì-trí		bá?\\
            \textsc{cl-}man	\textsc{dem}	\textsc{prog-}push	\textsc{q}\\
    \glt    ‘What is this man pushing?’
    \ex\label{ex:bisilki:4b}
    \gll    Ú		bì-trí		lóól	lέ.\\
            \textsc{3sg.sbj}	\textsc{prog-}push	car	\textsc{foc}\\
    \glt    ‘He is pushing a car.’
    \z
\z

\ea%5
    \label{ex:bisilki:5}
    \ea\label{ex:bisilki:5a}
    \gll    Ú-nìnjà	bì-trí		ú-táán.\\
            \textsc{cl-}man	\textsc{prog-}push	\textsc{cl-}horse\\
    \glt    ‘A man is pushing a horse.’
    \ex\label{ex:bisilki:5b}
    \gll    Dábí,	lóól	ńká	ú		bì-trí.\\
            No,	car	\textsc{foc}	\textsc{3sg.sbj}	\textsc{prog-}push\\
    \glt    ‘No, he is pushing a car.’
    \z
\z

The discourse function of focus in \REF{ex:bisilki:4b} is simply to lay emphasis or prominence on a car as the constituent bearing the relatively most salient information in the predication. On the other hand, the focus in \REF{ex:bisilki:5b} serves to show the contrast that it is a car (and not a horse) that the man is pushing.

Focus can also be broad or narrow depending on whether it is assigned to the entire sentence (or its truth value) or a particular constituent or complement \citep[44]{Dik1981}. See the Līkpākpáln (Līchábͻ́l) sentences in \REF{ex:bisilki:6} and \REF{ex:bisilki:7}:

\ea%6
    \label{ex:bisilki:6}
    \ea\label{ex:bisilki:6a}
    \gll    Bá	ŋá-ní?\\
            \textsc{q}	happen\textsc{-prog}\\
    \glt    ‘What is happening?’
    \ex\label{ex:bisilki:6b}
    \gll    Bū-sūb	lé	lír.\\
            \textsc{cl-}tree		\textsc{foc}	fall\textsc{.prf}\\
    \glt    ‘A tree has fallen.’
    \z
\z

\ea%7
    \label{ex:bisilki:7}
    \ea\label{ex:bisilki:7a}
    \gll    Bá	ú		trí?\\
            \textsc{q}	\textsc{3sg.sbj}	push\textsc{.ipfv}\\
    \glt    ‘What is s/he pushing?’
    \ex\label{ex:bisilki:7b}
    \gll    Ní	yí	lↄ՛r	lá.\\
            \textsc{3sg}	is	car	\textsc{foc}\\
    \glt    ‘It is a car.’
    \z
\z

\REF{ex:bisilki:6b} presents an instance of broad focus where the entire sentence serves to fill the information gap in the knowledge of the listener (addressee). \REF{ex:bisilki:7b}, however, exemplifies narrow focus as a car is the only focus bearing constituent in the sentence. More on broad and narrow focus can be found in \citet[96--97]{Hyman2010}.

Additionally, this study also draws on \citet[93]{VanPutten2016} in the light of the contradistinction between a focused or in focus constituent, on one hand and, on the other hand, a focus-marked constituent. The former is applicable to a situation where an element that constitutes the most crucial point of information is only so understood pragmatically without the use of overt linguistic devices to doing so. The latter case has to do with the situation where a focal element is explicitly marked for focus by any of a possible range of linguistic devices that may have a focus configuration function in a language. Another fact worth noting is that focus-marked elements are invariably in focus whilst an element can be in focus without necessarily being focus-marked. In the present discussion, my concentration is mainly on cases of focus-marked constituents as the presentation and analysis of data will show.

As already indicated (in \sectref{sec:bisilki:1}), the current analysis investigates some aspects of focus marking in Līkpākpáln from a dialectal perspective. This is in the sense that the study does not only describe focus marking in Līkpākpáln, but it also compares two actively spoken dialects of the language (Līnàjùúl and Līchábͻ́l), with respect to the phenomenon in question. There has been a preliminary attempt at investigating focus in Līkpākpáln by \citet{Schwarz2009}. Nevertheless, \citeauthor{Schwarz2009}’s study was limited to only Līchábͻ́l. Focus marking in Līchábͻ́l is here being re-examined and compared with focus marking as pertains in Līnàjùúl (which has no such previous study).

Beyond the agenda of providing linguistic description of Līkpākpáln, the immediate motivation for this study is also anchored on two issues. The first is to help settle some questions regarding the curiosity that is engendered by a constant, but cursory refrain in the few works on Līkpākpáln that the language is highly split into numerous clan dialects (\citealt[182]{Schwarz2009}; \citealt[107]{Hasselbring2006}). Although scholars have often been quick to point out that Līkpākpáln sub-divides into numerous dialectal forms along clan units, the state of linguistic convergence or divergence between the supposed variants of Līkpākpáln remains unexplored or, at best, little-researched. Secondly, there are currently proposals being made by the Līkpākpáln speaker community in Ghana to re-design an orthography that will have a more unified outlook for various speakers of the language. I am privy to this initiative as a native speaker of Līkpākpáln and member of the speaker community. Such a practical need further calls for studies that potentially reveal how similar or different the dialects of Līkpākpáln spoken by various Bikpakpaam clan groupings are. I look at focus marking in Līnàjùúl (see \sectref{sec:bisilki:5}) first, then focus marking in Līchábͻ́l (see \sectref{sec:bisilki:7}) before proceeding to compare the focus systems of the two in \sectref{sec:bisilki:8}.

\section{The data method}\label{sec:bisilki:4}

This study is based mainly on primary data sets collected from the native speakers of Līnàjùúl in the Nkwanta North District of (Northern) Volta and Līchábͻ́l speakers of Saboba in the Northern Region of Ghana. I used both observation (including participant and non-participant types) and direct elicitation techniques for data collection. The direct elicitation involved four consultants (two Līnàjùúl speakers and two Līchábͻ́l speakers; one male and one female for each dialect) purposively selected. The observation data covered varied communicative domains such as during arbitration proceeding at the chief’s court, religious ceremonies and family interactions.

With the direct elicitations, the prompts were a-10-minute video-clip, on one hand, and picture stimuli (some original; others adapted from \citealt{SkopeteasEtAl2006}), on the other hand. \citet[61]{Chelliah2013} attests to the advantage in using non-linguistic stimuli tasks such as video-clips and photographs. As \citet{Chelliah2013} puts it, “Non-linguistic stimuli have several advantages: speakers do not require special training to understand the tasks and responses are clearly linked to stimuli and are, therefore, less ambiguous.” The video and picture stimuli were designed based on local content in the Līkpākpáln speaker environment. For instance, I took pictures of different animals at different times, pictures of people engaged in different activities (e.g., during block laying at a construction site, cooking, etc.). The essence of using familiar stimuli was to avoid the situation where culturally foreign stimuli could lead to consultant confusion. A further benefit from the use of stimuli was that by taking the responses of Līnàjùúl and Līchábͻ́l speakers to the same prompts allowed for making easy contrasts between the two dialects (see \citealt[56]{Majid2012}).

Using the information structure questionnaire [QUIS] \citep{SkopeteasEtAl2006} as a guide, I also sometimes posed content questions to which consultants responded based on the stimuli. The use of question-answer pairs as a standard heuristic for determining focus constituents is also well established in the literature (see for example \citealt{Dik1978, Krifka2007, Watters1979}\todo{Please check whether this is the right Krifka work that you'd like cite or not}) Utterances were recorded with a digital video device. With the aid of Elan (4.9.4), the recorded speech was segmented and transcribed for analysis.

\section{Focus marking in Līnàjùúl }\label{sec:bisilki:5}

Focus marking in Līnàjùúl requires the use of special particles dedicated for marking foci elements. The use of particles for focus marking is also sometimes described as morphological (\citealt{Childs1997, HartmannZimmermann2009, Schwarz2009, VanPutten2016}). \todo{This was listed as FlemmingRochemont1986. Please verify if something by Flemming is not missing.}\citet{Rochemont1986} equally demonstrate the use of prosodic resources and syntactic means, respectively in focus assignment. The focus particles in Līnàjùúl include: ń, ńká and a sentence final focus particle that assumes varying shapes, depending on the sentence final consonantal involved (this is discussed in detailed in \sectref{sec:bisilki:5.3}).

\subsection{Focus particle ń }\label{sec:bisilki:5.1}

The particle, ń is employed to focus-mark constituents in the utterances of Līnàjùúl speakers. It is worthy to note that the ń particle anticipatorily undergoes homorganic assimilation, giving it other variants as ḿ and ŋ՜ in speech:

\ea%8
    \label{ex:bisilki:8}
    \ea\label{ex:bisilki:8a}
        \ea\label{ex:bisilki:8ai}
        \gll    ŋmá	tárí		kīyá?\\
                \textsc{q}	shout\textsc{.ipfv}	like.that\\
        \glt    ‘Who is shouting like that?’
        \ex\label{ex:bisilki:8aii}
        \gll    Mákīnyì	ḿ	bì-sìí		ū-pú.\\
                Name		\textsc{foc}	\textsc{prog-}insult	\textsc{poss-}wife\\
        \glt    ‘Mákīnyì is insulting his wife.’
        \z
    \ex\label{ex:bisilki:8b}
    \gll    Ú-nìmpū		ŋ՜	kpá	kī-nyↄ՛k.\\
            \textsc{cl.sg-}woman	\textsc{foc}	have	\textsc{cl.sg-}mouth\\
    \glt    ‘A woman is talkative (in a quarrel).’
    \ex\label{ex:bisilki:8c}
    \gll    Ú-bↄ՛r		ń	yì		kī-tìŋ.\\
            \textsc{cl.sg-}chief	\textsc{foc}	own\textsc{.hab}	\textsc{cl.sg-}land\\
    \glt    ‘The chief owns the land.’
    \ex\label{ex:bisilki:8d}
    \gll    Bīmá		ń	yór		ń-dàn.\\
            \textsc{3pl.sbj}	\textsc{foc}	take\textsc{.pfv}	drink\\
    \glt    ‘They took (to take away) the drink.’
    \z
\z

From the examples in \REF{ex:bisilki:8}, it can be noted that ń is used to mark focus on sentence initial subjects. Ń in its sentence initial subject focus constructions is restricted to the immediate post subject slot before the canonical verb. Apart from simple subject constituents (nouns and pronominals) as shown in \REF{ex:bisilki:8} above, ń can also be used to place focus on complex subject NPs as the sentences in \REF{ex:bisilki:9} reveal:

\ea%9
    \label{ex:bisilki:9}
    \ea\label{ex:bisilki:9a}
    \gll    Ú-ŋóó	mέn		wé	ń	júú		lī-núúl.\\
            \textsc{cl.sg-}goat	red	\textsc{dem}	\textsc{foc}	bite\textsc{.prf}	\textsc{cl.sg-}yam\\
    \glt    ‘This red goat bit the yam.’
    \ex\label{ex:bisilki:9b}
    \gll    Bī-nìnkpíí-b	         bī-tī-ká-nà	    ḿ           bán	 ń-dàn.\\
            \textsc{cl.pl-}elder\textsc{-cl.pl}   \textsc{3pl-loc-}sit\textsc{-def}	   \textsc{foc}	      want\textsc{.ipfv}	 \textsc{cl-}drink\\
    \glt    ‘The elders sitting over there want a drink.’
    \z
\z

Ń as a focus particle cannot be placed in an intervening position in complex focal NPs, but comes immediately after the last complement of the complex NP (i.e., it is placed at the right most edge of the focus phrase [FocP]). The subject focus role of Līkpākpáln ń makes it analogous to a similar subject focus marker (ń) in Dagbani and Gurenɛ (\citealt[4]{Dakubu2003}; \citealt[169]{Issah2013Focus}\todo{Please verify if this is the correct Issah (2013)}; \citealt[5]{IssahSmith2018}; \citealt[169]{Akrofi-Ansah2014}\todo{Please check the accuracy of the currently-working reference links}). The Dagbani and the Gurune data in \REF{ex:bisilki:10a} and \REF{ex:bisilki:10b} respectively confirm this observation:

\ea%10
    \label{ex:bisilki:10}
    \ea\label{ex:bisilki:10a}
    Dagbani\\
    \gll    Abu	ń	dá		gbáŋ	máá.\\
            Abu	\textsc{foc}	buy\textsc{.pfv}	book	\textsc{def}\\
    \glt    ‘Abu bought the book.’ \hfill (\citealt[5]{IssahSmith2018})
    \ex\label{ex:bisilki:10b}
    Gurenɛ\\
    \gll    A-nɪ	n	zàa		nyɛ		bùdáa	lá.\\
            a\textsc{-wh}	\textsc{foc}	yesterday	see\textsc{.pfv}	man	\textsc{def}\\
    \glt    ‘Who saw the man yesterday?’ \hfill (\citealt[4]{Dakubu2003})
    \z
\z

The deletion of ń from a sentence in Līnàjùúl, nevertheless, does not render such a construction ungrammatical. As such, any of the sentences in \REF{ex:bisilki:9} above can be re-presented grammatically as in \REF{ex:bisilki:11}, except that these sentences become neutral in their contextual meanings:

\ea%11
    \label{ex:bisilki:11}
    \ea{\label{ex:bisilki:11a}
    \gll    Ú-ŋóó	mέn		wé	júú		lī-núúl.\\
            \textsc{cl.sg-}goat	red	\textsc{dem}	bite\textsc{.prf}	\textsc{cl.sg-}yam\\}\jambox*{[neutral]}
    \glt    ‘This red goat bit the yam.’
    \ex{\label{ex:bisilki:11b}
    \gll   Bī-nìnkpíí-b          bī-tī-ká-nà	     bán	           ń-dàn.\\
           \textsc{cl.pl-}elder\textsc{-cl.pl}  \textsc{3pl-loc-}sit\textsc{-def}   want\textsc{.ipfv}   \textsc{cl-}drink\\}\jambox*{[neutral]}
    \glt    ‘The elders sitting over there want a drink.’
    \z
\z

Ń cannot be used to mark focus on non-subject constituents. As earlier indicated, an attempt to re-position ń in any part of the sentence different from the immediate post canonical subject slot results in ungrammaticality of the sentence. This accounts for the unacceptable forms in \REF{ex:bisilki:12}:

\ea%12
    \label{ex:bisilki:12}
    \ea[*]{\label{ex:bisilki:12a}
    \gll    Ú-bↄ՛r		yì		ń	kī-tìŋ.\\
            \textsc{cl.sg-}chief		own\textsc{.hab}	\textsc{foc}	\textsc{cl.sg-}land\\
    \glt    ‘The chief owns the land.’}
    \ex[*]{\label{ex:bisilki:12b}
    \gll    Bīmá		yór		ń-dàn		ń.\\
            \textsc{3pl.sbj}	take\textsc{.pfv}	drink		\textsc{foc}\\
    \glt    ‘They took (to take away) the drink.’}
    \ex[*]{\label{ex:bisilki:12c}
    \gll    Ń	Bīmá		yór		ń-dàn.\\
            \textsc{foc}	\textsc{3pl.sbj}	take\textsc{.pfv}	\textsc{cl-}drink\\
    \glt    ‘They took (to take away) the drink.’}
    \z
\z

Ń equally serves both +contrastive and -contrastive focus functions. The specific context of utterance determines whether ń is used for emphasis or to code a meaning of contrast.

\subsection{The particle, ńká as a focus marker}\label{sec:bisilki:5.2}

Ńká is used to focus-mark only fronted non-subject constituents. In this case, a focus phrase (i.e. comprising both the focus particle and the focal constituent) must be placed extra-clausally. Extracting the focus particle only or the focal constituent only leads to a distortion of the grammaticality of the sentence. Although Līkpākpáln is not a Kwa language, the requirement that ńká necessarily collocates with its focal target in the extra-clausal position falls in with \citegen{Ameka2010} observation that in some Kwa languages, both a focus particle and the focalised element must be placed together in a fronted position. Ńká can be used to focus-mark objects as the sentences in \REF{ex:bisilki:13} show:

\ea%13
    \label{ex:bisilki:13}
    \ea\label{ex:bisilki:13a}
    \gll    Ú-nìmpū	ká		ŋáándέ	tī-kpēn	nέ.\\
            \textsc{cl-}woman	sit\textsc{.ipfv}	boil\textsc{.ipfv}	\textsc{cl-}soup	\textsc{foc}\\
    \glt    ‘A woman is preparing soup.’
    \ex\label{ex:bisilki:13b}
    \gll    Tī-kpēn	ńká	ú-nìmpū	ká	ŋáándέ.\\
            \textsc{cl-}soup	\textsc{foc}	\textsc{cl-}woman	sit	boil\textsc{.ipfv}\\
    \glt    ‘A woman is preparing soup.’
    \ex{\label{ex:bisilki:13c}
    \gll    Ú		bī-nyↄ՛		ń-dám.\\
            \textsc{3sg.sbj}	\textsc{prog-}take	\textsc{cl-}drink\\}\jambox*{[canonical]}
    \glt    ‘S/he is taking (drinking) a drink.’
    \ex\label{ex:bisilki:13d}
    \gll    Ń-dám	ńká	ú		bī-nyↄ՛.\\
            \textsc{cl-}drink	\textsc{foc}	\textsc{3sg.sbj}	\textsc{prog-}drink\\
    \glt    ‘S/he is taking (drinking) a drink.’
    \z
\z

Also, the sentences \REF{ex:bisilki:14b} and \REF{ex:bisilki:14d} below provide instances of ńká marking focus on an adjunct and an adpositional respectively.

\ea%14
    \label{ex:bisilki:14}
    \ea{\label{ex:bisilki:14a}
    \gll    Kónjà	lán	fúnī		dín.\\
            Name	will	arrive		today\\}\jambox*{[canonical]}
    \glt    ‘Kónjà will arrive (here) today.’
    \ex\label{ex:bisilki:14b}
    \gll    Dín		ńká	Kónjà		lán	fú-nī.\\
            Today		\textsc{foc}	Konja		will	arrive-\\
    \glt    ‘Konja will arrive (here) today.’\todo{Missing gloss for the final morpheme. \textsc{loc}?}
    \ex{\label{ex:bisilki:14c}
    \gll    Ú	bī-kↄ՛r	kī-sáá-k	nē.\\
            \textsc{3sg.sbj}	\textsc{prog-}weed	\textsc{cl-}farm	in\\}\jambox*{[canonical sentence]}
    \glt    ‘S/he is weeding inside the farm.’
    \ex\label{ex:bisilki:14d}
    \gll    Kī-sáá-k	nē	ńká	ú		bī-kↄ՛r.\\
            \textsc{cl-}farm\textsc{-cl}	in	\textsc{foc}	\textsc{3sg.subj}	\textsc{prog-}weed\\
    \glt    ‘S/he is weeding in the farm.’
    \z
\z

With reference to the sentences cited so far, one would also realize that it stands to say that fronting constituents for focus assignment with ńká does not trigger a resumptive pronoun in the base clause. Syntactically, ńká takes the slot immediately after its focal host, but must also precede the subject argument in the canonical clause position, which can either be a pronominal or a lexical subject.

Unlike the ń focus marker, a deletion of the ńká particle from a focus construction renders it ungrammatical, unless such a deletion is concomitant with a re-positioning of the focal constituent in its base position (in situ). As such, \REF{ex:bisilki:14b} and \REF{ex:bisilki:14d} become ill-formed constructions as appear in \REF{ex:bisilki:15a} and \REF{ex:bisilki:15b} below:

\ea%15
    \label{ex:bisilki:15}
    \ea[*]{\label{ex:bisilki:15a}
    \gll    Dín		Kónjà		lán	fú-nī.\\
            Today		Konja		will	arrive\textsc{-loc}\\
    \glt    ‘Konja will arrive (here) today.’}
    \ex[*]{\label{ex:bisilki:15b}
    \gll    Kī-sáá-k	nē	ú	bī-kↄ՛r.\\
            \textsc{cl-}farm\textsc{-cl}		in	\textsc{3sg.subj}	\textsc{prog-}weed\\
    \glt    ‘S/he is weeding inside the farm.’}
    \z 
\z 

Nevertheless, \REF{ex:bisilki:15a} and \REF{ex:bisilki:15b} would have well-formedness if there were in situ object placement alongside the deletion of ńká. Hence ,\REF{ex:bisilki:15a} and \REF{ex:bisilki:15b} as re-presented in \REF{ex:bisilki:16a} and \REF{ex:bisilki:16b} stand as grammatically correct sentences:

\ea%16
    \label{ex:bisilki:16}
    \ea\label{ex:bisilki:16a}
    \gll    Kónjà		lán	fú-nī		dín.\\
            Name		will	arrive\textsc{-loc}	today\\
    \glt    ‘Konja will arrive (here) today.’
    \ex\label{ex:bisilki:16b}
    \gll    Ú		bī-kↄ՛r 		kī-sáá-k	nē.\\
            \textsc{3sg.subj}	\textsc{prog-}weed 		\textsc{cl-}farm\textsc{-cl}	in\\
    \glt    ‘S/he is weeding inside the farm.’
    \z
\z

Discourse contextually, it was observed that ńká is mostly used for a contrastive focus function. It appears that when a non-subject constituent is to be focused -contrastively, a sentence final particle (discussed in \sectref{sec:bisilki:5.3}) is preferred while the other way around calls for ńká.

\subsection{Sentence final focus particle}\label{sec:bisilki:5.3}

There is a phenomenon in Līnàjùúl where a focus particle is placed sentence finally for the marking of focus, mostly, on post verbal constituents. This is as shown in the sentences in \REF{ex:bisilki:17} below:

\ea%17
    \label{ex:bisilki:17}
    \ea\label{ex:bisilki:17a}
    \gll    Ú	jóó		lī-kú-l		lέ.\\
            \textsc{3sg}	hold\textsc{.ipfv}	\textsc{cl-}hoe\textsc{-cl}	\textsc{foc}\\
    \glt    ‘S/he is holding a hoe.’
    \ex\label{ex:bisilki:17b}
    \gll    Ú-pìì	bī-ŋmáán	bī-sū-b		áá-fár		rέ.\\
            \textsc{cl-}sheep	\textsc{prog-}chew	\textsc{cl-}tree\textsc{-cl}	\textsc{gen-}leaves	\textsc{foc}\\
    \glt    ‘A sheep is chewing leaves of a tree.’
    \ex\label{ex:bisilki:17c}
    \gll    Ńtáánáá	chá		kī-sáá-k		kέ.\\
            Name		go\textsc{.ipfv}	\textsc{cl-}farm\textsc{-cl}		\textsc{foc}\\
    \glt    ‘Ńtáánáá is going to the farm.’
    \ex\label{ex:bisilki:17d}
    \gll    Ńtáánáá	gáá		bī-sū-b	bέ.\\
            Name		cut\textsc{.prf}	\textsc{cl-}tree\textsc{-cl}	\textsc{foc}\\
    \glt    ‘Ńtáánáá has cut a tree.’
    \ex\label{ex:bisilki:17e}
    \gll    Ńtáánáá	bī-ŋmán	ńtúúm			mέ.\\
            Name		\textsc{prog-}chew	beans			\textsc{foc}\\
    \glt    ‘Ńtáánáá is eating beans.’
    \ex\label{ex:bisilki:17f}
        \ea\label{ex:bisilki:17fi}
        \gll    Chákún	dↄ՛		lá?\\
                Cat		be.lie		where\\
        \glt    ‘Where is the cat lying?’
        \ex\label{ex:bisilki:17fii}
        \gll    Chákún	dↄ՛		lī-jà-l		tàáb	bέ.\\
            Cat		lie\textsc{-ipfv}	\textsc{cl-}chair\textsc{-cl}	under	\textsc{foc}\\
        \glt    ‘It is lying [under a chair]\textsubscript{\textsc{foc}}.’
        \z
    \z
\z

While έ remains invariant in all instances of the sentence final focus particle, the consonants have varied, depending on the final consonant segment(s) in the sentence final word of the constituent in focus. This, therefore, means that the sentence final focus particle is constructed by retaining a sentence final consonant (where sentence final consonant refers to the word-final consonant before the focus particle) and adding έ to it. One may then state that the shape of a sentence final focus marker in Līnàjùúl is phonologically conditioned. The influence of phonological environment on the choice of focus particles is also found in Sissali (a sister Mabia language spoken in Upper West Ghana). \citet{Dumah2017} show that in Sissali, when a focal constituent ends in a consonant, nέ is used for focus while rέ is used where such a constituent ends in a vowel. The following Sissali sentences in \REF{ex:bisilki:18} from \citet{Dumah2017} illustrate the phenomenon:

\ea%18
    \label{ex:bisilki:18}
    Sissali \citep[84]{Dumah2017}
    \ea\label{ex:bisilki:18a}
    \gll    Gyinaŋi	nέ ti /*rέ	Dùmà		sί	gύnnὶ wὺjίŋ.\\
            Today		\textsc{foc} {} {}		Dùmá 		\textsc{fut} 	 learn 	        lesson\\
    \glt    ‘Today (and not any other day) that Dùmà will learn a lesson.’
    \ex\label{ex:bisilki:18b}
    \gll    Daarii		rέ/*nέ ti	yↄ́bↄ̀	 tèŋ.\\
            Name		\textsc{foc} {}		buy	book\\
    \glt    ‘Daari (and not any other person) has bought a book.’
    \z
\z

\todo[inline]{Please complete the missing glosses above.}

The inappropriateness of *rέ in \REF{ex:bisilki:18a} is because the focused constituent ends in a consonant and the reverse accounts for *nέ in \REF{ex:bisilki:18b}. However, in the case of Līnàjùúl, when a post verbal focal constituent ends in a vowel, a focus particle is not used. Instead, there is an increase in the duration/extra lengthening of the final vowel (although this still requires an acoustic investigation to be more formally established).

The sentence final focus particle can be used to focalise both simple and complex non-subject constituents, including even entire VPs as can be seen from the examples in \REF{ex:bisilki:17}. \REF{ex:bisilki:19} specifically illustrates VP focus with the sentence final focus particle.

\ea%19
    \label{ex:bisilki:19}
    \ea\label{ex:bisilki:19a}
    \gll    Áá-jàpúán	ŋáá	bá?\\
            \textsc{gen-}son	did	\textsc{q}\\
    \glt    ‘What did your son do?’
    \ex\label{ex:bisilki:19b}
    \gll    Ú		jↄ՛n		bī-sū-b	bέ.\\
            \textsc{3sg.sbj}	climb\textsc{.pfv}	\textsc{cl-}tree\textsc{-cl}	\textsc{foc}\\
    \glt    ‘He [climbed a tree]\textsubscript{\textsc{foc}}.’
    \z
\z

In \REF{ex:bisilki:19b}, we see a sentence final particle, bέ used to mark focus on an entire VP. The scenario is that the speaker \REF{ex:bisilki:19a} saw the addressee \REF{ex:bisilki:19b} knock her son (addressee’s son) on the head. This prompted the speaker’s question, leading to the addressee’s response \REF{ex:bisilki:19b} in which the entire VP structure is in focus. It must also be reiterated that the sentence final focus particle mainly has a -contrast discourse function. Thus, it serves more to give relative emphasis or prominence to a particular constituent rather that to contrast.

Also, the non-use or the deletion of a sentence final focus particle does not make a sentence ungrammatical. In this sense, sentence final focus particles behave like particle ń discussed in \sectref{sec:bisilki:5.1}. The sentences in \REF{ex:bisilki:20} are a representation of (\ref{ex:bisilki:17a}-b), except that they are now neutral forms.

\ea%20
    \label{ex:bisilki:20}
    \ea\label{ex:bisilki:20a}
    \gll    Ú	jóó		lī-kú-l.\\
            \textsc{3sg}	hold\textsc{.ipfv}	\textsc{cl-}hoe\textsc{-cl}\\
    \glt    ‘S/he is holding a hoe.’
    \ex\label{ex:bisilki:20b}
    \gll    Ú-pìì		bī-ŋmáán	bī-sū-b	áá-fár.\\
            \textsc{cl-}sheep	\textsc{prog.}chew	\textsc{cl-}tree\textsc{-cl}	\textsc{gen-}leaves\\
    \glt    ‘A sheep is chewing the leaves of a tree.’
    \z
\z

Thus, in \REF{ex:bisilki:20} we find that the sentences with sentence final focus particle in (\ref{ex:bisilki:17a}-b) are represented as grammatical forms without the focus markers.

\section{Any combinatorial permissibility between the Līnàjùúl focus particles}\label{sec:bisilki:6}

A careful analysis of the Līnàjùúl focus particles affirm that, to a large extent, they have a complementary distribution in clauses or sentences. A co-habitation of any two of the focus particles in the same clause or even respectively in conjunct clauses usually results in a grammatically weird form as can be seen in the examples in \REF{ex:bisilki:21} below.

\ea%21
    \label{ex:bisilki:21}
    \ea[*]{\label{ex:bisilki:21a}
    \gll    Chákún	ń	chú		ú-námpúl	lέ.\\
            cat		\textsc{foc}	catch\textsc{.pfv}	\textsc{cl-}mouse	\textsc{foc}\\
    \glt    ‘A cat caught a mouse.’}
    \ex[*]{\label{ex:bisilki:21b}
    \gll    Kónjà	ń	pēn		í-līk		kē kūn 		kī-sáák	kέ.\\
            Name		\textsc{foc}	borrow\textsc{-pfv}	\textsc{cl-}money	\textsc{conj} farm\textsc{.pfv}	\textsc{cl-}farm\textsc{-cl}	\textsc{foc}\\
    \glt    ‘Kónjà borrowed money and used it to make a farm.’}
    \z
\z

\REF{ex:bisilki:21a} is a simple clause while \REF{ex:bisilki:21b} is a compound clause, yet a concurrent hosting of two focus particles is unacceptable in any of the cases.

\section{Focus marking in Līchábͻ́ l}\label{sec:bisilki:7}

Two particles, lé and lá have been identified as the focus markers in Līchábͻ́l \citep{Schwarz2009}. In \citeauthor{Schwarz2009}’s study, emphasis was more on establishing the divergent status of lé and lá in the Līchábͻ́l grammatical system. \citeauthor{Schwarz2009} appears to have ended on the following key conclusions, inter alia:

\begin{itemize}
    \item[(i)]  Lé marks focus on constituents in the preverbal field while the realisation of focus on post-verbal constituents is the preserve of lá.
    \item[(ii)] Lé and lá have a mutually exclusive occurrence in a simple sentence.\todo{Is there no (iii)?}
    \item[(iv)] Both lé and lá are non-obligatory in sentence structures as their omission does not engender ungrammaticality in the sentence.
\end{itemize}

Much as I accede to \citeauthor{Schwarz2009}’s arguments, one of my points of disagreement lies with his claim about the non-obligatoriness of lé and lá in the sentence \citep[184--185]{Schwarz2009}. In my observation, it is only lá which is possibly non-obligatory in every context of its use as a focus particle. Lé, on the other hand, has an obligatory use in the case of certain pronominal subjects (which I tentatively typologize as strong, disjunctive pronouns) and also in a situation where a non-subject is placed sentence initial as data in \REF{ex:bisilki:22} and \REF{ex:bisilki:23} suggest. Furthermore, a new dimension that I offer in the present analysis is that lá can also be used to lay a special emphasis on the entire proposition of a clause, rather than on only constituents within the clause. This is illustrated in example \REF{ex:bisilki:23}.

\subsection{Particle lé}\label{sec:bisilki:7.1}

When marking focus on a focal subject, both lé and the focused constituent are located within the canonical clause as can be seen in \REF{ex:bisilki:22a}, \REF{ex:bisilki:22b}, and \REF{ex:bisilki:22c} respectively. However, when it is a non-subject focal constituent, both particle lé and a focalized element are fronted as in \REF{ex:bisilki:22e}:

\ea%22
    \label{ex:bisilki:22}
    \ea\label{ex:bisilki:22a}
    \gll    Mákīnyì	lé	bì-sìí	ū-pú.\\
            Name	\textsc{foc}	\textsc{prog-}insult		his-wife\\
    \glt    ‘Mákīnyì is insulting his wife.’
    \ex\label{ex:bisilki:22b}
    \gll    Ú-píí			lé	kpá	b-ūmͻ́-b.\\
            \textsc{cl.sg-}woman	\textsc{foc}	have	\textsc{cl.sg-}mouth\textsc{-cl}\\
    \glt    ‘A woman is talkative (in a quarrel).’
    \ex\label{ex:bisilki:22c}
    \gll    Úmáá		lé	nyún		ń-dáán.\\
            \textsc{3sg}		\textsc{foc}	drink\textsc{.pfv}	\textsc{cl-}drink\\
    \glt    ‘He is the one who took a drink.’
    \ex{\label{ex:bisilki:22d}
    \gll    Ń		wáá		ú-pìì.\\
            \textsc{1sg.sbj}	see\textsc{.ipfv}	\textsc{cl-}sheep\\}\jambox*{[canonical sentence]}
    \glt    ‘I see a sheep.’
    \ex\label{ex:bisilki:22e}
    \gll    Ú-pìì		lé	ń		wáá.\\
            \textsc{cl-}sheep	\textsc{foc}	\textsc{1sg.sbj}	see\textsc{.ipfv}\\
    \glt    ‘A sheep is what I see.’
    \z
\z

Noteworthy is that whether in subject or non-subject focus, lé invariably occupies the immediate slot after the focal constituent as can be seen from the sentences \REF{ex:bisilki:22}. Lé is not placed in an intervening position within the complements of a focal constituent (i.e. in the case of a complex constituent), even when it is used for a narrow focus on only a part of the complex constituent, illustrated in \REF{ex:bisilki:23}. The question \REF{ex:bisilki:23a} shows that the focus is narrowed to only ŋì-lé (two). Yet the placement of the lé focus marker \REF{ex:bisilki:23b} remains positioned in the same place as would be the case if the entire NP, ŋì-tà  ŋì-lé ‘two tyres’ were in focus:

\ea%23
    \label{ex:bisilki:23}
    \ea\label{ex:bisilki:23a}
    \gll    ŋì-tà		ŋìŋá		pú?\\
            \textsc{cl-}tyre	many		spoilt\textsc{.pfv}\\
    \glt    ‘How many tyres got spoilt?’
    \ex\label{ex:bisilki:23b}
    \gll    (ŋì-tà)		ŋì-lé	lé	pú.\\
            \textsc{cl-}tyre	two	\textsc{foc}	spoilt\\
    \glt    ‘Two (tyres)  got spoilt.’
    \z
\z

Contrary to \citet{Schwarz2009}, lé is found to be obligatory in certain focus conditions. This occurs when certain strong, disjunctive pronouns take the subject position and also when a non-subject constituent is moved to the left periphery. The examples \REF{ex:bisilki:24} further illustrate the use of lé:

\ea%24
    \label{ex:bisilki:24}
    \ea\label{ex:bisilki:24a}
    \gll    ú-píí		lé	ń		jóó.\\
            \textsc{cl-}woman	\textsc{foc}	\textsc{1sg.sbj}	marry\textsc{.prs}\\
    \glt    ‘I am married to a woman.’
    \ex\label{ex:bisilki:24b}
    \gll    Min	lé	ŋmán	ŋí-tùùn.\\
            \textsc{1sg.sbj}	\textsc{foc}	eat\textsc{.pfv}	\textsc{cl-}beans\\
    \glt    ‘I ate beans.’
    \ex\label{ex:bisilki:24c}
    \gll    Tìmīn		lé	jín		bī-sáá.\\
            \textsc{1pl.sbj}	\textsc{foc}	eat\textsc{.pfv}	\textsc{cl-}food\\
    \glt    ‘WE ate (the) food.’\todo{Is "WE" intentional? Perhaps it could be [We]\textsubscript{FOC} for consistency.}
    \z
\z

It is a result of the obligatory status of lé in contexts \REF{ex:bisilki:24} that the sentences \REF{ex:bisilki:24} become ungrammatical as re-presented \REF{ex:bisilki:25} below:

\ea%25
    \label{ex:bisilki:25}
    \ea[*]{\label{ex:bisilki:25a}
    \gll    Ú-píí	ń	ń	jóó.\\
            \textsc{cl-}woman	\textsc{1sg} {}	marry\textsc{.prs}\\
    \glt    ‘I am married to a woman.’\todo{Please check the gloss}}
    \ex[*]{\label{ex:bisilki:25b}
    \gll    Min		ŋmán		ŋí-tùùn.\\
            \textsc{1sg.sbj}	eat\textsc{.pfv}	\textsc{cl-}beans\\
    \glt    ‘I ate (the) beans.’}
    \ex[*]{\label{ex:bisilki:25c}
    \gll    Tìmīn	jín		bī-sáá.\\
            \textsc{1pl.sbj}	eat\textsc{.pfv}	\textsc{cl-}food\\
    \glt    ‘We ate (the) food.’}
    \z
\z

Lé \REF{ex:bisilki:25a} becomes necessary because of the fronted object (a non-subject constituent). Similarly, lé is indispensable (\ref{ex:bisilki:25b} and \ref{ex:bisilki:25c}) because of the particular pronominal subjects involved.

\subsection{Particle lá }\label{sec:bisilki:7.2}

The particle, lá as a focus marker in Līchábͻ́l is constrained to sentence final position in a similar way as the sentence final focus particle in Līnàjùúl. In its post canonical verb position, lá is immediately postposed to the constituents that it focus-marks. That is, the element in focus precedes lá in terms of nearness to the canonical verb. Lá can be used to mark focus on any non-subject constituent as can be noted \REF{ex:bisilki:26} below:

\todo[inline]{The subexamples in \REF{ex:bisilki:26} have been rearranged. There are 3 Q-R pairs, which were previously arranged as aQ, aR, bQ, bR, cQ, and dR. They are now rearranged as a to c with two subexamples each for the Q-R pairs.}

\ea%26
    \label{ex:bisilki:26}
    \ea\label{ex:bisilki:26a}
        \ea\label{ex:bisilki:26ai}
        \gll    Mbá	yúnl	nē		fī	cháá?\\
                \textsc{q}	time	\textsc{2pl.sbj}	\textsc{trm}	go\textsc{.ipfv}\\
        \glt    ‘What time are you leaving tomorrow?’
        \ex\label{ex:bisilki:26aii}
        \gll    Tī		gē	fī	búén	máláá		lá.\\
                \textsc{1pl.sbj}	\textsc{fut}	\textsc{trm}	go\textsc{.ipfv}  early	\textsc{foc}\\
        \glt    ‘We will be going early.’
        \z
    \ex\label{ex:bisilki:26b}
        \ea\label{ex:bisilki:26bi}
        \gll    Bá	ákēkēln	ú	gbáb?\\
                \textsc{q}	cloth		s/he	wear\textsc{.ipfv}\\
        \glt    ‘What type (colour) of cloth is s/he wearing?’
        \ex\label{ex:bisilki:26bii}
        \gll    Ú	gbáb	lī-kēkē	mēnl		lá.\\
                \textsc{3sg.sbj}	wear\textsc{.ipfv}	\textsc{cl-}cloth	red	\textsc{foc}\\
        \glt    ‘S/he is wearing a red cloth.’
        \z
    \ex\label{ex:bisilki:26c}
        \ea\label{ex:bisilki:26ci}
        \gll    Ú-jà		wé	bī-ŋánì	bá.\\
                \textsc{cl-}man	\textsc{dem}	\textsc{prog-}do	\textsc{q}\\
        \glt    ‘What is this man doing?’
        \ex\label{ex:bisilki:26cii}
        \gll    Ú		bī-máá		kī-díí-k		lá.\\
                \textsc{3sg.sbj}	\textsc{prog-}building	\textsc{cl-}room\textsc{-cl}		\textsc{foc}\\
        \glt    ‘He [is building a house]\textsubscript{\textsc{foc}}.’
        \z
    
    \z
\z

Lá \REF{ex:bisilki:26aii}, is used to mark focus on an adverbial. Lá marks focus on an adjective and \REF{ex:bisilki:26cii} marks focus on a complex VP.

Furthermore, lá also occurs when the element of focus is just the verb. \REF{ex:bisilki:27} is an example to this effect:

\ea%27
    \label{ex:bisilki:27}
    \ea\label{ex:bisilki:27a}
    \gll    Lá	bī		dá		ídↄ՛?\\
            \textsc{q}	\textsc{3pl.sbj}	buy\textsc{.pfv}	wood\\
    \glt    ‘Where did they buy the (fire)wood?’
    \ex\label{ex:bisilki:27b}
    \gll    Bī		sūn		lá.\\
            \textsc{3pl.sbj}	steal\textsc{.pfv}	\textsc{foc}\\
    \glt    ‘They stole it.’
    \z
\z

Additionally, \citet{Schwarz2009} hints of the fact that lá can be used to add a kind of emphasis to the meaning of a focal constituent. Perhaps, a further discovery, the present study brings on board that such emphasis by lá can also apply to the meaning of the sentence. This is observed to happen when, in discourse, a speaker wants to be sarcastic or, in earnest, indicate that the idea or situation being stated is beyond the ordinary. An example \REF{ex:bisilki:28} below illustrates this:

\ea%28
    \label{ex:bisilki:28}
    \gll    Jàgrì	kpↄ՛	ŋì-mↄ՛bìl	lá.\\
            Jàgrì	has	\textsc{cl-}money	\textsc{foc}\\
    \glt    ‘[Jàgrì has money]\textsubscript{\textsc{foc}}.’
\z

The discourse-contextual interpretation of the sentence \REF{ex:bisilki:27} is not to emphasize or contrast only a portion of the sentence. Rather, the contextual meaning is that “Jàgrì is, indeed, rich or he is richer than the ordinary.” One must also note that in cases \REF{ex:bisilki:28}, lá is still retained in the sentence. An interesting commonalty about every context use of the lá focus marker, is its optionality in the sentence. Hence, example \REF{ex:bisilki:28} and \REF{ex:bisilki:27b} are still grammatically correct (although their contextual meanings may become inappropriate) without lá \REF{ex:bisilki:29}.

\ea%29
    \label{ex:bisilki:29}
    \ea\label{ex:bisilki:29a}
    \gll    Jàgrì	kpↄ՛	ŋì-mↄ՛bìl.\\
            Jàgrì	has	\textsc{cl-}money\\
    \glt    ‘Jàgrì has money.’
    \ex\label{ex:bisilki:29b}
    \gll    Bī		sūn.\\
            \textsc{3pl.bj}		steal\textsc{.pfv}\\
    \glt    ‘They stole.’
    \z
\z

Finally on lá, \citet{Schwarz2009} acknowledges that there are similar particles like lá in Līchábͻ́l, but with different functions. Possibly, a more appropriate way to put this is to say that there are homophonous Lás in Līchábͻ́l-Kpakpaln. There is a focus marking lá and there is also an interrogative particle lá, meaning roughly “where” (see, for instance, data example \ref{ex:bisilki:27a}).

\section{Highlights of focus marking divergences between Līnàjùúl and Līchábͻ́l
}\label{sec:bisilki:8}

The foregoing discussions (in sections above) reveal that Līnàjùúl and Līchábͻ́l have intriguing similarities as well as differences, with respect to the phenomenon of focus marking. In the first place, the two dialects use special focus markers (in this case focus particles) for marking foci constituents. To that extent, both Līnàjùúl and Līchábͻ́l conform to the common linguistic phenomenon, where the focus systems of Mabia languages involve the use of focus marking particles. Nonetheless, whereas Līnàjùúl has three particles (Ń, ńká and a clause final particle of varying shapes), lé and lá are the only particles used for coding focus in Līchábͻ́l. However, Līnàjùúl further appears to draw on the prosodic feature of duration/sentence final vowel lengthening for focus assignment (see \sectref{sec:bisilki:5.3}), whereas this does not occur in Līchábͻ́l. This means that while focus marking remains mainly morphological in Līchábͻ́l, Līnàjùúl has both morphological and prosodic strategies for marking focus.

There is the temptation to state that the focus marking differences between Līnàjùúl and Līchábͻ́l owe to the fact that Līnàjùúl has innovated a more complex focus system, while also bearing decadence in that regard. This comes up somewhat clearly when one considers the focus marking system of Līnàjùúl vis-a-vis the larger Mabia framework. It can be said, for instance, that the use of prosody and a phonologically conditioned sentence final focus particle of varying shapes is currently not known to be prevalent among the Mabia languages. What comes close to the latter case in Līnàjùúl is the occurrence in Sissali where the focus markers, nέ and rέ alternate depending on whether the focalized constituent ends in a consonant or a vowel \citep{Dumah2017}. Also, lá which is a prevalent focus particle in the focus systems of many Mabia languages of Ghana, such as Dagbani, Dagaare, Moore, Kusaal, Mampruli (\citealt[93]{Bodomo1997}; \citealt{Dakubu2003, Issah2013Focus, Saanchi2005}) is synchronically not used for focalization in Līnàjùúl. The only trace of lá in Līnàjùúl is its use as a question particle (see example \ref{ex:bisilki:17f}).

There is a possibility that the lá focus marker existed in Līnàjùúl at a certain point, but only synchronically got lost due to linguistic evolution over time. The innovations presently noted in the Līnàjùúl focus system correlates with a pattern recently found with its noun class system (\citet{BisilkiAkpanglo-Nartey2017}). In a study of Līnàjùúl noun classes, \citet{BisilkiAkpanglo-Nartey2017} similarly found that Līnàjùúl is evolving further away from the prototypical Gurma noun class characteristics.

It must, however, be indicated that despite the attested focus marking incongruences between Līnàjùúl and Līchábͻ́l, no comprehension challenges obtain between the native speakers of these variants. Thus, the degree of mutual intelligibility between Līnàjùúl and Līchábͻ́l is high enough to warrant smooth intercommunication between their respective speakers.

\section{Conclusion}\label{sec:bisilki:9}

I have examined some salient aspects of focus marking in Līkpākpáln. In particular, I have discussed focus marking strategies and the syntax of focus constructions in the language. Perhaps, more in the article is comparable to the focus marking systems of the Līnàjùúl and the Līchábͻ́l dialects of Līkpākpáln. The comparison (see \sectref{sec:bisilki:8}) has revealed some generic similarities, but more intriguing divergence in the shape of number of focus particles. Within the Mabia focus systems, Līnàjùúl was also found to be a bit more diversified by using phonologically conditioned sentence final focus markers. Yet this finds a kind of analogous pattern in a sister Mabia language, Sissali (see \sectref{sec:bisilki:5.3}).

Another note of dissociation with Līnàjùúl is the non-focus function of lá, which is a common focus marker in several Mabia languages.

Finally, I recommend that investigation of focus marking in other dialects of Līkpākpáln be undertaken. This will help establish whether the focus system of Līnàjùúl is truly an isolate innovation or if the pattern is a shared linguistic tendency in Līkpākpáln. Similarly, more thorough studies on the phonological possibilities in focus marking in Mabia, needs to be pursued. Both the cases of Sissali and the Līnàjùúl dialect of Līkpākpáln raise this interest.

\section*{Abbreviations}
\begin{tabularx}{.45\textwidth}{>{\scshape}lQ}
cl & (Noun) class\\
gen & Genitive\\
cond & Conditional marker\\
conj & Conjunction\\
dem & Demonstrative\\
foc & Focus marker\\
fut & Future\\
hab & Habitual\\
ipfv & Imperfective\\
loc & Locative\\
\end{tabularx}
\begin{tabularx}{.45\textwidth}{>{\scshape}lQ}
obj & Object\\
pfv & Perfective\\
pl & Plural\\
prf & Perfect\\
prog & Progressive\\
prs & Present\\
q & Question marker\\
q & Question\\
r & Response\\
sbj & Subject\\
sg & Singular\\
sid & Subject identity\\
trm & Time reference marker\\
\end{tabularx}

\section*{Acknowledgements}



\printbibliography[heading=subbibliography,notkeyword=this]

\end{document}
